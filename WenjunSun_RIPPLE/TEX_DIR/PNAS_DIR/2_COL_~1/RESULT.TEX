%\section{Results and Discussion}

\vspace{0.128in}
\noindent
{\bf Results and Discussion}
\vspace{0.128in}

The absolute form factors $|F(h,k)|$ reported by Wack and Webb \cite{Wac89a} 
are shown in Table I along with two of our many model fits.
Our most striking result is that all of our models have the same
phases for all the (h,k) reflections displayed in Table I,
except for the pure ripple
reflections with h=0.  Figure 2 shows the Fourier map
of the electron density using the measured \cite{Wac89a} absolute form factors
$|F(h,k)|$, omitting the pure ripple reflections (0,k),
and our phases shown in Table I.
Since the F(0,k) are rather small, their inclusion with any phase
combination makes rather small differences to the electron
density map and negligible differences to the ripple profile.

The ripple profile is shown in Fig. 2 by the dotted lines
that are defined as the loci of maximum intensity in the headgroup
region.  This ripple profile clearly has the 
basic triangular shape
assumed in our modeling, but this

\begin{table}
\caption{Form factors.
\label{formfactor}}
\vspace{6pt}
\begin{tabular}{rrrrrr} 
h & k & 100$\times$q & Data$^{1}$ &
        \multicolumn{2}{c}{Model F(h,k)} \\ \cline{5-6}
& & (\AA$^{-1}$) & $|$F(h,k)$|$ & SDF & M1G \\ \hline
0 &  1 & 4.48  &  5.3  &  -1.4  &  7.9  \\
0 &  2 & 8.96  & 9.7  &  1.0  &  -4.6 \\
0 &  3 & 13.4  &  7.8  &  -0.4   &  0.9 \\
1 &  -2 & 13.0  &  ---  &  5.0  &  6.7   \\
1 &  -1 & 11.1  &  60.8  &  -42.8  &  -60.4 \\
1 &   0 & 10.8  &  100.0  &  -96.7  &  -99.1 \\
1 &   1 & 12.3  &  26.9  &  38.1  &  28.5 \\
1 &   2 & 15.0  &  ---  &  -23.6  &  -11.8 \\
1 &   3 & 18.5  &  7.6  &  13.9  &  4.8 \\
1 &   4 & 22.3  &  ---  &  -6.3  &  -2.4 \\
2 &  -3 & 23.8  &  ---  &  8.6  &  10.9 \\
2 &  -2 & 22.2  &  15.1  &  -9.0  &  -14.1 \\
2 &  -1 & 21.5  &  71.2  &  -66.9  &  -71.7 \\
2 &   0 & 21.7  &  39.7  &  -41.0  &  -39.9 \\
2 &   1 & 22.8  &  33.9  &  31.1  &  30.6 \\
2 &   2 & 24.6  &  22.7  &  -24.3  &  -22.8 \\
2 &   3 & 27.1  &  14.2  &  17.1  &  15.2 \\
2 &   4 & 30.1  &  7.8  &  -10.3  &  -9.1 \\
3 &  -3 & 33.3  &  ---  &  -6.1  &  -6.0 \\
3 &  -2 & 32.5  &  29.3  &  22.3  &  29.0 \\
3 &  -1 & 32.2  &  44.2  &  42.2  &  44.5 \\
3 &   0 & 32.5  &  12.0    &  2.2  &  0.7   \\
3 &   1 & 33.5  &  ---  &  -11.3  &  -10.5 \\
3 &   2 & 35.0  &  10.5  &  14.9  &  13.9 \\
3 &   3 & 37.0  &  14.9  &  -15.0  &  -13.4 \\
3 &   4 & 39.4  &  10.0  &  12.5  &  10.8 \\
4 &  -2 & 43.0  &  ---  &  -81.1  &  -39.2$^{2}$ \\
4 &  -1 & 43.0  &  ---  &  -63.5  &  -23.6$^{2}$ \\
4 &   0 & 43.4  &  ---  &  26.1  &  10.3$^{2}$ \\
\end{tabular}
{\small
\indent 1. From Wack and Webb [18].

\indent 2. Values are consistent with the densitometer trace shown in 
        Fig. 4 in [18].
}
\end{table}

\begin{table}
\caption{Ripple structural quantities.
\label{parameter}}
\vspace{6pt}
\begin{tabular}{ccccccc} 
Model & SDF & MDF & M1G \\ \hline
A(\AA) & 18.6 & 20  & 19 \\
x$_0$(\AA) & 103 & 103 & 103 \\
$\psi$($\deg$) & 5 & 9 & 9 \\
X$_h$(\AA) & 20.1 & 20.4 & 19.3 \\
f1 & 1(fixed) & 0.7 & 0.6 \\
f2 & 0(fixed) & -2 & -1 
\end{tabular}

\end{table}


\begin{figure}
\centerline {\psfig{figure=map_wack_34.ps,width=3.25in}}
\vspace{11pt}
\caption{Electron density map using measured absolute form factors 
[18] and phases from Table I omitting (0,k) reflections.  The smallest 
electron density is pure black and the largest electron density is pure 
white with a linear grey scale interpolating between.  The black dotted 
lines show two of the loci of maximum electron density. The electron 
density along the dashed lines A, B and C are shown in a conventional plot 
in Fig. 3.  The parallelogram shows the unit cell whose width is $\lambda_r$ 
which equals 141.7\AA.
\label{Fig2}}
\end{figure}

\noindent 
result is not tautological because the experimental 
absolute form factors are employed in Fig. 2.
The ripple profile shown in Fig. 2 is in excellent quantitative
agreement with the fitted model ripple profiles.  The parameters
A and $x_0$ that describe the ripple profile, as well as other
parameters in the models, are given in Table II for three models.
From this we conclude that the ripple amplitude is about 19\AA\
and the projection of the major 
side on the a axis is about 103\AA\
for the single DMPC sample for which Wack and Webb \cite{Wac89a}
reported absolute form factors.

To obtain a more quantitative picture of the thickness of the
bilayer in various parts of the ripple, electron density profiles
(using the experimental $|F(h,k)|$)
are plotted in Fig. 3 along the three slices shown by short straight dashed
lines in Fig. 2 and labelled A, B and C.
Slice A is along the normal of the major side and is centered in the
middle of the hydrocarbon region; it indicates that the head-head 
spacing is 38\AA\ and the water spacing is 21\AA\ on the major side.
Slice B is along the normal to the minor side and centered
in the middle of the hydrocarbon region; it indicates that
the bilayer head-head spacing on the minor side is 31\AA.
Slice C is along the normal to the minor side, but centered in
the water region; it indicates that the water spacing of the
minor side is 18\AA.
All these spacings are, of course, subject to Fourier truncation
errors since the intensity data only include up to three lamellar orders 
(h=1-3).

We also performed x-ray experiments on a DMPC sample with almost identical
hydration level as the data in  

\begin{figure}[t]
\centerline {\psfig{figure=gel_slc.ps,width=3.00in}}
\vspace{11pt}
\caption{Solid curves in A, B and C show electron density along the three 
dashed lines in Fig. 2 labelled by A, B and C.  The dotted curve in
A shows the electron density profile determined for the gel phase.
The dashed and dash-dot curves in B and C show the electron density when
the (0,k) reflections with phases $(-+-)$ and $(+-+)$, respectively, are
added.
\label{Fig3}}
\end{figure}

\noindent
\cite{Wac89a}.
The sample was partially dehydrated by a 40\% polyvinylpyrrolidone
(PVP) aqueous solution.  At 18$^{\circ}$C
this sample was in the ripple phase, with a d-spacing of 58.1\AA, comparing 
well with Wack and Webb's 57.94\AA.  We then took
low angle data at 6$^{\circ}$C at which the sample was clearly in the
gel phase with no ripple peaks present. The intensities of five 
orders $h=1-5$ were
recorded, but the electron density profile was reconstructed using
the first three orders only so as to form a valid comparison 
with the ripple phase electron densities that were obtained from 
experimental data limited to $h=1-3$.  
The dotted line in Fig. 3A shows this comparison.
The gel phase profile gives a head-head bilayer thickness of 37.0\AA\ and
a water spacing of 20.8\AA, agreeing well with 37.9\AA\ and 20.7\AA,
respectively, obtained for the ripple phase.

The preceding result strongly suggests that the lipid bilayer along the major
side of the ripple is similar to the gel phase bilayer.
In contrast, the bilayer along the minor side appears to be
thinner and this suggests the working hypothesis that the
minor side may be more like the fluid $L_{\alpha}$ bilayer.
A supporting argument for this hypothesis is that the electron density 
in the headgroup region along the minor side appears to be lower than 
along the major side as shown by the solid curves in Fig. 3B.  
This supporting argument is hardly affected by inclusion of the
F(0,k) for the two sets of phases $(-+-)$ and $(+-+)$ for the ripple 
reflections shown in Table I.  Inclusion makes too little difference 
to show in Fig. 3A for the major side.  For the minor side the 
$(-+-)$ phase set increases the height of the head groups, but not
enough to equal the height for the headgroups on the
major side, and not relative to the lower electron density of the 
chain region.

Wack and Webb \cite{Wac89a} argued that the (0,k) phases are $(+++)$ for 
k=1,2 and 3,
although the origin of their unit cell is translated by $\vec{b}/2$
compared to our unit cell in Fig. 1, so their
phases are equivalent to our phase set $(-+-)$ which we obtain
from the SDF model as shown in Table 1.  However, the small magnitudes
of the model F(0,k) motivated us to try the MDF and M1G models that
deliberately modulate the total electron density in the direction 
perpendicular to $\vec{a}$ in Fig. 1.  (This direction is also very nearly
along the major side since it makes an angle of about 10 degrees
with the b-axis which is nearly the same as $\gamma - 90^{\circ}$.)
The results for the M1G model given in Table 1, as well as for
the MDF model (not shown), show a dramatic
improvement in the fit to most of the F(h,k), not just the F(0,k), with
a factor of four decrease in the sum of the squares of the residuals.
(The crystallographic R-value is 0.195 for the SDF model and 
0.083 for the M1G model.)  It may also be noted that the (0,3) reflection 
has nearly the same q-value as the (1,-2) reflection.  Reassignment of the 
larger part of the value 7.81 for $|F(0,3)|$ in Table 1 to $|F(1,-2)|$ would
further improve the fit.
However, the set of ripple phases reverses to $(+-+)$ for both the M1G and
the MDF model and this result is
unchanged by using different starting values in the nonlinear
least squares fitting program.   We therefore favor the phases $(+-+)$
for the ripple reflections, although we recognize that none of
the models we have considered readily accommodates $|F(0,2)|$ larger
than $|F(0,1)|$.

Although the phases for the pure ripple (0,k) reflections
are not as well determined by our procedure as the h$\neq$0 phases,
the latter phases, which are very robustly determined using
our model method, already determine the ripple profile.  
Indeed, the effect of different (0,k) phases on the electron density
maps is less important than the effect of including the three h=4 
reflections shown in Table I for the M1G model.
Further progress toward understanding the molecular packing of lipids in the 
ripple phase should employ careful analysis of wide angle scattering data.
This study establishes the framework into which such additional information
must be incorporated.

\vspace{0.1in}
\small
Acknowledgment: This research was supported by the National Institutes 
of Health Grant GM44976.

