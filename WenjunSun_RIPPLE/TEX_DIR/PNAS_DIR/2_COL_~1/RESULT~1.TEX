\section{Results}

Parameters for the model ripple phase structure obtained from the fit
are shown in Table \ref{parameter}.  Additional ways to describe the degree
of the ripple asymmetry are the comparison of
the slope of the minor side ($\beta$=26$^{\circ}$) to the slope of
the major side ($\alpha$=10$^{\circ}$) and the 
ratio of the major side length to the minor side length; the latter
is 2.4 for the delta function transbilayer model and 2.6 for the hybrid
Gaussian model.

Table \ref{formfactor} compares the form factors obtained from the fit to
the delta function model and the hybrid transbilayer model.
Although there are differences between the magnitudes of the form factors 
for these two rather different transbilayer models, the phases are the same.  
In order to test the robustness of these phases 
additional variations of the models were used to fit the same data.
For example, when the monolayer shift angle, $\psi$, was set to either zero or
to the value of $\alpha$ (the slope of the major side), 
similar values of fitting
parameters and identical phases were obtained (not shown), although
the fits were slightly worse due to the constraints. 
(There will be more here, later.)

Table \ref{formfactor} also compares these model fits to
the absolute form factors obtained by Wack and Webb \cite{Wac89a}.
Although there are clearly differences, the general pattern of the
absolute form factors is the same except for the three purely ripple
peaks (0,k).  This supports using 
the model phases in Table \ref{formfactor} with the absolute form
factors of Wack and Webb in Table \ref{formfactor} to obtain
an electron density map.  This map is shown in Fig. \ref{wack_map}(A).

To obtain a more quantitative picture of the thickness of the
bilayer in various parts of the ripple, electron density profiles
are obtained along the three slices shown in Fig. \ref{wack_map}(A) 
and these are plotted in Fig. \ref{slice}.  (Remove the model
profiles from this figure.)  Slice B is along the normal of the
major side and indicates that the head-head spacing is 38\AA
and the water spacing is 21\AA.  For the minor side, two slices are taken,
one going through the center of the chain region (Slice C in
Fig. \ref{slice}) and another through the center of the water
region (Slice D in Fig. \ref{slice}). The electron density
profiles along slices C and D are shown in Fig. \ref{slice}(C) and
(D). The bilayer head-head spacing on the minor side is 31\AA, smaller
than that of the major side (37\AA). The water spacing of minor side
is 18\AA, also smaller than that of the major side (21\AA). 
All these spacings are, of course, subject to Fourier truncation
errors since the intensity data only include
up to three lamellar orders (k=1-3).

We also performed x-ray experiments on a DMPC sample with almost identical
hydration level as in Wack and Webb's \cite{Wac89a}. 
The sample was partially dehydrated by a 40\% Polyvinylpyrrolidone
(PVP) aqueous solution.  At 18$^{\circ}$C
this sample was in the ripple phase, with a d-spacing of 58.10\AA, comparing 
well with Wack and Webb's 57.94\AA.  We then took
low angle data at 6$^{\circ}$C at which the sample was clearly in the
gel phase with no ripple peaks present. The intensities of five orders were
recorded, but the electron density profile was reconstructed using
the first three orders only so as to form a valid comparison 
with the ripple profiles limited experimentally to three orders.  The dashed line in 
Fig. \ref{slice}(B) shows this comparison.
The gel phase profile gives a head-head bilayer thickness of 36.96\AA\ and
a water spacing of 20.79\AA, agreeing well with 37.86\AA\ and 20.73\AA,
respectively, obtained for the ripple phase.
\pagebreak
