%\section{Introduction}

\vspace{0.128in}
\noindent
{\bf Introduction}
\vspace{0.128in}

Lipids in water self-assemble into lamellar bilayers, which comprise the basic
structural element of biomembranes.   The supramolecular packing of the
bilayers, in turn, forms lyotropic liquid crystals with rich phase behavior
and diverse structures.  Among the thermodynamic phases observed in
lipid bilayer systems, the periodically modulated ripple (P$_{\beta '}$) phase
is especially fascinating as evidenced by the many theoretical papers that
have attempted to explain the formation of the ripples
\cite{Don79,Mar84,Haw86,Car87,Gol88,McCu90,Hon91,Lub93}. 
Such understanding has been delayed, however, because the structure has
not been as well characterized as for the lower temperature gel
(L$_{\beta '}$) phase and the higher temperature liquid crystalline
(L$_{\alpha}$) phase, despite many experimental studies 
\cite{Tar73,JanSS79,Luna77,Ino80,Cop80,Sta82,Rup83,Ale85,Sir88,%
Wac89a,Kat95}.  When an accurate ripple
structure is known, some indication of the energetics of the ripple formation
may be obtained by comparing the relative size of the ripple
wavelength and the ripple amplitude to the size of lipid molecules and
the bilayer thickness \cite{Gol88}.  Also, correlation 
between the chirality of the constituent lipid molecules and the asymmetry of 
the ripples may lead to elucidation of the microscopic origin 
of the macroscopic properties of the ripple phase \cite{Lub93,Kat95}.

Structural information about the ripple phase can be divided into two
categories. The first category consists of the 2D lattice parameters, namely, 
the ripple wavelength $\lambda_r$, the bilayer packing repeat distance $d$ and
the oblique angle $\gamma$ for the unit cell that are illustrated
in Fig. 1.  These well-established parameters have been accurately
obtained by indexing low-angle x-ray scattering peaks
\cite{Tar73,JanSS79,Ino80,Ale85,Wac89a}. 
For example, Wack and Webb \cite{Wac89a} reported for DMPC with
25\% water by weight at 18$^{\circ}$C that $\lambda_r$ = 141.7\AA, $d$ = 57.94\AA\
and $\gamma$ = 98.4$^{\circ}$.  The second category
consists of the structure within the ripple unit cell, including
the shape and the amplitude of the ripples and the bilayer thickness. 
This second category of structural information
has been quite uncertain.  Many ripple profiles have been 
proposed \cite{Don79,Mar84,Car87,Gol88,McCu90,Hon91,Lub93,Tar73,JanSS79,Sta82},
including sinusoidal, peristaltic and sawtooth and
estimates of the ripple trough-to-peak 
amplitude range widely from 15\AA\ to 50\AA\ 
\cite{Tar73,JanSS79,Sta82,Zas88a,Hata93}.

The second category of information involves electron density maps.  These
require the amplitudes of the form factors, which are just
the square roots of the Lorentz corrected intensities of the x-ray scattering peaks,
and the correct phases for these measured form factor amplitudes.
A traditional approach to solve the phase problem is the ``pattern recognition''
method employed by Tardieu et al. \cite{Tar73}; that method
selects those phase combinations which will give rise to electron density maps 
agreeing with known physical properties of the bilayer system.
Considering that there are typically 20 or so 
peaks in most of the ripple phase x-ray data, finding the right phase 
combination out of about 2$^{20}$ possibilities is a difficult task. 
The approach we employ involves modeling and nonlinear least square fitting, to 
which we now turn.

