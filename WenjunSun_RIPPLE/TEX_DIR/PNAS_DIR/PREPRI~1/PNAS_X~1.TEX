\documentstyle[aps,psfig,preprint]{revtex}
\begin{document}
\bibliographystyle{prsty}

\noindent
{\large \bf Biophysics \hspace{4in} MS XD2575}

\vspace{1.5in}

\begin{center}
{\Large \bf Structure of the Ripple Phase in Lecithin Bilayers}

(biomembranes/phospholipids/x-ray diffraction)

{W.-J. Sun$^{*}$, S. Tristram-Nagle$^{\dag}$, R. M. Suter$^{*}$ and J. F.
 Nagle$^{* \dag \ddag}$}

\textit{Department of Physics$^{*}$ and Department of Biological 
Sciences$^{\dag}$,
\protect\\ Carnegie Mellon University , Pittsburgh, PA 15213 }

(Revised March 22, 1996)

\end{center}

\vspace{4.0in}

\noindent
\begin{picture}
\put(0,0){\line(1,0){350}}
\end{picture}

{\footnotesize
\noindent Abbreviations: DMPC, dimyristoyl-phosphatidylcholine.

\noindent $^{\ddag}$To whom reprint requests should be addressed.
}

\pagebreak

% ********************** abst.tex ***********************************
\begin{abstract}

The phases of the x-ray form factors are derived for the ripple (P$_{\beta '}$)
thermodynamic phase in the lecithin bilayer system. By combining these phases 
with experimental intensity data, the electron density map of the 
ripple phase of DMPC is constructed. The phases are derived by fitting 
the intensity data to 2D electron density models, which are created
by convolving an asymmetric triangular ripple profile with a transbilayer 
electron density profile.  The robustness of the model method
is indicated by the result that many different models of the
transbilayer profile yield essentially the same phases, except
for the weaker, purely ripple (0,k) peaks.  Even with this
residual ambiguity, the ripple profile is well determined, resulting in
19\AA\ for the ripple amplitude and 10$^{\circ}$ and 26$^{\circ}$ 
for the slopes of the major and the minor sides,
respectively.   Estimates for the bilayer head-head spacings
show that the major side of the ripple is consistent with gel-like structure
and the minor side appears to be thinner with lower electron density.

\end{abstract}
\pagebreak

%--------------- Introduction ----------------------------------
\section{Introduction}

Lipids in water self-assemble into lamellar bilayers, which comprise the basic
structural element of biomembranes.   The supramolecular packing of the
bilayers, in turn, forms lyotropic liquid crystals with rich phase behavior
and diverse structures.  Among the thermodynamic phases observed in
lipid bilayer systems, the ripple (P$_{\beta '}$) phase in lecithin bilayers
is especially fascinating to a broad range of researchers in condensed matter
physics and physical chemistry as an example of periodically modulated phases.
Many theoretical papers have attempted to explain the formation of the ripples
\cite{Don79,Mar84,Haw86,Car87,Gol88,McCu90,Hon91,Lub93}. 
Such understanding has been delayed because, despite many
experimental studies
\cite{Tar73,JanSS79,Luna77,Ino80,Cop80,Sta82,Rup83,Ale85,Sir88,Wac89a,Kat95},
the structure has not been as well characterized as for the lower temperature 
gel (L$_{\beta '}$) phase \cite{Sun94} and the higher temperature liquid 
crystalline (L$_{\alpha}$) phase \cite{Nagle96}.  When an accurate ripple
structure is known, some indication of the energetics of the ripple formation
may be obtained by comparing the relative size of the ripple
wavelength and the ripple amplitude to the size of lipid molecules and
the bilayer thickness \cite{Gol88}.  Also, correlation 
between the chirality of the constituent lipid molecules and the asymmetry of 
the ripples may lead to elucidation of the microscopic origin 
of the macroscopic properties of the ripple phase \cite{Lub93,Kat95}.

Structural information about the ripple phase can be divided into two
categories. The first category consists of the 2D lattice parameters, namely, 
the ripple wavelength $\lambda_r$, the bilayer packing repeat distance $d$ and
the oblique angle $\gamma$ for the unit cell that are illustrated
in Fig. 1.  These well-established parameters have been accurately
obtained by indexing low-angle x-ray scattering peaks
\cite{Tar73,JanSS79,Ino80,Ale85,Wac89a}. 
For example, Wack and Webb \cite{Wac89a} reported for DMPC with
25\% water by weight at 18$^{\circ}$C that 
$\lambda_r$ = 141.7\AA, $d$ = 57.94\AA\
and $\gamma$ = 98.4$^{\circ}$.  The second category
consists of the structure within the ripple unit cell, including
the shape and the amplitude of the ripples and the bilayer thickness. 
This second category of structural information
has been quite uncertain.  Many ripple profiles have been 
proposed \cite{Don79,Mar84,Car87,Gol88,McCu90,Hon91,Lub93,Tar73,JanSS79,Sta82},
including sinusoidal, peristaltic and sawtooth and
estimates of the ripple trough-to-peak 
amplitude range widely from 15\AA\ to 50\AA\ 
\cite{Tar73,JanSS79,Sta82,Zas88a,Hata93}.

The second category of information involves electron density maps.  These
require the amplitudes of the form factors, which, ignoring
fluctuation corrections \cite{Nagle96}, are just the square roots of
the Lorentz corrected intensities of the x-ray scattering peaks,
and the correct phases for these measured form factor amplitudes.
A traditional approach to solve the phase problem is the 
``pattern recognition'' method employed by Tardieu et al. 
\cite{Tar73}; that method
selects those phase combinations which will give rise to electron density maps 
agreeing with known physical properties of the bilayer system.
Considering that there are typically 20 or so 
peaks in most of the ripple phase x-ray data, finding the right phase 
combination out of about 2$^{20}$ possibilities is a difficult task.
The approach we employ involves modeling and nonlinear least square fitting.
Possibly, this method may be generally useful when studying other disordered
and fluctuating biological systems which have intrinsically continuous 
electron density maps rather than an atomic resolution crystal structure.

%------------------- Theory ------------------------------------
\section{METHODS}

The modeling approach first creates a functional form for the electron density 
that incorporates known and plausible properties of the bilayer but that
contains free parameters to incorporate unknown detailed structure.
The success of the method depends on this modelling step which requires some care
and is therefore the emphasis of this section.
Then, the values of the free parameters in the real space model are 
determined by routine non-linear least squares procedures so that the Fourier
components of the model provide the best fit to the measured intensities.
This last step automatically determines the phases of the model, which are
then used with the original intensity data to produce an electron density map.

\subsection{Lattice structure}

It has been shown from x-ray studies \cite{Tar73,JanSS79,Ino80,Ale85,%
Wac89a,Kat95} that ripples in different bilayers are registered to form a 
2-D oblique lattice as shown by the unit cell in Fig. 1.
The unit cell vectors can be expressed (see Fig. 1) as
\begin{eqnarray}
\label{realunit}
%\left\{ \begin{array}{ll}
\vec{a} = d \cot \gamma\ \hat{x} + d\ \hat{z} \ \mbox{and}\ \vec{b} = \lambda_r\ \hat{x}.
%\end{array}
%\right.
\end{eqnarray}
The corresponding reciprocal lattice unit cell vectors (see Fig. 1) are 
\begin{eqnarray}
\label{reciunit}
%\left\{ \begin{array}{ll}
\vec{A} = \frac{2 \pi}{d}\ \hat{z}\ \mbox{and}\ \vec{B} = \frac{2 \pi}{\lambda_r}\ \hat{x} - \frac{2 \pi}{\lambda_r \tan \gamma}\ \hat{z}.
%\end{array}
%\right.
\end{eqnarray}
so the q-space vector $\vec{q}_{hk}$ for the Bragg peak with Miller indices $(h,k)$
is:
\begin{eqnarray}
\label{qvector}
\vec{q}_{hk} =&& h \vec{A} + k \vec{B} ,
%=&& \frac{2 \pi k}{\lambda_r}\ \hat{x} + \left( \frac{2 \pi h}{d} - \frac{2 \pi k}{\lambda_r \tan \gamma} \right) \hat{z},
\end{eqnarray}
with some examples shown in Fig. 1.
%----------------------------------------------------------------------------
\subsection{Electron density model and form factor}

The ripple profile is the line z=u(x) that describes the center of the bilayer 
in the (x,z) plane, such as the one shown in Fig. 1.  
The closely related contour function is defined as
\begin{eqnarray}
\label{contour}
C(x,z) = \delta [z - u(x)].
\end{eqnarray}
The transbilayer electron density function $T_{\psi}(x,z)$ expresses the
actual electron density at points (x,z) that lie on a straight line with
slope $\psi$ from the vertical as shown in Fig. 1.  An underlying electron
density profile will be chosen similar to models for flat bilayer phases.
Then, the electron density model for $\rho (x,z)$ within
the unit cell is described as the convolution of a ripple contour
function $C(x,z)$ and the trans-bilayer electron density profile $T_{\psi}(x,z)$:
\begin{eqnarray}
\label{model}
\rho (x,z) = C(x,z) \ast T_{\psi}(x,z).
\end{eqnarray}
This convolution moves the transbilayer electron density function so that
it is centered on the ripple profile.

The form factor $F(\vec{q})$ is the Fourier transform of the electron density
expressed in Eqn.\ref{model} and is the product
\begin{eqnarray}
\label{formall}
F(\vec{q}) =&& F_C(\vec{q}) F_T(\vec{q}),
\end{eqnarray}
of the Fourier transform $F_{C}(q)$ of $C(x,z)$ and of the Fourier transform
$F_{T}(q)$ of $T_{\psi}(x,z)$.

%----------------------------------------------------------------------------
\subsection{Ripple profile}

The results of freeze fracture electron microscopy \cite{Luna77,Cop80,Rup83} and
scanning tunneling microscopy \cite{Zas88a,Hata93} strongly suggest that the
ripple profile is asymmetric.  This conclusion is further supported by 
the result
of x-ray diffraction that the ripple unit cell is oblique
($\gamma$ not equal to 90$^{\circ}$) 
\cite{Tar73,JanSS79,Ino80,Ale85,Sir88,Wac89a,Kat95}.
For our model ripple profile
u(x) we choose a triangular shape as shown in
Fig. 1.  We shall call the longer side of the ripple
the {\it major} side, and the shorter side the {\it minor} side.
The triangular ripple profile is completely described by the amplitude
$A$ and $x_0$, the projection of the major side onto the $x$ axis,
with obvious relations for other quantities such as the slope angles
of the major and minor sides.  Choosing the origin at the center
of the unit cell shown in Fig. 1 gives the ripple profile
\begin{eqnarray}
\label{profile}
u(x) = \left\{ \begin{array}{ll}
		-\frac{A}{\lambda_r - x_0} (x + \frac{\lambda_r}{2})
\ \ \ \mbox{if}\ - \frac{\lambda_r}{2} \le x < - \frac{x_0}{2};\\
		\ \ \ \ \ \frac{A}{x_0}\ x\ \ \ \ \ \ \ \ \ \ \ \ \mbox{if}
\ \ - \frac{x_0}{2} \le x \le 
\frac{x_0}{2};\\
		-\frac{A}{\lambda_r - x_0} (x - \frac{\lambda_r}{2})
\ \ \ \mbox{if}\ \ \ \ \  \frac{x_0}{2} < x \le \frac{\lambda_r}{2}.
		\end{array}
	\right.
\end{eqnarray}
This particular choice shows explicitly the inversion
symmetry in the ripple profile, so that the form factors are real.

Using Eqn.\ref{profile} the contour part of the form factor is
\begin{eqnarray}
\label{formcon}
F_C(\vec{q}) =&& \frac{1}{\lambda_r}\ \int_{- \frac{\lambda_r}
{2}}^{\frac{\lambda_r}{2}} dx\ \mbox{e}^{i [q_x x + q_z u(x)]}\nonumber \\ 
=&& \frac{x_0}{\lambda_r}\ \mbox{sinc}(\omega)
+ \frac{\lambda_r - x_0}{\lambda_r}\ \frac{\cos \frac{1}{2} (\frac{1}{2}
q_x \lambda_r + \omega)}{\cos \frac{1}{2} (\frac{1}{2} q_x \lambda_r -
\omega)}\nonumber\\ 
&& \times\ \mbox{sinc}(\frac{q_x \lambda_r}{2} - \omega ),
\end{eqnarray}
where the intermediate variable $\omega$ is defined as
\begin{eqnarray}
\label{w}
\omega (\vec{q}) = \frac{1}{2}\ ( q_x x_0 + q_z A).
\end{eqnarray}
%Combining Eqn.\ref{qvector} and Eqn.\ref{formcon}, the discrete contour form
%factor for the Bragg peak with Miller indices $(h,k)$ can be expressed as:
%\begin{eqnarray}
%\label{discrete}
%F_C(h,k) =&& \frac{x_0}{\lambda_r}\ \mbox{sinc}
%(\omega_{hk}) + (-1)^k\ \frac{\lambda_r - x_0}{\lambda_r}\ 
%\mbox{sinc}(k \pi - \omega_{hk})\nonumber\\
%=&& \sin \omega_{hk}\ \left[\frac{k \pi x_0 - 
%\omega_{hk} \lambda_r}{\omega_{hk} \lambda_r (k \pi - \omega_{hk})} \right],
%\end{eqnarray}
%where
%\begin{eqnarray}
%\label{whk}
%\omega_{hk} =&& \frac{1}{2} (q_x^{hk} x_0 + q_z^{hk} A)\nonumber\\
%=&& \pi\ \left[ \frac{h A}{d} + \frac{k}{\lambda_r}\ (x_0 -
%\frac{A}{\tan \gamma}) \right].
%\end{eqnarray}
The first term on the right hand side of Eqn.\ref{formcon} is the contribution 
from the major side of the ripple, with scattering amplitude
\(\frac{x_0}{\lambda_r}\), proportional to its projected length onto the
x axis, while the last term is the contribution from the 
minor side of the ripple, with amplitude \(\frac{\lambda_r - x_0}{\lambda_r}\).
Eqn.\ref{formcon} also shows that the maximum scattering 
from the major side occurs along that $\vec{q}$ direction determined by
$\omega = 0$ in Eqn.\ref{w}.  In real space this is also the direction normal to
the major side of the ripple.  Similarly, the maximum scattering
direction of the minor side is given by \( \frac{q_x \lambda_r}{2} - \omega 
= 0 \), and this can be shown to be the direction normal to the minor side.
These directions are illustrated by the dashed lines in Fig. 1.  
This theoretical picture for the location of the strong (h,k) peaks
agrees qualitatively with the experimental intensity pattern of Wack and
Webb \cite{Wac89a} shown in Fig. 1 as well as other x-ray data
\cite{Ale85,Kat95}.  This general pattern would also be expected to
pertain to ripple profiles that are not perfectly triangular, for example,
with rounded corners.

%---------------------------------------------------------------------------
\subsection{Trans-bilayer electron density profile $T_{\psi}(x,z)$}

In a study of gel phase DPPC bilayers using the electron
density modeling approach \cite{WSN89} it was found
that the precise functional form of the electron density model did not
much affect the results provided that the essential structural
features of bilayers were incorporated into the model.
We therefore first utilized a very simple model for $T_{\psi}(x,z)$ in
which the two opposite headgroup regions consist of
two identical positive $\delta$-functions, with amplitude $\rho_H$ and separated
by the head-head spacing $2X_h$, and
the central methyl trough consists of a negative $\delta$-function, with amplitude
$\rho_M$.  The methylene region of bilayers has an electron density
close to that of water and hence is set equal to zero in the
following model (minus fluid \cite{WKMc73}) of $T_{\psi}(x,z)$,
\begin{eqnarray}
\label{transbi}
T_{\psi}(x,z) =&& \delta (x + z \tan \psi)\ \{ \rho_H 
	[\delta (z-X_h\cos \psi) \nonumber\\
&& +\delta (z+X_h \cos \psi)]-\rho_M \delta (z) \}.
\end{eqnarray}
The form factor corresponding to this trans-bilayer electron density profile is
\begin{eqnarray}
\label{form_trans}
F_T (\vec{q}) =&& \rho_M\ [R_{HM} \cos (q_z X_h \cos \psi - 
q_x X_h \sin \psi) -1],
\end{eqnarray}
where \(R_{HM} = \frac{2\ \rho_H}{ \rho_M}\).   We will call this the
simple delta function (SDF) model.
Following \cite{WSN89} we also utilized a more realistic model, called the 1G 
hybrid model that consists of Gaussian functions for the headgroups and
the terminal methyl trough and that also allows for differences between the electron
density of methylenes and of water.  We will call this the simple 1G (S1G) model.
We also considered more complex models that modulated the electron density
along the direction perpendicular to $\vec{a}$ in real space in Fig. 1; this is
also the direction $\vec{B}$ in reciprocal space, so this modulation strongly
affects the $(0,k)$ pure ripple form factors.  Our best ripple modulated 
models (MDF and M1G) had two additional parameters, one to allow for the 
electron density across the minor side to be different by a ratio $f_1$
from the electron density across the major side and a second parameter
$f_2$, which is multiplied by delta functions ${\delta}(x \pm x_0/2)$ to 
allow for a
different electron density near the kink between the major and the minor sides.

%----------------------------------------------------------------------------
\subsection{Determination of the phases and the electron density map}

Combining Eqns. \ref{formall}, \ref{formcon} and \ref{form_trans}, the 
Lorentz-corrected scattering intensity for the (h,k) reflection of the 
SDF model is
\begin{eqnarray}
\label{inten_dis}
I(h,k) && = |F(h,k)|^2 
= \rho_M^2\ \sin^2 \omega_{hk}\ \left[\frac{k \pi x_0 -
\omega_{hk} \lambda_r}{\omega_{hk} \lambda_r (k \pi 
- \omega_{hk})} \right]^2 \nonumber\\
&& \times\ [R_{HM} \cos (q_z^{hk} X_h \cos \psi - q_x^{hk} X_h \sin \psi) -1]^2,
\end{eqnarray}
which contains six independent parameters.
Two parameters are related to the ripple profile, namely
the ripple amplitude $A$ and the horizontal projection length of the major side
of the ripple, $x_0$. The trans-bilayer profile contains three
parameters, the headgroup to methyl trough electron
density ratio, $R_{HM}$, the monolayer shift angle $\psi$ and 
$X_h$, the headgroup position relative to the center of the bilayer. 
The sixth parameter is a common scaling factor for all the
$I(h,k)$ peaks.  For the S1G model, all the trans-bilayer profile parameters
were taken from gel phase data \cite{WSN89} except for the headgroup position
$X_h$.   The MDF and M1G models had 
the two additional parameters
$f_1$ and $f_2$ mentioned above.
(The parameters ${\lambda}_r$, d and $\gamma$ are known
from indexing the (h,k) peaks.)
Standard non-linear least squares fitting procedures were employed to
obtain unknown parameters by finding the best
fit of the intensities given by Eqn.\ref{inten_dis} to the $|F(h,k)|^2$ reported
by Wack and Webb \cite{Wac89a}. 
This procedure does not require knowing any phases in advance.  However,
once the best values of the parameters in the electron density model are
determined, the form factors, and especially the phases, are determined.
Using these phases and the amplitudes of the form factors $|F(h,k)|$ from 
the intensity data \cite{Wac89a},
an experimental relative electron density map $\rho(\vec{r})$ 
is obtained using
\begin{eqnarray}
\label{discretell}
\rho(x,z) = \sum_{h \geq 0} \sum_{k} F(h,k) \ \cos(q^{hk}_x x + q^{hk}_z z).
\end{eqnarray}

%------------------- Results and Discussion -------------------------
\section{Results and Discussion}

The absolute form factors $|F(h,k)|$ reported by Wack and Webb \cite{Wac89a} 
are shown in Table 1 along with two of our many model fits.
Our most striking result is that all of our models have the same
phases for all the (h,k) reflections displayed in Table 1,
except for the pure ripple
reflections with h=0.  Figure 2 shows the Fourier map
of the electron density using the measured \cite{Wac89a} absolute form factors
$|F(h,k)|$, omitting the pure ripple reflections (0,k),
and our phases shown in Table 1.
Since the F(0,k) are rather small, their inclusion with any phase
combination makes rather small differences to the electron
density map and negligible differences to the ripple profile.

The ripple profile is shown in Fig. 2 by the dotted lines
that are defined as the loci of maximum intensity in the headgroup
region.  This ripple profile clearly has the basic triangular shape
assumed in our modeling, but this result is not tautological because 
the experimental 
absolute form factors are employed in Fig. 2.
The ripple profile shown in Fig. 2 is in excellent quantitative
agreement with the fitted model ripple profiles.  The parameters
A and $x_0$ that describe the ripple profile, as well as other
parameters in the models, are given in Table 2 for three models.
From this we conclude that the ripple amplitude A is about 19\AA\
and the projection of the major side on the a axis is about 103\AA\
for the single DMPC sample for which Wack and Webb \cite{Wac89a}
reported absolute form factors.

To obtain a more quantitative picture of the thickness of the
bilayer in various parts of the ripple, electron density profiles
(using the experimental $|F(h,k)|$)
are plotted in Fig. 3 along the three slices shown by short straight dashed
lines in Fig. 2 and labelled A, B and C.
Slice A is along the normal of the major side and is centered in the
middle of the hydrocarbon region; it indicates that the head-head 
spacing is 38\AA\ and the water spacing is 21\AA\ on the major side.
Slice B is along the normal to the minor side and centered
in the middle of the hydrocarbon region; it indicates that
the bilayer head-head spacing on the minor side is 31\AA.
Slice C is along the normal to the minor side, but centered in
the water region; it indicates that the water spacing of the
minor side is 18\AA.
All these spacings are, of course, subject to Fourier truncation
errors since the intensity data only include up to three lamellar orders 
(h=1-3).  Furthermore, no corrections to the form factors have been
made to account for fluctuations.  For smectic liquid crystals such
corrections are considerably more complex than simple Debye-Waller
factors; they are expected to be smaller for the ripple phase than
for the fluid $L_{\alpha}$ phase, and even for this latter phase
these corrections change the head-head spacings negligibly
(see appendix to \cite{Nagle96}).

We also performed x-ray experiments on a DMPC sample with almost identical
hydration level as the data in \cite{Wac89a}. 
The sample was partially dehydrated by a 40\% polyvinylpyrrolidone
(PVP) aqueous solution.  At 18$^{\circ}$C
this sample was in the ripple phase, with a d-spacing of 58.1\AA, comparing 
well with Wack and Webb's 57.94\AA.  We then took
low angle data at 6$^{\circ}$C at which the sample was clearly in the
gel phase with no ripple peaks present. The intensities of five orders 
$h=1-5$ were
recorded, but the electron density profile was reconstructed using
the first three orders only so as to form a valid comparison 
with the ripple phase electron densities that were obtained from 
experimental data limited to $h=1-3$.  
The dotted line in Fig. 3A shows this comparison.
The gel phase profile gives a head-head bilayer thickness of 37.0\AA\ and
a water spacing of 20.8\AA, agreeing well with 37.9\AA\ and 20.7\AA,
respectively, obtained for the ripple phase.

The preceding result strongly suggests that the lipid bilayer along the major
side of the ripple is similar to the gel phase bilayer.
In contrast, the bilayer along the minor side appears to be
thinner and this is consistent with the working hypothesis
that the minor side may be more like the fluid $L_{\alpha}$ bilayer.
A supporting argument for this hypothesis is that the electron density 
in the headgroup region along the minor side appears to be lower,
as it should be for a fluid bilayer with larger area/lipd, than 
along the major side as shown by the solid curves in Fig. 3B.
This supporting argument is hardly affected by inclusion of the
F(0,k) for the two sets of phases $(-+-)$ and $(+-+)$ for the ripple 
reflections shown in Table 1.  Inclusion makes too little difference 
to show in Fig. 3A for the major side.  For the minor side the 
$(-+-)$ phase set increases the height of the head groups, but not
enough to equal the height for the headgroups on the
major side, and not relative to the lower electron density of the 
chain region.  This hypothesis is also consistent with the observation of 
anisotropic diffusion preferentially along the grooves \cite{SCM83}.

Wack and Webb \cite{Wac89a} argued that the (0,k) phases are $(+++)$ for 
k=1,2 and 3,
although the origin of their unit cell is translated by $\vec{b}/2$
compared to our unit cell in Fig. 1, so their
phases are equivalent to our phase set $(-+-)$ which we obtain
from the SDF model as shown in Table 1.  However, the small magnitudes
of the model F(0,k) motivated us to try the MDF and M1G models that
deliberately modulate the total electron density in the direction 
perpendicular to $\vec{a}$ in Fig. 1.  (This direction is also very nearly
along the major side since it makes an angle of about 10 degrees
with the b-axis which is nearly the same as $\gamma - 90^{\circ}$.)
The results for the M1G model given in Table 1, as well as for
the MDF model (not shown), show a dramatic
improvement in the fit to most of the F(h,k), not just the F(0,k), with
a factor of four decrease in the sum of the squares of the residuals.
(The crystallographic R-value is 0.195 for the SDF model and 
0.083 for the M1G model.)  It may also be noted that the (0,3) reflection 
has nearly the same q-value as the (1,-2) reflection.  Reassignment of the 
larger part of the value 7.81 for $|F(0,3)|$ in Table 1 to $|F(1,-2)|$ would
further improve the fit.
However, the set of ripple phases reverses to $(+-+)$ for both the M1G and
the MDF model and this result is
unchanged by using different starting values in the nonlinear
least squares fitting program.   We therefore favor the phases $(+-+)$
for the ripple reflections, although we recognize that none of
the models we have considered readily accommodates $|F(0,2)|$ larger
than $|F(0,1)|$.

Although the phases for the pure ripple (0,k) reflections
are not as well determined by our procedure as the h$\neq$0 phases,
the latter phases, which are very robustly determined using
our model method, already determine the ripple profile.  
Indeed, the effect of different (0,k) phases on the electron density
maps is less important than the effect of including the three h=4 
reflections shown in Table 1 for the M1G model.
Further progress toward understanding the molecular packing of lipids in the 
ripple phase should employ careful analysis of wide angle scattering data.
This study establishes the framework into which such additional information
must be incorporated.

Acknowledgment: This research was supported by NIH Grant GM44976.

%------------------ Bibliography ----------------------------------
\begin{thebibliography}{10}

\bibitem{Don79}
Doniach,~S. (1979) {\it J. Chem. Phys.} {\bf 70},  4587--4596.

\bibitem{Mar84}
Marder,~M., Frisch, H.~L., Langer, J.~S. and H.~M. McConnell, (1984)
{\it  Proc. Natl. Acad.  Sci. USA} {\bf 81},  6559-6561.

\bibitem{Haw86}
Hawton, M.~H. and Keeler, W.~J. (1986) {\it Phys. Rev. A} {\bf 33}, 3333--3340.

\bibitem{Car87}
Carlson, J.~M. and Sethna, J.~P. (1987) {\it Phys. Rev. A} {\bf 36}, 3359--3374.

\bibitem{Gol88}
Goldstein, R.~E. and Leibler, S. (1988) {\it Phys. Rev. Lett.} {\bf 61},  
2213--2216.

\bibitem{McCu90}
McCullough, W.~S.  and Scott, H.~L. (1990) {\it Phys. Rev. Lett.} {\bf 65},  
931--934.

\bibitem{Hon91}
Honda, K.  and Kimura, H. (1991) {\it J. Phys. Soc. Jpn.} {\bf 60}, 1212--1215.

\bibitem{Lub93}
Lubensky, T.~C. and MacKintosh, F.~C. (1993) {\it Phys. Rev. Lett.} {\bf 71},  
1565--1568.

\bibitem{Tar73}
Tardieu, A.,  Luzzati, V. and Reman, F.~C. (1973) {\it J. Mol. Biol.} {\bf 75},
711--733.

\bibitem{JanSS79}
Janiak, M.~J., Small, D.~M. and Shipley, G.~G. (1979) {\it J. Biol. Chem.} 
{\bf 254}, 6068--6078.

\bibitem{Luna77}
Luna, E.~J. and McConnell, H.~M. (1977) {\it Biochim. Biophys. Acta} {\bf 466},
381--392.

\bibitem{Ino80}
Inoko, Y., Mitsui, T., Ohki, K., Sekiya, T. and Y. Nozawa, (1980) {\it Phys. 
Stat. Sol. (a)} {\bf 61},  115--121.

\bibitem{Cop80}
Copeland, B.~R. and McConnell, H.~M. (1980) {\it Biochim. Biophys. Acta}  
{\bf 599},  95--109.

\bibitem{Sta82}
Stamatoff, J.~B., Feuer, H.~J., Guggenheim, B., Tellez, G. and T. Yamane, (1982)
{\it Biophys. J.} {\bf 38},  217--226.

\bibitem{Rup83}
Ruppel, D. and Sackmann, E. (1983){\it J. Physique} {\bf 44},  1025--1034.

\bibitem{Ale85}
Alecio, M.~R., Miller, A. and Watts, A. (1985) {\it Biochim. Biophys. Acta} 
{\bf 815},  139--142.

\bibitem{Sir88}
Sirota, E.~B., Smith, G.~S., Safinya, C.~R., Plano, F.~J.  and  Clark, N.~A. 
(1988) {\it Science} {\bf 242},  1406--1409.

\bibitem{Wac89a}
Wack, D.~C. and Webb, W.~W. (1989) {\it Phys. Rev. A} {\bf 40},  2712--2730.

\bibitem{Kat95}
Katsaras, J. and Raghunathan, V.~A. (1995) {\it Phys. Rev. Lett.} {\bf 74}, 
2022--2025.

\bibitem{Sun94}
Sun, W., Suter, R.~M., Knewtson, M.~A., Worthington, C.~R., 
Tristram-Nagle, S., Zhang, R., and Nagle, J.~F.
(1996) {\it Phys. Rev. E} {\bf 49}, 4665-4676.

\bibitem{Nagle96}
Nagle, J.~F., Zhang, R., Tristram-Nagle, S., Sun, W., Petrache, H.~I., and
Suter, R.~M. (1996) {\it Biophys. J.} {\bf 70}, 1419-1431.

\bibitem{Zas88a}
Zasadzinski, J.~A.~N., Schneir, J., Gurley, J., Elings, V. and Hansma, P.~K. 
(1988) {\it Science} {\bf 239},  1013--1014.

\bibitem{Hata93}
Hatta, I., Kato, S., and Takahashi, H. (1993) {\it Phase Transitions}
{\bf 45},  157--184.

\bibitem{WSN89}
Wiener, M.~C., Suter, R.~M., and Nagle, J.~F. (1989) {\it Biophys. J.}
{\bf 55},  315--325.

\bibitem{WKMc73}
Worthington, C., King, G.~I., and McIntosh, T.~J. (1973) {\it Biophys. J.}
{\bf 13}, 480--494.

\bibitem{SCM83}
Schneider, M.,~B., Chan, W.~K., and Webb, W.~W. (1983) 
{\it Biophys. J.} {\bf 43}, 157-165.

\end{thebibliography}

%-------------- Figures and Tables ----------------------------
\begin{figure}
%\centerline {\psfig{figure=q_model3_new.ps,width=4.00in}}
%\vspace{11pt}
\caption{Real space pattern (bottom) shows the asymmetric ripple profile by 
bold lines for two bilayers.  The q-space pattern (top) shows the (h,k) 
peak positions as the centers of the circles.  Larger magnitudes of the 
form factors for reflections that were observed [18] are indicated 
by darker circles.  The grey dot-dash lines show the perpendiculars to the 
major and the minor sides of the ripple profile.
\label{Fig1}}
\end{figure}

\begin{figure}
%\centerline {\psfig{figure=map_wack_34.ps,width=4.00in}}
%\vspace{11pt}
\caption{Electron density map using measured absolute form factors 
[18] and phases from Table 1 omitting (0,k) reflections.  The smallest 
electron density is pure black and the largest electron density is pure 
white with a linear grey scale interpolating between.  The black dotted 
lines show two of the loci of maximum electron density. The electron 
density along the dashed lines A, B and C are shown in a conventional plot 
in Fig. 3.  The parallelogram shows the unit cell whose width is $\lambda_r$ 
which equals 141.7\AA.
\label{Fig2}}
\end{figure}

\begin{figure}
%\centerline {\psfig{figure=gel_slc.ps,width=4.00in}}
%\vspace{11pt}
\caption{Solid curves in A, B and C show electron density along the three 
dashed lines in Fig. 2 labelled by A, B and C.  The dotted curve in
A shows the electron density profile determined for the gel phase.
The dashed and dash-dot curves in B and C show the electron density when
the (0,k) reflections with phases $(-+-)$ and $(+-+)$, respectively, are
added.
\label{Fig3}}
\end{figure}

\pagebreak
%-------------------------- Tables -------------------------------------
%------------------- Table I -----------------------------------
\begin{table}
\caption{Form factors.
\label{formfactor}}
\vspace{6pt}
{\small
\begin{tabular}{rrrrrr} 
h & k & 100$\times$q & Data$^{*}$ &
        \multicolumn{2}{c}{Model F(h,k)} \\ \cline{5-6}
& & (\AA$^{-1}$) & $|$F(h,k)$|$ & SDF & M1G \\ \hline
0 &  1 & 4.48  &  5.3  &  -1.4  &  7.9  \\
0 &  2 & 8.96  & 9.7  &  1.0  &  -4.6 \\
0 &  3 & 13.4  &  7.8  &  -0.4   &  0.9 \\
1 &  -2 & 13.0  &  ---  &  5.0  &  6.7   \\
1 &  -1 & 11.1  &  60.8  &  -42.8  &  -60.4 \\
1 &   0 & 10.8  &  100.0  &  -96.7  &  -99.1 \\
1 &   1 & 12.3  &  26.9  &  38.1  &  28.5 \\
1 &   2 & 15.0  &  ---  &  -23.6  &  -11.8 \\
1 &   3 & 18.5  &  7.6  &  13.9  &  4.8 \\
1 &   4 & 22.3  &  ---  &  -6.3  &  -2.4 \\
2 &  -3 & 23.8  &  ---  &  8.6  &  10.9 \\
2 &  -2 & 22.2  &  15.1  &  -9.0  &  -14.1 \\
2 &  -1 & 21.5  &  71.2  &  -66.9  &  -71.7 \\
2 &   0 & 21.7  &  39.7  &  -41.0  &  -39.9 \\
2 &   1 & 22.8  &  33.9  &  31.1  &  30.6 \\
2 &   2 & 24.6  &  22.7  &  -24.3  &  -22.8 \\
2 &   3 & 27.1  &  14.2  &  17.1  &  15.2 \\
2 &   4 & 30.1  &  7.8  &  -10.3  &  -9.1 \\
3 &  -3 & 33.3  &  ---  &  -6.1  &  -6.0 \\
3 &  -2 & 32.5  &  29.3  &  22.3  &  29.0 \\
3 &  -1 & 32.2  &  44.2  &  42.2  &  44.5 \\
3 &   0 & 32.5  &  12.0    &  2.2  &  0.7   \\
3 &   1 & 33.5  &  ---  &  -11.3  &  -10.5 \\
3 &   2 & 35.0  &  10.5  &  14.9  &  13.9 \\
3 &   3 & 37.0  &  14.9  &  -15.0  &  -13.4 \\
3 &   4 & 39.4  &  10.0  &  12.5  &  10.8 \\
4 &  -2 & 43.0  &  ---  &  -81.1  &  -39.2$^{\dag}$ \\
4 &  -1 & 43.0  &  ---  &  -63.5  &  -23.6$^{\dag}$ \\
4 &   0 & 43.4  &  ---  &  26.1  &  10.3$^{\dag}$ \\
\end{tabular}
}
{\small
\indent $^{*}$From Wack and Webb [18].

\indent $^{\dag}$Values are consistent with the densitometer trace shown in 
        Fig. 4 in [18].
}
\end{table}

\pagebreak

%------------------- Table II -----------------------------------
\begin{table}
\caption{Ripple structural quantities.
\label{parameter}}
\vspace{6pt}
\begin{tabular}{ccccccc} 
Model & SDF & MDF & M1G \\ \hline
A(\AA) & 18.6 & 20  & 19 \\
x$_0$(\AA) & 103 & 103 & 103 \\
$\psi$($\deg$) & 5 & 9 & 9 \\
X$_h$(\AA) & 20.1 & 20.4 & 19.3 \\
f1 & 1(fixed) & 0.7 & 0.6 \\
f2 & 0(fixed) & -2 & -1 \\
\end{tabular}
\end{table}

\end{document}
