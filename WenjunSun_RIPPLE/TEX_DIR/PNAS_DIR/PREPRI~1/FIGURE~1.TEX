%------------ Figures and Tables ----------------------------
%---------------------------- Figures -------------------------------
\begin{figure}
%\centerline {\psfig{figure=q_model3_new.ps,width=4.00in}}
%\vspace{11pt}
\caption{Real space pattern (bottom) shows the asymmetric ripple profile by 
bold lines for two bilayers.  The q-space pattern (top) shows the (h,k) 
peak positions as the centers of the circles.  Larger magnitudes of the 
form factors for reflections that were observed [18] are indicated 
by darker circles.  The grey dot-dash lines show the perpendiculars to the 
major and the minor sides of the ripple profile.
\label{Fig1}}
\end{figure}

\begin{figure}
%\centerline {\psfig{figure=map_wack_34.ps,width=4.00in}}
%\vspace{11pt}
\caption{Electron density map using measured absolute form factors 
[18] and phases from Table 1 omitting (0,k) reflections.  The smallest 
electron density is pure black and the largest electron density is pure 
white with a linear grey scale interpolating between.  The black dotted 
lines show two of the loci of maximum electron density. The electron 
density along the dashed lines A, B and C are shown in a conventional plot 
in Fig. 3.  The parallelogram shows the unit cell whose width is $\lambda_r$ 
which equals 141.7\AA.
\label{Fig2}}
\end{figure}

\begin{figure}
%\centerline {\psfig{figure=gel_slc.ps,width=4.00in}}
%\vspace{11pt}
\caption{Solid curves in A, B and C show electron density along the three 
dashed lines in Fig. 2 labelled by A, B and C.  The dotted curve in
A shows the electron density profile determined for the gel phase.
The dashed and dash-dot curves in B and C show the electron density when
the (0,k) reflections with phases $(-+-)$ and $(+-+)$, respectively, are
added.
\label{Fig3}}
\end{figure}

\pagebreak
%-------------------------- Tables -------------------------------------
%------------------- Table I -----------------------------------
\begin{table}
\caption{Form factors.
\label{formfactor}}
\vspace{6pt}
{\small
\begin{tabular}{rrrrrr} 
h & k & 100$\times$q & Data$^{*}$ &
        \multicolumn{2}{c}{Model F(h,k)} \\ \cline{5-6}
& & (\AA$^{-1}$) & $|$F(h,k)$|$ & SDF & M1G \\ \hline
0 &  1 & 4.48  &  5.3  &  -1.4  &  7.9  \\
0 &  2 & 8.96  & 9.7  &  1.0  &  -4.6 \\
0 &  3 & 13.4  &  7.8  &  -0.4   &  0.9 \\
1 &  -2 & 13.0  &  ---  &  5.0  &  6.7   \\
1 &  -1 & 11.1  &  60.8  &  -42.8  &  -60.4 \\
1 &   0 & 10.8  &  100.0  &  -96.7  &  -99.1 \\
1 &   1 & 12.3  &  26.9  &  38.1  &  28.5 \\
1 &   2 & 15.0  &  ---  &  -23.6  &  -11.8 \\
1 &   3 & 18.5  &  7.6  &  13.9  &  4.8 \\
1 &   4 & 22.3  &  ---  &  -6.3  &  -2.4 \\
2 &  -3 & 23.8  &  ---  &  8.6  &  10.9 \\
2 &  -2 & 22.2  &  15.1  &  -9.0  &  -14.1 \\
2 &  -1 & 21.5  &  71.2  &  -66.9  &  -71.7 \\
2 &   0 & 21.7  &  39.7  &  -41.0  &  -39.9 \\
2 &   1 & 22.8  &  33.9  &  31.1  &  30.6 \\
2 &   2 & 24.6  &  22.7  &  -24.3  &  -22.8 \\
2 &   3 & 27.1  &  14.2  &  17.1  &  15.2 \\
2 &   4 & 30.1  &  7.8  &  -10.3  &  -9.1 \\
3 &  -3 & 33.3  &  ---  &  -6.1  &  -6.0 \\
3 &  -2 & 32.5  &  29.3  &  22.3  &  29.0 \\
3 &  -1 & 32.2  &  44.2  &  42.2  &  44.5 \\
3 &   0 & 32.5  &  12.0    &  2.2  &  0.7   \\
3 &   1 & 33.5  &  ---  &  -11.3  &  -10.5 \\
3 &   2 & 35.0  &  10.5  &  14.9  &  13.9 \\
3 &   3 & 37.0  &  14.9  &  -15.0  &  -13.4 \\
3 &   4 & 39.4  &  10.0  &  12.5  &  10.8 \\
4 &  -2 & 43.0  &  ---  &  -81.1  &  -39.2$^{\dag}$ \\
4 &  -1 & 43.0  &  ---  &  -63.5  &  -23.6$^{\dag}$ \\
4 &   0 & 43.4  &  ---  &  26.1  &  10.3$^{\dag}$ \\
\end{tabular}
}
{\small
\indent $^{*}$From Wack and Webb [18].

\indent $^{\dag}$Values are consistent with the densitometer trace shown in 
	Fig. 4 in [18].
}
\end{table}

\pagebreak

%------------------- Table II -----------------------------------
\begin{table}
\caption{Ripple structural quantities.
\label{parameter}}
\vspace{6pt}
\begin{tabular}{ccccccc} 
Model & SDF & MDF & M1G \\ \hline
A(\AA) & 18.6 & 20  & 19 \\
x$_0$(\AA) & 103 & 103 & 103 \\
$\psi$($\deg$) & 5 & 9 & 9 \\
X$_h$(\AA) & 20.1 & 20.4 & 19.3 \\
f1 & 1(fixed) & 0.7 & 0.6 \\
f2 & 0(fixed) & -2 & -1 \\
\end{tabular}
\end{table}
