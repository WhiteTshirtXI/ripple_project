\section{Results}

\begin{figure}
\centerline {\psfig{figure=rppl_dir/wack_data.ps,height=3in}}
\caption{
Microdensitometer trace of film recording of high resolution
powder diffraction pattern for DMPC, by Wack and Webb [22].
Miller indices for each line are shown nearby. Inset shows indexing results 
for 2-D monoclinic lattice constants.
\label{wack_trace}}
\end{figure}

Fig. \ref{wack_trace} shows the microdensitometer trace reported by Wack 
and Webb \cite{Wac89a}.
The absolute form factors $|F(h,k)|$ obtained by them from that trace
are shown in Table \ref{formfactor} along with four of our many model fits
(the crystallographic R-value\footnote{\(R=\frac{\sum \left| |F_o| - |F_c|
\right|}{\sum |F_o|}\), where $|F_o|$ is the data form factor and $|F_c|$
is the model fit form factor.} is 0.195 for the SDF model and 
0.083 for the M1G model; S1G and MDF model have similar R-values as that
of SDF and M1G, respectively).
Our most striking result is that all of our models have the same
phases for all the (h,k) reflections displayed in Table \ref{formfactor},
except for the pure ripple reflections with h=0.  Figure \ref{wack_map} 
shows the electron density map Fourier-reconstructed by using the measured 
\cite{Wac89a} absolute form factors $|F(h,k)|$, omitting the pure ripple 
reflections (0,k), and our phases shown in Table \ref{formfactor}.
Since the F(0,k) are rather small, their inclusion with any phase
combination makes rather small differences to the overall electron
density map and negligible differences to the ripple profile.

\begin{table}
\caption{Form factors of the ripple phase of DMPC.
\label{formfactor}}
\vspace{6pt}
{\small
\tabcolsep=0.20in
\begin{tabular}{rrrrrrrr} \hline
h & k & 100$\times$q & Data$^{*}$ &
        \multicolumn{4}{c}{Model F(h,k)} \\ \cline{5-8}
& & (\AA$^{-1}$) & $|$F(h,k)$|$ & SDF & S1G & MDF & M1G \\ \hline
0 &  1 & 4.48  &  5.3  &  -1.4 & -2.1 & 5.7 &  7.9  \\
0 &  2 & 8.96  & 9.7  &  1.0  &  1.5 & -3.0 & -4.6 \\
0 &  3 & 13.4  &  7.8  &  -0.4  & -0.8 & 0.04 &  0.9 \\
1 &  -2 & 13.0  &  ---  &  5.0  & 5.6 & 5.7 &  6.7   \\
1 &  -1 & 11.1  &  60.8  &  -42.8  & -41.9 & -61.3 &  -60.4 \\
1 &   0 & 10.8  &  100.0  &  -96.7  & -96.9 & -99.6 &  -99.1 \\
1 &   1 & 12.3  &  26.9  &  38.1  & 37.4 & 27.6 &  28.5 \\
1 &   2 & 15.0  &  ---  &  -23.6  & -23.3 & -10.9 &  -11.8 \\
1 &   3 & 18.5  &  7.6  &  13.9  & 14.1 & 4.8 &  4.8 \\
1 &   4 & 22.3  &  ---  &  -6.3  & -6.9 & -3.5 &  -2.4 \\
2 &  -3 & 23.8  &  ---  &  8.6  & 8.0 & 11.3 &  10.9 \\
2 &  -2 & 22.2  &  15.1  &  -9.0  & -7.4 & -15.1 &  -14.1 \\
2 &  -1 & 21.5  &  71.2  &  -66.9  & -65.9 & -71.7 &  -71.7 \\
2 &   0 & 21.7  &  39.7  &  -41.0  & -41.1 & -40.4 &  -39.9 \\
2 &   1 & 22.8  &  33.9  &  31.1  & 30.7 & 30.6 &  30.6 \\
2 &   2 & 24.6  &  22.7  &  -24.3  & -24.3 & -22.2 &  -22.8 \\
2 &   3 & 27.1  &  14.2  &  17.1  & 17.7 & 14.6 &  15.2 \\
2 &   4 & 30.1  &  7.8  &  -10.3  & -11.2 & -9.1 &  -9.1 \\
3 &  -3 & 33.3  &  ---  &  -6.1  & -6.2 & -5.5 &  -6.0 \\
3 &  -2 & 32.5  &  29.3  &  22.3  & 20.5 & 28.6 &  29.0 \\
3 &  -1 & 32.2  &  44.2  &  42.2  & 41.6 & 43.6 &  44.5 \\
3 &   0 & 32.5  &  12.0    &  2.2  & 2.5 & 1.0 &  0.7   \\
3 &   1 & 33.5  &  ---  &  -11.3  & -11.1 & -10.6 &  -10.5 \\
3 &   2 & 35.0  &  10.5  &  14.9  & 14.9 & 13.6 &  13.9 \\
3 &   3 & 37.0  &  14.9  &  -15.0  & -15.6 & -12.9 &  -13.4 \\
3 &   4 & 39.4  &  10.0  &  12.5  & 13.7 & 10.3 &  10.8 \\
4 &  -2 & 43.0  &  ---  &  -81.1  & -28.5$^{\dag}$ & -108.8 &  -39.2$^{\dag}$ \\
4 &  -1 & 43.0  &  ---  &  -63.5  & -23.6$^{\dag}$ & -66.0 &  -23.6$^{\dag}$ \\
4 &   0 & 43.4  &  ---  &  26.1  & 9.2$^{\dag}$ & 28.6 &  10.3$^{\dag}$ \\ 
\hline
 & & & R-value & 0.195 & 0.208 & 0.080 & 0.083 \\ \hline
\end{tabular}
}
\\~\\
{\small
\indent $^{*}$From Wack and Webb \cite{Wac89a}.

\indent $^{\dag}$Values are consistent with the densitometer trace shown in 
        Fig. 4 in \cite{Wac89a}.
}
\end{table}

The ripple profile is shown in Fig. \ref{wack_map} by the dotted lines
that are defined as the loci of maximum intensity in the headgroup
region. This ripple profile clearly has the basic triangular shape
assumed in our modeling, but this result is not tautological because 
the experimental absolute form factors are employed in Fig. \ref{wack_map}.
From this map (Fig. \ref{wack_map}) the ripple amplitude A is measured
to be about 19\AA\ and the projection of the major side on the a axis is 
about 103\AA\ for the single DMPC sample for which Wack and Webb \cite{Wac89a}
reported absolute form factors.

\begin{figure}
\centerline {\psfig{figure=rppl_dir/map_wack_no0k.ps,height=3in}}
%\psfig{figure=rppl_dir/model3_map.ps,height=2.5in}
%\vspace{-0.2in}
%\hspace{1.5in} (A) \hspace{2.9in} (B)
%\vspace{0.2in}
\caption{
Electron density map using measured absolute form factors 
[22] and phases from Table 1 omitting (0,k) reflections.  The smallest 
electron density is pure black and the largest electron density is pure 
white with a linear grey scale interpolating between.  The black dotted 
lines show two of the loci of maximum electron density. 
The parallelogram shows the unit cell whose width is $\lambda_r$ 
which equals 141.7\AA.
\label{wack_map}}
\end{figure}

To obtain a more quantitative picture of the thickness of the
bilayer in various parts of the ripple, electron density profiles
(using the experimental $|F(h,k)|$)
are plotted in Fig. \ref{slice}(B) along the three slices shown by short 
straight dashed lines in Fig. \ref{slice}(A) and labeled A, B and C.
Slice A is along the normal of the major side and is centered in the
middle of the hydrocarbon region; it indicates that the head-head 
spacing is 38\AA\ and the water spacing is 21\AA\ on the major side.
Slice B is along the normal to the minor side and centered
in the middle of the hydrocarbon region; it indicates that
the bilayer head-head spacing on the minor side is 31\AA.
Slice C is along the normal to the minor side, but centered in
the water region; it indicates that the water spacing of the
minor side is 18\AA.  All these spacings are, of course, subject to 
Fourier truncation errors since the intensity data only include up 
to three lamellar orders (h=1-3).

\begin{figure}
\begin{center}
\leavevmode
%\hspace{0.5in}
\psfig{figure=rppl_dir/map_slice.ps,width=3.0in}
\hspace{3.2in} (A)
%\vspace{0.2in}
\end{center}
\begin{center}
\leavevmode
%\hspace{0.15in}
\psfig{figure=rppl_dir/slcabc.ps,width=4.0in}
\end{center}
\vspace{-0.2in}
\hspace{3.0in} (B) 
\caption{(A) The electron density map (also shown in Fig. \ref{wack_map}), 
with three slices along
the normals labeled by A, B and C. 
(B) Solid curves in A, B and C show electron density along the three 
dashed lines as shown in (A).  
The dashed and dash-dot curves in B and C show the electron density when
the (0,k) reflections with phases $(-+-)$ and $(+-+)$, respectively, are
added.
\label{slice}}
\end{figure}

We also performed x-ray experiments on a DMPC sample with almost identical
hydration level as the data in \cite{Wac89a}. 
The sample was partially dehydrated by a 40\% polyvinylpyrrolidone
(PVP) aqueous solution.  At 18$^{\circ}$C
this sample was in the ripple phase, with a d-spacing of 58.1\AA, comparing 
well with Wack and Webb's 57.94\AA.  We then took
low angle data at 6$^{\circ}$C at which the sample was clearly in the
gel phase with no ripple peaks present. The intensities of five orders 
$h=1-5$ were
recorded, but the electron density profile was reconstructed using
the first three orders only so as to form a valid comparison 
with the ripple phase electron densities that were obtained from 
experimental data limited to $h=1-3$.  
The dashed line in Fig. \ref{gel_rppl} shows this comparison.
The gel phase profile gives a head-head bilayer thickness of 37.0\AA\ and
a water spacing of 20.8\AA, agreeing well with 37.9\AA\ and 20.7\AA,
respectively, obtained for the ripple phase.

\begin{figure}
\begin{center}
\leavevmode
\psfig{figure=rppl_dir/gel_rppl.ps,height=3.5in}
\end{center}
\caption{Comparison of the gel phase and ripple phase electron density
profiles. Solid line is the major side slice electron density profile
as shown in Fig. \ref{slice}(B);
dashed line is the gel phase electron density
profile reconstructed using three orders of low angle gel-phase peaks.
\label{gel_rppl}}
\end{figure}

The preceding results do not depend on the details of the different
models, only on the phases that were derived from them. Now, however,
we show some results of the structural quantities obtained from four of
our model fits. Table \ref{parameter} shows the best fitting values
of the model structural parameters. Using the values for A, x$_0$ and
X$_h$ of S1G, a schematic ripple structure was constructed and is shown
in the bottom portion of Figure \ref{wack_data}; this fitted structure
is in excellent quantitative agreement with the ripple profile shown 
in Fig. \ref{wack_map}. The upper portion of 
Fig. \ref{wack_data} shows the scattering intensity data in Wack and
Webb \cite{Wac89a}, replotted in the 2D reciprocal space; in this bottom 
portion, the correlation between the normals of the two 
sides of the ripples and the angular scattering intensity distribution 
in the q-space is shown by the dot-dashed lines in Fig. \ref{wack_data}, 
as discussed in Section \ref{rppl_theory_3} and shown in Fig. \ref{q_pattern}.
From the structural quantities listed in Table \ref{parameter}, it can be
further calculated that the ratio of the major side length to the 
minor side length is about 2.4, which is a good parameter 
to describe the degree of the ripple asymmetry; the minor side
has a slope of about 26$^{\circ}$, steeper than 10$^{\circ}$ of the major side. 

\begin{table}
\caption{Ripple structural quantities.
\label{parameter}}
\vspace{6pt}
\tabcolsep=0.35in
\begin{tabular}{ccccccc} \hline 
Model & SDF & S1G & MDF & M1G \\ \hline
A(\AA) & 18.6 & 18.5 & 20  & 19 \\
x$_0$(\AA) & 103 & 105 & 103 & 103 \\
$\psi$($\deg$) & 5 & 5 & 9 & 9 \\
X$_h$(\AA) & 20.1 & 19.1 & 20.4 & 19.3 \\
f1 & 1(fixed) & 1(fixed) & 0.7 & 0.6 \\
f2 & 0(fixed) & 0(fixed) & -2 & -1 \\ \hline
\end{tabular}
\end{table}

\begin{figure}
\centerline {\psfig{figure=rppl_dir/q_wack_model3_new.ps,height=3in}}
\caption{Replotting of Wack and Webb's data [19] in the q-space (top) 
and the model fit real space ripple structure (bottom). 
The white regions are headgroups; the thick black lines represent
the central profiles of the rippled bilayers; the darker gray-scaled stripes 
denote the bilayers; the lighter gray-scale space
between the two adjacent bilayers is the water region. 
The dot-dashed lines are the normal directions of the major and
minor side. The big parallelogram represents the real space unit cell,
the small parallelogram the reciprocal space unit cell.
\label{wack_data}}
\end{figure}

%In order to test the robustness of this model and the phases obtained
%from it, a few variations of this model were used to fit the same data.
%When the monolayer shift angle, $\psi$, was set to either zero or
%$\alpha$ (the slope of the major side), similar values of fitting
%parameters and identical phases were obtained (not shown), although
%the fits were slightly worse due to the constraints. 

%The form factors, both from Wack and Webb's data \cite{Wac89a} and
%from the model fitting, are given in Table \ref{formfactor}. The phases
%of the form factors are outputs of the model fitting and listed
%in the last column of Table \ref{formfactor}. The agreements of the 
%stronger peaks between the model fit and the data are vary good.
%With the obtained phasing information, the electron density maps, both from 
%the data and the model fitting, are Fourier reconstructed and shown in 
%Fig. \ref{wack_map}(A) and Fig. \ref{wack_map}(B), respectively.
%Despite some minor differences in the details of the actual electron density,
%the agreements between the two maps are surprisingly well, especially
%the main structural features.

%Although the electron density maps are helpful in getting the
%overall view of the ripple phase structure, slice electron
%density profiles are needed to obtain accurate ripple structural
%quantities along some special directions. Three such slices
%are taken and the electron density profiles along them are
%shown in Figure \ref{slice}. Slice B is along the normal of the
%major side and the two electron density profiles (data and model)
%along it are shown in Fig. \ref{slice}(B), from which the bilayer
%head-head spacing cross the major side is found to be about 38\AA,
%and the water spacing is about 21\AA. Both the data and model maps
%give the same spacings. For the minor side, two slices are taken,
%one going through the center of the chain region (Slice C in
%Fig. \ref{slice}(A)) and another through the center of the water
%region (Slice D in Fig. \ref{slice}(A)). The electron density
%profiles along slices C and D are shown in Fig. \ref{slice}(C) and
%(D). The bilayer head-head spacing on the minor side is 31\AA, smaller
%than that of the major side (37\AA). The water spacing of minor side
%is 18\AA, also smaller than that of the major side (21\AA). Again,
%both the data and model maps give the same spacings. It is noticeable
%that all these spacings from the slice electron density profiles are
%slightly different from the corresponding spacings directly derived from the
%best values of fitting parameters (See Table \ref{parameter}). 
%These differences are caused by the well-known Fourier truncation error
%when only limited numbers of diffraction orders are used in the Fourier 
%reconstruction. Since in Wack and Webb's data, only the form factors 
%up to three lamellar orders were measured and reported, these truncation 
%effects are not unexpected.

%Another variation
%was to adopt the results of Wiener's \cite{MWTH} 1G hybrid model fit
%to the DPPC gel phase data. Most of Wiener's best fitting values were
%retained while the head-head distance was set free because DMPC molecules
%are two carbons short on each chain compared to DPPC. After incorporating
%this variation into our model as the transbilayer electron density, a fitting
%was performed with surprising results: almost identical values of the primary
%fitting parameters and identical phases were obtained (not shown). This 
%implies that the ripple phase has similar local structures as the gel phase. 
%Certainly it is desirable to replace the delta function model with the 1G 
%hybrid model, but because Wack and Webb's ripple phase data only reported
%three lamellar orders, while the 1G hybrid model requires more than that not 
%to be underdetermined, we stuck to the delta function model, which turned 
%out to be working well.

%We also performed x-ray experiments on a DMPC sample with almost identical
%hydration level as in Wack and Webb's \cite{Wac89a}. The aim was to test
%the hypothesis that the ripple phase has a gel-phase-like local bilayer
%structure. The sample was partially dehydrated by a 40\% PVP water solution
%(PVP is a polymer whose water solution exerts an osmotic pressure
%on the lipid bilayers, i.e., PVP molecules compete with the bilayers for
%water without entering into or in between the bilayers). At 18$^{\circ}$C, 
%this sample was in the ripple phase, with a d-spacing of 58.10\AA, comparing 
%well with Wack and Webb's 57.94\AA. This implies that our DMPC sample was at 
%about the same hydration level as Wack and Webb's sample. We then took
%the low angle data at 6$^{\circ}$C at which the sample was clearly in the
%gel phase with no ripple peaks present. Five orders were recorded. The electron
%density profile of this DMPC sample at 6$^{\circ}$C was reconstructed using
%the first three orders, in accord with the lamellar orders in Wack and Webb's
%data, and compares well with the major side slice profile of Wack and Webb's
%data electron density map (See Fig. \ref{gel_rppl}). The slight difference
%in the headgroup positions of the two profiles in Fig. \ref{gel_rppl} could
%be due to the small difference in the d-spacings. This gel phase profile
%gives a head-head bilayer thickness of 36.96\AA\ and a water spacing of
%20.79\AA, agreeing well with the ripple phase's 37.86\AA\ and 20.73\AA.
%The electron density profile reconstructed with the first four orders
%gives a head-head spacing of 40.42\AA\ and water spacing of 17.33\AA, 
%comparing with 40.0\AA\ and 18.5\AA\ given by the ripple phase model
%fitting results (see Table \ref{parameter}). Five order reconstruction
%gives 38.12\AA\ and 19.63\AA, close to those of the three order reconstruction.
