\section{Introduction}
\label{rppl_intro}

Among the thermodynamic phases observed in
lipid bilayer systems, the periodically modulated ripple (P$_{\beta '}$) phase
is especially fascinating as evidenced by the many theoretical papers that
have attempted to explain the formation of the ripples
\cite{Don79,Mar84,Haw86,Car87,Gol88,McCu90,Hon91,Lub93}. 
Such understanding has been delayed, however, because the structure has
not been as well characterized as for the lower temperature gel
(L$_{\beta '}$) phase and the higher temperature liquid crystalline
(L$_{\alpha}$) phase, despite many experimental studies 
\cite{Tar73,JanSS79,Luna77,Ino80,Cop80,Sta82,Rup83,Ale85,Sir88,%
Wac89a,Kat95}.  When an accurate ripple
structure is known, some indication of the energetics of the ripple formation
may be obtained by comparing the relative size of the ripple
wavelength and the ripple amplitude to the size of lipid molecules and
the bilayer thickness \cite{Gol88}.  Also, correlation 
between the chirality of the constituent lipid molecules and the asymmetry of 
the ripples may lead to elucidation of the microscopic origin 
of the macroscopic properties of the ripple phase \cite{Lub93,Kat95}.

Structural information about the ripple phase can be divided into two
categories. The first category consists of the 2D lattice parameters, namely, 
the ripple wavelength $\lambda_r$, the bilayer packing repeat distance $d$ and
the oblique angle $\gamma$ for the unit cell that are illustrated
in Fig. \ref{unitcells}.  These well-established parameters have been 
accurately obtained by indexing low-angle x-ray scattering peaks
\cite{Tar73,JanSS79,Ino80,Ale85,Wac89a}. 
For example, Wack and Webb \cite{Wac89a} reported for DMPC with
25\% water by weight at 18$^{\circ}$C that $\lambda_r$ = 141.7\AA, 
$d$ = 57.94\AA\ and $\gamma$ = 98.4$^{\circ}$.

The second category
consists of the structure within the ripple unit cell, including
the shape and the amplitude of the ripples and the bilayer thickness. 
This second category of structural information
has been quite uncertain.  Many ripple profiles have been 
proposed \cite{Don79,Mar84,Car87,Gol88,McCu90,Hon91,Lub93,Tar73,JanSS79,Sta82},
including sinusoidal, peristaltic and sawtooth and
estimates of the ripple trough-to-peak 
amplitude range widely from 15\AA\ to 50\AA\ 
\cite{Tar73,JanSS79,Sta82,Zas88a,Hata93}, obtained mostly from 
x-ray and freeze fracture electron microscopic work.

The information about this second category lies in the intensity distribution 
of the x-ray diffraction peaks. 
Electron density maps are needed to determine the 
structure of the ripple phase. In order to Fourier reconstruct the 
electron density 
maps from the experimentally measured form factor amplitudes, 
correct phases are essential. The amplitudes of the form factors are 
simply the square roots of the Lorentz-corrected x-ray intensities;
but the phasing information is not contained in the 
intensity data and has to be sought using other means.

There are two approaches to obtain the phasing information.
The more traditional approach is the ``pattern recognition''
method employed by Tardieu et al. \cite{Tar73}; that method 
selects those phase combinations which will give rise to electron density maps 
agreeing with known physical properties of the bilayer system, out of all 
possible phase combinations. Considering that there are typically 20 or so 
peaks in most of the ripple phase x-ray data, finding the right phase 
combination out of about 2$^{20}$ possibilities is a difficult task. 

The second approach, as we will employ in this chapter, is the modeling 
approach. This modeling approach first creates a functional form for the 
electron density that incorporates known and plausible properties of the 
bilayer but that contains free parameters to incorporate unknown detailed 
structure. The success of this method depends on this modeling step which 
requires some care and is therefore the emphasis of the next section.
Then, the values of the free parameters in the real space model are 
determined by routine non-linear least squares procedures so that the Fourier
components of the model provide the best fit to the measured intensities.
This last step automatically determines the phases of the model, which are
then used with the original intensity data to produce an electron density map.

% ----------------- Old text -----------------------------------------------
%It will also be helpful to compare the ripple structural features with that 
%of the gel and fluid phases, such as to see their relations and transitions.

%But it has been long overlooked that in a structure with periodic modulation
%like the ripple phase, the out-plane buckling of the bilayers can lead to
%an increase in the apparent d-spacing along the normal direction of the
%average membrane plane, while the structure along the local normal may remain
%less altered. The local d-spacing is simply the product of the apparent
%d-spacing and the cosine of the local ripple slope. A big increase in the
%apparent d-spacing may be just caused by a bigger slope, not necessarily
%implying a big change in the local d-spacings, which is much more relevant
%to the interbilayer interactions and the hydrocarbon chain configurations. 
%To get averaged local area per lipid molecular, lipid bilayer thickness and
%the water spacing, the amplitude and the profile of the ripples have to be
%accurately obtained.

%The second approach involves modeling and nonlinear least square fitting.
%This approach requires the following steps: 1). Build up an electron density 
%model by incorporating various known physical properties 
%of the bilayer; 2). Fourier transform the model electron density to get model 
%form factors (or scattering intensities); 3). Fit the experimentally measured
%amplitudes of form factors (or intensities) to the model and get the structural
%quantities and, as a by-product, the phases. It should be emphasized that in 
%this approach, phases are outputs instead of inputs as in the first approach
%and are not essential to obtaining the structural quantities, but are very
%useful in reconstructing the electron density map from the data and comparing
%that with the map from the model fitting.
%Wiener et al. \cite{WSN89} used this approach successfully to obtain many
%key structural quantities of gel phase DPPC bilayers. 
%Janiak et al. \cite{JanSS79} also followed this route and tried to solve 
%the ripple phase structures of DMPC at two different hydration levels. 
%But because their data were from low resolution measurements, which made
%the decomposition of overlapping peaks subject to large ambiguity, their
%results are not definitive. Nonetheless, they obtained a ripple amplitude
%of about 15\AA\ for DMPC at two hydration levels, which compares with
%15\AA\ for DLPC reported by Tardieu et al. \cite{Tar73}.

%The 
%difficulty associated with this approach can be seen from the reconstructed
%electron density map of DLPC published in \cite{Tar73} by Tardieu et al.,
%where one sees rather irregular bilayers with an arrangement of the
%ripples within the unit cell which does not give rise to the typical q-space
%ripple phase patterns shown in oriented sample x-ray data \cite{Ale85,Kat95}.

%The task of this chapter is to propose a more precise electron density model 
%which is consistent with both available experimental results and known 
%physical properties of the lipid bilayers, and to fit Wack and Webb's
%published high resolution x-ray form factor amplitude data to this model.
%Both accurate structural quantities and form factor phases are obtained 
%from the fitting. Ripple electron density maps, both from Wack and Webb's
%data and the model fitting, are Fourier reconstructed using the phases
%obtained from the model fitting. Comparisons with previous ripple phase
%works are presented. Implications of the model fitting results in 
%understanding the ripple phase structure will be discussed.
