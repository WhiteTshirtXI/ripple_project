\begin{figure}
\centerline {\psfig{figure=rppl_dir/lattice_dim.eps,height=3in}}
\vspace{11pt}
\caption{(i). Ripple unit cell lattice parameters; (ii). The real space and
the reciprocal space unit cells, and the 2D coordinate system used in this 
work.
\label{unitcells}}
\end{figure}

%----------------------------------------------------------------------------
\section{Modeling method}
\label{rppl_theory}

%The modeling approach first creates a functional form for the electron density 
%that incorporates known and plausible properties of the bilayer but that
%contains free parameters to incorporate unknown detailed structure.
%The success of the method depends on this modeling step which 
%requires some care
%and is therefore the emphasis of this section.
%Then, the values of the free parameters in the real space model are 
%determined by routine non-linear least squares procedures so that the Fourier
%components of the model provide the best fit to the measured intensities.
%This last step automatically determines the phases of the model, which are
%then used with the original intensity data to produce an electron density map.

%----------------------------------------------------------------------------
\subsection{Lattice structure}
\label{rppl_theory_1}

It has been shown from x-ray experiments \cite{Tar73,JanSS79,Ino80,Ale85,%
Wac89a,Kat95} that ripples in different bilayers are registered to form a 
2-D oblique lattice, which can be adequately described by three parameters: 
the ripple wavelength $\lambda_r$, the oblique angle $\gamma$ and the stacking
repeat $d_s$ (See Figure \ref{unitcells}(i)). The lamellar repeat along 
the average bilayer
normal direction, $d$, is related to $\gamma$ and $d_s$ by
\begin{eqnarray}
\label{dspace}
d = d_s\ \sin \gamma.
\end{eqnarray}

By choosing the ripple wave vector direction as the $x$ axis and the average
bilayer normal as the $z$ axis, the ripple unit cell vectors can be 
expressed as (See Figure \ref{unitcells}(ii)):
\begin{eqnarray}
\label{realunit}
\left\{ \begin{array}{ll}
\vec{a} = d_s \cos \gamma\ \hat{x} + d\ \hat{z},\\
\vec{b} = \lambda_r\ \hat{x}.
\end{array}
\right.
\end{eqnarray}

The corresponding reciprocal lattice unit cell vectors are 
(See Fig. \ref{unitcells}(ii)):
\begin{eqnarray}
\label{reciunit}
\left\{ \begin{array}{ll}
\vec{A} = \frac{2 \pi}{d}\ \hat{z},\\
\vec{B} = \frac{2 \pi}{\lambda_r}\ \hat{x} - \frac{2 \pi}{\lambda_r 
\tan \gamma}\ \hat{z}.
\end{array}
\right.
\end{eqnarray}
So the q-space vector for the Bragg peak with Miller indices $(h,k)$
is:
\begin{eqnarray}
\label{qvector}
\vec{q}_{hk} =&& h \vec{A} + k \vec{B} \nonumber\\
=&& \frac{2 \pi k}{\lambda_r}\ \hat{x} + \left( \frac{2 \pi h}{d} -
\frac{2 \pi k}{\lambda_r \tan \gamma} \right) \hat{z}.
\end{eqnarray}

%----------------------------------------------------------------------------
\subsection{Electron density and form factor}
\label{rppl_theory_2}

The overall electron density $\rho (x,z)$ within the ripple unit cell 
can be described as the convolution of the ripple contour function $C(x,z)$ 
and the trans-bilayer electron density profile $T(x,z)$:
\begin{eqnarray}
\label{model}
\rho (x,z) = C(x,z) \ast T(x,z),
\end{eqnarray}
where `$\ast$' stands for convolution. $C(x,z)$ is related to the ripple 
profile $u(x)$, which is the trajectory of the center of the bilayer (See
Fig. \ref{rppl_profile}), by:
\begin{eqnarray}
\label{contour}
C(x,z) = \delta [z - u(x)].
\end{eqnarray}

The form factor $F(\vec{q})$ is the Fourier transform of the electron density
expressed in Eqn. \ref{model}. Using the Convolution Theorem in Fourier
Transform, we have:
\begin{eqnarray}
\label{formall}
F(\vec{q}) =&& {\cal F} \left[ C(x,z) * T(x,z) \right]\nonumber\\
=&& {\cal F} \left[ C(x,z) \right] \times {\cal F} \left[ T(x,z) \right]\nonumber\\
=&& F_C(\vec{q}) \times F_T(\vec{q}),
\end{eqnarray}
where ${\cal F} [\ ]$ represents the Fourier transform. Eqn. \ref{formall}
shows that the overall form factor can be conveniently separated into
the contour part $F_C(\vec{q})$ and the trans-bilayer part $F_T(\vec{q})$. 
It will be shown in later sections that generally $F_C(\vec{q})$ determines
the angular scattering intensity distribution in q space.
%and $F_T(\vec{q})$ determines the radial 
%intensity distribution in the 2-D q-space.

In the following sections, detailed calculations will be performed using an
asymmetric triangular ripple profile and a simple trans-bilayer electron 
density profile.

%----------------------------------------------------------------------------
\subsection{Ripple profile}
\label{rppl_theory_3}

\begin{figure}
\centerline {\psfig{figure=rppl_dir/ripple_profile.eps,height=3in}}
\vspace{11pt}
\caption{(i). Parameters describing the ripple profile; (ii). 
The origin of the coordinate system is chosen to be at the center
of the major side, to utilize the inversion symmetry. The ripple unit
cell is shown as indicated.
\label{rppl_profile}}
\end{figure}

The results of freeze fracture electron microscopy \cite{Luna77,Cop80,Rup83} 
and scanning tunneling microscopy \cite{Zas88a,Hata93} strongly suggest that 
the ripple profile is asymmetric.  This conclusion is further supported by 
the results of x-ray diffraction that the ripple unit cell is oblique
($\gamma$ not equal to 90$^{\circ}$) 
\cite{Tar73,JanSS79,Ino80,Ale85,Sir88,Wac89a,Kat95}, and that the angular 
scattering intensity distribution is uneven in the 2D oriented q plane
\cite{Ale85,JanSS79}.

In the light of these experimental evidences, an asymmetric, 
triangular ripple profile u(x) is chosen as shown in Fig. \ref{rppl_profile}.
In the rest of this chapter, we shall call the longer side of the ripple 
the {\it major} side, and the shorter side the {\it minor} side 
(See Fig. \ref{rppl_profile}).  As shown in Fig. \ref{rppl_profile}(i), the 
triangular ripple profile can be well described by the amplitude $A$, 
the slope angle of the major side, $\alpha$, and $x_0$, the projection of 
the major side onto the $x$ axis. Of these three parameters, only two of 
them are independent and they are related by the following equation:
\begin{eqnarray}
\label{tripara}
A = x_0\ \tan \alpha.
\end{eqnarray}
We will choose $A$ and $x_0$ as the independent parameters of this model.
By using these parameters and choosing the origin at the center of the 
major side (see Fig.\ref{rppl_profile}(ii)), the ripple profile can be 
expressed as
\begin{eqnarray}
\label{profile}
u(x) = \left\{ \begin{array}{ll}
		-\frac{A}{\lambda_r - x_0} (x + \frac{\lambda_r}{2})
\ \ \ \mbox{if}\ - \frac{\lambda_r}{2} \le x < - \frac{x_0}{2};\\
		\ \ \ \ \ \frac{A}{x_0}\ x\ \ \ \ \ \ \ \ \ \ \ \ \mbox{if}
\ \ - \frac{x_0}{2} \le x \le 
\frac{x_0}{2};\\
		-\frac{A}{\lambda_r - x_0} (x - \frac{\lambda_r}{2})
\ \ \ \mbox{if}\ \ \ \ \  \frac{x_0}{2} < x \le \frac{\lambda_r}{2}.
		\end{array}
	\right.
\end{eqnarray}
This particular choice of the origin shows explicitly the inversion
symmetry in the ripple profile, so that the form factors are real.

Now, the contour part of the form factor, $F_C(\vec{q})$ 
(see Eqn.\ref{formall}), is
\begin{eqnarray}
\label{formcon}
F_C(\vec{q}) =&& \frac{1}{\lambda_r}\ \int_{- \frac{\lambda_r}{2}}^{ 
\frac{\lambda_r}{2}} dx
\int_{- \frac{d}{2}}^{\frac{d}{2}} dz\ C(x,z)\ \mbox{e}^{i \vec{q} \cdot \vec{r}} \nonumber\\
=&& \frac{1}{\lambda_r}\ \int_{- \frac{\lambda_r}{2}}^{\frac{\lambda_r}{2}} dx
\int_{- \frac{d}{2}}^{\frac{d}{2}} dz\ \delta [z - u(x)] \ \mbox{e}^{i (q_x x + q_z z)} \nonumber\\
=&& \frac{1}{\lambda_r}\ \int_{- \frac{\lambda_r}{2}}^{\frac{\lambda_r}{2}} dx\ \mbox{e}^{i [q_x x + q_z u(x)]}.
\end{eqnarray}
By introducing an intermediate variable
\begin{eqnarray}
\label{w}
\omega (\vec{q}) = \frac{1}{2}\ ( q_x x_0 + q_z A),
\end{eqnarray}
and using the triangular ripple profile given in Eqn. \ref{profile}, 
Eqn. \ref{formcon} can be integrated out to be
\begin{eqnarray}
\label{formcon1}
F_C(\vec{q}) = \frac{x_0}{\lambda_r}\ \mbox{sinc}(\omega)
+ \frac{\lambda_r - x_0}{\lambda_r}\ \frac{\cos \frac{1}{2} (\frac{1}{2}
q_x \lambda_r + \omega)}{\cos \frac{1}{2} (\frac{1}{2} q_x \lambda_r -
\omega)}\ 
\mbox{sinc}(\frac{q_x \lambda_r}{2} - \omega ).
\end{eqnarray}
The first term in the parentheses of Eqn. \ref{formcon1} is the contribution 
from the major side of the ripple, with scattering amplitude 
\(\frac{x_0}{\lambda_r}\), proportional to its projection length onto the 
x axis, while the second term is from the minor side of the ripple, with 
amplitude \(\frac{\lambda_r - x_0}{\lambda_r}\). 
If $x_0 > \frac{\lambda_r}{2}$, the major side should scatter more strongly
than the minor side, and vice versa. Eqn. \ref{formcon1} also shows that
the maximum scattering of the major side occurs along the direction determined 
by $\omega = 0$ in Eqn. \ref{w}; in real space this also the direction
normal to the major side of the ripple.  Similarly, the maximum scattering
direction of the minor side is given by \( \frac{q_x \lambda_r}{2} - \omega 
= 0 \), and this can be shown to be the direction normal to the minor side.
These observations can be seen in Fig. \ref{q_pattern}.

\begin{figure}
\centerline {\psfig{figure=rppl_dir/q_profile.ps,height=3in}}
\vspace{11pt}
\caption{Real space ripple structure (only the ripple profile part) and its 
corresponding q-space pattern. The dot-dashed lines show the normals
of the major and minor sides. Notice that the two maximum
scattering directions are along the two normals, and that
the scattering along the normal of the major side is stronger than
along the normal of the minor side. Darker spots represent stronger
scatterings.
\label{q_pattern}}
\end{figure}

Some important conclusions can be drawn from the analysis up to now:
(i) The q-space angular distribution of the scattering intensity from 
ripples is 
determined by the geometry of the ripple profile. Angular scattering maxima
occur at the average normal directions of the two sides of the ripple.
(ii) The uneven intensity distribution among the two maximum scattering
directions is due to the asymmetry of the ripple 
profile, namely, the major side of the ripple contains more scattering
materials (lipid molecules), thus scatters more strongly along its normal 
direction compared to the maximum scattering of the minor side, which 
has fewer lipid molecules. (iii) The experimentally observed uneven angular 
distribution of 
scattering intensity among Bragg peaks \cite{Ale85,Wac89a,Kat95} can be used 
to deduce the average ripple profile. These conclusions would also
be expected to pertain to ripple profiles that are not perfectly triangular,
for example, with rounded corners. 

Combining Eqn. \ref{qvector} and Eqn. \ref{formcon1}, the discrete contour form
factor for the Bragg peak with Miller indices $(h,k)$ can be expressed as:
\begin{eqnarray}
\label{discrete}
F_C(h,k) =&& \frac{x_0}{\lambda_r}\ \mbox{sinc}
(\omega_{hk}) + (-1)^k\ \frac{\lambda_r - x_0}{\lambda_r}\ 
\mbox{sinc}(k \pi - \omega_{hk})\nonumber\\
=&& \sin \omega_{hk}\ \left[\frac{k \pi x_0 - 
\omega_{hk} \lambda_r}{\omega_{hk} \lambda_r (k \pi - \omega_{hk})} \right],
\end{eqnarray}
where
\begin{eqnarray}
\label{whk}
\omega_{hk} =&& \frac{1}{2} (q_x^{hk} x_0 + q_z^{hk} A)\nonumber\\
=&& \pi\ \left[ \frac{h A}{d} + \frac{k}{\lambda_r}\ (x_0 -
\frac{A}{\tan \gamma}) \right].
\end{eqnarray}

The q-space pattern of $|F_C(h,k)|^2$ is shown in Fig. \ref{q_pattern}. 
The two maximum scattering directions and the uneven intensity distribution 
among them can be seen clearly, as well as their
relations to the ripple profile. 
This theoretical picture for the location of the strong (h,k) peaks
agrees qualitatively with the experimental x-ray intensity data
\cite{Ale85,Wac89a,Kat95}. 

%---------------------------------------------------------------------------
\subsection{Trans-bilayer electron density profile $T(x,z)$}
\label{rppl_theory_4}

\begin{figure}
\centerline{\psfig{figure=rppl_dir/transedp.ps,height=3.0in}}
\caption{
Two transbilayer electron density profiles we use in our model.
The solid line indicates the simple Delta function model; the dashed line
shows the Gaussian model.
\label{transedp}}
\end{figure}

\begin{figure}
\begin{center}
\leavevmode
\raggedleft 
\hspace{0.2in}
\psfig{figure=rppl_dir/psi1.ps,height=2.0in}
\leavevmode
\raggedright 
\hspace{0.7in}
\psfig{figure=rppl_dir/psi2.ps,height=2.0in}
\end{center}
\vspace{-0.2in}
\hspace{1.2in} (A) \hspace{2.7in} (B)
\vspace{0.2in}
\caption{Illustrations of the trans-bilayer structure. The monolayer
shift angle $\psi$ and headgroup position X$_h$ are defined as
indicated. The halftone stripes denote the bilayers; The thick black lines 
represent the central profiles of the rippled bilayers; The white spaces
between the adjacent bilayers are the water regions. The thin solid
lines represent the vertical direction and the dot-dashed lines are the
connecting lines of the deflection points of the upper and lower monolayers.
The white dots are the centers of the above mentioned connecting lines. 
(A). A possibility for $\psi$; (B). An unlikely possibility for $\psi$.
\label{trans}}
\end{figure}

Because of the bilayer nature of the lipid water system, it is natural 
to start with a simple, symmetric transbilayer electron density profile. 
In a study of gel phase DPPC bilayers using the electron
density modeling approach \cite{WSN89} it was found
that the precise functional form of the electron density model did not
much affect the results provided that the essential structural
features of bilayers were incorporated into the model.
We therefore first utilized a very simple model for $T(x,z)$ in
which the two opposite headgroup regions consist of 
two identical positive $\delta$-functions, with amplitude $\rho_H$ and
separated by the head-head spacing $2X_h$ (See Fig. \ref{transedp}), and
the central methyl trough consists of a negative $\delta$-function, with 
amplitude $\rho_M$. In order to avoid possible conflict between adjacent 
rippled bilayers, the two opposing monolayers of the same bilayer in this 
model are allowed to shift by a angle $\psi$ (See Fig. \ref{trans}).
The methylene region of bilayers has an electron density
close to that of water and hence is set equal to zero in the
following model (minus fluid \cite{WKMc73}) of $T_{\psi}(x,z)$,
\begin{eqnarray}
\label{transbi}
T(x,z) = \delta (x + z \tan \psi)\ \left\{ \rho_H [\delta (z-X_h\cos \psi)
+\delta (z+X_h \cos \psi)]-\rho_M \delta (z) \right\},
\end{eqnarray}
Then the form factor corresponding to this trans-bilayer 
electron density profile is:
\begin{eqnarray}
\label{form_trans}
F_T (\vec{q}) =&& \int_{- \infty}^{\infty} dx \int_{- \infty}^{\infty} dz
\ T(x,z) \ \mbox{e}^{i \vec{q} \cdot \vec{r}} \nonumber\\
=&& 2 \rho_H\ \cos (q_z X_h \cos \psi - q_x X_h \sin \psi) - \rho_M \nonumber\\
=&& \rho_M\ [R_{HM} \cos (q_z X_h \cos \psi - q_x X_h \sin \psi) -1],
\end{eqnarray}
where \(R_{HM} = \frac{2\ \rho_H}{ \rho_M}\). $F_T (\vec{q})$ is also
a real quantity, as $F_C (\vec{q})$, because of the inversion symmetry of the
trans-bilayer electron density profile relative to the center of the bilayer.
$R_{HM}$, $\psi$ and $X_h$ are the independent parameters which will enter
the model fitting. The discrete form of Eqn. \ref{form_trans} is:
\begin{eqnarray}
\label{dis_trans}
F_T (h,k) = \rho_M\ [R_{HM} \cos (q_z^{hk} X_h \cos \psi 
		- q_x^{hk} X_h \sin \psi) -1].
\end{eqnarray}

We will call the above model the simple delta function (SDF) model.
Following \cite{WSN89} we also utilized a more realistic model, called the 1G 
hybrid model that consists of Gaussian functions for the headgroups and
the terminal methyl trough and that also allows for differences between 
the electron density of methylenes and of water. We will call this the simple 
1G (S1G) model. We also considered more complex models that modulated the 
electron density along the direction perpendicular to $\vec{a}$ in real space 
in Fig. \ref{unitcells}; this is also the direction $\vec{B}$ in reciprocal 
space, so this modulation strongly affects the $(0,k)$ pure ripple form 
factors. Our best ripple modulated models (MDF and M1G) have two additional 
parameters, one to allow for the electron density across the minor side to 
be different by a ratio $f_1$ from the electron density across the major 
side and a second parameter $f_2$, which is multiplied by delta functions 
${\delta}(x \pm x_0/2)$ to allow for a different electron density near 
the kink between the major and the minor sides.

%----------------------------------------------------------------------------
\subsection{Overall form factor and the scattering intensity}
\label{rppl_theory_5}

Combining Eqns.\ref{formall}, \ref{discrete} and \ref{dis_trans}, the overall 
form factor for the (h,k) peak of the SDF model is:
\begin{eqnarray}
\label{form_dis}
F(h,k) =&& F_C (h,k) \times F_T (h,k) \nonumber\\
=&& \sin \omega_{hk} \left[\frac{k \pi x_0 -
\omega_{hk} \lambda_r}{\omega_{hk} \lambda_r (k \pi - \omega_{hk})} \right]
 \rho_M \nonumber\\ 
&& \times\ [R_{HM} \cos (q_z^{hk} X_h \cos \psi - q_x^{hk} X_h \sin \psi) -1].
\end{eqnarray}
$F(h,k)$ is real because the overall electron density function has the 
inversion symmetry as can be seen from the last three subsections.
The overall Lorentz-corrected scattering intensity of the (h,k) peak
of the SDF model is:
\begin{eqnarray}
\label{inten_dis}
I(h,k) =&& |F(h,k)|^2 \nonumber\\
=&& \rho_M^2\ \sin^2 \omega_{hk}\ \left[\frac{k \pi x_0 -
\omega_{hk} \lambda_r}{\omega_{hk} 
\lambda_r (k \pi - \omega_{hk})} \right]^2 \nonumber\\ 
&& \times\ [R_{HM} \cos (q_z^{hk} X_h \cos \psi - q_x^{hk} X_h \sin \psi) -1]^2.
\end{eqnarray}

For the SDF model, there will be six independent parameters when the 
non-linear least square fitting is performed. Two of them are related to the 
ripple profile, namely the ripple amplitude $A$ (contained in $\omega_{hk}$, 
see Eqn.\ref{whk}) and the horizontal projection length of the major side of 
the ripple, $x_0$. The trans-bilayer part contains three, the headgroup to 
methyl trough electron density ratio, $R_{HM}$, the monolayer shift angle 
$\psi$ and the headgroup position $X_h$. the sixth parameter is an overall 
scaling factor for $F(h,k)$ or $I(h,k)$ when actually fitting to the data. 

For the S1G model, all the trans-bilayer profile parameters
were taken from gel phase data \cite{WSN89} except for the headgroup position
$X_h$. The MDF and M1G models had the two additional parameters $f_1$ and 
$f_2$ mentioned above. (The parameters ${\lambda}_r$, d and $\gamma$ are known
from indexing the (h,k) peaks.)
 
%----------------------------------------------------------------------------
\subsection{Determination of the phases and the electron density map}
\label{rppl_theory_6}

In order to reconstruct the electron density maps of the ripple phase
from the experimentally measured amplitudes of the form factors $|F(h,k)|$, 
correct phases have to be found out first. There are two ways of getting 
the phases, as discussed in the introduction section. The first is 
the pattern recognition method. This approach
is difficult when the number of peaks is large. The second approach,
which we follow here, is to get phases from model fitting. 
In our approach, standard non-linear least squares fitting procedures were 
employed to obtain unknown parameters by finding the best
fit of the intensities given by Eqn. \ref{inten_dis} to the $|F(h,k)|^2$ 
reported by Wack and Webb \cite{Wac89a}. 
This procedure does not require knowing any phases in advance.  However,
once the best values of the parameters in the electron density model are
determined, the form factors, and especially the phases, are determined
from Eqn. \ref{form_dis}.
Using these phases and the amplitudes of the form factors $|F(h,k)|$ from 
the intensity data \cite{Wac89a},
an experimental relative electron density map $\rho(x,z)$ can be obtained. 

The formula for reconstructing the electron density maps
from discrete form factors is derived as follows. It is easy to see that 
the general form factor satisfies 
\begin{eqnarray}
\label{formallcon2}
F(-\vec{q}) = \ F^{*}(\vec{q}),
\end{eqnarray}
as it should be because $F(\vec{q})$ is the Fourier transform of a real
function (the electron density).

The electron density is the inverse Fourier transform of $F(\vec{q})$
(Keep in mind that we are dealing with 2D structures here, so that
\( \vec{r} = x \hat{x} + z \hat{z}, \vec{q} = q_x \hat{x} + q_z \hat{z}\)),
\begin{eqnarray}
\label{edrecon}
\rho(\vec{r}) =&& \int d \vec{q}\ F(\vec{q})\ 
\mbox{e}^{- i \vec{q} \cdot \vec{r}}\nonumber\\
=&& \int_{- \infty}^{\infty} dq_x \int_{- \infty}^{\infty} dq_z
F(\vec{q})\ \mbox{e}^{- i \vec{q} \cdot \vec{r}}\nonumber\\
=&& \int_{- \infty}^{\infty} dq_x \int_{0}^{\infty} dq_z
\left[ F(\vec{q})\ \mbox{e}^{- i \vec{q} \cdot \vec{r}} +
F(-\vec{q})\ \mbox{e}^{i \vec{q} \cdot \vec{r}}\right]\nonumber\\
=&& \int_{- \infty}^{\infty} dq_x \int_{0}^{\infty} dq_z
\left[ F(\vec{q})\ \mbox{e}^{- i \vec{q} \cdot \vec{r}} +
F^{*}(\vec{q})\ \mbox{e}^{i \vec{q} \cdot \vec{r}}\right]\nonumber\\
=&& \int_{- \infty}^{\infty} dq_x \int_{0}^{\infty} dq_z
\ 2\ Re \left[ F(\vec{q})\ \mbox{e}^{- i \vec{q} \cdot \vec{r}}\right]\nonumber\\
=&& 2\ \int_{- \infty}^{\infty} dq_x \int_{0}^{\infty} dq_z
\ F(\vec{q}) \cos( \vec{q} \cdot \vec{r}).
\end{eqnarray}
In the last step, we used the result that $F(\vec{q})$ is real.

Ignoring the irrelevant constant prefactor, the discrete form for 
the relative electron density $\rho(\vec{r})$ is 
\begin{eqnarray}
\label{discretell}
\rho(x,z) = \sum_{h \geq 0} \sum_{k} F(h,k) \ \cos(q^{hk}_x x + q^{hk}_z z).
\end{eqnarray}
Eqn. \ref{discretell} will be the formula used to reconstruct the electron 
density maps presented in this chapter. 

\pagebreak
