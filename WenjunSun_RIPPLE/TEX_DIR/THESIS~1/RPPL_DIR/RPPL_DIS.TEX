\section{Discussion}

\begin{figure}
\begin{center}
\leavevmode
\raggedleft
\hspace{0.1in}
\psfig{figure=rppl_dir/wack_map.ps,height=2.5in}
\leavevmode
\raggedright
\hspace{0.6in}
\psfig{figure=rppl_dir/tardieu_map.ps,height=2.5in}
\end{center}
\vspace{-0.2in}
\hspace{1.2in} (A) \hspace{2.9in} (B)
\vspace{0.2in}
\begin{center}
\leavevmode
\raggedleft
\hspace{0.1in}
\psfig{figure=rppl_dir/janiak_map1.ps,height=2.5in}
\leavevmode
\raggedright
\hspace{0.6in}
\psfig{figure=rppl_dir/janiak_map2.ps,height=2.5in}
\end{center}
\vspace{-0.2in}
\hspace{1.2in} (C) \hspace{2.9in} (D)
\vspace{0.2in}
\caption{Electron density maps of the ripple phase. 
(A) Reconstructed from the form factor amplitude
data in Wack and Webb [17] with our model fit phases (without using
the (0,k) ripple peaks); same as in Fig. 6.5. The sample was
DMPC with 25\% water (by weight) at 18$^{\circ}$C. The width of the image is
283.4\AA. Brightness corresponds to higher electron density.
(B) Reconstructed
from the results in Tardieu et al. [1]. The sample was DLPC with 23\%
water (by weight) at -7$^{\circ}$C. The width of the image is 170.6\AA.
(C) Reconstructed from the results in Janiak et al. [3]. The sample was
DMPC with 21\% water (by weight) at 20$^{\circ}$C. The width of the image is
321.8\AA.
(D) Reconstructed from the results in Janiak et al. [3]. The
sample was DMPC with 30\% water (by weight) at 20$^{\circ}$C. The width
of the image is 236.0\AA. 
\label{others}}
\end{figure}

The first result of this chapter is to obtain an electron density map
for the ripple phase. This requires accurate measurement of the amplitudes
of the form factors, as has been done by Wack and Webb \cite{Wac89a},
and the phases obtained from our model fittings. 
As mentioned in Section
\ref{rppl_intro}, the `pattern recognition' method employed by Tardieu et al. 
\cite{Tar73} is subject to large ambiguity and difficulty when the number
of peaks is large. We assert that the modeling approach is better.
To compare with previous ripple phase work and
the importance of obtaining the correct phases, electron density
maps reconstructed from previous work are shown in Fig. \ref{others}.
Fig. \ref{others}(A) shows the same electron density map as in 
Fig. \ref{wack_map}, reconstructed from the form factor amplitude
data of Wack and Webb [17] with our model fit phases (without using
the (0,k) ripple peaks).
Fig. \ref{others}(B) shows Tardieu et al.'s result, in which 
one can still see a bilayer structure, although the headgroup regions 
are rather irregular. Figs. \ref{others}(C) and (D) are from 
Janiak et al.'s modeling results; it is difficult to see from these two maps
the typical bilayer structure of a lipid/water system. 
The correctness of the phases determined by this work
can be seen clearly from the comparisons of our map with others.

%Fig. \ref{others}(D) was reconstructed from Wack and 
%Webb's form factor amplitude data \cite{Wac89a} with random phase assignment. 
%By comparing Fig. \ref{others}(D) with Fig. \ref{wack_map}, one can see the 
%importance of the determination of the correct phases.

Our result of slice electron density profiles (see Fig. \ref{slice}) 
strongly suggests that the lipid bilayer along the major
side of the ripple is similar to the gel phase bilayer, with similar
head-head spacings. This conclusion is strongly supported by the resulting
excellent fits of 1G models (S1G and M1G), where all of the key parameters
were fixed to the best fitting values for gel phase DPPC obtained from
\cite{MWTH}, except for the head-head spacing 2X$_h$; this parameter
should be different for the shorter chain lipid DMPC. 

In contrast, the bilayer along the minor side appears to be
thinner and this suggests the working hypothesis that the
minor side may be more like the fluid $L_{\alpha}$ bilayer.
A supporting argument for this hypothesis is that the electron density 
in the headgroup region along the minor side appears to be smaller than 
along the major side as shown by the solid curves B in Fig. \ref{slice}(B).  
This supporting argument is hardly affected by inclusion of the
F(0,k) for the two sets of phases $(-+-)$ and $(+-+)$ for the ripple 
diffractions shown in Table \ref{formfactor}.  Inclusion makes too 
little difference to show in traces A in Fig. \ref{slice}(B) 
for the major side.  For the minor side the 
$(-+-)$ phase set increases the height of the head groups, but not
enough to equal the height for the headgroups on the
major side, and not relative to the lower electron density of the 
chain region.

Wack and Webb \cite{Wac89a} argued that the (0,k) phases are $(+++)$ for 
k=1,2 and 3,
although the origin of their unit cell is translated by $\vec{b}/2$
compared to our unit cell in Fig. \ref{rppl_profile}, so their
phases are equivalent to our phase set $(-+-)$ which we obtain
from the SDF model as shown in Table \ref{formfactor}.  However, the 
small magnitudes
of the model F(0,k) motivated us to try the MDF and M1G models that
deliberately modulate the total electron density in the direction 
perpendicular to $\vec{a}$ in Fig. \ref{rppl_profile}.  (This direction 
is also very nearly
along the major side since it makes an angle of about 10 degrees
with the b-axis which is nearly the same as $\gamma - 90^{\circ}$.)
The results for the M1G model given in Table \ref{formfactor}, as well as for
the MDF model (also given in Table \ref{formfactor}), show a dramatic
improvement in the fit to most of the F(h,k), not just the F(0,k), with
a factor of four decrease in the sum of the squares of the residuals.
It may also be noted that the (0,3) diffraction 
has nearly the same q-value as the (1,-2) diffraction.  Reassignment of the 
larger part of the value 7.81 for $|F(0,3)|$ in Table \ref{formfactor} 
to $|F(1,-2)|$ would further improve the fits.
However, the set of ripple phases reverses to $(+-+)$ for both the M1G and
the MDF model and this result is
unchanged by using different starting values in the nonlinear
least squares fitting program.  We therefore favor the phases $(+-+)$
for the ripple diffractions, although we recognize that none of
the models we have considered readily accommodates $|F(0,2)|$ larger
than $|F(0,1)|$.

\begin{figure}
\begin{center}
\leavevmode
\raggedleft
\hspace{0.1in}
\psfig{figure=rppl_dir/wack_map.ps,height=2.5in}
\leavevmode
\raggedright
\hspace{0.6in}
\psfig{figure=rppl_dir/map_non0k_4k.ps,height=2.5in}
\end{center}
\vspace{-0.2in}
\hspace{1.2in} (A) \hspace{2.9in} (B)
\caption{Electron density maps of the ripple phase. 
(A) Reconstructed from the form factor amplitude
data in Wack and Webb [17] with our model fit phases (less
the (0,k) ripple peaks); same as in Fig. 6.5. 
(B) Reconstructed from the form factor amplitude
data in Wack and Webb [17] with our model fit phases (less
the (0,k) ripple peaks), and (4,0), (4,-1) and (4,-2) peaks from
model S1G fit. 
\label{4kmap}}
\end{figure}

Although the phases for the pure ripple (0,k) diffractions
are not as well determined by our procedure as the h$\neq$0 phases,
the latter phases, which are very robustly determined using
our model method, already determine the ripple profile.  
Indeed, because of the small form factor magnitudes of the (0,k) pure
ripple peaks, the effect of different (0,k) phases on the electron density
maps is less important than the effect of including the three h=4 
diffractions, which have much larger amplitudes, as shown in the
densitometer trace in Wack and Webb \cite{Wac89a} and in Table \ref{formfactor} 
for the M1G and S1G model. Fig. \ref{4kmap}(B) shows the electron density
map reconstructed by combining Wack and Webb's form factor amplitude
data (less (0,k) peaks) and the form factors of (4,0), (4, -1) and
(4, -2) peaks from S1G model. Comparing Fig. \ref{4kmap}(B) with
Fig. \ref{4kmap}(A), which is a duplicate of Fig. \ref{wack_map} and
has no (4,k) components, one can see that the contribution of (4,k) peaks 
is mainly to make the methyl trough region more distinguishable from
the methylene region.

The other important result of this work is the structural quantities
of the ripple phase extracted from the model fits (See Table \ref{parameter}). 
As far as we know, this work provides the most complete and accurate 
determination of the ripple structure to date. The
ripple amplitude from this work, 19\AA (from M1G model, other models give
similar values), compares well with the estimated values of 15\AA\ from 
Tardieu et al. \cite{Tar73} for DLPC and 15\AA\ from 
Janiak et al. \cite{JanSS79} for DMPC. Our work also indicates that the 
major side of the ripple is about 2.4 times that of the minor side; this
quantitatively shows the degree of ripple asymmetry. Furthermore, the closeness
of the three angles (see Figs. \ref{trans}, \ref{wack_map}, \ref{wack_data} 
and Table \ref{parameter}), $\gamma$-90$^{\circ}$ (8.4$^{\circ}$), the major 
side slope $\alpha$ (10.2$^{\circ}$ from M1G ) and the 
monolayer shift angle $\psi$ (9$^{\circ}$ from M1G), implies that the unit 
cell obliquity originates from the asymmetric ripple profile and the way 
two opposing monolayers are shifted relative to each other. This combination
of the three angles enables the inter-bilayer packing to be balanced and 
physically plausible; this point can be seen clearly from the reconstructed 
electron density map (Fig.\ref{wack_map}). In contrast, one highly unlikely 
possibility of intra-bilayer monolayer packing is shown
is Fig. \ref{trans}(B). As mentioned in Section \ref{rppl_intro}, 
Tardieu \cite{Tar72} and Katsaras et al. \cite{Kat95} made
a qualitative connection between the ripple asymmetry and the obliquity 
of the ripple unit cell. Our work puts that connection on a more solid 
quantitative ground.

The success of the model approach is due to capturing
the most essential features of the rippled bilayers. 
This agrees with one conclusion
of the gel-phase modeling work of Wiener, Suter and Nagle \cite{WSN89} 
that the results of the electron density modeling are basically independent 
of the particular analytical forms of the electron density profile, as long
as the models have certain key structural features of the bilayer. 
For the gel phase, the key structural quantities include 
the head-head spacing, two identical positive headgroup peaks, a negative
methyl trough, and a constant methylene region with an electron 
density close to that of water. For the ripple phase, we must add parameters
to describe the ripple profile.

It is highly desirable to systematically study the dependence of
the ripple structure on various physical and chemical variables,
e.g. the temperature, the hydration level and the chain length.
But up to now, only the dependence of the ripple unit cell
lattice parameters, i.e. the ripple wavelength, the bilayer stacking
repeat and the unit cell obliquity angle, on some of these variables
were studied and reported \cite{Wac89a,Matu90,Cev91}. With the help
of our models, the detailed ripple features can now be obtained and used
to help elucidate the mechanism of the ripple phase formation. 

In summary, we have proposed reasonably simple, but realistic, electron density 
models for the ripple phase of lecithin bilayers. Excellent agreement 
between the experimental data and our model fits has been
achieved. Many key structural
quantities of the ripple phase and the phases of the form factors
have been accurately extracted. For the first time, an accurate electron 
density map of the ripple phase, which agrees with available experimental 
results and known physical properties of the lipid bilayers, has been obtained. 
This model is very robust when subjected to various variations. 
The rippled bilayers have an asymmetric profile and gel-phase like structure, 
with the two opposing monolayers in the same bilayer shifted relative to 
each other in a way which gives rise to the obliquity of the unit cell. 
Further progress toward understanding the molecular packing of lipids in the 
ripple phase should employ careful analysis of wide angle scattering data.
This study establishes the framework into which such additional information
must be incorporated.

%----------------------------------------------------------------------------
%Close scrutiny showed
%that they were rather unsuccessful. The electron density maps reconstructed
%from their published model fitting form factors suggested no bilayer
%feature which should be a part of the lipid system. Given the lack of
%details of their models in the paper \cite{JanSS79}, one can nonetheless
%guess the reasons for their unsuccessful attempts: (1) They used a symmetric
%ripple profile, without realizing that the obliquity of the ripple unit cells
%suggested asymmetric ripples; (2) Their data were low resolution x-ray
%scattering intensity measurements from unoriented samples, which made the
%indexing and decomposition of overlapping peaks very difficult. Considering
%the fact that many later high quality oriented sample data \cite{Ale85,Kat95}
%and high resolution x-ray data \cite{Wac89a} were not available back then, their
%efforts are rather respectable.
%
%Knowing accurately the detailed ripple structure within the unit cell is 
%essential to the determinations of other key structural parameters of the 
%ripple phase, like the area per lipid molecule, inter-bilayer water spacing. 
%Incorrect ripple shapes and profiles or the lack of them will lead to
%wrong or incomplete answers. Parsegian \cite{Pars83} first proposed using 
%both a sinusoidal and a symmetric triangular 
%profiles to evaluate average local area per lipid. But he used the ripple 
%amplitude (50\AA\ peak-to-trough) given by Stamatoff et al. \cite{Sta82}, 
%which was obtained from an incorrect analysis, as pointed out by Wack and 
%Webb \cite{Wac89a}. Parsegian's calculation gave an averaged local area per 
%lipid molecule of about 68\AA$^2$ for ripple phase DPPC, directly comparable 
%to the area per lipid in the fluid phase, 71\AA$^2$ \cite{Pars83}, but much 
%larger than the gel phase area per lipid, 47.9\AA$^2$ \cite{Sun94}. 
%Detailed analysis 
%showed that Parsegian's local area/lipid will require a chain tilt of about 
%50$^{\circ}$ relative to the local bilayer normal. This tilt angle is bigger 
%than the slope of the ripple (about 35$^{\circ}$ using a ripple amplitude 
%of 50\AA\ and a symmetric ripple profile), thus contradicts 
%the model of Stamatoff et al. \cite{Sta82} in which the chains are proposed 
%to be perpendicular to the average bilayer plane. Using gravimetric method 
%\cite{Luz68} , Janiak et al. \cite{JanSS79} obtained the apparent
%area per DPPC molecule in the ripple phase of about 53.5\AA$^2$, and Wack
%et al.\cite{Wac89a} gave a value of about 52.7\AA$^2$, but no average local
%area per lipid was given in either of these two works \cite{JanSS79,Wac89a},
%due to the lack of information on the ripple profile. 

%This conclusion agrees well with the volumetric \cite{NW78} and the
%calorimetric \cite{Cev87} results that the volume change and enthalpy
%change at the pretransition (gel to ripple transition) have little
%chain length dependence, which imply that the chains are not very much
%involved in the pretransition, in contrast to the chain-melting or
%main transition, at which both the volume and enthalpy changes showed
%strong chain length dependence \cite{NW78,Cev87}.

%Flat lipid bilayers, such as in the gel and fluid phase, are uniform
%when viewed along the membrane plane. This uniformity is broken
%in the ripple phase.
%This can be seen in more details from the slice electron density profiles taken
%from the major and the minor sides of the rippled bilayer 
%(See Fig. \ref{slice}). Fig. \ref{slice} shows that the major side has 
%a bigger head-head spacing and a wider water spacing than that of the 
%minor sides.

%It is surprising that simple models like the ones
%presented in this chapter can be very powerful and robust. This can
%be seen from the excellent quantitative agreement 
%between the experimental form factor amplitudes 
%\cite{Wac89a} and the model fit 
%form factors (See Table \ref{formfactor}). The electron density map
%(Fig. \ref{wack_map}), reconstructed by combining the phases obtained
%from our model fits and the experimental form factor amplitudes \cite{Wac89a}, 
%has all the features agreeing well with known physical properties of 
%lipid bilayers, such as the bilayer head-head spacing, the uniform water
%spacing and the relative electron density of the constituent regions
%of bilayers. This map (Fig. \ref{wack_map}) further supports the correctness 
%of our models. 
