\chapter{Summary}
\label{summary_chap}

Section \ref{recap} is a recapitulation of the main results
reported in previous chapters. Section \ref{theme} presents
the main theme of this thesis and my arguments of the biophysical
relevance and the connection of the previous four chapters.

\section{Recapitulation of the main results}
\label{recap}

This thesis can be basically divided into three parts. The first part,
Chapter \ref{gel_model_chap}, dealt with the order and the disorder in
the regular gel phase lipid bilayers. Our analysis of the wide-angle 
x-ray scattering data from unoriented samples showed that, with the 
background scatterings due to air and water subtracted, the overall 
scattering from lipid bilayers consists of the Bragg scattering (order) and a 
large, broad diffuse scattering (disorder). We successfully employed a 
refined electron density model of the in-plane molecular packing to fit 
the Bragg scattering, including the conventional (20), (11) peaks, and the 
satellites. From our model fit, we were able to extract many important 
structural quantities of the gel phase bilayers from the unoriented sample 
data; these structural quantities had been thought to be only obtainable 
from oriented sample data before our work. The diffuse scattering was 
found to be larger than the Bragg scattering, implying that there
is more disorder than order even in the gel phase which had been thought
to be well ordered. Our further analysis showed that the headgroup region
is basically disordered, and that chain region has to be relatively 
disordered in order to account for all the observed diffuse scattering.

The second part of this thesis consists of two chapters, Chapters 
\ref{gel_anol_chap} and \ref{gel_long_chap}. This part showed our
studies on the chain ordered phases of long chain lecithin bilayers.
Systematically varying the chain length changes the balance between the 
headgroup interaction energy and the chain interaction energy; this variation
may induce systematic changes in the structures of the regular gel phase,
and may even lead to new phases. Chapter \ref{gel_anol_chap} showed that
for lecithins with chain length longer than (inclusive) 20, two new
chain ordered phases coexist with the regular gel phase at both lower
and higher temperatures. Although Gibbs' phase rule and our studies
showed that these new phases are not thermodynamically stable phases,
they do extend over broad temperature ranges and are quite stable
within the time scale in which typical x-ray measurements are
performed. Identifying and understanding these phases are necessary for
and should be obtained prior to systematic studies on long chain lecithins.
Chapter \ref{gel_long_chap} showed such studies. In Chapter 
\ref{gel_long_chap}, the chain length and temperature dependence of
the regular gel phase was reported. Low angle x-ray data showed that
the interfacial region (headgroups+water) between two adjacent bilayers 
changes little as the chain length and the temperature are varied; the
changes along the bilayer normal are mainly due to the hydrocarbon
chain region. Wide angle data indicated that longer chain lecithins
have more compacted and distorted chain packing than shorter chain lengths.
As a result of the opposing effects of temperature on area per chain A$_c$
and tilt angle $\theta_t$, the area per lipid A changes little as
a function of chain length and temperature. Our results can be accounted
for by a qualitative theoretical model based on headgroup and chain
interactions.

The third part, Chapter \ref{rppl_chap}, shows what we consider
the best structural determination of the ripple phase bilayers so far. 
By using a modelling approach, we were able to solve the phase problem
which had withheld the elucidation of the ripple structure. We have
tried a hierarchy of models based on a simple but essence-catching model
and found that these models were all very robust in phase determination.
By combining our phases with the best x-ray data of the ripple phase 
\cite{Wac89a}, we reconstructed the electron density map of of the 
ripple phase. The ripples were found to consist of a gel-phase-like
major side and a thinner minor side which is consistent with being
in the fluid phase. This work lays the foundation of future theoretical
work and systematic structural studies of the ripple phase or other
2D modulated phases.

\section{Connection and relevance of the result chapters}
\label{theme}

The three parts of this thesis as mentioned above may seem uncorrelated 
at the first glance, but there is indeed an underlying unified theme 
connecting them, which is the investigation of the structures of the chain 
ordered phases of lipid bilayer systems consisting of a single species of 
saturated lecithins and water. Although real biomembranes are known to 
have disordered, fluid-like chains and to be composed of more than 
one lipid species, studies of the chain ordered phases of pure lipid bilayers 
are not idle exercises and biologically irrelevant. The reasons are as follows.

First, in order to understand the thermodynamic properties and the structures
of biomembranes, many physical techniques have to be employed. Except some
real space microscopic techniques such as Transmission Electron Microscopy
and Scanning Tunneling Microscopy, other physical techniques, such as x-ray 
diffraction, scanning calorimetry and infrared spectroscopy, require large 
quantities of pure samples and high enough order in the system to obtain 
detectable signals and reproducible results. It would be difficult to use 
the latter techniques directly on biomembranes because of their intrinsic 
inhomogeneity, disorder and complex phase behaviors. The knowledge learnt 
from pure lipid bilayer systems serves as the foundation of understanding 
more complex lipid mixtures and finally real biomembranes.

Second, fluid phase is the most biologically relevant phase among all the 
lamellar phases of pure lipid bilayers. Elucidation of the fluid phase
structure holds the most promise compared to the studies of non-fluid
phases such as the gel and the ripple phase. But just as in real biomembranes,
the biologically relevant disorder in the fluid phase bilayers renders their 
structural determination rather difficult since disorder leads to broader
and weaker x-ray scattering peaks than that of the chain ordered phases. 
Wide angle x-ray data of the fluid phase provides little information
about the in-plane packing of the headgroups. After careful experiments and
theoretical analysis carried out recently by this lab \cite{N96}, low angle 
data of the fluid phase does provide enough information to the determination 
of the electron density profile along the bilayer normal, but not the in-plane
structure. In contrast, as we have shown in Chapter \ref{gel_anol_chap}, 
chain ordered phases have reliable wide angle data and more low angle data,
from which both in-plane and transbilayer structures can be better determined.
By bootstrapping from the gel phase structure obtained in this thesis, the
in-plane structure of the fluid phase has been determined \cite{N96}.

Third, the better x-ray determined structures of the chain ordered phases 
are important to Molecular Dynamics (MD) simulations of lipid 
bilayers \cite{Tu96}. MD simulation has become more and more important
a technique in investigating the structure and dynamics of biomembranes. 
The x-ray structures of the chain ordered phases can be used both to help 
set up the initial configurations of the MD simulations and to serve as a 
criterion of the correctness of the simulated structures and the potentials 
employed in the simulations. This is not just a precautious measure since
things do go wrong sometimes with the simulations. One example is reported
in Appendix A of this thesis. In Appendix A, I show that the pleated
chain structure obtained from Tu et al.'s \cite{Tu96} MD simulation
of gel phase bilayer can not give rise to the experimentally observed
x-ray diffraction pattern. They checked their simulation and
found that the cause of their problem was the incorrect initial configuration 
in which the chains of the two monolayers were pleated to start with.
Concuring with us this finding, they have started a new MD run in
order to further check their MD potentials.

Fourth, our studies on the chain ordered phases of long chain lecithins,
as reported in Chapters \ref{gel_anol_chap} and \ref{gel_long_chap}, are
related to another biologically relevant field of lipid research, namely
the detection and measurement of phase separation and microdomain formation
in lipid mixtures \cite{Sny95}. Snyder et al. \cite{Sny96} employed IR
spectroscopy on the long chain lecithins we have studied and were able to
identify the same anomalous chain ordered phases we observed. Based on 
the new understanding achieved from the combination of our x-ray result 
and their IR result, further investigation of the binary mixtures of long 
chain and short chain lecithins will be performed.

Finally but not lastly, lipid bilayers are not only model biomembranes,
but also lyotropic liquid crystals and closely related to surfactants. 
One aspect of biological physics is how we can learn new physics from
biological systems; lipid bilayers do provide new physics and sometimes
serve as test grounds for types of physics which had previously no real world 
realization, such as low dimension physics and the physics of weakly 
interacting systems. The ripple phase, in this aspect, is a good example.  
As an example of 2D modulated phases, It has attracted the attention of many 
physicists, both theorists and experimentalists 
\cite{Don79,Mar84,Haw86,Car87,Gol88,McCu90,Hon91,Lub93,Sir88,Wac89a}.
The accurate determination of the ripple phase structure, as reported
in Chapter \ref{rppl_chap} of this thesis, will greatly enhance the
understanding of this enigmatic phase.
