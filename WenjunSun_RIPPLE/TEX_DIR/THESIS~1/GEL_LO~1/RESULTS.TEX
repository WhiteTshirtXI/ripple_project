\section{Results and Discussion}
 
\subsection{Low-angle Scattering}

Figure \ref{gel_d95} shows the interlamellar D spacings.  
The D spacings exhibit a smooth
dependence upon both temperature T and chain length N.  The
values of D at 25$^{\circ}$C and the slopes $dD/dT$ are reported in Table 
\ref{gel_table_master}.

\begin{figure}[h]
\centerline {\psfig{figure=gel_long_dir/lgel_d95.ps,width=4in}}
\caption{Lamellar D-spacing as a function of T and N.
Solid lines are linear least-square fits. 
\label{gel_d95}}
\end{figure}

\begin{table}
\caption{Thermodynamic and structural quantities of the gel phase.
\label{gel_table_master}}
\begin{tabular}{ccccccccccc}
\\ \hline \hline
N & D & $\frac{dD}{dT}$ & D$_{HH}$ & A$_c$ & $\frac{dA_c}{dT}$ &
$\frac{dV_c}{dT} \times 10^4$ & $\theta$ & $\frac{d\theta}{dT}$ & 
A & $\alpha_A \times 10^4$ \\ 
% & & & & & & ($\times 10^{-4}$) &  & & & ($\times 10^{-4}$) \\ 
 & \AA & \AA /$^\circ$C & \AA & \AA$^2$ & \AA$^2$/$^\circ$C & 
ml/(gm $^\circ$C) & $^\circ$ & 
$^\circ$/$^\circ$C & \AA$^2$ & $^\circ$C$^{-1}$ \\ \hline
16 & 63.8 & 0.037(6) & 42.8 & 20.2(2) & 0.026(3) 
& 9(1) & 31.6$^a$ & -0.10(2) & 47.3 & 2(3) \\
& & & & & & & 31.6$^b$ & & 47.4 & \\ \hline
18 & 67.5 & 0.040(4) & 47.0 & 19.8(1) & 0.029(4) 
& 10(1) & 32.5$^a$ & -0.09(1) & 47.3 & 4(3) \\
& & & & & & & 32.1$^b$ & & 46.8 & \\ \hline
20 & 71.0 & 0.039(2) & 50.6 & 19.6(1) & 0.025(3) 
& 9(1) & 34.2$^a$ & -0.08(1) & 47.6 & 3(2) \\
& & & & & & & 33.6$^b$ & & 47.2 & \\ \hline
22 & 74.2 & 0.038(2) & 53.9 & 19.5(2) & 0.027(4) 
& 10(1) & 35.7$^a$ & -0.07(1) & 47.6 & 5(2) \\
& & & & & & & 35.2$^b$ & & 47.8 & \\ \hline
24 & 78.1 & 0.044(4) & 57.9 & 19.3(2) & 0.029(3) 
& 11(1) & 36.5$^a$ & -0.07(1) & 47.6 & 6(2) \\ 
& & & & & & & 35.4$^b$ & & 47.4 & \\
\hline \hline
\end{tabular}
{\small
\indent 1. All values are for T=25$^{\circ}$C.\\
\indent 2. $^{a}$Measured on oriented films.\\
\indent 3. $^{b}$Using Eq. \ref{tilt} to obtain $\theta$.
}
\end{table}

Figure \ref{gel_edp} shows the electron density profiles ${\rho}(z)$ along the
bilayer normal for all chain lengths at 25$^{\circ}$C except C24 for which the
low angle data for the usual gel phase are too obscured by 
scattering from the new
phases \cite{Sun95}; the measured and Lorentz-corrected form factors with
conventional phasing used for Fourier reconstruction are shown in 
Table \ref{long:form}.  The uncorrected head-head spacing
$D_{HH}^{4th}$ is calculated as the distance between headgroup peaks in the
same bilayer (from zero to around 
50\AA) in Fig. \ref{gel_edp}.
Results for $D_{HH}$ reported in Table \ref{gel_table_master} include 
systematic corrections
to account for the effects of Fourier truncation as described in
Materials and Methods.
In this chapter we will define the water spacing $D_W$ to be the distance
between 
headgroup peaks on adjacent bilayers (from zero to about -20\AA\ 
in Fig. \ref{gel_edp}), so $D_{W} = D - D_{HH}$.  This definition of $D_W$ overestimates 
the space that is exclusively occupied by water because 
the lipid heads obviously
extend some 4-5\AA\ beyond the headgroup peaks and one would wish to
subtract 8-10\AA\ from our $D_{W}$ in order to discuss interbilayer
forces \cite{Mc&S86B,McI89}.
McIntosh and Simon \cite{Mc&S86B} reported
a fluid layer thickness of 11.7\AA\ for fully hydrated C16 at 20$^{\circ}$C;
using their definition for the fluid layer thickness, which is D$_W$-10\AA, 
we obtain a fluid layer thickness of 11.0\AA\ for fully hydrated C16 at 
25$^{\circ}$C.

\begin{table}
\caption{Form factors up to 4th order for N=16, 18, 20, 22, 24 as 
functions of the temperature. |F(1)| is normalized to 1.
\label{long:form}}
\begin{center}
\raggedleft
\hspace{0.1in}
\begin{tabular}{ccccc}
\\ \hline \hline
 & & C$_{16}$ & & \\ \hline
T ($^{\circ}$C) & F(1) & F(2) & F(3) & F(4) \\ \hline
25 & -1.0 & -0.88 & 0.49 & -0.38 \\
30 & -1.0 & -0.88 & 0.48 & -0.45 \\
\hline \hline
\end{tabular}
\raggedright
\hspace{0.2in}
\begin{tabular}{ccccc}
\\ \hline \hline
 & & C$_{18}$ & & \\ \hline
T ($^{\circ}$C) & F(1) & F(2) & F(3) & F(4) \\ \hline
25 & -1.0 & -0.78 & 0.53 & -0.66 \\
30 & -1.0 & -0.74 & 0.53 & -0.63 \\
35 & -1.0 & -0.75 & 0.49 & -0.56 \\
40 & -1.0 & -0.64 & 0.43 & -0.59 \\
45 & -1.0 & -0.67 & 0.49 & -0.57 \\
\hline \hline
\end{tabular}
\end{center}
\begin{center}
\raggedleft
\hspace{0.1in}
\begin{tabular}{ccccc}
\\ \hline \hline
 & & C$_{20}$ & & \\ \hline
T ($^{\circ}$C) & F(1) & F(2) & F(3) & F(4) \\ \hline
25 & -1.0 & -0.71 & 0.56 & -1.0 \\
30 & -1.0 & -0.73 & 0.65 & -0.92 \\
35 & -1.0 & -0.61 & 0.59 & -0.81 \\
40 & -1.0 & -0.57 & 0.45 & -0.69 \\
45 & -1.0 & -0.57 & 0.48 & -0.71 \\
50 & -1.0 & -0.52 & 0.44 & -0.69 \\
55 & -1.0 & -0.61 & 0.51 & -0.72 \\
60 & -1.0 & -0.70 & 0.49 & -0.76 \\
\hline \hline
\end{tabular}
\raggedright
\hspace{0.2in}
\begin{tabular}{ccccc}
\\ \hline \hline
 & & C$_{22}$ & & \\ \hline
T ($^{\circ}$C) & F(1) & F(2) & F(3) & F(4) \\ \hline
25 & -1.0 & -0.47 & 0.45 & -0.93 \\
30 & -1.0 & -0.51 & 0.43 & -0.92 \\
35 & -1.0 & -0.48 & 0.42 & -0.88 \\
40 & -1.0 & -0.54 & 0.49 & -0.89 \\
45 & -1.0 & -0.51 & 0.44 & -0.86 \\
50 & -1.0 & -0.50 & 0.39 & -0.79 \\
55 & -1.0 & -0.53 & 0.42 & -0.76 \\
60 & -1.0 & -0.53 & 0.39 & -0.80 \\
65 & -1.0 & -0.53 & 0.44 & -0.80 \\
70 & -1.0 & -0.50 & 0.40 & -0.73 \\
\hline \hline
\end{tabular}
\end{center}
\begin{center}
\hspace{0.1in}
\begin{tabular}{ccccc}
\\ \hline \hline
 & & C$_{24}$ & & \\ \hline
T ($^{\circ}$C) & F(1) & F(2) & F(3) & F(4) \\ \hline
40 & -1.0 & -0.42 & 0.29 & -1.2 \\
45 & -1.0 & -0.37 & 0.36 & -0.93 \\
50 & -1.0 & -0.51 & 0.43 & -0.94 \\
55 & -1.0 & -0.40 & 0.37 & -0.90 \\
60 & -1.0 & -0.46 & 0.42 & -0.96 \\
65 & -1.0 & -0.45 & 0.38 & -0.94 \\
\hline \hline
\end{tabular}
\end{center}
\end{table}

\begin{figure}
\centerline {\psfig{figure=gel_long_dir/lgel_edp.ps,width=4in}}
\caption{Electron density profiles at 25$^{\circ}$C Fourier constructed using
h=1--4 peaks. 
Solid line, C16; 
short-dashed line, C18; long-dashed line, C20; dot-dashed line, C22.
The heights of the headgroup peaks have been normalized to the same
value. One headgroup peak has been placed at the origin for each chain
length. 
\label{gel_edp}}
\end{figure}

Figure \ref{gel_spc} shows $D_{HH}/2$ and $D_W$ as a function of temperature
for four chain lengths.  As would be expected, the thickness of the
bilayers $D_{HH}$ systematically increases as the chain 
length increases. In contrast, the water spacing is 
nearly the same for all chain lengths.  From
the results in Fig. \ref{gel_spc} the slope $dD_{HH}/dT$ is at least 5 times
greater than the slope $dD_{W}/dT$, so most of the thermal increase in D
is due to thickening of the bilayer rather than to thickening of
the water layer. It may be noted that the effect of our 
Fourier truncation corrections only increases $dD_{HH}/dT$ by a factor
of 1.1.  Simon et al. \cite{Simon95} obtained values of $D_{HH}$ for
C22 at 23$^{\circ}$C and 50$^{\circ}$C that agree within
0.1\AA\ with ours, although the temperature 
dependence of their
water spacing ($dD_{W}/dT$) was about twice as large as ours before we applied
our correction.
Our result that the water thickness is nearly independent of 
temperature 
and chain length has previously been obtained by Kirchner and Cevc
\cite{Kir94}, although their values for dD/dT and other quantities to be
discussed differ somewhat from those in Table \ref{gel_table_master}.

\begin{figure}[h]
\centerline {\psfig{figure=gel_long_dir/lgel_spc.ps,width=4in}}
\caption{Corrected D$_{HH}$/2 (solid symbols) and D$_W$ (open symbols) as 
a function of temperature. Circles, C16;
squares, C18; triangles, C20; diamonds, C22. 
Lines are linear least-square fits.
\label{gel_spc}}
\end{figure}

\subsection{Wide-angle Scattering}

For the ordinary $L_{{\beta}^{'}}$ gel phase, wide angle scattering shows 
the usual sharp (20) peak and a broad (11) peak for all chain lengths N.  
As has been known for a long time, these peaks
indicate orthorhombic packing of all-trans hydrocarbon chains tilted
towards nearest neighbors \cite{Sun94,LevTh,McI80}.
Figure \ref{gel_d20} shows
the temperature and chain length dependence of the d spacings from
the two wide angle peaks.  As N increases, the curves are
displaced to higher temperatures, but retain their shape as a function
of T.  

\begin{figure}
\centerline {\psfig{figure=gel_long_dir/lgel_d20.ps,width=4in}}
\caption{Temperature and chain length dependence 
of the wide angle d$_{11}$ (open symbols) and d$_{20}$ (closed symbols)
in {\AA}. Circles, C16; squares, C18; triangles, C20; diamonds,
C22; inverted triangles, C24. Lines are to guide the eye.
\label{gel_d20}}
\end{figure}

The area per chain in the plane perpendicular to the chains, Ac, 
is calculated from d$_{20}$ and d$_{11}$ assuming orthorhombic packing 
with two chains per unit cell using the 
formula, 2$A_{c} = a_c b_c$, and Eq. \ref{gel_lattice} for $a_c$ and $b_c$.
As expected, the chains are packed more tightly at lower temperatures,
so $A_c$ expands with temperature as shown in Fig. \ref{gel_lac}.
Assuming that the chains are in nearly all-trans conformations,
the volume $V_c$ of the hydrocarbon chain region is given by
$V_c = 2N(1.27\AA)A_c$, where 1.27\AA\ is the distance between
adjacent methylenes projected onto the chain axis.
The wide angle data therefore yield the hydrocarbon
volume expansion coefficients $dV_c/dT$ shown in Table \ref{gel_table_master}.  
For the volume $V_L$ of the entire lipid, dilatometry \cite{NW78}
has given
$dV_L/dT = 8.3\times 10^{-4}ml/gm{^{\circ}C}$
for C16 and C18.  Within error, $dV_L/dT$ and $dV_c/dT$ are
the same, indicating that thermal changes in lipid volume are accounted
for by the hydrocarbon chain region, as originally suggested by Nagle 
and  Wilkinson \cite{NW78} from much less extensive x-ray data. It may
also be noted that the sign of the small difference for
C18 would require a small volume shrinkage of the headgroup and glycerol
parts of the lipid bilayer with increasing temperature.

\begin{figure}
\centerline {\psfig{figure=gel_long_dir/lgel_lac.ps,width=4in}}
\caption{Temperature and chain length dependence of
the area per chain, $A_c$.
Solid lines are linear least-square fits with the slopes fixed
to the averaged value (0.027\AA$^2$/$^{\circ}$C) of all chain
lengths. 
\label{gel_lac}}
\end{figure}

The widths of the sharp (20) wide angle peaks, shown in Fig. \ref{gel_hwh},
decrease with increasing temperature.  The leveling out of 
the apparent half-widths at the higher temperatures is an artifact of the 
instrumental resolution of about $0.065^{\circ}$(HWHM) as shown in 
Fig. \ref{gel_hwh}. A higher resolution 
study \cite{Sun94} obtains $0.015^{\circ}$(HWHM) for the intrinsic 
half-width of the (20) peak for C16 at 25 $^{\circ}$C.  
Since the intrinsic half-width of the
(20) peak is usually inversely proportional to the correlation length of the 
sample, it is surprising that the half-width decreases with increasing 
temperature since one would expect the sample to become more disordered, with 
smaller correlation lengths at higher temperatures.  It is also
remarkable that longer chain lipids with stronger cohesive
van der Waals forces between chains should have shorter correlation lengths.
Our temperature reversibility study on C24 \cite{Sun95} showed that the 
HWHM result in Fig. \ref{gel_hwh} was reproducible when going 
up and down in temperature,
so the temperature dependence in Fig. \ref{gel_hwh} is not due to 
gradual annealing as the samples are warmed.

\begin{figure}
\centerline {\psfig{figure=gel_long_dir/lgel_hwh.ps,width=4in}}
\caption{Temperature and chain length dependence of
the half-width at half maximum (HWHM) of the (20) peaks (in degrees).
The dashed line shows the instrumental resolution.
\label{gel_hwh}}
\end{figure}

Fig. \ref{gel_d20} also shows that as temperature increases, 
d$_{20}$-d$_{11}$ decreases,
which means that the chain packing moves towards hexagonal for which
$d_{20}=d_{11}$.  A principal reason that the chain packing is not hexagonal
is that hydrocarbon chains are not simple cylinders.
Sirota et. al. \cite{Sir93} have identified an 
appropriate measure of the orthorhombic distortion from hexagonal
packing, namely the distortion order parameter, defined as

\begin{eqnarray}
\label{distort}
\epsilon = 1 - \frac{a_c}{\sqrt{3} b_c}, 
\end{eqnarray}
where $a_c$ and $b_c$ are the orthorhombic lattice parameters defined in 
Eq. \ref{gel_lattice}. The magnitude
of $\epsilon$ measures the amount of distortion, and the sign shows
the direction of the distortion; a negative sign means stretching
along $a_c$ and a positive sign means stretching along $b_c$.
Our results for $\epsilon$ are plotted in Fig. \ref{gel_dis}; a similar 
temperature
dependence has been reported by Sirota et al. \cite{Sir93} for long 
chain alkanes.
For each chain length N, the 
magnitude of the distortion $|\epsilon|$ decreases with increasing temperature.
This can be understood because the non-cylindrical steric forces between
hydrocarbon chains are relatively stronger at closer packing (smaller $A_c$) 
at lower temperatures than at higher temperatures where the 
attractive van der Waals interaction, which is more cylindrically symmetric, 
plays a relatively larger role.
We find it especially intriguing that the lipids of all chain lengths undergo a
transition out of the gel phase when the magnitude of the distortion 
decreases to about 0.025, as seen in Fig. \ref{gel_dis}. We know of no 
fundamental
theory that has predicted this.  This result looks like a clue
for understanding gel phase stability, but one must remember that
the transition out of the gel phase is a first order transition,
so the thermal behavior of the higher temperature phase, which
is a ripple phase for lower values of N and is thought
to be an $L_{\alpha}$ phase for C24 \cite{Lew87}, should also 
play a role.

\begin{figure}
\centerline {\psfig{figure=gel_long_dir/lgel_dis.ps,width=4in}}
\caption{Temperature and chain length dependence 
of the distortion parameter $\epsilon$ (defined in the text).
\label{gel_dis}}
\end{figure}

\subsection{Tilt Angle $\theta$}
The tilt angle $\theta$ of the hydrocarbon chains in the $L_{\beta'}$
phase has been measured directly on fully hydrated oriented films 
in this and previous work \cite{STN92}.  For C16 it has 
also been possible to obtain
the tilt angle in unoriented MLV powder samples 
\cite{Sun94} like those employed here and the value 
$\theta = 31.6^{\circ}$ obtained at $25^{\circ}$C agrees
with our results from oriented samples.  
Measured values of $\theta$ from oriented films at or near 25$^{\circ}$C
are shown by the first entry under $\theta$ in Table \ref{gel_table_master} 
with superscript a.  For the temperature dependence of $\theta$, 
Fig. \ref{gel_tlt} shows data for C16 taken over a larger temperature range. 
Although it is the subgel phase rather than the gel phase that is 
thermodynamically stable
at the lower three temperatures in Fig. \ref{gel_tlt}, the subgel phase 
does not
form in C16 unless the temperature is decreased below 7$^{\circ}$C
for a long enough time to incubate it \cite{NW82}. The wide and
low angle patterns confirmed that our oriented samples were in the
gel phase at all temperatures in Fig. \ref{gel_tlt}.
Fig. \ref{gel_tlt} gives
a best fitted value $d{\theta}/dT = -0.16^{\circ}/^{\circ}C$. 
Furthermore, there is no doubt that $d{\theta}/dT$ is negative 
\cite{JanSS76,Kir94};
the least temperature dependent value that can be justified from
Fig. \ref{gel_tlt} is $d{\theta}/dT = -0.10^{\circ}/^{\circ}C$.
This latter value is close to the value of -0.11$^{\circ}/^{\circ}C$
for C16 that we estimate from the $\theta$ versus T data
that Janiak et al. \cite{JanSS76} obtained using the gravimetric method.

\begin{figure}
\centerline {\psfig{figure=gel_long_dir/lgel_tlt.ps,width=4in}}
\caption{Temperature dependence of the tilt angle $\theta$ of
C16 obtained from oriented samples. The solid line is
the best linear least-square fit to the data with 
$d{\theta}/dT = -0.16^{\circ}/^{\circ}C$; the dashed line has 
$d{\theta}/dT = -0.10^{\circ}/^{\circ}C$.
Data obtained by Dr. Tristram-Nagle.
\label{gel_tlt}}
\end{figure}

In this paragraph we shall analyze our low angle data to obtain an indirect
determination of $d{\theta}/dT$ for all chain lengths.  From 
Fig. \ref{gel_spc} we find that
$dD_{HH}/dT$ is practically the same as $dD/dT$ shown in 
Table \ref{gel_table_master}.
Assuming that the change in $D_{HH}$ is due to changes in chain tilting, then
\begin{eqnarray}
\label{tilt}
D_{HH} = 2N(1.27\AA) \cos{\theta} + 2D_{H},
\end{eqnarray}
where 2$D_{H}$ accounts for the headgroup thickness which is assumed not to 
change with temperature.  Then,
\begin{eqnarray}
\label{Dtilt}
dD_{HH}/dT = 2N(1.27\AA) \sin{\theta}(d{\theta}/dT) .
\end{eqnarray}
The values of $d{\theta}/dT$ calculated from Eq. \ref{Dtilt} are shown
in Table \ref{gel_table_master}.  

Although the value of $d{\theta}/dT$ obtained in the preceding paragraph
is consistent with the extreme upper end of the range for $d{\theta}/dT$ 
obtained from our direct measurements on C16,
we are concerned with the possible discrepancy
and so we have considered possible changes in the model that might
bring them into better agreement.  However, all the most plausible changes
we have thought of, such as the one in the last paragraph of this
subsection, make the 
disagreement worse.  The best model we can derive that changes our calculated 
value shown in Table \ref{gel_table_master} to
our best measured value of -0.16$^{\circ}/^{\circ}$C would require that 
the headgroup
thickness $D_{H}$ decrease with increasing temperature.

In this paragraph let us turn from temperature dependence to chain length
dependence.  Our best established tilt angle is $\theta = 31.6^{\circ}$ for
C16 at 25$^{\circ}$C.  Together with $D_{HH}$ from Fig. \ref{gel_spc}, 
Eq. \ref{tilt} yields $D_{H}$ = 4.12\AA.  Using this value of $D_{H}$ for 
other chain lengths
yields the second set of $\theta$ values in Table \ref{gel_table_master} 
with the superscript b.
Agreement with directly measured tilt angles indicated by superscript
a is satisfactory.  

Another model that has been considered by Kirchner and Cevc \cite{Kir94}
assumes that the heads are tilted with the same angle as the chains.  
In place of Eq. \ref{tilt} one then has

\begin{eqnarray}
\label{mtilt}
D_{HH} = (2N(1.27\AA) + 2D_{H})cos{\theta}.
\end{eqnarray}
Following the same procedure as in the preceding paragraph we obtain,
for N=18, 20, 22 and 24, $\theta$ = 31.8$^{\circ}$, 33.1$^{\circ}$,
34.6$^{\circ}$ and 34.9$^{\circ}$, respectively, using Eq. \ref{mtilt}.  
Since these are in poorer agreement with our direct 
measurements of $\theta$, we suggest that the effective tilt 
of the headgroups is not rigidly coupled to the tilt of the chains
and that Eq. \ref{tilt} is superior to Eq. \ref{mtilt}.  Using
Eq. \ref{mtilt}, Kirchner and Cevc \cite{Kir94} obtained a 
$d{\theta}/dT$ of about $-0.18^{\circ}/^{\circ}C$.  The sign of their
result concurs with our earlier \cite{STN92} and 
present study, but
the absolute value, which would be even larger if Eq. \ref{Dtilt} were
used, is about twice as large as our values presented in 
Table \ref{gel_table_master}.

\subsection{Interfacial Area A}

The area A that each lipid occupies on average at the interface with
the water space is given by
\begin{eqnarray}
\label{area}
A = 2A_c/cos\theta .
\end{eqnarray}
Table \ref{gel_table_master} shows two sets of values of A for 
T = 25$^{\circ}$C obtained
using our two determinations of $\theta$.  
As was emphasized in 
our earlier
work \cite{STN92}, A is nearly constant as a 
function of chain length.

The temperature dependence of A is obtained by differentiating Eq. \ref{area}.
Then, the area expansivity defined by ${\alpha}_A = (dA/dT)/A$ is given by
\begin{eqnarray}
\label{darea}
{\alpha}_A = [(dA_c/dT)/A_c] + [tan{\theta}(d{\theta}/dT)] .
\end{eqnarray}
Values of ${\alpha}_A$ calculated using our data and Eq. \ref{darea}
 are shown in Table \ref{gel_table_master}.  The most remarkable result
is how small ${\alpha}_A$ is for all chain lengths.  Even for a
50$^{\circ}$C increase in T, A would only increase by 1.4\AA$^2$ for C24.
A significant reason for small ${\alpha}_A$ is the competition
of thermal expansion of the chain packing and of the chain tilting.
As would be expected, the chain lattice expands thermally so that
the first term in Eq. \ref{darea}, $[(dA_c/dT)/A_c]$, is positive.
However, the second term in Eq. \ref{darea} is negative because
chain tilt decreases with increasing temperature.  Because the
magnitudes (about 0.001/$^{\circ}$C) of the two nearly equal terms is 
comparable,
the magnitude of ${\alpha}_A$ is smaller than one would obtain
from either term alone.  Also, this cancellation of terms, each of which
has errors, suggests that the fairly large relative increase of
${\alpha}_A$ with N seen in Table \ref{gel_table_master} may not be 
significant.  In this
regard, it is worth noting that, for C16, our direct measurements 
of $d{\theta}/dT=-0.16^{\circ}/^{\circ}C$ would even yield negative 
values for ${\alpha}_A$.

Values of ${\alpha}_A$ from $+5 \times 10^{-4}$ \cite{Evans82} to
$+3 \times 10^{-3}$ \cite{Need88} have been reported
from studies of giant unilamellar vesicles of C14.
It is quite impossible that our multilamellar vesicles could have
values of ${\alpha}_A$ greater than $1.5 \times 10^{-3}/^{\circ}$C, 
since this number
comes from the first term in Eq. \ref{darea} determined directly
from our wide angle results for $A_c$ 
and the effect of tilt angle in the last term in Eq. \ref{darea}
only decreases ${\alpha}_A$.  Although one might wish to consider 
that ${\alpha}_A$
might be different for gel phases in giant unilamellar vs. multilamellar
vesicles or for C14 versus longer chain lengths, these papers emphasize 
that obtaining
results for the gel phase of giant unilamellar vesicles is much 
more problematical than for the
higher temperature phases. 

Dynamical light scattering data on small unilamellar vesicles
led Kirchner and Cevc \cite{Kir94} to conclude that 
${\alpha}_A$ is negative and they then based a new theory of
the pretransition on this conclusion.  They argued that this conclusion
was plausible because the tilt angle decreases with increasing temperature.  
Their argument, however, neglected the increase in the chain packing area
$A_c$ with increasing temperature which adds a positive term to
${\alpha}_A$ as shown in Eq. \ref{darea} and
compensates for the negative contribution to ${\alpha}_A$ from the decreasing
chain tilt.  
