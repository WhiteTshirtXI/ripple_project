\section{Introduction}

In the last chapter, the anomalous chain ordered lamellar phases
in longer chain lecithins, especially for C24 and partially for C22,
have been carefully studied and documented. These new phases have
been studied and confirmed by infrared spectroscopy \cite{Sny96}.
With these new phases being identified, the chain length and temperature 
dependence of the regular gel phase can now be carried out.  The thorough 
understanding of these regular and anomalous phases will be valuable for 
what they can reveal about the competition between different forces 
that organize the gross and fine structures of bilayers.

This chapter reports our systematic study on the chain length and temperature
dependence of the regular gel phase. The chain lengths we studied are
N=16, 18, 20, 22 and 24. It will be shown that the thermal volume expansion 
of these lipids is accounted for by the expansion in the hydrocarbon chain 
region. Electron density profiles are constructed using four orders of low 
angle lamellar peaks. These show that most of the increase in D with 
increasing T is due to thickening of the bilayers that is consistent with 
a decrease in tilt angle $\theta$ and with little change in water spacing 
with either T or N.  Due to the opposing effects of temperature on area per 
chain A$_c$ and tilt angle $\theta$, the area expansivity $\alpha_A$ is 
quite small. A qualitative theoretical model based on competing head and 
chain interactions accounts for our results.

%The study of lipid bilayers has been and will continue to be greatly enriched
%by investigating how the structure and thermodynamic properties vary as the 
%lipids are varied, including both naturally occurring and specifically
%synthesized lipids \cite{JanSS76,Lew87,Huang94}.
%One particularly appropriate strategy
%is to vary the chain length, since this variation systematically alters the
%balance between the interaction energy involving the headgroups,
%which remains the same, and the total interaction between the chains, which 
%increases with chain length. As is well known, this variation yields increased 
%main transition temperatures with increased chain length, 
%but other structural changes, especially as a function of temperature,
%have not been so well documented.

%Increasing chain length may involve dramatic changes.
%For example, for the phosphatidylethanolamines, decreasing chain length takes
%one from phase diagrams having stable gel phases to phase diagrams in which
%the gel phase is merely metastable at all temperatures 
%\cite{Chang83,WN84}. 

%In a recent study of a sequence of di-saturated phosphatidylcholines from
%this laboratory, it was found that the wide angle pattern in the gel phase
%started to become qualitatively different as the chain length N was increased
%beyond 20 carbons \cite{STN92} and 

%We have also been
%able to obtain and study the ordinary $L_{{\beta}^{'}}$ gel phase for all 
%chain lengths over extended temperature ranges.  This kind of study, which 
%is the subject of this chapter, is valuable for
%revealing the competition between different forces that organize
%the fine structure of the gel phase.  
