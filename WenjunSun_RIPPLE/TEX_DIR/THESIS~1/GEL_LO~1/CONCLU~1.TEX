\section{Theoretical Interpretation}

In our earlier paper \cite{STN92} we had shown that the 
area A and
the water region were nearly independent of chain length.  We 
advanced a theory for this that will apply equally well to our
new results that the area A and the water region are also nearly
independent of temperature, while the thickness $D_{HH}$ of the bilayer and
the chain tilt angle $\theta$ have significant temperature variations.
That theory is based on the explanation \cite{Nag76,McI80}
that chains tilt due to 
competing interactions in the head and chain regions of the lipid
molecule.  Specifically, there is a strong steric repulsive interaction between
headgroups for areas A less than 48\AA$^2$ \cite{STN92}.
However, the natural area 2$A_c$
for the packing of all-trans hydrocarbon chains is only about 40\AA$^2$.  To
minimize the total energy due to both headgroup and chain interactions, the
chains tilt at an angle $\theta$ so that both the heads and the chains 
can fit into the geometric constraint of a flat bilayer.  However, there is
still competition between heads and chains because the chains, by themselves,
would prefer not to tilt as much as is required by the steric interaction
of the heads.  The chains then exert a pressure on the headgroups
that pushes them together so that they are in the strongly repulsive, and
therefore less elastic, portion of the head-head interaction energy function.  
Therefore, the area A changes little as the external forces on the heads
change, e.g. due to changes in chain length or temperature.  However, the 
chains are in the much more elastic, and probably attractive, portion of the
chain-chain interaction energy function.  As the temperature rises, the chain
packing area $A_c$ naturally increases significantly.  This allows the chain
tilt to decrease, while keeping the head group spacing nearly constant on the 
highly inelastic portion of the head-head interaction energy function. 
%A schematic of the repulsive and the attractive regions in the interaction
%energy is shown in Figure \ref{elastic}.

%\begin{figure}[h]
%\centerline {\psfig{figure=gel_long_dir/elastic.ps,width=5in}}
%\caption{Schematic of the interaction energy. The repulsive, less
%elastic region and the attractive, thus more elastic region are
%marked.
%\label{elastic}}
%\end{figure}

Our result that the water spacing D$_{W}$ in the gel phase does not depend 
significantly upon temperature is also understandable if
the interbilayer forces do not depend much upon temperature.  
Recently, Simon et al. \cite{Simon95} showed that the principal 
repulsive force 
in the gel phase of C22 is the hydration force and that it is nearly
independent of temperature.  The other principal interbilayer force is the
van der Waals force \cite{Isr85}, which is essentially given as
\begin{eqnarray}
\label{vdw}
F_W = - \frac{W}{6 \pi} [ \frac{1}{D_{W}^3}
- \frac{2}{(D_{W}+D_{HH})^3} + \frac{1}{(D_{W} + 2D_{HH})^3} ].
\end{eqnarray}
Although the magnitude of the van der Waals force should become larger
with increasing temperature due to changing D$_{HH}$, the increase 
due to that change is
less than 0.5\% over a temperature interval of 45$^{\circ}$C
using our measured changes in D$_{HH}$.

We therefore conclude that the major thermal changes in lecithin
bilayers within the usual $L_{\beta '}$ thermodynamic phase are 
changes in chain packing area $A_c$ and in tilt angle $\theta$.
Because the headgroups are always packed together very strongly, much smaller
changes occur in the headgroup area. Because the lipid/water
interface does not change, it is then not
surprising that the water spacing does not change much with
temperature or with chain length.
