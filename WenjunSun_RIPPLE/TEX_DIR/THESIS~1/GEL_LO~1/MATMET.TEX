\section{Methods}

Lamellar D spacings were obtained quantitatively from the fourth order
peaks.  This peak was well defined and was also less affected by slit
smear than the lower order peaks.  Taking into account slit smear,
the lower order peaks indexed well with the fourth order peak except
for C24 at low and high temperatures \cite{Sun95}.
Electron density profiles were obtained from the Lorentz corrected intensities
of the first four orders of low angle diffraction in the usual way 
\cite{WSN89}.

\begin{figure}[h]
\centerline {\psfig{figure=gel_long_dir/lgel_cor.ps,width=4in}}
\caption{Corrections $\Delta D_{HH} = D^{4th}_{HH} - D^{true}_{HH}$ 
versus $D^{4th}_{HH}$, obtained from the electron density
profile that was  Fourier constructed using h=1--4 lamellar peaks for a 
known gel-phase electron density profile. Both axes are scaled by 1/D
to account for all chain lengths and temperatures.
The straight line is a linear least-square fit to the points (open circles).
\label{gel_cor}}
\end{figure}

Uncorrected head-head distances $D_{HH}^{4th}$ were obtained from the peak-peak
distances in the 4th order electron density profiles.  We have found that
$D_{HH}^{4th}$ is remarkably accurate, compared to larger and smaller
numbers of Fourier components \cite{N96}. Nevertheless,
we are interested in small differences as a function of temperature,
so systematic corrections to $D_{HH}$ were estimated.  We began with
a known electron density profile of the one Gaussian hybrid type with
parameters determined for the gel phase of C16 \cite{WSN89}.
The fourth order Fourier reconstruction from this model yielded values of 
$D_{HH}^{4th}$ for various values of $D_{HH}^{true}$.
The relative error $(D_{HH}^{4th} - D_{HH}^{true})/D$ is plotted in Fig. 
\ref{gel_cor} as a
function of $D^{4th}_{HH}/D$, which can be obtained experimentally.  
Although the
errors are fairly small, one sees that they vary systematically with 
$D^{4th}_{HH}/D$.
Fig. \ref{gel_cor} was then used to obtain corrected values of $D_{HH}$ from
experimental determinations of $D_{HH}^{4th}$ and D.

The orthorhombic lattice parameters for the chain packing, $a_c$ 
and $b_c$, were 
obtained using the standard formula,
\begin{eqnarray}
\label{gel_lattice}
a_c = 2 d_{20}\ \mbox{and}\ b_c = {d_{11} \over 
\sqrt{1-({d_{11} \over 2 d_{20}})^2}}\ ,
\end{eqnarray}
where $d_{20}$ and $d_{11}$ are obtained using Bragg's law from the two wide
angle reflections.

%\section{Materials and methods}
%
%All lecithins were purchased in lyophilized form from Avanti Polar Lipids
%(1,2- dipalmitoyl-$sn$-glycero-3-phosphatidylcholine (to be abbreviated C16); 
%1,2-distearoyl-$sn$-glycero-3-phosphatidylcholine (C18); 
%1,2-diarachidoyl-$sn$-glycero-3-phosphatidylcholine (C20);
%1,2-dibehenoyl-$sn$-glycero-3-phosphatidylcholine (C22);
%1,2-dilignoceroyl-$sn$-glycero-3-phosphatidylcholine (C24))
%and used without further purification.
%Oriented samples were prepared and analyzed as described in \cite{STN92}.
%
%Most of the work in this chapter was on lipid/water dispersions which were 
%placed in
%1mm$\times$4cm capillaries following standard procedures \cite{STN92}.
%Upon brief centrifugation, these dispersions separate
%into a lipid rich phase and a clear
%water rich phase, thus demonstrating full hydration.  
%Following scattering measurements, the lipid was 
%assayed for radiation and thermal damage by thin layer chromatography. 
%The sample chamber holds a cassette with slots for 10 x--ray capillaries. 
%The chamber is connected
%to a motorized 3D translational unit, which allows easy access of the x--ray
%beam to the capillaries and new spots on each capillary; this facilitates 
%sample 
%loading and allows us to minimize radiation damage by frequently translating
%unexposed sample into the beam. The temperature was controlled as described
%in \cite{Sun95}.
%
%Our principle measurements were carried out using a rotating anode x-ray
%source interfaced with a four-circle diffractometer and a Braun linear 
%position sensitive detector (PSD) as previously described \cite{Sun94}.
%Accounting for the finite beam size, the finite sample size and the divergence
%of the incoming beam leads to an estimate for the instrumental resolution
%of 0.065$^{\circ}$ (HWHM) in $2 \theta$ or $\Delta Q = 5 \times 10^{-3}$\AA$
%^{-1}$.
%The maximum exposure time on one spot 
%was about 4 hours with x-ray 
%power of 5.25 kW using a graphite monochromator. No radiation damage,
%as detected by changes in peak shapes, was detected in this time;
%thin layer chromatography showed less than 1\% lysolecithin for all the 
%samples.
