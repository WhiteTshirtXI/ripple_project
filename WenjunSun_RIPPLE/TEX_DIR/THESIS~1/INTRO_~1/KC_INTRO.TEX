
%--------------------------------------------------------%
%                                                        %
%   Chapter 1   Introduction                             %
%   ========================                             %
%                                                        %
%--------------------------------------------------------%
%\documentstyle{pthesis}
%\documentstyle[12pt]{article}
%\input ptmacro
%\begin{document}

\section{NATURE OF BILAYER MEMBRANES}
Lipids, together with proteins and small amounts of carbohydrate, 
form an enveloping semi-permeable bilayer sheet  which is often
called the cell membrane.
They are highly selective permeability barriers which
partly separate a certain volume of a solution
with its solutes from the environment.
They regulate 
the composition of the intracellular medium by controlling the flow of
nutrients, waste products, ions, etc., into and out of the cell
through membrane-embedded ''pumps'' and ''gates''.
Membranes also control the flow of information
between cell and their environment. They contain specific receptors
for external stimuli, and in turn, some membranes also generate
signals. Thus membranes play a central role in biological
communication.
In addition, two most important energy conversion processes in
biological system involve membranes.
The first is the oxidative phosphorylation processes which generate
energy in the form of adenosine triphosphate (ATP) by the oxidation
of nutrients. This takes place in the inner membrane of
mitochondria.
The second is photosynthesis, in which light energy is converted
into chemical-bond energy by forming carbohydrates from $H_2O$ and
$CO_2$. This occurs in the inner membranes of chloroplasts (Voet \&
Voet, 1990)


Phospholipids are one of the  major lipid components in the
biological membrane. The chemical structure of a common phospholipid,
phosphatidylcholine (PC), which is also called ''lecithin'', is given in
Figure~\ref{fig:dppcmol}. 
\begin{figure}
\label{fig:dppcmol}
\vspace{6.0in}
\caption{The chemical structure of phosphatidylcholine}
\end{figure}
Phospholipids are amphipathic molecules, i.e., they
contain both water-soluble (hydrophilic) and water-insoluble
(hydrophilic) parts. It is well known that when such molecules
are added in to water in sufficient quantity, they self-associate
into aggregates in order to minimize contact between water and
hydrophobic group. When this happens, the
hydrocarbon chains are forced into close contact and each chain is
limited in its conformational freedom by the close presence of
neighboring chains. Some of them form micelles, others form bilayers.
Phospholipid molecules which have two hydrocarbon chains
tend to form bilayer structures (Voet \& Voet, 1990).

The phospholipid bilayer matrix is often employed as a model membrane
in experimental and theoretical research.
A prerequisite for an understanding of the bulk behavior and
functional properties of complex membrane lipids is a detailed
knowledge of the structural and dynamical properties of the lipid
bilayer. 
The specific phospholipid studied in this thesis is
1,2-palmitoyl-sn-glycerol-3-phosphatidylcholine, which is commonly
abbreviated as DPPC. DPPC has a phosphatidylcholine head group ether linked
by a glycerol backbone to two saturated fatty acid hydrocarbon chains, 
containing 16 carbon atoms each (Small, 1986). It serves as a benchmark 
for lipid bilayer studies
because its structure and phase behavior are very stable and clean
(Zhang, 1995)

\section{EXPERIMENTAL STUDIES ON THE PHOSPHOLIPID BILAYER} 

The phospholipid-water systems have been studied by many experimental 
disciplines. On one hand, the bulk properties can be viewed in terms of the 
overall thermodynamic or geometric changes occurring during cooperative 
phase transitions.  On the other hand, the microscopic properties at the 
molecular level are considered either as a static feature related to 
the arrangement or order of the molecules, or as a dynamic property 
related to the motion of individual components.

It is well known that there are three major first order transitions 
in fully hydrated DPPC bilayers,
namely sub-, pre- and main- transitions (Small, 1986).
First, there is the polymorphic sub-transition at about $25^\circ C$, 
i.e., from the $L_c$ crystalline to the $L_\beta'$ 
gel phase (Magni \& Sheridan, 1982).
The hydrocarbon chain in the $L_c$ crystalline phase is characterized
by an orthorhombic hybrid cell with area per chain about $19\AA^2$, 
which is the most dense form. In
this phase, there are two to ten water molecules per lipid, which
leads to about $59\AA$ in the lamellar spacing and $48\AA$ in the  
bilayer thickness (Ruocco \& Shipley, 1982a; 1982b).
In the $L_\beta'$ gel phase,
the hydrocarbon chains are mostly in all-trans configurations, and
are close-packed in a rotationally disordered orthorhombic hybrid
cell with an area per chain of about $20\AA^2$, and they are tilted 
at an angle of about $30^\circ$ to the bilayer normal. 
In the gel phase, there are  has 10 to 18 water molecules
per lipid, and the bilayer thickness and the lamellar spacing are
$45\AA$ and $64\AA$ respectively (Janiak et. al, 1979).
The pre-transition to the rippled gel phase involves
a transition from a distorted hexagonal to a true hexagonal 
chain packing with area per chain of about 20.5 \AA
(Janiak et. al, 1979).
Finally, the sharp endothermic main-transition 
to the $L_\alpha$ or liquid crystalline phase occurs at about $41^\circ
C$. In the biological most
relevant $L_\alpha$ phase, the hydrocarbon chains are
conformationally disordered and the lipid molecules are even free to
diffuse in the plane of the bilayer. The area per chain jumps to
over $28\AA^2$, and the bilayer thickness and the lamellar spacing
are $38\AA$ and $67\AA$, respectively (Small, 1986).

On a more microscopic level, the $L_\beta'$ gel and $L_\alpha$ liquid phases
have been intensively studied in the past twenty or so years. 
The main insight comes from neutron and X-ray diffraction experiments.
Neutron scattering  method is a direct and sensitive way
to trace the lipid structure along the bilayer normal.
In fact, neutron scattering with lipid molecules deuterated
at certain segments  can provide the mean position
of the deuterated segments and the width of the distribution of
those segments, along the bilayer normal direction.
For example, the experiments carried out by Buldt $et al.$ indicate
that in the $L_\beta'$ and $L_\alpha$ phases, the deuteriums attached
to the choline carbons (C$\gamma$,C$\beta$, C$\alpha$) are all about
the same distance from the center of the bilayer. In the gel phase,
this distance is $24-25\AA$, and it is $21\AA$ in the liquid phase.
Moreover, the mean positions of certain methylene groups show
that the hydrocarbon chain in the $L_\beta'$ gel phase is tilted,
almost in trans conformation, and on the contrary, it is melted in
the $L_\alpha$ liquid phase.
X-ray diffraction has yielded complementary information
on the profile structure via electron density profiles fit to low angle
diffraction data. 
Some of the structural features could be extracted,
such as the bilayer thickness and the lamellar spacing.
Moreover, together with the specific volume obtained by neutral
density centrifugation and model calculations, some most important 
structural properties, such as area per lipid in the $L_\alpha$ phase, 
can be estimated (Wiener et al., 1989; Nagle et al., 1995).
In addition, in some cases,
the lateral organization of the lipids could be obtained via
2D lattice modeling of wide angle diffraction data (Levine, 1973).

For the $L_\beta'$ gel phase, which contains sufficient order to allow
this kind of quantitative structural work, 
the models used to interpret X-ray data have attained
increasing sophistication over the years.  The most detailed to
date is that of Sun et al. (Sun et al., 1994), consisting of Gaussian electron
density distributions for the headgroups, and tilted rods of electron
density for the hydrocarbon chains, which are assumed to be packed in
a distorted hexagonal lattice and tilted toward nearest neighbors.
The model contains a gap (methyl trough) in the middle of the bilayer
and allows for separate offsets of the headgroups and chains in
opposing monolayers.

The disorder in the liquid crystal phase is undoubtedly important to
biological function, but it has made atomic-scale
structure determination impossible with present techniques.
Wide angle scattering gives limited information, namely, a single
broad peak, centered on the average interchain separation,
reflecting the absence of long-range order in the melted chains.
Most of what we know about the structure of the chains in liquid
crystal phase has come from deuterium NMR studies (Seelig \& Seelig,
1974). 
The deuterium quadrupole splitting is mainly
determined by the average conformation and the amplitude of
oscillations of individual segments (i.e., $C-D$ bonds),
and an order parameter, $S_{CD}$, defined to be proportional to
the splitting, is routinely used to characterize the behavior
of the hydrocarbon chains in membranes. In spite of a great deal
of experimental research on the average structure of liquid
crystal phase bilayers, there is still considerable variability
in the values for such basic structural parameters as the surface
area/lipid, bilayer thickness, and number of water molecules/lipid
(Nagle, 1993),
and this complicates setting up a atomic-scale
simulation (Nagle et al., 1995).

Compared to the large number of experimental structural investigations,
there have been relatively few studies of dynamics in lipid bilayers.
While information on lipid dynamics can be obtained by a variety
of spectroscopic techniques (e.g. fluorescence, NMR, infrared),
perhaps the most fruitful has been incoherent quasi-elastic neutron
scattering.
Analysis of the structure factors in terms of various dynamical models
has yielded timescales and extent of motion for a variety of dynamical
process including normal, lateral, and rotational diffusion (Pfeiffer et
al., 1989),
and chain defect motion (K\"{o}nig et al., 1995).
One of the more important results for simulators is that the
diffusion constant for lateral diffusion is 2 $\times$ 10$^{-7}$
cm$^2$/s, which implies that lateral diffusion will not be observable
in simulations of less than nanosecond timescales.

\section{MOLECULAR DYNAMICS SIMULATION}

Molecular dynamics (MD) simulation is a potentially
valuable tool to investigate biological system.
It has been proven powerful in studies
of proteins and nucleic acids (McCammon \& Harvey, 1987).
In principle, MD simulation can provide
atomic-scale structural and dynamical information for the
discussion of the structure and function of biological membranes,
since what emerges from this method is the trajectories of each
atom in this system (Allen \& Tildesley, 1989).

Ultimately, one would like to model a portion of a real biological
membrane, complete as a mixture of different lipids with proteins,
solvent, and ions, at the atomic level.
However, this goal is not likely to be realizable any time soon since,
with presently available algorithms and computer hardware,
the size of system of biological molecules that can be simulated
for a reasonable amount of time ($\approx$ 1 ns, still short on
the biological timescale!) is O(10$^5$) atoms.
However, simulations can potentially contribute to our understanding
to the function of biological membranes in the present
by refining the experimentally derived picture of the structure
and dynamics of model membranes, that is, hydrated lipid bilayers
alone and with cholesterol and/or small membrane bound protein
incorporated in them.  Indeed, many simulations of such model membranes
have been carried out recently under this premise.

The application of MD simulations to lipid bilayers with explicit
solvent was pioneered by Egberts and Berendsen in 1988 (Egberts \&
Berendsen, 1988).
Their initial study was on a ternary alcohol/fatty acid/water
system as a model membrane. As computer power subsequently increased,
several simulation groups turned their attention to membrane simulations,
and in the last three years, there have been many publications of
simulations of more realistic membranes.  Most of the work to date has
focused on liquid crystal phase phospholipid bilayers
(Damodaran et al., 1992; Venable et al., 1993; Heller et al., 1993;
Stouch, 1993; Egberts et al, 1994; Huang et al., 1994; Robinson et al.,
1994; Zhou \& Schulten, 1993; Shinoda et al., 1995; Tu et al., 1995b), 
but there has also been one published simulation of a gel phase bilayer
(Tu, et al., 1995c), one of a mixed phospholipid/cholesterol bilayer
(Robinson et al., 1995), and one of the peptide ion channel Gramicidin in
a membrane (Woolf \& Roux, 1994).

While many interesting results came from these simulations,
there is plenty of room for skepticism since membrane
simulations are hard to do, for several reasons, and it
is clear that much of the work to-date has not been adequately
validated.
Lipid potentials are complicated (see below) and,
in many cases, membrane simulations were based on potentials that
were derived after small modifications and without scrutiny,
from protein and nucleic acid potentials.
However, there is no guarantee that these potentials will work well
in simulations of membranes where the relative importance of
different types of interactions (hydrophobic/lipophilic,
electrostatic, etc.) are likely to be different.
Furthermore, many of the membrane simulations to date have used serious,
untested approximations, i.e. in the handling of long-ranged electrostatic
forces (Alper et al., 1993) and/or boundary conditions (Heller et al.,
1993)
for computational convenience and/or expediency.
Most of the lipid simulations to date have employed
truncation of electrostatic forces.
However, there is no well-prescribed procedure for determining the
optimal truncation scheme, and it has been shown that
poorly chosen schemes can introduce serious artifacts into
the simulation results (Alper et al., 1993).
Most of the membrane simulations have been carried out
at constant volume,
using experimental values for the system dimensions (surface
area/lipid, bilayer thickness, interlamellar spacing), although
there is considerable disagreement on the correct values for
these parameters even for the most studied lipid (Nagle, 1993).
It is difficult to assess the situation in most published constant
volume simulations because the pressure is rarely reported.
Pressures changes of roughly 1000 atmospheres, not uncommon in
simulations of polar molecules using standard force fields,
can induce phase transitions in lipid bilayers (Jonas and Jonas, 1994).
The first simulations of lipid bilayers at constant pressure in a
flexible simulation cell produced relatively short trajectories,
and convergence of the cell parameters was not demonstrated
convincingly (Egberts et al., 1994; Huang et al., 1994; Shinoda et al.,
1995). Finally, most membrane
simulations have been relatively short (few hundred picoseconds or
less), although it seems obvious that long equilibration times
should be necessary in systems with long, fluid hydrocarbon chains
and flexible, hydrated polar groups, especially in the absence of
well-prescribed initial configurations.

The general goal of our recent work in this field has been to show that
it is possible to perform converged constant pressure MD simulations
that yield stable lipid bilayers whose structures are in accord with
established experimental results.
Then we can use the results of these simulations to evaluate
assumptions used in the interpretation of
experimental data and, therefore, discern the best values
for key structural parameters such as the area/lipid and bilayer
thickness in the liquid crystal phase. Given the sensitivity
of constant pressure simulations to the details of the potential,
the first step of our approach has been to validate the potential.
To this end, we have performed constant pressure and temperature
simulations to test the all-atom potentials that we employ on
a series of systems with well-known structures: solid and liquid alkanes
(Tobias et. al., 1995), crystals of the phospholipid fragments 
dilauroylglycerol (DLG),
glycerylphosphorylcholine (GPC), and cyclopentylphosphorylcholine (CPPC)
monohydrate ( Tu et al., 1995a; 1995d), and a fully hydrated
gel phase dipalmitoylphosphatidylcholine (DPPC) bilayer.
The important structural quantities (liquid densities, crystal
lattice parameters, and bilayer area/lipid and lamellar spacing)
were reproduced to within 3$\%$ in every case.

This thesis is organized as follows.
The molecular dynamics simulation techniques are described in
Chapter~2. The potential models for the alkane and phospholipid 
fragments are present in Chapter~3 and Chapter~4.
In Chapter~4, we will summarize the  results of simulations on on
the fully hydrated $L_\beta'$ gel phase. The Chapter~5 discusses the
simulations of the  $L_\alpha$ liquid phase. The thesis will be
concluded in Chapter~6 with a brief description of further studies
that are presently underway.

\clearpage
\parindent=0.0in
\vspace*{0.5truecm}

{\bf{REFERENCES}}

\vspace*{0.5truecm}

H.~E. Alper, D.~Bassolino, and T.~R. Stouch.
{\em J. Chem. Phys.}, 98:9798, (1993).

M.~P. Allen and D.~J. Tildesley.
{\em Computer Simulation of Liquids}.
Oxford University Press, New York, 1989.

K.~V. Damodaran, K.~M. Merz, and B.~P. Gaber.
Structure and dynamics of the dilauroylphosphatidylethanolamine lipid
  bilayer.
{\em Biochemistry}, 31:7656--7664, (1992).

E.~Egberts and H.~J.~C. Berendsen.
Molecular dynamics simulation of a smectic liquid crystal with atomic
  detail.
{\em J. Chem. Phys.}, 89:3718--3732, (1988).

E.~Egberts, S.-J. Marrink, and H.~J.~C. Berendsen.
Molecular dynamics simulation of a phospholipid membrane.
{\em Eur. Biophys. J.}, 22:423--436, (1994).

P.~Huang, J.~J. Perez, and G.~H. Loew.
Molecular dynamics simulations of phospholipid bilayers.
{\em J. Biomol. Struct. Dyn.}, 11:927--956, (1994).

H.~Heller, M.~Schaefer, and K.~Schulten.
Molecular dynamics simulation of a bilayer of 200 lipids in the gel
  and in the liquid-crystal phases.
{\em J. Phys. Chem.}, 97:8343--8360, (1993).

M.~J. Janiak, D.~M. Small, and G.~G. Shipley.
{\em J. Biol. Chem.}, 254:6068, (1979).

S.~{K\"{o}nig}, T.~M. Bayerl, G.~Coddens, D.~Richter, and E.~Sackmann.
{\em Biophys. J.}, 68:1871, (1995).

Y.~K. Levine.
X-ray diffraction studies of membranes.
{\em Prog. Surf. Sci.}, 3:279--352, (1973).

J.~A. McCammon and S.~C. Harvey.
{\em Dynamics of Proteins and Nucleic Acids}.
Cambridge University Press, New York, 1987.

R.~Magni and J.~P. Sheridan.
{\em Biophys. J}, 37, 1982.

J.~F. Nagle.
Area/lipid of bilayers from {NMR}.
{\em Biophys. J.}, 64:1476--1481, (1993).

J.~F. Nagle, R.~Zhang, W.-J. Sun, S.~Tristram-Nagle, R.~L. Headrick, T.~Irving
and  R.~M.  Suter,
Manuscript in preparation, 1995.

W.~Pfeiffer, Th. Henkel, E.~Sackmann, W.~Knoll, and D.~Richter.
Local dynamics of lipid bilayers studied by incoherent quasi-elastic
  neutron scattering.
{\em Europhys. Lett.}, 8:201--206, (1989).

A.~J. Robinson, W.~G. Richards, P.~J. Thomas, and M.~M. Hann.
Head group and chain behavior in biological membranes: a molecular
  dynamics computer simulation.
{\em Biophys. J.}, 67:2345--2354, (1994).

A.~J. Robinson, W.~G. Richards, P.~J. Thomas, and M.~M. Hann.
{\em Biophys. J.}, 66:164, (1995).

M.~J. Ruocco and G.~G. Shipley.
{\em Biochim. Biophys. Acta}, 684:59, (1982).

M.~J. Ruocco and G.~G. Shipley.
{\em Biochem. Biophys. Acta}, 691:309--320, (1982).

W.~Shinoda, T.~Fukada, S.~Okazaki, and I.~Okada.
{\em Chem. Phys. Lett.}, 232:308--312, (1995).

D.~M. Small.
{\em The physical chemistry of lipids}.
Plenum Press, New York, 1986.

A.~Seelig and J.~Seelig.
{\em Biochemistry}, 13:4839--4845, (1974).

W.-J. Sun, R.~M. Suter, M.~A. Knewtson, C.~R. Worthington, S.~Tristram-Nagle,
  R.~Zhang, and J.~F. Nagle.
{\em Phys. Rev. E}, 49:4665, (1994).

T.~R. Stouch.
{\em Molec. Simulation}, 10:335--362, (1993).

D.~J. Tobias, K.~Tu, and M.~L. Klein.
Molecular dynamics simulation of solid and liquid alkanes using an
  all-atom model.
{\em J. Phys. Chem.}, in press, 1995.

K.~Tu, D.~J. Tobias, and M.~L. Klein.
Constant pressure and temperature molecular dynamics simulation of a
  fully hydrated liquid crystal phase DPPC bilayer. {\em Biophys. J.}, in
  press, (1995b). 

K.~Tu, D.~J. Tobias, and M.~L. Klein.
Molecular dynamics investigation of the structure of a fully hydrated
  gel phase DPPC bilayer. {\em Biophys. J.}, in press, 1995c.

K.~Tu, D.~J. Tobias, and M.~L. Klein.
Unpublished results, 1995d.

K.~Tu, D.~J. Tobias, and M.~L. Klein.
{\em J. Phys. Chem.}, 99:10035, (1995)a.

D.~Voet and J.~G. Voet.
{\em Biochemistry}.
John Wiley \& Sons, New York, 1990.

R.~M. Venable, Y.~Zhang, B.~J. Hardy, and R.~W. Pastor.
{\em Science}, 262:223--226, (1993).

T.~B. Woolf and B.~Roux.
{\em Proc. Natl. Acad. Sci. USA}, 91:11631, (1994).

M.~C. Wiener, R.~M. Suter, and J.~F. Nagle.
{\em Biophys. J.}, 55:315--325, (1989).

R.~Zhang.
{\em Lecithin Bilayers in Fluid Phase: Effect of Fluctuations on
  X-ray Determination of Structure}.
PhD thesis, Carnegie Mellon University, 1995.

F.~Zhou and K.~Schulten.
{\em J. Phys. Chem.}, 99:2194--2207, (1993).


