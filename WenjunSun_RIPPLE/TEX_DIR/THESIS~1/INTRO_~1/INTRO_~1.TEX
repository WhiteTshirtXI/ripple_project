\chapter{Introduction}
\label{intro_chap}

\section{Biomembranes and their functions}
\label{intro_chap_biom}

Biomembranes are the walls or barriers separating a living organism,
be it a cell or a subcell organelle, from its environment. Besides
maintaining the structural integrity of the living entity, biomembranes
have to be flexible to adjust to environmental changes. Biomembranes
are also the places where essential living processes are carried out;
these include the exchange of substances and information with the
surroundings, and the generation of energy \cite{Voet90}.

Living organisms maintain and regulate their internal compositions
through biomembranes. This is realized because biomembranes are
highly-selective semi-permeable barriers which control the flows of
nutrients, waste products, ions into and out of the organism
through membrane-embedded pumps and gates.

Biomembranes also control the flow of information between organisms and 
their environment. They contain specific receptors
for external stimuli, and in turn, some membranes also generate
signals. Thus membranes play a central role in biological
communication.

In addition, two most important energy conversion processes in
biological systems involve biomembranes.
The first is oxidative phosphorylation process which generates
and store energy in the form of ATP (Adenosine TriPhosphate) by oxidizing
nutrients \cite{Voet90}; this process takes place in the inner membrane of
mitochondria.  The second is well-known photosynthesis, in which 
light energy is converted into chemical-bond energy by forming 
carbohydrates from H$_2$O and CO$_2$; this process occurs in the inner 
membranes of chloroplasts \cite{Voet90}.

\section{Phospholipids and lipid bilayers as model biomembranes}

Biomembranes are essentially a 2-D matrix composed of lipids, proteins and 
small amounts of carbohydrate. Phospholipids are one of the major lipid 
components in biological membranes. A phospholipid molecule basically
consists of a polar, hydrophilic headgroup and apolar, hydrophobic
hydrocarbon tails; because of this special feature, lipid molecules
are also called amphiphiles. When sufficient amounts of phospholipid
molecules are dispersed into water, they spontaneously assemble into 
several forms of aggregates, including micelles, lamellar bilayers
and cubic phases. The driving force for these aggregate formations
is the tendency of hydrophobic hydrocarbon tails to be away from water
and for the heads to be in water in order to minimize the free energy.

Among the most studied phospholipids are the ones with PC (phosphatidylcholine)
headgroups; this category of phospholipids are also called lecithins. Two
lecithin molecules with 14 saturated carbons on each of the 
two chains are shown in Figure~\ref{intro:dmpc}; this molecule is named DMPC 
(Dimyristoylphosphatidylcholine or C14, following a convention later chapters 
of this thesis will use.). 
Saturated lecithins like DMPC and others with longer chain lengths 
are the focus of this thesis.

\begin{figure}[h]
\label{intro:dmpc}
\centerline {\psfig{figure=intro_dir/dmpc_pair.ps,width=4in}}
\caption{Space-filling model of a pair of DMPC (C14) molecules.
By courtesy of Dr. Kechuan Tu.}
\end{figure}

Phospholipid bilayers (see Fig.~\ref{intro_bilayer}) are often employed as 
model biomembranes, both in experiments and computer simulations,
because pure lipid bilayers have less complex thermodynamic behaviors
than real biomembranes. Elucidating the structures and thermodynamics
of these model bilayer systems is essential for understanding the bulk 
behavior and functional properties of real, multi-component biological
membranes.

\begin{figure}
\label{intro_bilayer}
\centerline {\psfig{figure=intro_dir/bilayer_model.ps,width=4in}}
\caption{Space-filling model of two gel phase DPPC (C16) lipid bilayers.
In between the two bilayers is water.
By courtesy of Dr. Kechuan Tu.}
\end{figure}

\section{Phases and structures of lipid bilayers}
\label{intro_chap_phas}

The supramolecular packing of the bilayers forms lyotropic smectic
liquid crystals with rich phase behavior and diverse structures. The 
richness and diversity of phases and structures result from the 
competitions and the balance among many of the interactions present in 
lipid bilayer systems, including chain-chain and interbilayer van der 
Waals interactions, interbilayer hydration interaction, head-head 
electrostatic interactions \cite{Isr85}. Extensive structural and 
thermodynamic studies of model membrane systems have been performed [1,6-27].
%\cite{WSN89,BMc86,Buldt79,Hui76,HenRus91,JanSS76,Kat92,LevTh,McI80,%
%Mit78,Nag80,NW78,Smi88,Sta79,Tar73,TW76,STN92,Wac89a,WW6,WoK78,Sta82,%
%HenHos83,Mit80}. 

Noncrystalline lamellar phases observed in lecithin bilayer systems
include the subgel (L$_{c '}$) phase, the gel (L$_{\beta '}$) phase,
the ripple (P$_{\beta '}$) phase and the fluid (L$_{\alpha}$) phase
\cite{Sma86}. The L$_{c '}$ phase has a crystalline-like intra-bilayer 
structure, in which both the hydrocarbon chains and headgroups are 
ordered. But this in-plane order doesn't correlate with that of the 
adjacent bilayers; because of this, the  L$_{c '}$ is not a crystalline 
phase. Undergoing a so-called sub-transition when the temperature is 
increased, the bilayers go into the L$_{\beta '}$ phase. In this phase, 
the hydrocarbon chains are basically in the all-trans conformation and 
the headgroups are disordered. At higher temperature, after the 
pre-transition (also called the lower-transition by some researchers), 
the system is in the ripple (P$_{\beta '}$) phase, in which the bilayers 
are not flat but modulated, while most of the chains are as ordered
as in the gel phase. As the temperature is further increased and 
passes the main-transition (also called the chain-melting transition), 
the system is in the fluid phase; in this fluid phase, both chains and 
headgroups are disordered and the whole bilayers fluctuate around their 
equilibrium positions. Since nearly all real biomembranes are fluid-like, 
the fluid phase is the most biologically relevant phase.
Schematics of the L$_{\beta '}$, P$_{\beta '}$ and L$_{\alpha}$ phases
are shown in Fig. \ref{intro_phase} \cite{Sma86}.

This thesis focusses on the chain ordered phases, particularly the gel and 
ripple phases in lipid bilayers. 
These phases have not been fully understood and there are still 
disagreements in the literature about structures of these phases, 
especially the relatively less studied ripple phase. In addition, these 
chain ordered phases are not crystals; they also possess various degrees 
of disorder; understanding the disorder in these phases can serve as 
a prelude to the study of disorder in the fluid phase.

From the point of view of biological relevance, the strongest reason to
study the chain ordered phases is that the determination of the
biologically relevant fluid phase structure requires the gel phase 
structure. The major difficulty with obtaining fluid phase structure
is the paucity of structure data. In contrast to the fluid phase,
for the chain ordered phases, there is usable wide angle x-ray data and
more low angle data, so better structure can be obtained. By bootstrapping
from the much better determined gel phase structure obtained in this
thesis, reliable fluid phase structure can be obtained. This has been 
demonstrated in recent work of this lab \cite{N96}.

%Disorders and fluctuations are important to real biological systems 
%\cite{Voet90,N96}. Biomembranes at physiological temperatures are 
%two-dimensional fluids in which lipid molecules are disordered and 
%undergo constant lateral diffusions and much rare transverse diffusions 
%(flip-flop). Cell fusion, division, endocytosis and exocytosis are all 
%vital biological processes which require fluid and flexible membranes.

\begin{figure}
\label{intro_phase}
\centerline {\psfig{figure=intro_dir/lamellar.eps,width=6in}}
\caption{Three major phases of lipid bilayers and the phase
transitions between them.}
\end{figure}

In order to understand the different roles of headgroups and hydrocarbon
chains in the structure and thermodynamic properties of lipid
bilayers, one strategy is to vary the chain length. This variation 
systematically alters the balance between the interaction energy 
involving the headgroups, which remains the same, and the total 
interaction between the chains, which increases with chain length. 
Chapters \ref{gel_anol_chap} and \ref{gel_long_chap} will show
the results of chain-length dependence of gel phase structure 
and thermodynamic properties. 

\section{Structure determination using the x-ray scattering method}
\label{intro_chap_xray}

X-ray scattering is our primary tool in investigating order and
disorder in lipid bilayer systems. Ordered and periodic
structures can be determined from the Bragg diffraction.
Some of the key structural parameters of gel phase lipid bilayers 
are shown in Fig. \ref{intro_struc}. Using Bragg's law and 
Fourier reconstruction of the transbilayer electron density profiles, 
the lamellar packing structural parameters can be determined from the 
low angle data. Wide angle data analysis determines the 
chain packing structure. 
In contrast, the information about disorder in the lipid bilayer system
is contained in the broad diffuse scattering underlying the
Bragg peaks. 

%As shown in \cite{N96}, the bilayer fluctuation
%in the broad diffuse scatterings and fluctuations in the non-Bragg
%portion of the scattering peaks \cite{N96}.

\begin{figure}
\label{intro_struc}
\centerline {\psfig{figure=intro_dir/schematic.ps,width=5in}}
\caption{Definitions of main structural parameters of the gel phase
lipid bilayers. Shown in the right picture is the cross section
perpendicular to the chains, with each black dot representing
a hydro-carbon chain.}
\end{figure}

X-ray scattering provides the lineshapes of the diffraction peaks
and the amplitudes of the form factors, but not the phase information. 
In order to determine the detailed and accurate electron density
distributions either across the bilayer or intra-bilayer (i.e. the
lateral distribution), electron density modeling and nonlinear least
square fitting are sometimes necessary. In Chapter \ref{gel_model_chap},
we propose and apply an electron density model to fit the wide
angle Bragg scattering data of an unoriented sample of gel phase DPPC
bilayers, excellent fit was achieved and many key structural parameters
were extracted; these parameters include the chain tilt angle, 
effective chain length and offsets of the two monolayers. It had been
thought that only from oriented samples could those parameters be obtained;
our modeling results show that with a correct model and careful data
analysis, powder samples yield no less information than the oriented
samples. In Chapter \ref{rppl_chap}, we solve the structure of the
ripple phase by another modeling approach. Our ripple phase model fit 
provides the phase information of 20 or so peaks typical in ripple phase data.
It was the difficulty of obtaining the correct phase combination which
hindered early work on the ripple phase from revealing the ripple
structure.

\section{Description of thesis}
\label{intro_chap_desc}

The rest of this thesis is organized as follows. 
Chapter \ref{matmet_chap} is the Materials and Methods section.
Chapter \ref{gel_model_chap} shows our model analysis on the wide
angle results of the gel phase of fully hydrated unoriented 
MLVs (multilamellar vesicles) of DPPC. Chapters \ref{gel_anol_chap} 
and \ref{gel_long_chap} show systematic
low-angle and wide-angle x-ray scattering studies on fully hydrated
unoriented MLVs of saturated lecithins with even chain
lengths N = 16, 18, 20, 22 and 24 as a function of temperature in the 
chain ordered phases; Chapter \ref{gel_anol_chap} shows the anomalous 
phase behavior of the longer chain lengths $N \geq 20$; Chapter 
\ref{gel_long_chap} shows the temperature and chain length dependence 
of the normal gel phase.  Chapter \ref{rppl_chap} shows the study on 
the structure of the ripple (P$_{\beta '}$) phase.
