\chapter{Materials and Methods}
\label{matmet_chap}

\section{Phospholipid sample preparation}

For the work in Chapter \ref{gel_model_chap} DPPC (Lot \#160 PC-176) 
was purchased from Avanti Polar Lipids (Birmingham, AL) and used
without further purification.  Samples were prepared for X-ray scattering
by weighing water and lipid in a 3:1 (w:w) ratio into a small Nalgene vial. 
The lipid was hydrated by 
cycling it three times between $80^{\circ}$C and $5^{\circ}$C with 5 minutes 
of vortexing at each temperature. After hydration the sample was loaded into 
thin walled 1.0 mm glass X-ray capillaries (Charles Supper Co.).  The 
capillaries had been precleaned by sequentially washing with chromic acid, 
acetone and copious amounts of deionized water. After drying with nitrogen 
and flame sealing the capillaries at one end, they were filled with hydrated 
lipid using a 1.0 ml Hamilton syringe. Upon standing, these dispersions 
separate into a lipid rich phase and a clear water rich phase.  
In order to remove air bubbles, the capillaries
were centrifuged for 10 minutes at 1100g at room temperature.
This amount of centrifugation did not overly compress the lipid, 
since upon additional standing for one week, the lipid settled 
further as indicated by the presence of a larger volume of water rich phase at
the top of the capillary.  X-ray scattering was obtained from the
lipid rich phase and samples with either weight ratio gave the same low 
angle D-spacing of 63.4\AA. In addition, the low angle D-spacing was 
unaffected by centrifugation. After centrifugation the capillaries were 
flame-sealed above the water layer, and this seal was dipped in Duco cement.  
After completion of the experiment, the continued presence of an excess
water layer above the lipid was observed, confirming that the sample was
fully hydrated during the course of the experiment. Then the lipid was 
removed from the capillary, dried under nitrogen and analyzed by thin layer 
chromatography using the solvent system chloroform:methanol:water (60:30:5).
The chromatogram showed less than 0.2\% lysolecithin formed during irradiation.
Also, the positions and half-widths of the first and second order low angle 
peaks were identical before and after 48 hours of irradiation, indicating 
that any degradation did not affect gel phase structure.

For the work reported in Chapters \ref{gel_anol_chap} and \ref{gel_long_chap},
all lecithins were purchased in lyophilized form from Avanti Polar Lipids
(1,2-dipalmitoyl-$sn$-glycero-3-phosphatidylcholine (to be 
abbreviated C16) Lot\# 160PC-188; 
1,2-distearoyl-$sn$-glycero-3-phosphatidylcholine (C18) 
Lot\# 180PC--76; 1,2-diarachidoyl-$sn$-glycero-3-phosphatidylcholine (C20)
Lot\# 200PC--17; 1,2-dibehenoyl-$sn$-glycero-3-phospha- tidylcholine (C22)
Lot\# 220PC--24; 1,2-dilignoceroyl-$sn$-glycero-3-phosphatidylcholine (C24)
Lot\# 240PC--24) and used without further purification.
Lipid/water dispersions were placed in $1mm \times 4cm$ capillaries following 
standard procedures \cite{STN92}. Upon brief centrifugation, these dispersions 
separate into a lipid rich phase, from which X--ray scattering was performed, 
and a clear water rich phase, thus demonstrating full hydration.  
%The sealed sample was further annealed by incubating at 80-95$^{\circ}$C 
%overnight. 
Following scattering measurements, the lipid was dried under nitrogen
and assayed for radiation and thermal damage by thin layer chromatography 
using the solvent system chloroform:methanol:7N ammonium hydroxide (46:18:3).
Less than 1\% lysolecithin was detected for all the samples.

\section{X-ray experiments}

Our principal measurements were carried out using a rotating anode x-ray 
source run at 35 kV and 150 mA and interfaced with a Huber four circle
diffractometer via evacuated beampaths (see Fig. \ref{matmet_xray}).
Monochromatic copper $K_{\alpha}$ radiation was obtained
by diffraction from a vertically bent graphite monochromator.
Several different input and diffracted beam collimation and
detection schemes were used, as described below. On the input side,
two sets of xy-slits, separated by 500 mm, define the beam size
and divergence in the in-plane and out-of-plane directions.
The four slit openings will be referred to as $\mbox{S}_{2h}$, 
$\mbox{S}_{2v}$, $\mbox{S}_{3h}$ and $\mbox{S}_{3v}$; as can be seen
in Fig. \ref{matmet_xray}, $\mbox{S}_{2}$ slits are the ones right
after the monochromator, and $\mbox{S}_{3}$ slits are right before
the sample, {\em h} is the horizontal (in-plane) opening, 
and {\em v} is the vertical (out-of-plane) opening. Slits S4 after the sample 
eliminated extraneous scattering from air.  An xy-slit set S5 before the 
detector was also used to reduce extraneous radiation and to define the
out-of-plane divergence.

For most measurements in the wide angle region, a Braun linear
position sensitive detector (PSD) (see Fig. \ref{matmet_xray}) was placed 
on the detector arm of the diffractometer at a distance of 572 mm from 
the sample center. 

\begin{figure}[h]
\label{matmet_xray}
\centerline{\psfig{figure=matmet_dir/psdhi.ps,width=6in}}
\caption{Schematic top view of the x-ray instrumental setup. MH, monochromator;
PSD, Positional Sensitive Detector; S1 --- 5 are slits.}
\end{figure}

For the work reported in Chapter \ref{gel_model_chap},
all vertical slit openings were 4 mm. $\mbox{S}_{2h}$ and $\mbox{S}_{3h}$ 
were set to 0.4 mm.  Thus the beam size was less than the 1 mm
capillary diameter and the in-plane divergence was $0.092^{\circ}$ full width.
Accounting for the finite beam size, the finite sample size and the
divergence of the incoming beam leads to an estimate for the resolution
half width at half maximum (HWHM) of $0.08^{\circ}$ in 2$\theta$ which
is expected to be an overestimate. 
Some additional measurements used two other setups. First,
to verify the background scattering obtained with the PSD,
low resolution data ($0.14^{\circ}$ HWHM) were
also taken for 2$\theta$ from $1^{\circ}$ to $55^{\circ}$ using a 
graphite crystal analyzer. To fully resolve the sharp (20) peak, 
the F-3 beamline at CHESS was used with silicon monochromator, silicon 
analyzer crystal and photomultiplier detector. This setup had a resolution 
of $0.004^{\circ}$ HWHM at $\lambda = 1.2148$ \AA.

We carried out a limited set of measurements in the low angle region
to verify that samples were in the fully hydrated state and to check
for effects of radiation damage.  These measurements used the PSD
with slits configured in the same way as for the wide angle
measurements.  Some smearing of peaks on the low angle side occurs in
this configuration; we obtained best estimates for the D-spacing by
extrapolating the fitted peak positions to the higher order limit using
model calculations of the smearing effect \cite{STN92}. While this procedure
yields only approximate values, it is reliable for comparing fresh
samples with those which have been exposed to the x-ray beam for
substantial periods.  All data reported here are from samples for
which no discernable shift in D-spacing or broadening of the low or wide
angle peaks can be detected after measurements were completed. 

For Chapters \ref{gel_anol_chap} and \ref{gel_long_chap},
the instrumental (longitudinal) resolution was estimated to be
0.065$^{\circ}$ (HWHM) in $2 \theta$ or $\Delta q = 5 \times 
10^{-3}$\AA$^{-1}$. The maximum exposure time on one spot was about 
4 hours with X--ray power of 5.25 kW; no changes in the diffraction 
peaks were detected in this time. In addition, 
for Chapter \ref{gel_anol_chap}, some film data were collected in 17 hours 
using a Philips Norelco X-ray source at 0.2 kW. The beam was pinhole 
collimated with a sample-to-film distance of 66 mm, and Kodak DEF5 
film was used.

\section{Sample chamber}

For the work reported in Chapter \ref{gel_model_chap},
the lipid capillary was held upright in a custom built sample chamber made 
from aluminum with $1.5\mu$ thick mylar windows (DuPont).  A calibrated 
silicon diode (Type DT-470-CU-13) coated with Dow Corning heat sink compound 
was seated in a pocket in the aluminum block directly next to the capillary. 
The diode was connected to a Lake Shore Cryotronics Model DRC 84C Temperature
Controller which monitored the temperature.

For Chapters \ref{gel_anol_chap} and \ref{gel_long_chap},
the sample chamber is a custom-built aluminum can 
with ultrathin (1.5$\mu$) mylar windows.  It
holds a cassette with slots for 10 X--ray capillaries. The chamber is connected
to a motorized 3D translational unit via 
a stainless steel rod that can be moved 2 inches vertically and 1 inch 
horizontally; this allows easy access of the X--ray
beam to the capillaries and new spots on each capillary, thus greatly
facilitating sample loading and allowing us to minimize radiation damage 
by frequently translating unexposed sample into the beam. 
The 3D translation unit consists of two low-profile Newport translations, 
series 436 and 426 (Irvine, CA) equipped with 860A-2 and 860A-1 motorizers, 
respectively, and a third manual translation, series 426, which can move 
the chamber 1 inch in the direction of the beam with a Newport SM-1
vernier micrometer.

To control the temperature, heating strips are symmetrically arranged on the 
outside of the chamber in order to minimize the temperature gradient. 
The temperature was monitored and controlled via a Rosemount Model 
146MA1000F platinum resistance thermometer (PRT, Eagen, MN) that had 
been factory calibrated. Since the PRT was centrally located in the cassette, 
a careful measurement of the temperature gradient between the PRT and each 
capillary position was carried out using a platinum thin film detector 
(TFD) (Omega Engineering, Inc., Stamford, CT) that had been calibrated to 
the Rosemount PRT. The temperature gradient was found to
be within 0.05$^{\circ}$C/mm for all the capillaries except the one closest 
to each heating strip. For this reason these two slots were not used in these 
experiments. The temperature gradient was determined by a separate experiment 
that observed the proportion of the two phases in two-component (DPPC/DSPC)
lipid bilayers near the main transition.

%Fig. \ref{samhold_overview} shows and overview of the whole sample
%manipulation system. 

%\begin{figure}[h]
%\label{samhold_overview}
%\centerline{\psfig{figure=matmet_dir/samhold_overview.ps,width=6in}}
%\caption{Vertical cross-section view of the sample manipulation
%system.} 
%\end{figure}

%aluminum cassette holder with a $1\times 6.5$ cm heating strip on each side
%(Minco, Minneapolis, MN). The rectangular cassette holder was contained within
%an air--filled beryllium cannister. This chamber was attached to a stainless 
%steel rod that can be moved 2 inches vertically and 1 inch horizontally
%with two low-profile Newport translations, series 436 and 426 (Irvine, CA)
%equipped with 860A-2 and 860A-1 motorizers, respectively. A third translation,
%series 426, can move the chamber 1 inch manually in the direction of the
%beam with a Newport SM-1 vernier micrometer. 

%The temperature was monitored by a platinum
%resistance thermometer at the center of the cassette and 
%temperature was automatically controlled through
%heating strips symmetrically arranged on the outside of the chamber in order
%to minimize the temperature gradient.
%the temperature differences
%between different capillaries and different parts of each capillary
%Temperature gradients were less than 0.05$^{\circ}$C/mm 
%along each capillary, as determined by a separate experiment that
%observed the proportion of the two phases in two-component (DPPC/DSPC)
%lipid bilayers near the main transition.
