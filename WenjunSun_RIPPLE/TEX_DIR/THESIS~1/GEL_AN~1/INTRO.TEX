This chapter and the next chapter study the chain length
and temperature dependence of the chain ordered lamellar phases.
This chapter mainly deals with the anomalous phase behavior
of long chain lecithins. The next chapter reports our study
on the regular gel phase.

\section{Introduction}

The study of lipid bilayers has been and will continue to be greatly enriched
by investigating how the structure and thermodynamic properties vary as the 
lipids are varied, including both naturally occurring and specifically
synthesized lipids. One particularly appropriate strategy
is to vary the chain length \cite{NW78,LMc87,Huang94,Cev91},
since this variation systematically alters the
balance between the interactions involving the headgroups,
which remain the same, and the total interactions between the chains, which 
increase with increasing chain length.
This variation yields increased main transition temperatures with 
increased chain length \cite{NW78,LMc87,Huang94,Cev91}.  It also
may involve more subtle changes; e.g., decreasing chain length 
for the phosphatidylethanolamines alters the phase diagram from one having
stable gel phases to one in which the gel phase is merely metastable at all 
temperatures \cite{WN84}.

In a recent study from this laboratory, it was found that the wide angle 
pattern in the gel phase started to become qualitatively different as 
the chain length n was increased beyond 20 carbons \cite{STN92}.  
When C$_{22}$ and C$_{24}$ chains were investigated, rather different wide 
angle patterns were observed compared to those for shorter chain lengths 
$14 < n < 20$, and so a detailed study of the longer chain lengths was not 
included in a previous report \cite{STN92}. 

This chapter reports our detailed study on the anomalous phases of 
the longer chain lecithins. This chapter concentrates on C$_{24}$, with
some results on C$_{22}$.  Although the history and time dependence 
shows that equilibrium was not always achieved, it appears that there 
is a second gel-like phase G2 below 40$^{\circ}$C for C$_{24}$. The G2 
phase has a small tilt angle and opposite hexagonal symmetry breaking 
from the usual G1 gel phase. Also, as T is raised above 45$^{\circ}$C 
for C$_{24}$, the wide angle data suggest the appearance of a phase 
G3 with hexagonal chain packing and small chain tilt angle. C$_{22}$
also has similar phase behaviors.
