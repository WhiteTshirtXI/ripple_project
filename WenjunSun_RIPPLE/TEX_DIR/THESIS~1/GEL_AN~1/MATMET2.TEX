\section{MATERIALS AND METHODS}

%\subsection{Phospholipid Samples}

All lecithins were purchased from Avanti Polar Lipids (1,2-dipalmitoyl-$sn$-glycero-3-ph--osphatidylcholine (to be abbreviated C$_{16}$): Lot\# 
160PC-188; 1,2-distearoyl-$sn$-glycero-3-ph--osphatidylcholine (C$_{18}$): 
Lot\# 180PC--76; 1,2-diarachidoyl-$sn$-glycero-3-phosphatidylcholine (C$_{20}$)
: Lot\# 200PC--17; 1,2-dibehenoyl-$sn$-glycero-3-phosphatidylcholine (C$_{22}$):
Lot\# 220PC--24; 1,2-dilignoceroyl-$sn$-glycero-3-phosphatidylcholine (C$_{24}$)
: Lot\# 240PC--24) and used 
without further purification.
Lipid/water dispersions were placed in
$1mm \times 4cm$ capillaries following standard procedures \cite{STN92}. 
Upon brief centrifugation, these dispersions separate
into a lipid rich phase, from which x--ray scattering was performed, and a clear
water rich phase, thus demonstrating full hydration. Following scattering 
measurements, the lipid was dried under nitrogen
and assayed for radiation and thermal damage by thin layer chromatography 
using the solvent system chloroform:methanol:7N ammonium hydroxide (46:18:3).

%\subsection{Sample Chamber}

The sample chamber is a custom-built aluminum can with mylar windows.  It
holds a cassette with slots for 10 x-ray capillaries. The chamber is connected
to a motorized 3D translational unit, which allows easy access of the X-ray
beam to the capillaries and new spots on each capillary; this minimizes sample 
loading effort and allows us to minimize radiation damage by frequently translating
unexposed sample into the beam. The temperature was monitored by a platinum
resistance thermometer and temperature was automatically controlled through
heating strips symmetrically arranged on the outside of the chamber in order
to minimize the temperature gradient.
%the temperature differences
%between different capillaries and different parts of each capillary
%Temperature gradients were less than 0.05$^{\circ}$C/mm 
%along each capillary, as determined by a separate experiment that
%observed the proportion of the two phases in two-component (DPPC/DSPC)
%lipid bilayers near the main transition.

%The sample capillaries were held vertically in a custom-built aluminum
%cassette with 11 slots for capillaries.  The cassette was loaded into an 
%aluminum cassette holder with a $1\times 6.5$ cm heating strip on each side
%(Minco, Minneapolis, MN). The rectangular cassette holder was contained within
%an air--filled beryllium cannister. This chamber was attached to a stainless 
%steel rod that could be moved 2 inches vertically and 1 inch horizontally
%with two low-profile Newport translations, series 436 and 426 (Irvine, CA)
%equipped with 860A-2 and 860A-1 motorizers, respectively. A third translation,
%series 426, could move the chamber 1 inch manually in the direction of the
%beam with a Newport SM-1 vernier micrometer. The temperature was monitored
%and controlled via a Rosemount Model 146MA1000F platinum resistance 
%thermometer (PRT, Eagen, MN) that had been factory calibrated. Since the PRT
%was centrally located in the cassette, a careful measurement of the temperature
%gradient between the PRT and each capillary position was carried out using a 
%platinum thin film detector (TFD) (Omega Engineering, Inc., Stamford, CT) that
%had been calibrated to the Rosemount PRT. The temperature gradient was found to
%be within 1$^{\circ}$C for all the capillaries except the one closest to each
%heating strip. For this reason these two slots were not used in these 
%experiments.

%\subsection{X--ray Scattering}

Our principle measurements were carried out using Cu $K_{\alpha}$ X rays
from a rotating anode source interfaced with a four-circle diffractometer 
and a Braun linear 
position sensitive detector (PSD) as previously described \cite{Sun94}. 
Accounting for the finite beam size, the finite sample size and the divergence
of the incoming beam leads to an estimate for the instrumental (longitudinal)
resolution
of 0.075$^{\circ}$ (HWHM) in $2 \theta$ or $\Delta q = 5 \times 10^{-3}$\AA$
^{-1}$.  The maximum exposure time on one spot was about 4 hours with x-ray 
power of 5.25 kW using a graphite monochromator; no radiation damage was 
detected in this time.


