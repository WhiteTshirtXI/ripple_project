%\documentclass[titlepage,12pt]{article}
\documentstyle[preprint,aps,epsf]{revtex}
\begin{document}
%\bibliographystyle{prsty}
%\bibliographystyle{unsrt}
\bibliographystyle{bba}

\begin{titlepage}
\title{ANOMALOUS PHASE BEHAVIOR OF\protect\\
LONG CHAIN SATURATED LECITHIN BILAYERS
}
\author{W.-J. Sun$^{\ a}$, S. Tristram-Nagle$^{\ b}$, R. M. Suter$^{\ a}$ 
and J. F. Nagle$^{\ a,b,}$\footnote{\ Corresponding author. Fax: +1 (412) 
681--0648}} 
\address{$^{\ a}$Departments of Physics and $^{\ b}$Biological Sciences,
\protect\\
 Carnegie Mellon University , Pittsburgh, PA 15213
}
\date{\today}
This is page 0. Please discard.
\maketitle
\vspace{360pt}
\begin{flushleft}
\textit{Key words:} Lipid bilayer; Phosphatidylcholine; X-ray scattering;
Gel phase
\end{flushleft}
\end{titlepage}

% ********************** abst.tex ***********************************
\begin{abstract}
X--ray scattering has been performed on fully hydrated unoriented
multilamellar vesicles of lecithins with even chain lengths n from 16 to 24
as a function of temperature in chain ordered phases. The longer chain lengths, 
$n \geq 20$, show anomalous behavior compared to the shorter chain lengths,
$n< 20$.  This report concentrates on $n = 24$.  Although the history
and time dependence shows that equilibrium was not always achieved, it
appears that there is a second gel-like phase G2 below 40$^{\circ}$C.  
The G2 phase has a small tilt angle and opposite hexagonal symmetry breaking 
from the usual G1 gel phase. Also, as T is raised above 45$^{\circ}$C, the 
wide angle data suggest the appearance of a phase with hexagonal chain packing 
and small chain tilt angle.
\end{abstract}
\newpage

%---------------------- Introduction -------------------------------
\section{Introduction}

The study of lipid bilayers has been and will continue to be greatly enriched
by investigating how the structure and thermodynamic properties vary as the 
lipids are varied, including both naturally occurring and specifically
synthesized lipids. One particularly appropriate strategy is to vary the chain 
length \cite{NW78,LMc87,Huang94,Cev91}, since this variation systematically 
alters the balance between the interactions involving the headgroups,
which remain the same, and the total interactions between the chains, which 
increase with increasing chain length. This variation yields increased main 
transition temperatures with increased chain length \cite{NW78,LMc87,Huang94,%
Cev91}. It also may involve more subtle changes; e.g., decreasing chain length 
for the phosphatidylethanolamines alters the phase diagram from one having
stable gel phases to one in which the gel phase is merely metastable at all 
temperatures \cite{WN84}.

In a recent study of a sequence of di-saturated phosphatidylcholines from
this laboratory, it was found that the wide angle pattern in the gel phase
started to become qualitatively different as the chain length n was increased
beyond 20 carbons \cite{STN92}. When C$_{22}$ and C$_{24}$ chains were 
investigated, rather different wide angle patterns were observed compared to 
those for shorter chain lengths $14 < n < 20$, and so a detailed study of the 
longer chain lengths was not included in a previous report \cite{STN92}. The 
purpose of the present study is to report some details of this anomalous 
behavior of the longer chain lecithins.

%-------------- Materials and methods -----------------------------------
\section{Matherial and methods}

All lecithins were purchased in lyophilized form from Avanti Polar Lipids
(1,2-dipalmitoyl-$sn$-glycero-3-phosphatidylcholine (to be abbreviated 
C$_{16}$) Lot\# 160PC-188; 1,2-distearoyl-$sn$-glycero-3-phosphatidylcholine 
(C$_{18}$) Lot\# 180PC--76; 1,2-diarachidoyl-$sn$-glycero-3-phosphatidylcholine 
(C$_{20}$) Lot\# 200PC--17; 1,2-dibehenoyl-$sn$-glycero-3-phospha- tidylcholine
(C$_{22}$) Lot\# 220PC--24; 1,2-dilignoceroyl-$sn$-glycero-3-phosphatidylcholine
(C$_{24}$) Lot\# 240PC--24) and used without further purification.
Lipid/water dispersions were placed in $1mm \times 4cm$ capillaries following 
standard procedures \cite{STN92}. Upon brief centrifugation, these dispersions 
separate into a lipid rich phase, from which X--ray scattering was performed, 
and a clear water rich phase, thus demonstrating full hydration.  The sealed 
sample was further annealed by incubating at 80-95$^{\circ}$C overnight. 
Following scattering measurements, the lipid was dried under nitrogen
and assayed for radiation and thermal damage by thin layer chromatography 
using the solvent system chloroform:methanol:7N ammonium hydroxide (46:18:3).

The sample chamber is a custom-built aluminum can with ultrathin (1.5$\mu$) 
mylar windows. It holds a cassette with slots for 10 X--ray capillaries. The 
chamber is connected to a motorized 3D translational unit, which allows easy 
access of the X--ray beam to the capillaries and new spots on each capillary; 
this facilitates sample loading and allows us to minimize radiation damage by 
frequently translating unexposed sample into the beam. The temperature was 
monitored by a platinum resistance thermometer and temperature was 
automatically controlled through heating strips symmetrically arranged on the 
outside of the chamber in order to minimize the temperature gradient.

Our principle measurements were carried out using Cu $K_{\alpha}$ X rays
from a rotating anode source interfaced with a four-circle diffractometer 
and a Braun linear position sensitive detector (PSD) as previously described 
\cite{Sun94}. We used a graphite monochromator and a pair of xy slit sets to 
further restrict the beam size and divergence. Accounting for the finite beam 
size, the finite sample size and the divergence of the incoming beam leads to 
an estimate for the instrumental (longitudinal) resolution of 0.075$^{\circ}$
 (HWHM) in $2 \theta$ or $\Delta q = 5 \times 10^{-3}$\AA$ ^{-1}$.  
The maximum exposure time on one spot was about 4 hours with X--ray 
power of 5.25 kW; no changes in the diffraction peaks were detected
in this time. In addition, some film data were collected in 17 hours using a 
Philips Norelco X-ray source at 0.2 kW.  The beam was pinhole collimated with 
a sample-to-film distance of 66 mm, and Kodak DEF5 film was used.

%---------------------- Results -----------------------------------------
\section{Results}
 
Figure 1 shows wide angle scattering data at 25$^{\circ}$C for different chain
lengths.  For the shorter chain lengths (n$\leq$20) there is a sharp (20) peak 
and a broad (11) peak; this pattern is well understood to be due to the 
following properties of the chain packing \cite{Sun94,HenHos83,HenRus91,Smi88}:
(1) chains are packed in a centered rectangular (often called
orthorhombic or distorted hexagonal) lattice; 
(2) chains are tilted towards nearest neighbors;
(3) chains of the two opposing monolayers are parallel to each other, not 
pleated against each other;
(4) and chains in different bilayers are not in registry.  As n increases,
the angular separation between the (20) and (11) peaks increases.
For n=24 there are four peaks in Fig. 1; these will be
labeled A through D.  For n=22 there is a small peak that corresponds to
peak B for n=24; under different conditions there is also a clearly
visible peak for n=22 that corresponds to the D peak
for n=24, but it is too small to be seen in the data in Fig. 1.
Following the systematic trend of the shorter chain
lengths, Fig. 1 suggests that A and C are the (20) and (11) peaks from the 
usual chain tilted gel phase as in the shorter chain length lipids.

Figure 2 shows the d-spacings $d_{20}$ from peak A and $d_{11}$ from peak
C as a function of T.
The systematic trend with chain length n of these d-spacing curves supports 
the suggestion that the A and
C peaks for n=22 and n=24 correspond to the usual gel phase.  
Quantitative analysis of the wide angle spacings and additional aspects 
of this usual gel phase as a function of n and T
will be discussed in detail in another paper.  

Let us now turn to a more quantitative characterization of the new
B and D peaks shown in Fig. 1 and let us focus upon C$_{24}$ for which the 
effect is more pronounced.  Figure 3 illustrates the variety of 
wide angle scattering that can be obtained from nominally similar samples.
Traces 1 and 2 in Fig. 3 were measured on the same sample,
with the difference that the sample was heated to 80$^{\circ}$C for
4 hours between the two measurements. Trace 3 in Fig. 3 is from a sample that 
had been prepared by hydrating the lipid in the capillary and then annealing 
between 5$^{\circ}$C and 90$^{\circ}$C several times. These latter data show 
almost pure B and D peaks, but when this sample was heated to and held at 
80$^{\circ}$C and then brought back to 25$^{\circ}$C, its wide angle 
scattering more closely resembled trace 2 in Fig. 3.  This suggests that 
trace 2 may be the more stable result.  Most importantly, the data in Fig. 3 
clearly show that the sizes of peaks B and D are proportional to each other 
and the sizes of peaks A and C are proportional to each other. However, the 
relative sizes of peaks A and B are clearly highly variable.
This supports the identification of peaks A and C belonging to the usual gel 
phase that we will call G1 and peaks B and D belonging to a new phase that 
we will call G2.

Further support for identifying two coexisting phases comes from the 
temperature dependence of the wide angle scattering in Fig. 4 which shows
larger B and D peaks and smaller A and C peaks at the lower temperatures.
As the temperature is increased, the ratio of the A and C peaks relative
to the B and D peaks increases until there are only A and C peaks at 
40$^{\circ}$C. All of these data were obtained from samples that had been 
raised to 80$^{\circ}$C in the capillary, similar to trace 2 in Fig. 3.  
We note the appearance of an additional peak E shown in the data at 
35$^{\circ}$C in Fig. 4.  The size of this E peak was proportional to the 
size of the A peak and decreased as the size of the B peak increased, although 
peak E was not always present when the sample was freshly prepared.
When the temperature was only varied in the range below 45$^{\circ}$C, the 
magnitudes of the wide angle peaks were not so irreversible as in Fig. 3, and 
the irreversibility went in the opposite direction.  Specifically, we observed 
a small increase in the A and C peaks relative to the B and D peaks 
(data not shown) (i) when a sample was maintained at 35$^{\circ}$C after being 
heated to 80$^{\circ}$C and (ii) when a sample was cycled from 25$^{\circ}$C to 
40$^{\circ}$C and back to 25$^{\circ}$C.  However, we observed no change with
time at 25$^{\circ}$C, even when the total elapsed time was 33 hours.

The A and C peaks have been identified as the (20) and (11)
peaks of the usual tilted-chain gel phase G1.  Since the phase associated with
the B and D peaks becomes more prominent at lower temperature, one might
guess that it is a subgel phase.  Subgel phases have several
additional weak diffraction rings at angles between 7 and 17 degrees
\cite{RuoS82A,RuoS82B,Stum83,STN94}; 
these extra rings could be due to larger unit cells as well as 
to symmetry breaking between different chains in each unit cell 
so that, unlike the usual gel phase, peaks with an odd sum of the two 
in-plane indices would not have to be extinct. Figure 5 shows film data 
for a sample with relatively stronger BD peaks compared to the AC peaks.
This film does not show any weak rings between the low angle lamellar rings
and the ABCD rings, suggesting that the BD phase is not a subgel phase.
However, two extra very weak rings were observed at $2\theta = 36.3^{\circ}$ 
and 40.9$^{\circ}$ in Fig. 5.  These extra rings were weaker for samples with
weaker BD peaks compared to the AC peaks (data not shown).
Assuming an orthorhombic lattice and the more intense ring B as (11) plus 
(1$\bar{1}$) and the less intense ring D as (20) yields lattice constants 
a= 7.45\AA\ and b= 4.97\AA ; then, the two extra weak rings nicely index 
as (31) and (02).  The non-appearance of 
any rings with odd sums of indices suggests that the chains in the unit cell 
are still basically equivalent, as in the usual gel phase,
so we classify the G2 phase as a second gel-like phase.

The sharpness of the B and D peaks suggests that the chains have little 
tilt relative to the bilayer normal \cite{Smi88,McI80}. The G2 phase has a
more compact chain packing than the usual G1 gel phase;
the area in the plane perpendicular to the
chains is $A_c = 18.5$ \AA$^2$ at 25$^{\circ}$C, smaller
than for the usual gel phase which has $A_c = 19.3$ \AA$^2$ at 25$^{\circ}$C.
A notable difference between the G2 phase and the G1 phase is that the
order of the (20) and (11) peaks is reversed.  This means that, when
compared to hexagonal packing of chains, the chains in the
G2 phase are more compressed along the (10) direction (i.e., along a 
direction between two neighboring chains) whereas the chains in the usual gel 
phase are more compressed along the (01) direction (i.e., along a 
direction towards a neighboring chain).

It has been emphasized that there is a great deal of thermal hysteresis
and sample preparation dependence (Fig. 3) in the scattering data
taken below 40$^{\circ}$C. We will also see additional complications above 
45$^{\circ}$C. In the midst of these complications, it is fortunate 
that there appears to be reproducibility for temperatures between 40 and 
45$^{\circ}$C, for which we always obtained a rather normal gel phase pattern 
with peaks A and C. This suggests that it is appropriate to split the study of 
C$_{24}$ into two temperatures regimes. We now turn to the high temperature 
regime, T$ > 45^{\circ}$C.  

The wide angle data of the high temperature regime are shown in Fig. 6.  
At 85$^{\circ}$C this sample of C$_{24}$ enters the fluid L$_\alpha$ phase 
with only broad wide angle scattering centered below 20$^{\circ}$.  For lower
temperatures, it first appears that the broad C peak that occurs at 
40$^{\circ}$C in the G1 phase gradually sharpens as T is increased to 
80$^{\circ}$C.  However, detailed peak fitting reveals a different 
interpretation 
that is shown in Fig. 7.  Just as for the shorter chain lengths, the broad C
peak moves close to the sharp A peak with increasing temperature,
and the C peak remains broad as for the shorter chain lengths.
In addition, a new sharp peak that we will call the F peak grows in 
with increasing temperature as indicated in Fig. 6, suggesting that another 
new phase coexists with the usual gel phase in the high temperature region.  
We will tentatively name this
the G3 phase.  The appearance of only one new peak in 
the wide angle scattering requires that the G3 phase has 
hexagonal chain packing.  The HWHM of the F peak goes from about twice
the instrumental resolution at 65$^{\circ}$C to resolution limited at
higher temperatures, and this requires that the chains have little tilt 
($< 11^{\circ}$) with respect to the bilayer normal at the highest temperatures.
Surprisingly, the chain packing density of the G3
phase is greater than the usual gel phase; at 75$^{\circ}$C,
this new phase has an area per chain perpendicular to the chains 
$A_c = 20.1$\AA$^2$, compared to 20.7\AA$^2$ for the usual G1 gel phase,
using the formula for $A_c$ as in \cite{STN92,Sun94}.

A time and temperature sequence of low angle scattering is shown in Fig. 8.
The peak near $2 \theta = 4.5^{\circ}$ is the highest observable order and
it appears to be the fourth order of the lamellar stacking.
Surprisingly, for most temperatures this fourth peak is larger
than the second and third peaks and is often even better defined 
than the first peak.  Because the fourth peak persists through all the
temperature ranges, we suggest that it is associated with the G1 phase.
This conclusion is further supported by the linear temperature dependence
of the lamellar D-spacing (data not shown) that are calculated assuming the
fourth peak to be the h=4 order peak for lamellar stacking.
After the sample had been heated to 80$^{\circ}$C and then returned to 
35$^{\circ}$C, trace 7 in Fig. 8 was promptly taken;
the wide angle data (not shown) had larger B and D peaks and smaller A and C 
peaks than those wide angle data that corresponded to the first 
35$^{\circ}$C data (trace 2 in Fig. 8).

Trace 8 in Fig. 8 was also obtained at 35$^{\circ}$C but after waiting 20 hours 
after trace 7 was taken; the wide angle scan (data not shown) had smaller B 
and D peaks and larger A and C peaks than the wide angle scans for trace 7 and
a similar wide angle peak pattern to the one corresponding to trace 2.
The four low angle peaks in trace 8 of Fig. 8 index as lamellar orders 1-4 
rather well.  Trace 8 in Fig. 8 at 35$^{\circ}$C is very similar to the trace 
at 45$^{\circ}$C 
where there is only the G1 phase as determined by the wide angle scattering.
The G1 electron density profile (not shown) obtained from trace 8 by standard 
methods \cite{McI89}, with phases $(--+-)$, is typical of gel phases.

In contrast to trace 8 of Fig. 8, the behavior of the first three peaks
is more complex for some of the other traces.
The broad, low second and third peaks in trace 7 in Fig. 8 
do not index with the fourth peak or with the first peak and especially
not with the shoulder on the first peak, which is the extra part of that
peak compared to trace 8.  In the high temperature regime the low angle data 
also show broadening and shifting of the second and third peaks
(trace 6 in Fig. 8), accompanied by the onset and growth of the wide angle F 
peak in
Fig.6.  At 75$^{\circ}$C, the third peak gives a D-spacing of about 86\AA\ 
if one were to assume that it is the h=3 order, but this is inconsistent with
the D-spacing of 80\AA\ that is obtained from the fourth peak assuming that it 
is the h=4 order.  The widths of the second and third peaks are about twice as
broad as the instrumental resolution, while the fourth peak is resolution 
limited. 

To study thermal reversibility of the G3 phase which gives
rise to the F peak, the temperature of the sample was raised from 
65$^{\circ}$C to 80$^{\circ}$C, then lowered back to 65$^{\circ}$C, with 
the results shown in Fig. 9.   Trace 1 in Fig. 9 (65$^{\circ}$C) shows an 
initially
small F peak compared to the A peak, signifying that the C$_{24}$ sample is 
mainly in the usual G1 gel phase. Upon raising the temperature to 80$^{\circ}$C,
the AC peaks disappear and the F peak grows, indicating that the sample is 
then purely in the F phase. This transition is not immediately reversible as 
seen in traces 4 to 6 in Fig. 9.  Although the AC peaks grow back upon 
reducing the temperature, they remain smaller than the F peak at 65$^{\circ}$C,
in contrast to trace 1, even after waiting nearly a day at 65$^{\circ}$C.
The dramatic differences between the data in Fig. 6 and in Fig. 9 are also
consistent with having non-equilibrium in the high temperature regime.
It should also be mentioned that we found that long X-ray exposure of the
lipid at high temperature induced phospholipid breakdown as determined by 
thin layer chromatography (TLC), and so efforts were made to minimize such 
exposure.  TLC on the sample in Fig. 9 showed 1\% lysolecithin compared to a 
standard curve.

%---------------------- Discussion -------------------------------------
\section{Discussion}

Compared to chain lengths n smaller than 20,
there appear to be two distinct novel phenomena occurring in the 
X--ray diffraction data from saturated longer chain phosphatidylcholines, in
particular C$_{24}$.  The first novel phenomenon is the appearance of extra 
scattering peaks, B, D and E at low temperature. 
The second is the appearance of an additional scattering peak F above 
45$^{\circ}$C. Only for temperatures 40-45$^{\circ }$C does C$_{24}$ have a 
reproducible, normal chain-tilted gel phase with only peaks A and C in the wide 
angle region.

In the low temperature regime, T $<$ 40$^{\circ }$C,
the intensities of the two extra peaks (B and D in Fig. 3) decrease
in the same proportion, when temperature or sample history are varied, 
but the ratio of the A to B peaks varies widely.  Furthermore, compared to the
changes in relative intensities of the A and B peaks, there is relatively 
little shift in each peak position with temperature (Fig. 4) compared to
the separation between the peaks.  This behavior strongly suggests the
coexistence of two phases. 

There is, however, a major thermodynamic difficulty with the explanation of the
data in terms of coexisting G1 and G2 phases in the low temperature
regime.  According to Gibbs' phase rule, in the presence of an excess
water phase and at constant pressure, lipid bilayers composed of a single
lipid component may exhibit two phase coexistence only at a single temperature 
if the system is in equilibrium.  The apparent phase coexistence over
wide temperature ranges in Figs. 4 and 6 is only permitted if the system is 
not in equilibrium.  Of course, the existence of 
considerable hysteresis (compare traces 1 and 2 of Fig. 3) and
transient behavior (compare the bottom two traces of Figs. 8 and 9) imply
that the system is not in thermal equilibrium. This will be our tentative
rationale for not rejecting, on thermodynamic
grounds, the identification of the G2 and G3 phases that are only
observed in coexistence with the G1 phase over wide temperature ranges.
While the presence of non-equilibrium samples obviously
makes it impossible to determine
equilibrium phase diagrams, we nevertheless believe that the $C_{22}$ and
$C_{24}$ lecithins have two additional chain ordered phases, G2 and G3,
which will be discussed in detail after this paragraph.
The G1 phase that scatters into the A and C peaks would
appear to be an ordinary gel phase, whose wide angle d-spacings fit
smoothly with the gel phase for shorter chain lengths $n < 20$ 
(Fig. 2) and whose low angle intensities yield a typical gel phase electron
density profile.  The only anomaly associated with the G1 phase is the
appearance of the E peak near 35$^{\circ }$C (trace 4 in Fig. 4).

Scattering into the B and D wide angle peaks is interpreted as due to
the G2 phase. Both the B and D peaks are resolution limited at
25$^{\circ}$C.  Such narrow lines require that the tilt angle be
smaller than 11 degrees and that the intrinsic coherence length
be larger than 500\AA.  
The fact that the B peak is larger than the D peak can be explained as
an orthorhombic breaking of the hexagonal symmetry such that peak B is the 
sum of the
(11) and (1$\bar{1}$) peaks and peak D is the (20) peak.
This is easily accomplished by stretching the hexagonal chain packing lattice
along the (01) direction; this is perpendicular to the direction of
stretching for the G1 phase where the (20) peak appears at lower scattering 
angles than the (11) peak.  This difference may also be characterized by
different signs for the distortion parameter defined by Sirota et al.
 \cite{Sir93}. This indexing of the G2 phase wide angle peaks is consistent 
with the appearance 
of two additional weak wide angle peaks (Fig. 5) whose
intensities are proportional to the B and D peaks and that
index as (31) and (02).  The absence of any peaks between the low
angle peaks and the B peak and of any peaks whose sum of indices is odd
suggests that the G2 phase should be described as another gel phase rather
than a subgel phase.  This conclusion is also consistent with the
observations of Lewis et al. \cite{LMc87} that the incubation time for
initiating formation of subgel phases increases dramatically with chain length,
reaching the order of weeks for C$_{22}$.
The one disturbing aspect of this interpretation of
the wide angle scattering of the G2 phase is that the ratio of 
intensities of the B peak to the D peak is closer to four than to the value
of two that one would expect if the B peak is the doubly degenerate (11)
and (1$\bar{1}$) peaks.  As has been mentioned previously \cite{Sun94}, this
intensity could vary by 40\%, although this would require chain packing similar
to crystal or subgel phases.  Hopefully, other techniques, such as infrared
spectroscopy, can clarify these possibilities.

The area per chain A$_c$= 18.5\AA$^2$ of the G2 phase at
25$^{\circ}$C yields, when multiplied by 1.27\AA, the volume per
methylene V$_{CH_{2}}$ = 23.6\AA$^2$. These values are smaller than the 
corresponding
values for the C$_{24}$ gel phase G1 with A$_c$ = 19.3\AA$^2$\ and V$_c$ = 24.5
\AA$^2$\ also at 25$^{\circ}$C; this means that the G2 phase
is denser than the G1 phase, which is consistent with it becoming more
pronounced at lower temperatures.  The apparently small tilt angle in
the G2 phase requires, however, that the
area/headgroup $A = 2A_c = 37\text{\AA}^2$ be considerably smaller than for the 
G1 phase for which $A = 47-48\text{\AA}^2$ \cite{STN92}.  One possibility for 
alleviating headgroup crowding is to offset the headgroups along the bilayer
normal in an alternating pattern; this kind of pattern has been suggested
for the subgel phase of DPPC(C$_{16}$) \cite{RuoS82A}.  This kind of packing 
would presumably 
cost free energy in the headgroup region and gain free energy in the
hydrocarbon chain region.   The longer chains in C$_{24}$ would make this
option relatively more favorable than for shorter chains.
Another possibility to avoid headgroup crowding in the G2 phase
would be untilted interdigitated lipids because then $A$ would 
be essentially 4$A_c$, even larger than the $A = 2A_c$/cos$\theta_{t}$.  
Interdigitation would be expected to decrease the low angle lamellar 
d-spacing \cite{McI84}.  

The low angle pattern of the G2 phase is complicated by the history
dependence.  One possibility is that the first order low angle peak is the
shoulder at higher angle to the usual G1 first order peak (see trace 7 in
Fig. 8, which corresponds in the wide angle region to more G2 phase).  
This would be consistent with the G2 phase being interdigitated
with D spacing about 65\AA\ compared to a D spacing of about 80\AA\
for the G1 phase.  Comparison of trace 7 in Fig. 8 with trace 8, which
corresponds to more G1 phase, suggests that the G2 phase is associated with the 
broad second and third peaks.  Unfortunately,
these broad peaks do not index with the shoulder on the h=1 peak, nor do
they index with either the h=1 peak or with the well-defined h=4 peak that
appears to be
associated with the G1 phase. Therefore, the low angle pattern for the
G2 phase is unclear at present.  We have considered the possibility that the
G2 phase is not multilamellar, but hexagonal or cubic phases would seem to
be inconsistent with the sharp wide angle peaks B and D.  

In contrast to the uncertainty in the low angle pattern for the G2 phase, the
G1 phase appear be rather well characterized.  The G1 phase
corresponds to the four sharp peaks in trace 8 in
Fig. 8 which are similar to the four peaks at 45$^{\circ}$C in trace 3 in 
Fig. 8 for which the wide angle pattern has only the A and C peaks of the
G1 phase.  The electron density profile obtained from traces 3 or 8
of Fig. 8 indicates trans-bilayer headgroup spacings of roughly
58\AA.  This is consistent with $C_{24}$ chains tilted at about 36 degrees
as will be discussed in detail in another paper.
The electron density profile also indicates the usual terminal methyl trough 
in the center of the bilayer.

In the high temperature regime, T $>$ 45$^{\circ }$C, there appears to be a 
different new phase G3, identified by the F peak, that coexists with the usual 
G1 phase over a wide temperature range.  This coexistence has the same
thermodynamic difficulties as the G2/G1 phase coexistence.
We have also observed time dependence in the proportions of this coexistence
(see Fig. 9), so we believe that the coexistence is due to non-equilibrium and
not due to any fundamental difficulty with Gibbs' phase rule.

The G3 phase has only one rather symmetrical F peak in
the wide angle region.  This indicates that this
is a phase with basically hexagonal packing of the chains.
The relative sharpness of this peak suggests that the chains in this
phase have rather small tilt angles.  A consequence is that the
headgroup area A is nearly 2$A_c$ = 40.16\AA$^2$.  As with the G2 phase,
this value of A is rather small for lecithins.  A possible 
resolution of this problem is that the G3 phase might be a
ripple phase; if so, then the effective headgroup area is larger, being
proportional to the amplitude of the ripples and inversely
proportional to the ripple repeat distance.  
Support for the possibility that G3 is a ripple phase include: 
(1) G3 has hexagonal chain packing, as in the ripple phase of shorter chain 
length lipids \cite{HenRus91};
(2) the D-spacing measured from the second and third peaks at 75$^{\circ}$C
in Fig. 8 is 86\AA.  This is about 6\AA\ larger than the D spacing of 80\AA\ 
measured from the fourth peak that we interpret as the fourth order of the
the G1 phase.  This difference agrees with the D-spacing differences between 
the gel phase and the ripple phase of the shorter chain lengths; 
(3) The widths of the second and third peaks are about twice as broad as the
instrumental resolution, while the fourth peak is resolution limited.  Previous
ripple phase studies \cite{Wac89a} show that as the chain length increases,
and as full hydration is reached, the ripple scattering peaks weaken,
broaden, and intermingle with the lamellar peaks. When the resolution is
not high enough, one would expect to see broad peaks composed of ripple and 
lamellar peaks.

We have considered various explanations for the observed non-equilibrium
behavior.  High temperature annealing could induce chemical 
disintegration of the hydrated lipid.  However, this would be 
expected to inhibit
G2 phase formation because impurities usually retard formation of more ordered
phases, such as the subgel phase in DPPC \cite{STN94}, but high
temperature results in more G2 phase.
A different non-equilibrium hypothesis that we have explored involves
the possibility of some extra variable that makes the sample 
intrinsically heterogeneous.  A particular way to realize this
general hypothesis would be to have a distribution in the size $R$ of
the multilamellar vesicles and to postulate that larger vesicles have higher 
temperatures for G2--$>$G1 transitions.  There are two trends in our
observations that are consistent with this hypothesis.
First, we have noticed that there is usually more G1 phase (larger AC peaks) 
in the low temperature regime below 40$^{\circ}$C
upon first loading a sample into capillaries using a Hamilton syringe, where 
the shearing forces might break up large MLVs.  Second, raising the
temperature to 80$^{\circ}$C could anneal the sample to larger MLVs,
thereby explaining the differences between traces 1 and 2 in Fig. 3.
The occurrence of more G2 phase in trace 3 of Fig. 3 would be due
to formation of large MLVs in the capillary.
However, we also have data that do not fit neatly into this hypothesis,
so we will not claim to have established any comprehensive model for the 
non-equilibrium phase coexistence.  

Despite these non-equilibrium complications in preparations of
long chain lecithins, our data clearly indicate that they have a propensity
for forming phases in addition to the usual gel phase.  These new
phases could be interesting to elucidate new ways for lipids to
interact.  Additional experimental techniques should be employed to 
supplement the
X--ray data in this paper and to test our tentative suggestions for the
nature of the G2 and G3 phases.  Finally, recognizing the existence of these
anomalous phases is essential in order to complete future work on the
systematic temperature and chain length studies of the usual G1 gel phase of 
saturated lecithin bilayers.

%-------------------- Acknowledgments ---------------------------------
\section{Acknowledgments} 

We thank R. G. Snyder for informing us prior to publication
of infrared absorption data that also indicates new phase
behavior in these long chain lipids.  This research was supported by
the US National Institutes of Health grant GM-44976.

%------------------- References ---------------------------------------
\begin{thebibliography}{10}

\bibitem{NW78}
Nagle, J.~F. and Wilkinson, D.~A. (1978) Biophys. J. 23,  159--175.

\bibitem{LMc87}
Lewis, R.~H. A.~H.,  Mak, N.  and McElhaney, R.~N. (1987) Biochemistry 26,  
6118--6126.

\bibitem{Huang94}
Huang, C., Wang, Z.-Q., Lin, H.-N., Brumbaugh, E.~E. and Li, S. (1994) 
Biochim. Biophys. Acta 1189, 7--12.

\bibitem{Cev91}
Cevc, G. (1991) Biochim. Biophys. Acta 1062, 59--69.

\bibitem{WN84}
Wilkinson, D.~A. and Nagle, J.~F. (1984) Biochemistry 23, 1538--1541.

\bibitem{STN92}
Tristram-Nagle, S., Zhang, R., Suter, R.~M., Worthington, C.~R., Sun, W.-J. and
  Nagle, J.~F. (1993) Biophys. J. 64, 1097--1109.

\bibitem{Sun94}
Sun, W.-J., Suter, R.~M., Knewtson, M.~A., Worthington, C.~R., 
Tristram-Nagle, S., Zhang, R., and Nagle, J.~F. (1994) Phys. Rev. E 49, 
4665--4676.

\bibitem{HenHos83}
Hentschel, M. and Hosemann, R. (1983) Mol. Cryst. Liq. Cryst. 94, 291--316.

\bibitem{HenRus91}
Hentschel, M. and Rustichelli, F. (1991) Phys. Rev. Lett. 66, 903--906.

\bibitem{Smi88}
Smith, G.~S., Sirota, E.~B., Safinya, C.~R., and Clark, N.~A. (1988) 
Phys. Rev. Lett. 60,  813--816.

\bibitem{RuoS82A}
Ruocco, M.~J. and Shipley, G.~G. (1982) Biochim. Biophys. Acta 691, 309--320.

\bibitem{RuoS82B}
Ruocco, M.~J. and Shipley, G.~G. (1982) Biochim. Biophys. Acta 684,  59--66.

\bibitem{Stum83}
Stumpel, J., Eibl, H. and Nicksch, A. (1983) Biochim. Biophys. Acta 727, 
246--254.

\bibitem{STN94}
Tristram-Nagle, S., Suter, R.~M., Sun, W.-J. and Nagle, J.~F. (1994) Biochim. 
Biophys. Acta. 1191, 14--20.

\bibitem{McI80}
McIntosh, T.~J. (1980) Biophys. J. 29, 237--245. 

\bibitem{McI89}
McIntosh, T., Magid, A.~D. and Simon, S.~A. (1989) Biochemistry 28, 17--25.

\bibitem{Sir93}
Sirota, E.~B., King, H.~E., Singer, D.~M. and Shao, H.~H. (1993) J. Chem. 
Phys. 98, 5809--5824.

\bibitem{McI84}
McIntosh, T.~J., Simon, S.~A., Ellington, J.~C. and Porter, N.~A. (1984)
Biochemistry 23, 4038-4044.

\bibitem{Wac89a}
Wack, D.~C. and Webb, W.~W. (1989) Phys. Rev. A 40, 2712--2730.

\end{thebibliography}

%-------------------- Figure captions ------------------------------------%
\begin{figure}
\caption{Wide angle scattering data at $25^{\circ}$C for different chain 
lengths. The (20) and (11) peaks are present in the usual tilted gel phase. 
The four peaks present in $C_{24}$ are named A, B, C and D.
\label{fig1}}
\end{figure}

\begin{figure}
\caption{Temperature and chain length dependence of the usual
gel phase wide angle spacings d$_{11}$ and d$_{20}$ in {\AA}.
\label{fig2}}
\end{figure}

\begin{figure}
\caption{Wide angle scattering data from C$_{24}$ at 25$^{\circ}$C. 
Traces 1 and 2 are from the same sample, with the second trace taken after high
temperature annealing for 4 hours at 80C. The third trace is from a 
different sample (described in the text).
\label{fig3}}
\end{figure}

\begin{figure}
\caption{Wide angle data from C$_{24}$ in the low temperature regime. Peaks
A and C belong to the usual G1 phase, while B and D peaks belong to a new
phase, G2. The E peak is associated with the G1 phase.
\label{fig4}}
\end{figure}

\begin{figure}
\caption{Film data from C$_{24}$ at 22$^{\circ}$C. This sample had been
raised to 70$^{\circ}$C for one hour and allowed to cool slowly to room 
temperature before this picture. The four prominent wide angle rings shown in 
this film correspond to peaks A, B, C and D as described in the text. The 
shadow to the right of the beam was from scattering off the beam stop.
\label{fig5}}
\end{figure}

\begin{figure}
\caption{Wide angle data from C$_{24}$ in the high temperature regime. 
The F peak appears to be associated with a new phase, G3.
Each trace was taken two hours after the preceding one.
\label{fig6}}
\end{figure}

\begin{figure}
\caption{Decomposition of wide angle data (open circles) from C$_{24}$ 
at 75$^{\circ}$C into A, C and F components indicated by dashed lines.
The solid line is the sum of the three components.
\label{fig7}}
\end{figure}

\begin{figure}
\caption{Logarithm of low angle data from the same sample of C$_{24}$ in both
the low and high temperature regimes.  Data are shown in chronological order
beginning at the top.   Trace 8 was taken 20 hours after trace 7.  
\label{fig8}}
\end{figure}

\begin{figure}
\caption{Wide angle scattering showing the time and temperature dependence
of the A and F peaks for the same sample of C$_{24}$.  
Each trace was taken one hour after the preceding one with the exception
of the last trace.
\label{fig9}}
\end{figure}

\end{document}
