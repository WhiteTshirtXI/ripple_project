\section{Results}
 
Figure \ref{anol:fig1} shows wide angle scattering data at 25$^{\circ}$C 
for different chain
lengths.  For the shorter chain lengths (n$\leq$20) there is a sharp (20) 
peak and
a broad (11) peak; this pattern is well understood to be due to the following
properties of the chain packing \cite{Sun94,HenHos83,HenRus91,Smi88}:
(1) chains are packed in a centered rectangular (often called
orthorhombic or distorted hexagonal) lattice; 
(2) chains are tilted towards nearest neighbors;
(3) chains of the two opposing monolayers are parallel to each other, not 
pleated against each other;
(4) and chains in different bilayers are not in registry.  As n increases,
the angular separation between the (20) and (11) peaks increases.
For n=24 there are four peaks in Fig. \ref{anol:fig1}; these will be
labeled A through D.  For n=22 there is a small peak that corresponds to
peak B for n=24; under different conditions there is also a clearly
visible peak for n=22 that corresponds to the D peak
for n=24, but it is too small to be seen in the data in Fig. \ref{anol:fig1}.
Following the systematic trend of the shorter chain
lengths, Fig. \ref{anol:fig1} suggests that A and C are the (20) and (11) peaks from the 
usual chain tilted gel phase as in the shorter chain length lipids.

\begin{figure}[t]
\centerline {\psfig{figure=gel_anol_dir/lfig1.ps,width=4in}}
\caption{Wide angle scattering data at $25^{\circ}$C for different chain
lengths. The (20) and (11) peaks are present in the usual tilted gel phase. The
four
peaks present in $C_{24}$ are named A, B, C and D.
\label{anol:fig1}}
\end{figure}

%Figure \ref{anol:fig2} shows the d-spacings $d_{20}$ from 
%peak A and $d_{11}$ from peak
%C as a function of T.
%The systematic trend with chain length n of these d-spacing curves supports 
%the suggestion that the A and
%C peaks for n=22 and n=24 correspond to the usual gel phase.  
%Quantitative analysis of the wide angle spacings and additional aspects 
%of this usual gel phase as a function of n and T
%will be discussed in detail in Chapter \ref{gel_long_chap}.  

%\begin{figure}[t]
%\centerline {\psfig{figure=gel_anol_dir/d2011.ps,width=4in}}
%\caption{Temperature and chain length dependence of the usual
%gel phase wide angle spacings d$_{11}$ and d$_{20}$ in {\AA}.
%\label{anol:fig2}}
%\end{figure}

Let us now turn to a more quantitative characterization of the new
B and D peaks shown in Fig. \ref{anol:fig1} and let us focus upon C$_{24}$ 
for which the 
effect is more pronounced.  Figure \ref{anol:fig3} illustrates the variety of 
wide angle scattering that can be obtained from nominally similar samples.
Traces 1 and 2 in Fig. \ref{anol:fig3} were measured on the same sample,
with the difference that the sample was heated to 80$^{\circ}$C for
4 hours between the two measurements. Trace 3 in Fig. \ref{anol:fig3} is from a sample that 
had been prepared by hydrating the lipid in the capillary and then annealing 
between 5$^{\circ}$C and 90$^{\circ}$C several times. These latter data show 
almost pure B and D peaks, but when this sample was heated to and held at 
80$^{\circ}$C and then brought back to 25$^{\circ}$C, its wide angle 
scattering more closely resembled trace 2 in Fig. \ref{anol:fig3}.  This 
suggests that 
trace 2 may be the more stable result.  Most importantly, the data in 
Fig. \ref{anol:fig3} 
clearly show that the sizes of peaks B and D are proportional to each other 
and the sizes of peaks A and C are proportional to each other. However, the 
relative sizes of peaks A and B are clearly highly variable.
This supports the identification of peaks A and C belonging to the usual gel 
phase that we will call G1 and peaks B and D belonging to a new phase that 
we will call G2.

\begin{figure}[t]
\centerline {\psfig{figure=gel_anol_dir/lfig3.ps,width=4in}}
\caption{Wide angle scattering data from C$_{24}$ at 25$^{\circ}$C.
Traces 1 and 2 are from the same sample, with the second trace taken after high
temperature annealing for 4 hours at 80C. The third trace is from a
different sample (described in the text).
\label{anol:fig3}}
\end{figure}

Further support for identifying two coexisting phases comes from the 
temperature dependence of the wide angle scattering in 
Fig. \ref{anol:fig4} which shows
larger B and D peaks and smaller A and C peaks at the lower temperatures.
As the temperature is increased, the ratio of the A and C peaks relative
to the B and D peaks increases until there are only A and C peaks at 
40$^{\circ}$C. All of these data were obtained from samples that had been 
raised to 80$^{\circ}$C in the capillary, similar to trace 2 in 
Fig. \ref{anol:fig3}.  
We note the appearance of an additional peak E shown in the data at 
35$^{\circ}$C in Fig. \ref{anol:fig4}.  The size of this E peak was 
proportional to the 
size of the A peak and decreased as the size of the B peak increased, although 
peak E was not always present when the sample was freshly prepared.
When the temperature was only varied in the range below 45$^{\circ}$C, the 
magnitudes of the wide angle peaks were not so irreversible as in 
Fig. \ref{anol:fig3}, and 
the irreversibility went in the opposite direction.  Specifically, we observed 
a small increase in the A and C peaks relative to the B and D peaks 
(data not shown) (i) when a sample was maintained at 35$^{\circ}$C after being 
heated to 80$^{\circ}$C and (ii) when a sample was cycled from 25$^{\circ}$C to 
40$^{\circ}$C and back to 25$^{\circ}$C.  However, we observed no change with
time at 25$^{\circ}$C, even when the total elapsed time was 33 hours.

\begin{figure}[t]
\centerline {\psfig{figure=gel_anol_dir/lfig4.ps,width=4in}}
\caption{Wide angle data from C$_{24}$ in the low temperature regime. Peaks
A and C belong to the usual G1 phase, while B and D peaks belong to a new
phase, G2. The E peak is associated with the G1 phase.
\label{anol:fig4}}
\end{figure}

The A and C peaks have been identified as the (20) and (11)
peaks of the usual tilted-chain gel phase G1.  Since the phase associated with
the B and D peaks becomes more prominent at lower temperature, one might
guess that it is a subgel phase.  Subgel phases have several
additional weak diffraction rings at angles between 7 and 17 degrees
\cite{RuoS82A,RuoS82B,Stum83,STN94}; 
these extra rings could be due to larger unit cells as well as 
to symmetry breaking between different chains in each unit cell 
so that, unlike the usual gel phase, peaks with an odd sum of the two 
in-plane indices would not have to be extinct. Figure \ref{anol:fig5}
 shows film data 
for a sample with relatively stronger BD peaks compared to the AC peaks.
This film does not show any weak rings between the low angle lamellar rings
and the ABCD rings, suggesting that the BD phase is not a subgel phase.
However, two extra very weak rings were observed at $2\theta = 36.3^{\circ}$ 
and 40.9$^{\circ}$ in Fig. \ref{anol:fig5}.  These extra rings were 
weaker for samples with
weaker BD peaks compared to the AC peaks (data not shown).
Assuming an orthorhombic lattice and the more intense ring B as (11) plus 
(1$\bar{1}$)
and the less intense ring D as (20) yields lattice constants 
a= 7.45\AA\ and b= 4.97\AA ;
then, the two extra weak rings nicely index as (31) and (02).  
The non-appearance of 
any rings with odd sums of indices suggests that the chains in the unit cell 
are still basically equivalent, as in the usual gel phase,
so we classify the G2 phase as a second gel-like phase.

\begin{figure}[t]
\centerline {\psfig{figure=gel_anol_dir/film_data.ps,width=4in}}
\caption{Film data from C$_{24}$ at 22$^{\circ}$C. This sample had been
raised to 70$^{\circ}$C for one hour and allowed to cool slowly to room
temperature before this picture. The four prominent wide angle rings shown in
this film correspond to peaks A, B, C and D as described in the text. 
The shadow to the right of the beam was from scattering off the beam stop.
Data obtained by Dr. Tristram-Nagle.
\label{anol:fig5}}
\end{figure}

The sharpness of the B and D peaks suggests that the chains have little 
tilt relative to the bilayer normal \cite{Smi88,McI80}. The G2 phase has a
more compact chain packing than the usual G1 gel phase;
the area in the plane perpendicular to the
chains is $A_c = 18.5$ \AA$^2$ at 25$^{\circ}$C, smaller
than for the usual gel phase which has $A_c = 19.3$ \AA$^2$ at 25$^{\circ}$C.
A notable difference between the G2 phase and the G1 phase is that the
order of the (20) and (11) peaks is reversed.  This means that, when
compared to hexagonal packing of chains, the chains in the
G2 phase are more compressed along the (10) direction (i.e., along a 
direction between two neighboring chains) whereas the chains in the usual gel 
phase are more compressed along the (01) direction (i.e., along a 
direction towards a neighboring chain).

It has been emphasized that there is a great deal of thermal hysteresis
and sample preparation dependence (Fig. \ref{anol:fig3}) in the scattering data
taken below 40$^{\circ}$C. We will also see additional complications above 
45$^{\circ}$C. In the midst of these complications, it is fortunate 
that there appears to be reproducibility for temperatures between 40 and 
45$^{\circ}$C, for which we always obtained a rather normal gel phase pattern 
with peaks A and C. This suggests that it is appropriate to split the study of 
C$_{24}$ into two temperatures regimes. We now turn to the high temperature 
regime, T$ > 45^{\circ}$C.  

The wide angle data of the high temperature regime are shown in 
Fig. \ref{anol:fig6}.  
At 85$^{\circ}$C this sample of C$_{24}$ enters the fluid L$_\alpha$ 
phase with only
broad wide angle scattering centered below 20$^{\circ}$.  For lower
temperatures, it first appears that the broad C peak that occurs at 
40$^{\circ}$C
in the G1 phase gradually sharpens as T is increased to 80$^{\circ}$C.
However, detailed peak fitting reveals a different interpretation 
that is shown in Fig. \ref{anol:fig7}.  Just as for the shorter chain 
lengths, the broad C
peak moves close to the sharp A peak with increasing temperature,
and the C peak remains broad as for the shorter chain lengths.
In addition, a new sharp peak that we will call the F peak grows in 
with increasing temperature as indicated in Fig. \ref{anol:fig6}, 
suggesting that another 
new phase coexists with the usual gel phase in the high temperature region.  
We will tentatively name this
the G3 phase.  The appearance of only one new peak in 
the wide angle scattering requires that the G3 phase has 
hexagonal chain packing.  The HWHM of the F peak goes from about twice
the instrumental resolution at 65$^{\circ}$C to resolution limited at
higher temperatures, and this requires that the chains have little tilt 
($< 11^{\circ}$) with respect to the
bilayer normal at the highest temperatures.
Surprisingly, the chain packing density of the G3
phase is greater than the usual gel phase; at 75$^{\circ}$C,
this new phase has an area per chain perpendicular to the chains 
$A_c = 20.1$\AA$^2$, compared to 20.7\AA$^2$ for the usual G1 gel phase,
using the formula for $A_c$ as in \cite{STN92,Sun94}.

\begin{figure}[t]
\centerline {\psfig{figure=gel_anol_dir/anof6n.ps,width=4in}}
\caption{Wide angle data from C$_{24}$ in the high temperature regime.
The F peak appears to be associated with a new phase, G3.
Each trace was taken two hours after the preceding one.
\label{anol:fig6}}
\end{figure}

\begin{figure}[t]
\centerline {\psfig{figure=gel_anol_dir/lfig8b.ps,width=4in}}
\caption{Decomposition of wide angle data (open circles) from C$_{24}$
at 75$^{\circ}$C into A, C and F components indicated by dashed lines.
The solid line is the sum of the three components.
\label{anol:fig7}}
\end{figure}

A time and temperature sequence of low angle scattering is shown in 
Fig. \ref{anol:fig8}.
The peak near $2 \theta = 4.5^{\circ}$ is the highest observable order and
it appears to be the fourth order of the lamellar stacking.
Surprisingly, for most temperatures this fourth peak is larger
than the second and third peaks and is often even better defined 
than the first peak.  Because the fourth peak persists through all the
temperature ranges, we suggest that it is associated with the G1 phase.
This conclusion is further supported by the linear temperature dependence
of the lamellar D-spacing (data not shown) that are calculated assuming the
fourth peak to be the h=4 order peak for lamellar stacking.
After the sample had been heated to 80$^{\circ}$C and then returned to 
35$^{\circ}$C,
trace 7 in Fig. \ref{anol:fig8} was promptly taken;
the wide angle data (not shown) had larger B and D peaks and smaller A and C 
peaks than those wide angle data that corresponded to the first 
35$^{\circ}$C data (trace 2 in Fig. \ref{anol:fig8}).

\begin{figure}[t]
\centerline {\psfig{figure=gel_anol_dir/lfig6.ps,width=4in}}
\caption{Logarithm of low angle data from the same sample of C$_{24}$ in both
the low and high temperature regimes.  Data are shown in chronological order
beginning at the top.   Trace 8 was taken 20 hours after trace 7.
\label{anol:fig8}}
\end{figure}

Trace 8 in Fig. \ref{anol:fig8} was also obtained at 35$^{\circ}$C but after 
waiting 20 hours 
after trace 7 was taken; the wide angle scan (data not shown) had smaller B 
and D peaks and larger A and C peaks than the wide angle scans for trace 7 and
a similar wide angle peak pattern to the one corresponding to trace 2.
The four low angle peaks in trace 8 of Fig. \ref{anol:fig8} index as 
lamellar orders 1-4 rather well.  
Trace 8 in Fig. \ref{anol:fig8} at 35$^{\circ}$C is very similar to the 
trace at 45$^{\circ}$C 
where there is only the G1 phase as determined by the wide angle scattering.
The G1 electron density profile (not shown) obtained from trace 8 by standard 
methods \cite{McI89}, with phases $(--+-)$, is typical of gel phases.

In contrast to trace 8 of Fig. \ref{anol:fig8}, the behavior of the first three peaks
is more complex for some of the other traces.
The broad, low second and third peaks in trace 7 in Fig. \ref{anol:fig8} 
do not index with the fourth peak or with the first peak and especially
not with the shoulder on the first peak, which is the extra part of that
peak compared to trace 8.  In the high temperature regime the low angle data 
also show broadening and shifting of the second and third peaks
(trace 6 in Fig. \ref{anol:fig8}), accompanied by the onset and growth 
of the wide angle F peak in
Fig.6.  At 75$^{\circ}$C, the third peak gives a D-spacing of about 86\AA\ 
if one were to assume that it is the h=3 order, but this is inconsistent with
the D-spacing of 80\AA\ that is obtained from the fourth peak assuming that 
it is
the h=4 order.  The widths of the second and third peaks are about twice as
broad as the instrumental resolution, while the fourth peak is resolution 
limited. 

To study thermal reversibility of the G3 phase which gives
rise to the F peak, the temperature of the sample was raised from 
65$^{\circ}$C to 
80$^{\circ}$C, then lowered back to 65$^{\circ}$C, with the results
shown in Fig. \ref{anol:fig9}.   Trace 1 in Fig. \ref{anol:fig9} 
(65$^{\circ}$C) shows an initially
small F peak compared to the A peak, signifying that the C$_{24}$ sample is 
mainly in the usual G1 gel phase. Upon raising the temperature to 
80$^{\circ}$C,
the AC peaks disappear and the F peak grows, indicating that the sample is 
then purely in the F phase. This transition is not immediately reversible as 
seen in traces 4 to 6 in Fig. \ref{anol:fig9}.  
Although the AC peaks grow back upon 
reducing the temperature, they remain smaller than the F peak at 65$^{\circ}$C,
in contrast to trace 1, even after waiting nearly a day at 65$^{\circ}$C.
The dramatic differences between the data in Fig. \ref{anol:fig6} 
and in Fig. \ref{anol:fig9} are also
consistent with having non-equilibrium in the high temperature regime.

\begin{figure}[t]
\centerline {\psfig{figure=gel_anol_dir/c24rvw.ps,width=4in}}
\caption{Wide angle scattering showing the time and temperature dependence
of the A and F peaks for the same sample of C$_{24}$.
Each trace was taken one hour after the preceding one with the exception
of the last trace.
\label{anol:fig9}}
\end{figure}

All three phases, G1, G2 and G3, were observed for C$_{22}$. 
Figure \ref{c22data} shows the reversibility and stability study
of those phases. At 25$^{\circ}$C, peaks A, B, C and D can be clearly
seen from the top trace of Fig. \ref{c22data}(A) and signify the coexistence
of G1 and G2 phases. Traces 2 and 3 in Fig. \ref{c22data}(A) show peaks
A, C and the sharp F peak, thus demonstrating the coexistence of G1 and
G3 phases at 50$^{\circ}$C and 40 $^{\circ}$C. Then from the temperature 
range from 30$^{\circ}$C to 1.4$^{\circ}$C (traces 4--8 in 
Fig. \ref{c22data}(A)), the sample goes back to the regime of coexistence of
G1 and G2 phases. Comparing the two 10$^{\circ}$C traces separated by
13 hours, one sees that both G1 and G2 phases changed little within that
time interval. The bottom trace of Fig. \ref{c22data}(A) shows the sample
in the fluid phase, as signified by the absence of all A, B, C, D and F
peaks. Fig. \ref{c22data}(B) show the corresponding low angle scattering
data of Fig. \ref{c22data}(A). Just like  C$_{24}$, these low angle data
are not very informative at discerning the structures of G1, G2 and G3
phases, as can be seen by the similarities among all the traces in
Fig. \ref{c22data}(B) except the bottom fluid phase trace.
 

\begin{figure}[t]
\begin{center}
\leavevmode
\raggedleft 
%\hspace{0.2in}
\psfig{figure=gel_anol_dir/c222w.ps,width=2.6in}
\leavevmode
\raggedright 
\hspace{0.2in}
\psfig{figure=gel_anol_dir/c222l.ps,width=2.6in}
\end{center}
\vspace{-0.2in}
\hspace{1.2in} (A) \hspace{2.7in} (B)
\vspace{0.2in}
\caption{X-ray scattering data for C$_{22}$: (A) Wide angle data; (B) 
corresponding low angle data with logarithmic vertical axis. Data are 
shown in chronological order beginning at the top. Temperatures and 
time are shown to the right of the traces.
\label{c22data}}
\end{figure}

%It should also be mentioned that we found that long X-ray exposure of the
%lipid at high temperature induced phospholipid breakdown as determined by 
%thin layer
%chromatography (TLC), and so efforts were made to minimize such exposure.
%TLC on the sample in Fig. \ref{anol:fig9} showed 1\% lysolecithin 
%compared to a standard curve.
