\section{Discussion}

Compared to chain lengths smaller than 20,
there appear to be two distinct novel phenomena occurring in the 
X--ray diffraction data from saturated longer chain phosphatidylcholines, in
particular C$_{24}$.  The first novel phenomenon is the appearance of extra scattering 
peaks, B, D and E at low temperature. 
The second is the appearance of an additional scattering peak F above 
45$^{\circ}$C. Only for temperatures 40-45$^{\circ }$C does C$_{24}$ have a 
reproducible, normal chain-tilted gel phase with only peaks A and C in the wide 
angle region.

In the low temperature regime, T $<$ 40$^{\circ }$C,
the intensities of the two extra peaks (B and D in Fig. \ref{anol:fig3}) decrease
in the same proportion, when temperature or sample history are varied, 
but the ratio of the A to B peaks varies widely.  Furthermore, compared to the
changes in relative intensities of the A and B peaks, there is relatively 
little shift in each peak position with temperature (Fig. \ref{anol:fig4}) compared to
the separation between the peaks.  This behavior strongly suggests the
coexistence of two phases. 

There is, however, a major thermodynamic difficulty with the explanation of the
data in terms of coexisting G1 and G2 phases in the low temperature
regime.  According to Gibbs' phase rule, in the presence of an excess
water phase and at constant pressure, lipid bilayers composed of a single
lipid component may exhibit two phase coexistence only at a single temperature 
if the system is in equilibrium.  The apparent phase coexistence over
wide temperature ranges in Figs. 4.3 and 4.5 is only permitted if the 
system is not in equilibrium.  Of course, the existence of 
considerable hysteresis (compare traces 1 and 2 of Fig. \ref{anol:fig3}) and
transient behavior (compare the bottom two traces of Figs. 8 and 9) imply
that the system is not in thermal equilibrium. This will be our tentative
rationale for not rejecting, on thermodynamic
grounds, the identification of the G2 and G3 phases that are only
observed in coexistence with the G1 phase over wide temperature ranges.
While the presence of non-equilibrium samples obviously
makes it impossible to determine
equilibrium phase diagrams, we nevertheless believe that the $C_{22}$ and $C_{24}$
lecithins have two additional chain ordered phases, G2 and G3,
which will be discussed in detail after this paragraph.
The G1 phase that scatters into the A and C peaks would
appear to be an ordinary gel phase, whose wide angle d-spacings fit
smoothly with the gel phase for shorter chain lengths $n < 20$ 
%(Fig. \ref{anol:fig2}) 
and whose low angle intensities yield a typical gel phase electron
density profile.  The only anomaly associated with the G1 phase is the
appearance of the E peak near 35$^{\circ }$C (trace 4 in Fig. \ref{anol:fig4}).

%Scattering into the B and D wide angle peaks is interpreted as coming
%from the G2 phase.
%Although the B and D peaks are wider than resolution, they both have
%the same width and this is much narrower than the usual off-equator peaks that 
%correspond to chains tilted by ca. 30 degrees.
%The fact that both peaks have widths greater than instrumental resolution could
%be caused by small tilt angles in some non-nearest neighbor direction
%or it could be due to small intrinsic coherence lengths.  In either case, 
%the first characteristic of the G2 phase would appear to be that the tilt 
%angle is small.

Scattering into the B and D wide angle peaks is interpreted as due to
the G2 phase. Both the B and D peaks are resolution limited at
25$^{\circ}$C.  Such narrow lines require that the tilt angle be
smaller than 11 degrees and that the intrinsic coherence length
be larger than 500\AA.  
The fact that the B peak is larger than the D peak can be explained as
an orthorhombic breaking of the hexagonal symmetry such that peak B is the sum of the
(11) and (1$\bar{1}$) peaks and peak D is the (20) peak.
This is easily accomplished by stretching the hexagonal chain packing lattice
along the (01) direction; this is perpendicular to the direction of
stretching for the G1 phase where the (20) peak appears at lower scattering 
angles than the (11) peak.  This difference may also be characterized by
different signs for the distortion parameter defined by Sirota et al. \cite{Sir93}.
This indexing of the G2 phase wide angle peaks is consistent with the appearance 
of two additional weak wide angle peaks (Fig. \ref{anol:fig5}) whose
intensities are proportional to the B and D peaks and that
index as (31) and (02).  The absence of any peaks between the low
angle peaks and the B peak and of any peaks whose sum of indices is odd
suggests that the G2 phase should be described as another gel phase rather
than a subgel phase.  This conclusion is also consistent with the
observations of Lewis et al. \cite{LMc87} that the incubation time for
initiating formation of subgel phases increases dramatically with chain length,
reaching the order of weeks for C$_{22}$.
The one disturbing aspect of this interpretation of
the wide angle scattering of the G2 phase is that the ratio of 
intensities of the B peak to the D peak is closer to four than to the value
of two that one would expect if the B peak is the doubly degenerate (11)
and (1$\bar{1}$) peaks.  As has been mentioned previously \cite{Sun94}, this
intensity could vary by 40\%, although this would require chain packing similar
to crystal or subgel phases.  Hopefully, other techniques, such as infrared
spectroscopy, can clarify these possibilities.

%(Should compare to polyethylene crystals.)

The area per chain A$_c$= 18.5\AA$^2$ of the G2 phase at
25$^{\circ}$C yields, when multiplied by 1.27\AA, the volume per
methylene V$_{CH_{2}}$ = 23.6\AA$^2$. These values are smaller than the corresponding
values for the C$_{24}$ gel phase G1 with A$_c$ = 19.3\AA$^2$\ and V$_c$ = 24.5
\AA$^2$\ also at 25$^{\circ}$C; this means that the G2 phase
is denser than the G1 phase, which is consistent with it becoming more
pronounced at lower temperatures.  The apparently small tilt angle in
the G2 phase requires, however, that the
area/headgroup $A = 2A_c = 37\mbox{\AA}^2$ be considerably smaller than for the 
G1 phase for which $A = 47-48\mbox{\AA}^2$ \cite{STN92}.  One possibility for alleviating 
headgroup crowding is to offset the headgroups along the bilayer
normal in an alternating pattern; this kind of pattern has been suggested
for the subgel phase of DPPC(C$_{16}$) \cite{RuoS82A}.  This kind of packing would presumably 
cost free energy in the headgroup region and gain free energy in the
hydrocarbon chain region.   The longer chains in C$_{24}$ would make this
option relatively more favorable than for shorter chains.
Another possibility to avoid headgroup crowding in the G2 phase
would be untilted interdigitated lipids because then $A$ would 
be essentially 4$A_c$, even larger than the $A = 2A_c$/cos$\theta_{t}$.  
Interdigitation would be expected to decrease the low angle lamellar 
d-spacing \cite{McI84}.  

The low angle pattern of the G2 phase is complicated by the history
dependence.  One possibility is that the first order low angle peak is the
shoulder at higher angle to the usual G1 first order peak (see trace 7 in
Fig. \ref{anol:fig8}, which corresponds in the wide angle region to more G2 phase).  
This would be consistent with the G2 phase being interdigitated
with D spacing about 65\AA\ compared to a D spacing of about 80\AA\
for the G1 phase.  Comparison of trace 7 in Fig. \ref{anol:fig8} with trace 8, which
corresponds to more G1 phase, suggests that the G2 phase is associated with the 
broad second and third peaks.  Unfortunately,
these broad peaks do not index with the shoulder on the h=1 peak, nor do
they index with either the h=1 peak or with the well-defined h=4 peak that
appears to be
associated with the G1 phase. Therefore, the low angle pattern for the
G2 phase is unclear at present.  We have considered the possibility that the
G2 phase is not multilamellar, but hexagonal or cubic phases would seem to
be inconsistent with the sharp wide angle peaks B and D.  

In contrast to the uncertainty in the low angle pattern for the G2 phase, the
G1 phase appears be rather well characterized.  The G1 phase
corresponds to the four sharp peaks in trace 8 in
Fig. \ref{anol:fig8} which are similar to the four peaks at 45$^{\circ}$C in trace 3 in Fig. \ref{anol:fig8}
for which the wide angle pattern has only the A and C peaks of the
G1 phase.  The electron density profile obtained from traces 3 or 8
of Fig. \ref{anol:fig8} indicates trans-bilayer headgroup spacings of roughly
58\AA.  This is consistent with $C_{24}$ chains tilted at about 36 degrees
as will be discussed in detail in Chapter \ref{gel_long_chap}.
The electron density profile also indicates the usual terminal methyl trough in the 
center of the bilayer.

In the high temperature regime, T $>$ 45$^{\circ }$C, there appears to be a 
different new phase G3, identified by the F peak, that coexists with the usual 
G1 phase over a wide temperature range.  This coexistence has the same
thermodynamic difficulties as the G2/G1 phase coexistence.
We have also observed time dependence in the proportions of this coexistence
(see Fig. \ref{anol:fig9}), so we believe that the coexistence is due to non-equilibrium and
not due to any fundamental difficulty with Gibbs' phase rule.

The G3 phase has only one rather symmetrical F peak in
the wide angle region.  This indicates that this
is a phase with basically hexagonal packing of the chains.
The relative sharpness of this peak suggests that the chains in this
phase have rather small tilt angles.  A consequence is that the
headgroup area A is nearly 2$A_c$ = 40.16\AA$^2$.  As with the G2 phase,
this value of A is rather small for lecithins.  A possible 
resolution of this problem is that the G3 phase might be a
ripple phase; if so, then the effective headgroup area is larger, being
proportional to the amplitude of the ripples and inversely
proportional to the ripple repeat distance.  
Support for the possibility that G3 is a ripple phase include: 
(1) G3 has hexagonal chain packing, as in the ripple phase of shorter chain 
length lipids \cite{HenRus91};
(2) the D-spacing measured from the second and third peaks at 75$^{\circ}$C
in Fig. \ref{anol:fig8} is 86\AA.  This is about 6\AA\ larger than the D spacing of 80\AA\ 
measured from the fourth peak that we interpret as the fourth order of the
G1 phase.  This difference agrees with the D-spacing differences between 
the gel phase and the ripple phase of the shorter chain lengths; 
(3) The widths of the second and third peaks are about twice as broad as the
instrumental resolution, while the fourth peak is resolution limited.  Previous
ripple phase studies \cite{Wac89a} show that as the chain length increases,
and as full hydration is reached, the ripple scattering peaks weaken,
broaden, and intermingle with the lamellar peaks. When the resolution is
not high enough, one would expect to see broad peaks composed of ripple and lamellar peaks.

We have considered various explanations for the observed non-equilibrium
behavior.  High temperature annealing could induce chemical 
disintegration of the hydrated lipid.  However, this would be expected to inhibit
G2 phase formation because impurities usually retard formation of more ordered
phases, such as the subgel phase in DPPC \cite{STN94}, but high
temperature results in more G2 phase.
A different non-equilibrium hypothesis that we have explored involves
the possibility of some extra variable that makes the sample 
intrinsically heterogeneous.  A particular way to realize this
general hypothesis would be to have a distribution in the size $R$ of
the multilamellar vesicles and to postulate that larger vesicles have higher 
temperatures for G2--$>$G1 transitions.  There are two trends in our
observations that are consistent with this hypothesis.
First, we have noticed that there is usually more G1 phase (larger AC peaks) 
in the low temperature regime below 40$^{\circ}$C
upon first loading a sample into capillaries using a Hamilton syringe, where 
the shearing forces might break up large MLVs.  Second, raising the
temperature to 80$^{\circ}$C could anneal the sample to larger MLVs,
thereby explaining the differences between traces 1 and 2 in Fig. \ref{anol:fig3}.
The occurrence of more G2 phase in trace 3 of Fig. \ref{anol:fig3} would be due
to formation of large MLVs in the capillary.
However, we also have data that do not fit neatly into this hypothesis,
so we will not claim to have established any comprehensive model for the 
non-equilibrium phase coexistence.  

Despite these non-equilibrium complications in preparations of
long chain lecithins, our data clearly indicate that they have a propensity
for forming phases in addition to the usual gel phase.  These new
phases could be interesting to elucidate new ways for lipids to
interact.  Additional experimental techniques should be employed to supplement the
X--ray data in this chapter and to test our tentative suggestions for the
nature of the G2 and G3 phases.  Finally, recognizing the existence of these
anomalous phases is essential in order to complete future work on the
systematic temperature and chain length studies of the usual G1 gel phase of 
saturated lecithin bilayers.
