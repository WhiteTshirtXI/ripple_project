% ********************** abst.tex ***********************************
\prefacesection{Abstract}
This dissertation shows the results of x-ray scattering investigation on
the structures of the gel (L$_{\beta '}$) and ripple (P$_{\beta '}$)
phases.

Chapter \ref{intro_chap} is a general introduction to biomembranes,
lipid bilayers and x-ray scattering techniques. Chapter \ref{matmet_chap} 
shows the instrumental setups.

% ********************** gel_model.abs *******************************
In Chapter \ref{gel_model_chap}, we show
the wide angle results of the gel phase of fully hydrated unoriented 
MLVs (multilamellar vesicles) of DPPC (C16) that quantitate
two satellite peaks in addition to the usual (20) and (11) peaks.
All the peaks in the wide angle region are adequately fit using an
electron density model consisting of straight chains with a terminal methyl
gap and a head group term.  The fit yields chains of length close to 
20\AA \ that are tilted by ${\theta}_{t}=31.6^{\circ}$ toward nearest 
neighbors at $24^{\circ}$C. The fit also requires that the two monolayers 
in the bilayer are slightly offset rather than collinear.  Subtraction of the 
fitted peak scattering and
the background scattering obtained from samples with no
lipid indicates that there is considerable broad diffuse scattering 
underlying the prominent peaks, thus providing a measure of the disorder
in gel phase bilayers.  The result that the head group term required
to fit the wide angle peaks is smaller 
than the corresponding headgroup term required to fit the low angle
reflections suggests that there is less order in the headgroups than
in the chains.  However, the large amount of diffuse scattering appears
to require disorder in both the chain and head regions.

% *********************** gel_anoal.abs *******************************
Chapters \ref{gel_anol_chap} and \ref{gel_long_chap} show systematic 
low-angle and wide-angle x-ray scattering studies on fully hydrated 
unoriented MLVs of saturated lecithins with even chain 
lengths N = 16, 18, 20, 22 and 24 as
a function of temperature in the chain ordered phases. 
As reported in Chapter \ref{gel_anol_chap}, the longer chain lengths, 
$N \geq 20$, show anomalous behavior compared to the shorter chain lengths,
$N< 20$.  This chapter concentrates on $N = 24$.  Although the history
and time dependence shows that equilibrium was not always achieved, it
appears that there is a second gel-like phase G2
below 40$^{\circ}$C.  The G2 phase has a small tilt angle
and opposite hexagonal symmetry breaking from the usual G1 gel phase.
Also, as T is raised above 45$^{\circ}$C, the wide angle data suggest
the appearance of a phase with hexagonal chain packing and small chain 
tilt angle.

% ********************** gel_long.abs ***********************************
With Chapter \ref{gel_anol_chap} focusing on the anomalous phase behaviors
of longer chain lengthes, Chapter \ref{gel_long_chap} shows the 
temperature and chain length dependence of the normal gel phase. For all 
N the area per chain A$_{\mbox c}$ increases linearly with T 
with an average slope \(dA_c/dT=0.027{\mbox \AA}^2/^{\circ}{\mbox C}\),
and the lamellar D-spacings also increase linearly 
with an average slope \(dD/dT=0.040{\mbox \AA}/^{\circ}{\mbox C}\).
At the same T, longer chain length lecithins have more densely packed 
chains, i.e. 
smaller A$_{\mbox c}$'s, than shorter chain lengths. 
The chain packing of longer chain lengths is found to be more distorted 
from hexagonal packing than that of smaller N, and the 
distortion $\epsilon$ of all N approaches the same value at the 
respective transition temperatures. The thermal volume expansion of 
these lipids 
is accounted for by the expansion in the hydrocarbon chain region. 
Electron density profiles are constructed using four orders of low angle 
lamellar peaks.  These show that most of the increase in D with increasing
T is due to thickening of the bilayers that is consistent with a
decrease in tilt angle $\theta$ and with little change in water spacing
with either T or N.  Due to the opposing effects of
temperature on area per chain A$_c$ and tilt angle
$\theta$, the area expansivity $\alpha_A$ is quite small.
A qualitative theoretical model based on competing head and chain 
interactions accounts for our results.

%************************* ripple abstract *****************************
Chapter \ref{rppl_chap} shows the study on the structure of the ripple
(P$_{\beta '}$) phase. The phases of the x-ray form factors are 
derived for this thermodynamic phase in the lecithin bilayer system. 
By combining these phases with experimental intensity data, the electron 
density map of the ripple phase of DMPC is constructed. The phases are 
obtained by fitting the intensity data to 2D electron density models, 
which are created
by convolving an asymmetric triangular ripple profile with a transbilayer 
electron density profile.  The robustness of the model method
is indicated by the result that many different models of the
transbilayer profile yield essentially the same phases, except
for the weaker, purely ripple (0,k) peaks.  Even with this
residual ambiguity, the ripple profile is well determined, resulting in
19\AA\ for the ripple amplitude and 10$^{\circ}$ and 26$^{\circ}$ 
for the slopes of the major and the minor sides,
respectively.   Estimates for the bilayer head-head spacings
show that the major side of the ripple is consistent with gel-like structure
and the minor side appears to be thinner with lower electron density.
