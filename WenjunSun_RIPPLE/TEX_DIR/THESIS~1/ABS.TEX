% ********************** abst.tex ***********************************
\prefacesection{Abstract}
X-ray scattering is employed to elucidate the structures of the 
chain-ordered lipid bilayers.
Chapter \ref{intro_chap} is a general introduction to biomembranes,
lipid bilayers and x-ray scattering techniques. Chapter \ref{matmet_chap} 
describes the sample preparation and the instrumental setups.
% ********************** gel_model.abs *******************************
Chapter \ref{gel_model_chap} reports the wide-angle results of the gel phase 
of fully hydrated unoriented MLVs (Multilamellar Vesicles) of DPPC 
(Dipalmtoylphosphatidylcholine) that quantitate two satellite 
peaks in addition to the usual (20) and (11) peaks. All the peaks in the 
wide-angle region are adequately fit using an electron density model 
consisting of straight chains with a terminal methyl gap and a head group 
term.  The fit yields chains of length close to 20\AA \ that are tilted by 
${\theta}_{t}=31.6^{\circ}$ toward nearest neighbors at $24^{\circ}$C. 
Subtraction of the fitted peak scattering and the background scattering 
obtained from samples with no lipid indicates that there is considerable 
broad diffuse scattering underlying the prominent peaks, thus providing a 
measure of the disorder in gel phase bilayers. The large amount of diffuse 
scattering appears to require disorder in both the chain and head regions.
% *********************** gel_anoal.abs *******************************
Chapters \ref{gel_anol_chap} and \ref{gel_long_chap} show systematic 
low-angle and wide-angle x-ray scattering studies on fully hydrated 
unoriented MLVs of saturated lecithins with even 
chain lengths N = 16, 
18, 20, 22 and 24 as a function of temperature in the chain ordered phases. 
Chapter \ref{gel_anol_chap} reports that the longer chain lengths
$N \geq 20$, show anomalous behavior compared to the shorter chain lengths,
$N< 20$.  This chapter concentrates on $N = 24$.  Although the history
and time dependence shows that equilibrium was not always achieved, it
appears that there is a second gel-like phase G2 below 40$^{\circ}$C. The 
G2 phase has a small tilt angle and opposite hexagonal symmetry breaking 
from the usual G1 gel phase. Also, as T is raised above 45$^{\circ}$C, the 
wide-angle data suggest the appearance of a phase G3 with hexagonal chain 
packing and small chain tilt angle.
% ********************** gel_long.abs ***********************************
Chapter \ref{gel_long_chap} presents the temperature and chain length 
dependence 
of the normal gel phase. The thermal volume expansion of these lipids is 
accounted for by the expansion in the hydrocarbon chain region. Electron 
density profiles are constructed using four orders of low angle lamellar 
peaks.  These show that most of the increase in D with increasing T is due 
to thickening of the bilayers that is consistent with a decrease in tilt 
angle $\theta$ and with little change in water spacing with either T or N.  
Due to the opposing effects of temperature on area per chain A$_c$ and tilt 
angle $\theta$, the area expansivity $\alpha_A$ is quite small. A qualitative 
theoretical model based on competing head and chain interactions accounts for 
our results.
%************************* ripple abstract *****************************
Chapter \ref{rppl_chap} focusses on the structure of the ripple
(P$_{\beta '}$) phase. The phases of the x-ray form factors are derived for 
this thermodynamic phase in the lecithin bilayer system. By combining these 
phases with experimental intensity data, the electron density map of the 
ripple phase of DMPC is constructed. The phases are obtained by fitting the 
intensity data to 2D electron density models, which are created by convolving 
an asymmetric triangular ripple profile with a transbilayer electron density 
profile. The ripple profile is well determined, resulting in 19\AA\ for the 
ripple amplitude and 10$^{\circ}$ and 26$^{\circ}$ for the slopes of the major 
and the minor sides, respectively. Estimates for the bilayer head-head 
spacings show that the major side of the ripple is consistent with gel-like 
structure and the minor side appears to be thinner with lower electron density.
