\section{X-ray scattering data}
\label{gel_model_data}

\begin{figure}[ht]
\centerline {\psfig{figure=gel_model_dir/shdpt.ps,width=4in}}
\caption{Wide-angle scattering intensity versus scattering angle
2$\theta$ (open circles)
taken with graphite monochromator and PSD 572mm from the sample.
Slits were configured to yield resolution $0.068^{\circ}$ HWHM, which is the
width of the unresolved (20) peak.  The area in black is interpreted
as the peak scattering.  The temperature was $24^{\circ}$C.
At $2 \theta = 22.3^{\circ}$ the solid square shows the scattering
from a capillary filled with water and the solid triangle shows the
scattering from a capillary in air.
\label{model:fig1}}
\end{figure}

Our best resolved data over a broad range, shown in Fig.\ \ref{model:fig1}, 
were obtained at T = $24^{\circ}$C with the rotating anode source using 
the PSD. The sharp peak is usually identified 
as the (20) peak; the use of only two Miller indices indicates that this
is scattering along a Bragg rod due to a basically two-dimensional system
rather than scattering from a Bragg peak due to three dimensional order,
as will be elaborated in Sec. \ref{gel_model_model}.
The (20) peak was always centered very close to $20.9^{\circ}$,
corresponding to $d_{20}$ = 4.25\AA . In these data, the (20) peak has HWHM of 
$0.06^{\circ}$, which is close to our estimated instrumental resolution.  
Data that focussed on the (20) peak were also obtained at CHESS with resolution
$0.004^{\circ}$ HWHM.  These data (not shown) yield an intrinsic linewidth of 
$\Delta q$ that corresponds to $0.015^{\circ}$ HWHM, indicating that order in the 
(20) direction within the membrane plane persists over length scales greater
than 2900\AA = $2{\pi}/\Delta q$.  A much broader peak occurs with a maximum 
near 2$\theta$ equal to $21.3^{\circ}$, although for one of our three samples this peak
had moved to about $21.5^{\circ}$.  This is usually identified as the (11) peak.

Fig.\ \ref{model:fig1} shows two smaller peaks in the 2$\theta$ range of $20.5^{\circ}$ to 
$18.5^{\circ}$.
Below $18.7^{\circ}$ the scattering is smooth and gradually decreasing (data only
shown to $18.0^{\circ}$).
The two smaller peaks will be called ``satellites'' because they are due to
additional scattering from the (11) Bragg rod.
These satellites were first reported by Mitsui \cite{Mit78} and they can be discerned
in other data \cite{Sta82} but they have usually been ignored in data
from unoriented samples.
Although the signal to noise is not high for such small features,
we have consistently seen these two satellites at the same angles for
different data sets and different samples.  It should also be noted that the
two satellites appear to be sharper with more prominent maxima when observed on
film or when the film is densitometered and the densitometer
data are viewed on a computer screen.  However, this sharpening is an optical
illusion since numerical plots of the digitized densitometer data are very similar
to the data shown in Fig.\ \ref{model:fig1}.  

Background scattering arises from the glass capillary, air and the water.
The background scattering curve shown in Fig.\ \ref{model:fig1} was obtained
from samples consisting of only water in the capillary and of only
air in the capillary; data points at one angle are shown in 
Fig. \ \ref{model:fig1}.
Since the ratio of water to lipid in the lipid samples
was about 3 to 1, the true background was chosen to lie 3/4 of the way from
the air+capillary data to the water+capillary data.
Care was taken to normalize counting times and beam intensity for the control
samples.
The background scattering taken with the PSD and shown in Fig.\ \ref{model:fig1} are in
satisfactory agreement with background scattering obtained using a graphite
analyzer crystal (data not shown).
The difference between the total scattering and the background scattering
becomes practically zero between $25^{\circ}$ and $30^{\circ}$. It then
increases between $35^{\circ}$ and $40^{\circ}$, though 
it remains less than 20\% as large as the difference
near $21^{\circ}$.  The difference between total scattering and
the background scattering also continues to 
decrease for smaller $2 \theta$ than shown in Fig.\ \ref{model:fig1}.

\section{Strategy}
\label{gel_model_strategy}

The sharpness of the (20) peak suggests that there is order in these samples
that extends over rather long distances; 2900\AA\ is over 300 molecular nearest
neighbor distances.  This suggests that a reasonable
starting point is to model the system as a crystal.  As is already well-known,
however, the order that is reported by the wide angle data in Fig.\ \ref{model:fig1}
is not three-dimensional but consists of in-plane or two-dimensional
order \cite{LevTh,Tar73,Hui76,HenHos83,Smi88,STN92}.
Thus, the starting point will be a two-dimensional crystal model.
Strictly speaking, it is not possible to have true long range crystalline order
in two dimensions \cite{Mermin}.  However, if the decay distance of the short
range order is large compared to the inverse instrumental resolution width, 
then the scattering will be close \cite{Dutta} to that from 
two-dimensional crystal models, which therefore remain the
appropriate starting point.

The strategy we used was to start with the simplest
basic two-dimensional crystal model (details of the 2-d crystal models
will be given in the next section) to 
see how well it could account for the data in Fig.\ \ref{model:fig1}.
As is well-known \cite{LevTh,Tar73,Hui76,McI80} this model predicts a sharp (20) peak and a broad
(11) peak.  As is not as well-known, this model also predicts satellite
peaks.  This reinforces the hypothesis that the 2-d crystal model
is an appropriate starting point.  However, there were two major
quantitative discrepancies.  The first was that the exact positions of
the satellite peaks were not in good agreement with the data.  This is not surprising
since the detailed nature of lipid bilayers suggests many refinements
to the simplest 2-d crystal model.  As we will show in Section 
\ref{gel_model_fit}, our
refinements overcome this discrepancy completely.  Specifically, we are able
to obtain a model that gives that portion of the scattering indicated by
the black areas in Fig.\ \ref{model:fig1}.

The second major discrepancy was that there is a large amount of scattering 
that is not accounted for, either by the simplest crystal model, or by 
any refinement of the 2-d crystal model that we can imagine.  This scattering 
is indicated by the dark gray area in Fig.\ \ref{model:fig1}.  Because this 
scattering is 
very broad with no sharp peaks, we suggest that it be identified as
diffuse scattering.  

Formally, we are therefore dividing the total scattering I
(after background subtraction) into peak scattering I$_{peak}$ and
I$_{diffuse}$ where, as usual,

\begin{equation}
\label{nagle1}
I({\bf q}) = \int e^{i {\bf q} \cdot {\bf r}} <\rho ({\bf r}_{1}) 
\rho ({\bf r}_{2})> d {\bf r}_{1} d {\bf r}_{2}~,
\end{equation}
where \( {\bf r} = {\bf r}_{1} - {\bf r}_{2} \).  Writing

\begin{equation}
\label{nagle2}
<\rho ({\bf r}_{1}) \rho ({\bf r}_{2})> = <\rho ({\bf r}_{1})> <\rho ({\bf r}_{2})> + [<\rho ({\bf r}_{1}) \rho ({\bf r}_{2})> - <\rho ({\bf r}_{1})> <\rho ({\bf r}_{2})>]~, 
\end{equation}
where the term in square brackets is the pair correlation function 
\( C({\bf r}_{1},{\bf r}_{2}) \) (which can also be written as
\( <\delta ({\bf r}_{1}) \delta({\bf r}_{2})> \) 
where \( \delta({\bf r})=\rho({\bf r}) - <\rho({\bf r})>) \),
the total scattering I({\bf q}) can be written as a sum \cite{Zach} of I$_{peaks}$ 
and I$_{diffuse}$ where 
\begin{equation} 
\label{nagle3} 
I_{peaks} = | \int e^{i {\bf q} \cdot {\bf r}} <\rho({\bf r})> d{\bf r} |^{2}~,
\end{equation}
depends only upon the average structure $<\rho({\bf r})>$, 
which is assumed to be accurately described by a 2-d crystal model, and
\begin{equation}
\label{nagle4}
I_{diffuse} = \int d {\bf r}_{1} \int d {\bf r}_{2} e^{i {\bf q} \cdot {\bf r}} C({\bf r}_{1},{\bf r}_{2})~,
\end{equation}
which depends upon pair correlations which report fluctuational disorder in the
bilayer.  

The formalism in the preceding paragraph raises two issues.  
The first issue refers back to the first paragraph in this section:
the possibility that no formal separation between peak scattering (i.e. Bragg rod
scattering) and diffuse scattering can be performed because there are no
2-d crystals due to long wavelength phonons.  The consequences for scattering
have been shown \cite{Dutta} to be power law tails added on to basically
Gaussian peaks.  These tails are important aspects of partially ordered
low-dimensional physical systems.  However, such tail scattering due
to long wavelength phonons associated with the correlation length greater than
2900\AA\ would be expected to be highly concentrated near the Bragg rods and 
the total scattering in the tails would be expected to be small compared
to the scattering under the central peaks.  It would not appear that such
tails would be
able to account for the broad diffuse scattering indicated in Fig.\ \ref{model:fig1}.  
Instead, we suggest that most of the broad scattering is due to breakdown of order
at much shorter distances.  With molecules as complex as lipids,
some portions may be packed in arrays fairly well ordered over large
distances while other parts may be, at the same time, rather disordered.
This disorder would probably include
many other fluctuations that are not directly coupled to 
the (20) peak order, such as headgroup orientations and rotations 
of the molecules and/or the hydrocarbon chains.  This
picture leads us back to the separation formalism embodied in the preceding paragraph.

If we now accept the separation formalism, then the second issue is a practical one.
Where should one draw the curve between peak and diffuse scattering?
Clearly, this curve is arbitrary unless one has a model that will give
either the peak scattering or the diffuse scattering.  Our 
approach indicated above is to obtain the best peak scattering from a
2-d crystal model and to subtract it from the data to obtain the
diffuse scattering.  We turn first, in the next section, to a description of
the 2-d crystal model that we employed and then in Secs. \ref{gel_model_sate}
and \ref{gel_model_fit}
to the problem of obtaining the best values of the model parameters.
Then, the subtraction of the peak scattering can be performed to obtain
the diffuse scattering shown in Fig.\ \ref{model:fig1}.

