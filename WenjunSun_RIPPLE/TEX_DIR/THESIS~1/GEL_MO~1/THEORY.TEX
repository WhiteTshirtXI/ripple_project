\section{Model}
\label{gel_model_model}

The gel phase of lipid bilayers has been characterized for many years as
having conformationally ordered, nearly {\em all-trans} hydrocarbon chains
that pack into ordered arrays that give rise to the (20) and (11)
peaks in Fig.\ \ref{model:fig1} \cite{LevTh,Tar73}.
These ordered hydrocarbon chains may be tilted in various directions with respect 
to the nearly hexagonal lattice of packed chains \cite{Smi88}.  However, it 
has been known for a long time that the fully hydrated gel phase of DPPC is 
characterized by the hydrocarbon chains being tilted towards nearest neighboring 
chains \cite{LevTh}.  It has also been known that each bilayer scatters 
independently in the wide angle region due to lack of registry of individual 
molecules across the aqueous spaces \cite{LevTh,Tar73}.  Since our data will 
fully confirm these two results, the exposition of the theory will assume them.

\begin{figure}
\begin{center}
%\leavevmode
\raggedleft
\hspace{0.5in}
\psfig{figure=gel_model_dir/modela.ps,height=3.0in}
%\leavevmode
\raggedright
\hspace{1.2in}
\psfig{figure=gel_model_dir/modelb.ps,height=3.0in}
\end{center}
\begin{center}
\leavevmode
\psfig{figure=gel_model_dir/modelc.ps,width=2.5in}
\end{center}
\caption{Cross-sectional views of the model to fit the peak scattering.
(a) Projection onto the (x,z) plane.  The chains are represented
by long thin black rectangles and the head groups are represented by shorter
gray rectangles.  ${\Delta x}_{c}$ is the chain offset, ${\Delta x}_{h}$ is the
head offset, $Z_{h}$ is the peak position of the headgroup Gaussian and a is the
unit cell dimension in the x-direction.
(b) Projection onto the (y,z) plane showing the tilt angle ${\theta}_{t}$
and the length L of the chains.  ${\Delta y}_{c}$ is the chain offset,
${\Delta y}_{h}$ is the head offset and b is the unit cell
dimension in the y-direction.
(c) Projection onto the (x,y) midplane showing the body-centered
two-dimensional unit cell.  The filled (open) circles are the projections of
the hydrocarbon chains from the upper (lower) monolayer, respectively, to show
the offsets ${\Delta x}_{c}$ and ${\Delta y}_{c}$.  The scale for
this projection is
three times the scale for the preceding two projections.
Only the z-component of the headgroup electron density distribution is
represented, so this figure does not imply that the headgroups are
perpendicular to the bilayer.
\label{model:fig2}}
\end{figure}

Fig.\ \ref{model:fig2} provides three views of the model of the gel phase lipid bilayer
used in this chapter.  This model consists of chains and heads.
The chains are straight, thin rods of electron density that are tilted
by ${\theta}_{t}$.  Straight ({\it all-trans}) hydrocarbon chains have
methylenes ($CH_{2}$) spaced a distance 1.27\AA ~apart along the
chain axis \cite{Bunn}.  However, the terminal methyl ($CH_{3}$) occupies roughly 
twice as much volume as the methylenes \cite{NW88,WW5} while having one additional 
electron, so each chain is modelled with a gap of \( {7 \over 8} 1.27\)\AA~at 
the center of the bilayer. The length of each rod will be given as L.
There is no reason for precise collinearity of pairs of chains from the opposing
monolayers, so offsets, $\Delta x_{c}$ and $\Delta y_{c}$, will be allowed
as shown in Fig.\ \ref{model:fig2}.
Use of a body centered rectangular unit cell (often referred to as orthorhombic)
with sides $a$ and
$b$ then yields the following electron density function for the
chains for values of $|z|$ between \( L_{1}={7 \over 8} 1.27\mbox{\AA} \cos {\theta}_{t} 
\) and 
\( L_{2}={L \over \cos {\theta}_{t}} \),

\begin{equation}
\label{et1}
{\rho}_{c}(x,y,z) = {8 \over 1.27\mbox{\AA}~\cos {\theta}_{t}} {\rho}_{cr}(x,y,z)~,
\end{equation}

where for \( z > 0 \)

\begin{eqnarray*} {\rho}_{cr}(x,y,z) =&& \delta (x - {\Delta x_{c} \over 2},~ y - 
{\Delta y_{c} \over 2} - z 
\tan {\theta}_{t}) \\
&&+ \delta (x -{\Delta x_{c} \over 2} - {a \over 2},~ y - {\Delta y_{c} 
\over 2} - {b \over 2} - z \tan {\theta}_{t})~,
\end{eqnarray*}

and for \( z < 0 \)

\begin{eqnarray*} {\rho}_{cr}(x,y,z) =&& \delta (x + {\Delta x_{c} \over 2},~ y + 
{\Delta y_{c} \over 2} - z 
\tan {\theta}_{t}) \\
&& + \delta (x + {\Delta x_{c} \over 2} - {a \over 2},~ y + {\Delta y_{c} 
\over 2} - {b \over 2} - z \tan {\theta}_{t})~. 
\end{eqnarray*}

The electron density ${\rho}_{H}(x,y,z)$ for headgroups is represented by
Gaussians centered at $\pm z_{H}$ with widths ${\sigma}_{H}$ in the z-direction.
Other data \cite{Buldt79} indicate that the headgroups are oriented more parallel than
perpendicular to the bilayer, so the portrayal of the headgroups in Fig.\ \ref{model:fig2}
may be visually unsettling.  However, our data do not allow determination of the orientation
of the headgroups, i.e. the widths of the headgroups in the x and y directions.
We can only determine the z-direction distribution of electron density, so
only this is shown in Fig.\ \ref{model:fig2}, which therefore does not imply that the headgroups
are oriented perpendicular to the bilayer. 
Offsets, $\Delta x_{H}$ and $\Delta y_{H}$, are also allowed for the headgroups.
Headgroups are only placed on the half of the chains associated with the
corners of the rectangular unit cell.  Quantitatively, we take
\begin{eqnarray}
\label{et2}
{\rho}_{H} (x,y,z) = &&{n_{H} \over {\sigma}_{H} \sqrt{2 \pi}}~\{\exp[{- (z - z_{H})^{2} 
\over 2 {\sigma}_{H}^{2}}]~\delta (x - {\Delta x_{H} \over 2})~\delta (y - {\Delta y_{H} 
\over 2}) \nonumber \\ 
&& + \exp[{-(z + z_{H})^{2} \over 2 {\sigma}_{H}^{2}}]~\delta (x + {\Delta x_{H} 
\over 2})~\delta (y + {\Delta y_{H} \over 2})\}~.
\end{eqnarray}

Periodicity with spacing $a$ in the x direction and $b$ in the y direction
gives two Laue equations for x-ray scattering
\begin{equation}
\label{et3}
a q_{x} = 2 \pi N \mbox{ and } b q_{y} = 2 \pi M~,
\end{equation}
which define the Bragg rods. If there are two equivalent substituents per unit 
cell that form a rectangular body centered lattice, as in Eq.\ \ref{et1},
then N+M must be even and the two major wide angle peaks
are for (MN) equal to (20) and (11).  This even constraint could be
broken by inequivalent ordering of the chains as in polymethylene, but
even there the (20) and (11) peaks are by far the strongest peaks \cite{Bunn}.  
The N+M even constraint could also be broken by ordered headgroups as embodied
in Eq.\ \ref{et2} and Fig. \ref{model:fig2}; we will return to this 
possibility later.  For
now let us note that this makes no difference to the calculation of the
intensities along the (20) and (11) Bragg rods.

To obtain the x-ray scattering from the electron density, a final integration must 
be performed in the z-direction. Apart from an inconsequential factor
corresponding to the x and y integrations, this yields a form factor
amplitude $F(q_{z})$ along the $q_{z}$ direction for each Bragg rod,
\begin{equation}
\label{et4}
F(q_{z}) = \int_{0}^{\infty} {\rho}_{z}(z) e^{i (\gamma z + \phi)} dz + \int_{- \infty}^{0} 
{\rho}_{z} (z) e^{i (\gamma z - \phi)} dz~,
\end{equation}
where \( \gamma = q_{z}+q_{y} \tan \theta \) and \( \phi = {q_{x} \Delta x + q_{y} \Delta y 
\over 2} \) , with $q_{x}$ and $q_{y}$ given by Eq.\ \ref{et3}.
The integration in Eq.\ \ref{et4} is best performed separately for heads and
chains, each with their separate offsets, $\Delta x$ and $\Delta y$, with the result
\begin{eqnarray}
\label{et5}
F(q_{z}) =&& {16 \over 1.27\mbox{\AA}~\gamma \cos {\theta}_{t}}~[\sin(\gamma L_{2} + 
\phi_{c}) - \sin(\gamma L_{1} + \phi_{c})] \nonumber \\
&& + 2 n_{H} \cos(\gamma z_{H} + \phi_{H})~e^{-(q_{z} {\sigma}_{H})^{2}/2}~,
\end{eqnarray}
where \( \phi_{c} = q_{x} \Delta x_{c} + q_{y} \Delta y_{c} \), and \( \phi_{H} = q_{x} 
\Delta x_{H} + q_{y} (\Delta y_{H} - z_{H} \tan {\theta}_{t}) \).

Our samples are unoriented dispersions so the {\bf q}-space diffraction pattern
given by Eq.\ \ref{et5} was powder averaged and the usual Lorentz correction
was applied.  This was accomplished by multiplying the square of
Eq.\ \ref{et5} by the factor
\( { {1+{\cos}^2 (2 \theta)} \over q q_{z}} \), where the components of q are $q_{z}$
in Eq.\ \ref{et5} and $q_{x}$ and $q_{y}$ in Eq.\ \ref{et3}.
The final $1/q_{z}$ factor, due to powder averaging the Bragg rods, theoretically
produces square root singularities when the rod crosses the equator at $q_{z}=0$,
but observation of such singularities would require zero intrinsic widths
of the Bragg rods as well as perfect instrumental resolution.  


\section{Origin of the satellite peaks}
\label{gel_model_sate}

The essential scattering pattern for the model in the previous section can be
most easily seen in the simple limit when headgroups, offsets and methyl
gaps in the middle of the bilayer are ignored.  Then the scattering is from
a planar array of rods of finite length 2L all tilted with angle $\theta$
towards nearest neighbors.  The q-space pattern for such an array is
shown in Fig.\ \ref{model:fig3}.  Along each Bragg rod the intensity variation is
just the square of a sinc function as one sees from Eq.\ \ref{et5}.  The maximum
intensity along the (20) Bragg rod occurs on the equator, \( q_{z} = 0 \).
Because the rods are tilted, however, the maximum intensity along the (11) and (1,-1)
rods are displaced from the equator and are given by the relation $\gamma = 0$ 
where $\gamma$ is defined after Eq.\ \ref{et4}.

\begin{figure}[ht]
\centerline {\psfig{figure=gel_model_dir/fig3.ps,width=4in}}
\caption{The q-space pattern for the simplest model of chains tilted towards
nearest neighbors.  The locations of the Bragg rods are shown as vertical
dashed lines.  The intensities along these rods are shown as the
horizontal distances between the solid curves and their underlying dashed
lines.  The angle $\hat{\theta}$ appears in Eq.\ \protect\ref{et6}.
\label{model:fig3}}
\end{figure}

The powder averaged (1,1) and (1,-1) central peaks are broad because
the magnitude of the q-vectors to various parts of these peaks
along the Bragg rod are substantially different.
In contrast, the powder averaged (20) peak is much narrower
because the magnitudes of all the q-vectors to various parts of that
peak are nearly the same.  

The ``single slit diffraction pattern'' along each rod shown in 
Fig. \ref{model:fig3}
also has secondary maxima, and some of these are the satellite peaks observed
in Fig.\ \ref{model:fig1}.  The secondary maxima along the (1,1) and (1,-1) Bragg rods
that are closer to the equator than the central peaks are the satellites
that occur at smaller angles than the (20) and (11) peaks in Fig.\ \ref{model:fig1}.
The \(1 / q_{z}\) factor that appears in the powder averaging enhances the
apparent intensity of these peaks, which is another way of saying that
the intensity is compressed into a smaller angular range because the
difference in magnitudes of the q-vectors for various parts of these
secondary maxima is smaller the closer they are to the equator (\(q_{z} = 0\)).
Conversely, secondary maxima further from the equator will contribute
broader and therefore lower peaks to powder scattering.  With film data we 
observed one such satellite at higher angles than the (11) peak,
but quantitative intensities could not be 
obtained.  Similarly, satellites on the (20) Bragg rod are expected to be
much less prominent than the central equatorial peak.
