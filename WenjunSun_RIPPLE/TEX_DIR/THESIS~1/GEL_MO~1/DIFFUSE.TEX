\section{Diffuse scattering}
\label{gel_model_diff}

Our success in fitting the peak scattering in Fig.\ \ref{model:fig1} suggests that
our program of separating the total scattering into peak scattering
and diffuse scattering is reasonable.  It also suggests that the
actual quantitative diffuse scattering curve is now reasonably
reliable and can be used as a starting point for the discussion
of diffuse scattering.  Indeed, precision in the determination of 
the diffuse scattering curve will probably be much less important in
discussing the diffuse scattering than it was in fitting the peak scattering.
At this time the most remarkable aspect of the
diffuse scattering is the large amount of it as shown in Fig.\ \ref{model:fig1}.  

At this point it may be useful to compare to the data in Fig. \ref{model:fig2}
of reference \cite{Smi88}
which shows one nicely resolved satellite peak for oriented lipid samples.
Although the lipid and the degree of hydration were different, it is
still perhaps meaningful that there is a clear minimum between the central
peak and the satellite peak on each Bragg rod in the scattering
from the oriented samples.  For the same magnitude of the scattering angle
in Fig. \ref{model:fig1} (2$\theta$ = $20.4^{\circ}$) there is, in contrast,
only a plateau in the
scattering.  Of course, the scattering in Fig. \ref{model:fig1}
comes from all q-vectors
with the same magnitude, so there is no necessity that there should be
a minimum in powder samples even if there is a minimum along the
Bragg rods.  The two sets of data together suggest that a major
part of the diffuse scattering comes from other parts of q-space than
along the Bragg rods.  Since the Bragg rods are dominated by the
chain packing and
since the results in the preceding section suggest that the headgroup
region is relatively more disordered at the 4-5\AA~length scale,
it is reasonable that headgroups may be a
plausible starting point for this other diffuse scattering.

Let us therefore initiate a first exploratory search for the origin
of the diffuse scattering by focussing upon the head group region.  One
reason for greater disorder in the headgroup region is simply the molecular
nature of the lipid molecule, with one head for every two chains, so
that, even if the chains were perfectly packed in an hexagonal array, the
heads could be represented as a disordered array of dimers connecting pairs
of chains, as indicated in the inset to Fig.\ \ref{model:fig7}.  Furthermore, the 
most electron dense
part of the headgroup is the phosphate group.  The position of this group,
when projected onto the bilayer plane, is closer to the sn-1 chain than to
the sn-2 chain.  This suggests a simple model as a first
approximation for headgroup diffuse scattering.  Each headgroup represented
by a dimer in the inset to Fig.\ \ref{model:fig7} will have a differential scattering 
center located at 
only one end on the projection of the hexagonal array of chains. 
The array of dimers will be considered to be random with respect to orientation
of the dimers and with respect to the electron dense ends.

\begin{figure}[ht]
\centerline {\psfig{figure=gel_model_dir/diffuse.ps,width=4in}}
\caption{Diffuse scattering S($2 \theta$) versus scattering angle $2 \theta$
predicted by a simple model for phosphate headgroups indicated by the inset
which represents an x-y plane (see Fig. \ref{model:fig2}c) at the z level 
of the headgroups.
Each hydrocarbon chain projects to one of the lattice sites in the
inset.  Each phosphate group is represented by a solid circle near one
chain and
the position of the other chain in the same molecule is
represented by an open circle connected to the phosphate group by a
broad grey line.
\label{model:fig7}}
\end{figure}

For any model the total scattering is given by Eq.\ \ref{nagle1}.
For the simple model considered here, ${\bf r}_{1}$ may be taken to be the origin
and the double integral becomes a single sum over the triangular lattice,
\begin{equation}
\label{D3}
I({\bf q}) = \sum_{n} e^{i {\bf q} \cdot {\bf r}} <\rho(0) \rho({\bf r}_{n})>~.
\end{equation}
The correlation function at the origin,
$<\rho(r_{0}) \rho(r_{0})>$ equals 1/2 because this is the probability that the electron
dense phosphate end of the headgroup is at the origin.  For nearest neighbor
lattice sites on the triangular lattice the correlation function
is $<\rho(r_{0}) \rho(r_{nn})>$ which can be calculated as a factor of 1/2 for the dense
end at the origin and a factor of 5/12 for having a dense end at the nearest
neighbor site.  The reason that this latter probability is less than 1/2 is
due to the fact that, if there is a dense end at the origin, then one of the
nearest neighbor sites must have a non-dense end while the other five neighbors
have probability 1/2 of having a dense end.  Further neighbor correlations
$<\rho(r_{0}) \rho(r_{fn})>$ all have values of (1/2)(1/2)=(1/4).  

The sum in Eq.\ \ref{D3} may now be separated into two pieces.  The first piece
is a sum over all lattice sites n,
\begin{equation}
\label{D4}
I_{P} = \frac{1}{4} \sum_{n} e^{i {\bf q} \cdot {\bf r}}~,
\end{equation}
which yields a Bragg rod that has already been included in the peak analysis in
the preceding section.  The remaining terms can be written as
a sum over nearest neighbors nn,
\begin{equation}
\label{D5}
I_{D} = \frac{1}{24} \sum_{nn} [1 - e^{i {\bf q} \cdot {\bf r} }]~,
\end{equation}
and yield diffuse scattering.

In order to compare to Fig.\ \ref{model:fig1} the diffuse scattering in Eq.\ \ref{D5} must
be powder averaged.  Ignoring atomic or molecular form factors, this yields 
\begin{equation}
\label{D6}
I_{D} = {1 - sinc(bq) \over 2}~,
\end{equation}
where b is the nearest neighbor distance which is also the
unit cell dimension b used in the preceding section.  Fig.\ \ref{model:fig7} shows
S(q) from Eq.\ \ref{D6}.  Unfortunately, this result is not in good agreement with
the diffuse scattering inferred from Fig.\ \ref{model:fig1}, suggesting that this simple
model of positional disorder in the phosphate group is not the only cause
of diffuse scattering.  

An indication that the diffuse scattering is not likely to be due 
entirely to the headgroup region comes from the
estimate that the wide angle scattering under the peaks is
only about 30-40\% as large as the scattering labelled diffuse
in Fig.\ \ref{model:fig1}, so the integrated peak scattering is only about 25\% of the
total scattering.  As a rough approximation, let us suppose that 
this fraction $f_{P}=0.25$ of the scattering under the peaks can be
represented as a superposition of the scattering from the heads and
chains, as follows,
\begin{equation}
\label{D1}
f_{P} = { f_{c}n_{c}+f_{H}n_{H} \over n_{P}}~,
\end{equation}
where $n_{P}=412$ is the number of electrons in DPPC,
$n_{c}=248$ is the number of electrons in the chains and $n_{H}=164$ is the
number of electrons in the heads.  In this approximation the fractions
$f_{c}$ and $f_{H}$ give estimates of the degree of order in the chain
and head regions, respectively.
From the fitting in the preceding section we have $f_{H} n_{H}=45 f_{c}$,
so $f_{c}=103/(248+45)=0.35$ and $f_{H}=0.10$.
This rough estimate yields larger values of $f_{c}$ than $f_{H}$, in
agreement that the headgroups are more disordered than the chains.
The small value of $f_{c}$ also suggests that even the chains
are relatively disordered and this disorder must be included in order to account
for the diffuse scattering.  The appearance of a diffuse scattering
peak centered near the (20) and (11) peaks is also an indication 
of chain disorder.
