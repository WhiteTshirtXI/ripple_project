\section{Fitting the peaks}
\label{gel_model_fit}

While the simple model described in the preceding section
gives the overall qualitative features seen in Fig.\ \ref{model:fig1}, 
it does not give very good quantitative agreement with the scattering
angles for the satellite peaks.  In this section we
demonstrate that the physically plausible refinements
of the model described in Section \ref{gel_model_model} 
can yield quantitative agreement with all peak scattering.

Fitting the peaks in Fig.\ \ref{model:fig1} with a model poses a problem because there are
a number of parameters in any realistic model whose practical
determination requires non-linear least squares fitting, but
such fitting can not be performed on the total measured scattering.
We therefore used calculations for our model
to establish criteria that allow reasonable guesses
for the diffuse scattering curve.  One
key criterion was that peak scattering should fall to low values (though not 
necessarily exactly zero, {\it vide infra}) between the satellite peaks 
and between the first satellite and the (20) peak.  Also, peak scattering 
should be small for $2 \theta$ larger than the (11) peak.  
Another key criterion is that the 
integrated intensity in the (11) peak should be
about twice as large as the integrated intensity of the (20) peak.
This ratio, RII, should be exactly 2 if the electron density of the
hydrocarbon chains is modelled as a delta function rod as in Section 
\ref{gel_model_model}.
We also performed calculations modelling the tails more accurately as 
stereochemically accurate zigzag chains.  Then, RII could vary by as much 
as 40\%, but this much deviation of RII from 2 requires full rotational order
of the chains about their long axis in specific 
directions.  There is evidence for some rotational order in the
gel phase \cite{Nag93a}, but the rotational order is closer
to fully disordered than to ordered, so the deviation of RII from 2
is probably much smaller than 40\%.  For RII to be close to 2, the 
diffuse scattering curve must have a broad peak under the (11) peak such
as the one drawn in Fig.\ \ref{model:fig1}.  Subtraction of this diffuse
scattering curve from the measured scattering yields the
peak scattering curve shown in Fig.\ \ref{model:fig4}.

\begin{figure}[ht]
\centerline {\psfig{figure=gel_model_dir/fit_new.ps,width=4in}}
\caption{Peak scattering, solid circles with error bars,
from Fig.\ \protect\ref{model:fig1} and
model fits from Section \ref{gel_model_strategy}. Longer counting times in 
the satellite
region account for the smaller error bars there. The solid line running
within the error bars of most
of the data points shows our best fit (${\chi}^{2}$=2.03). The dashed line
shows the separate
contribution from the (11) Bragg rod. The left hand inset expands
the vertical scale for the
satellite region with scattering angles between $18^{\circ}$ and $20.5^{\circ}$.
The right hand inset expands the region between $21.0^{\circ}$
and $21.4^{\circ}$.
The dot-dash lines in the insets show the suboptimal fit when the
offsets were set to zero (${\chi}^{2}$=5.79).
\label{model:fig4}}
\end{figure}

In addition to the parameters defined in the model
in the previous section, a parameter ${\sigma}_{inst}$ for the
instrumental resolution and a parameter
${\sigma}_{11}$ for the intrinsic width of the (11) Bragg rod were allowed.
Since the intrinsic (20) peak width is much narrower than the
instrumental resolution, it was assumed to be zero.

Fits to the peak scattering in Fig.\ \ref{model:fig4} were driven by the IMSL non-linear 
least squares library routine UNLSF.  
The reduced ${\chi}^{2}$ of the fit shown in Fig.\ \ref{model:fig4}
was 2.03, which was quite good considering the intrinsic uncertainty in
subtraction of the diffuse scattering.  The largest systematic
error in the fit is for $2 {\theta}$ between $20.7^{\circ}$ and
$20.8^{\circ}$ on the low angle side of the (20) peak.  This
could be due to omission from the fitting of the power law tails mentioned in 
Sec. \ref{gel_model_strategy}.
The values of the basic parameters determined by the fit 
shown in Fig.\ \ref{model:fig4} are given in Table\ \ref{table1}.  Two kinds of errors are
shown in this table.  The first set of errors was determined
for one particular diffuse scattering curve in the usual way, namely,
fits were performed in which one parameter was held fixed and
all other parameters were then optimized.  Those values of the fixed parameter
that gave increases of 1 in ${\chi}^{2}$ yield the first errors shown in Table\ \ref{table1}.
The second set of errors was estimated by choosing two different diffuse
scattering curves and comparing the values of the best fit parameters.

\begin{table}
\caption{Results for Fitted Parameters to Wide Angle Data
\label{table1}}
\begin{center}
\begin{tabular}{cc}
\hline \hline \\
$d_{20}$                        & 4.2440$\pm$ 0.0001$\pm$ 0.0004\AA\\
$d_{11}$                        & 4.182$\pm$ 0.003 $\pm$0.006\AA\\
${\theta}_{t}$                  & 31.6$\pm$ 0.1 $\pm$ $0.2^{\circ}$\\
L                               & 21.32$\pm$ 0.07 $\pm$0.5\AA\\
${\sigma}_{inst}$               & 0.0578$\pm$ 0.0005 $\pm$ $0.0014^{\circ}$\\
${\sigma}_{11}$                 & 0.09$\pm$ 0.01 $\pm$ $0.04^{\circ}$\\
${\Delta x}_{c}$                & 0.47$\pm$ 0.03 $\pm$0.05\AA\\
${\Delta y}_{c}$                & 0.64$\pm$ 0.07 $\pm$0.13\AA\\
$n_{\mbox{H}}$                  & 45$\pm$ 1 $\pm$11 electrons\\
$z_{\mbox{H}}$                  & 20.38$\pm$ 0.07 $\pm$0.25\AA\\
${\sigma}_{\mbox{H}}$           & 1.9$\pm$ 0.2 $\pm$0.1\AA\\
${\Delta x}_{\mbox{H}}$         & 0.73$\pm$ 0.06 $\pm$1.0\AA\\
${\Delta y}_{\mbox{H}}$         & 0.4$\pm$ 0.1 $\pm$ 0.3\AA\\
$RII$                           & 2.01$\pm$ 0.08 $\pm$0.12\\
\hline \hline \\
\end{tabular}
\end{center}
\end{table}

While the fit to the data is a complex interaction of all the parameters,
there are some particular regions of the data that most strongly
influence certain parameters. The values of $d_{20}$ and ${\sigma}_{inst}$ 
are quite well determined by the sharp (20) peak.
The breadth of the central (11) peak is largely determined by twice the length
L of the chains, which establishes the width of the central peak along
the Bragg rod, and by the tilt angle $\theta$ which determines how
far this peak is displaced from the equator ({\it vide infra}).  
The correct value of L for all-trans hydrocarbon
chains consisting of 14 $CH_{2}$ groups and a terminal $CH_{3}$ is
16(1.27\AA )=20.3\AA. That the fitted value of L is slightly
larger is quite satisfactory; it is consistent with the model chain including
some of the carbonyl group.

A subtle but important relation concerns
the ``edge'' angle $2 {\theta}_{e}$ corresponding to the q
value where the (11) Bragg rod crosses the equator.  
For \( \theta < {\theta}_{e} \) there is no peak scattering except for resolution
broadening of the square root singularity at ${\theta}_{e}$.  The fact that
our data are very smooth from $18.8^{\circ}$ down to $15.5^{\circ}$ (not all data shown
in Fig.\ \ref{model:fig1}) suggests that $2 {\theta}_{e}$ is close to $18.9^{\circ}$.  
Identifying ${\theta}_{e}$ provides the equatorial component \(  q_{r,11} = {4 \pi 
\over \lambda} \sin {\theta}_{e} \) of the
(11) peak.  This allows one to determine ${\theta}_{t}$ from the
well-known relation \cite{LevTh,Smi88,HenHos83}
\begin{equation}
\label{et6}
\sin \hat{\theta} = \sqrt{1-({q_{20} \over 2 q_{11}})^2}~ \sin {\theta}_{t}~,
\end{equation}
where $\hat{\theta}$ is the angle of the (11) peak in {\bf q}-space
defined by \( \cos \hat{\theta} = {q_{r,11} \over q_{11}} \).
Although there is some uncertainty in estimating $q_{11}$ because of the foot
on the (20) peak between $21.1-21.3^{\circ}$,
even rough estimates without detailed fitting yield values
of ${\theta}_{t}$ of 31--32$^{\circ}$.  This agrees very well with our recent work 
\cite{STN92} on fully hydrated oriented DPPC bilayers which obtained the tilt angle
${\theta}_{t}$ to be $32.0 \pm 0.5^{\circ}$ at $19^{\circ}$C.  Even the slightly smaller
values obtained in this fit to data at $24^{\circ}$C can be attributed to our
earlier observation that the tilt angle decreases with increasing temperature
\cite{STN92}.
Our earlier work also verified directly that the tilt is towards nearest neighbors.  
With such good agreement with known quantities, we believe that some of the
values for additional parameters obtained for the first time are worthy of
consideration.  

The intrinsic linewidth ${\sigma}_{11}$ of the (11) Bragg rod is another parameter
that can be directly estimated by the breadth of the edge at $2 {\theta}_{e}$
of the second satellite.
At first, we were concerned that the (11) Bragg rod requires a 
considerably larger intrinsic width than the (20) Bragg rod, but
the data in Fig. \ref{model:fig2}c in \cite{Smi88} show the same comparison 
for oriented
samples.  An explanation
for this is that there is likely to be a distribution of tilt angles ${\theta}_{t}$.
Since the tilt angle probably depends upon competing interactions \cite{STN92} that
are small relative to the strong cohesive interactions that establish the
wide angle $d_{11}$ and $d_{20}$ spacings, the distribution
of these latter spacings may be taken to be very narrow and can be ignored
compared to the distribution of ${\theta}_{t}$.  Therefore, the
distribution of ${\theta}_{t}$ gives rise to a distribution of
(11) Bragg rods that cut the equator at different angles ${\theta}_{e}$ 
and this distribution in ${\theta}_{e}$ determines ${\sigma}_{11}$.  Using the 
relations in the preceding paragraph yields mean fluctuations in ${\theta}_{t}$
of  $\pm 1.3^{\circ}$ about its average value given in Table\ \ref{table1}.  Notice
that the size of these fluctuations is distinct from the estimated error
in the average value of ${\theta}_{t}$.

The insets in Fig.\ \ref{model:fig4} show two regions
where the fit is adversely affected by not allowing any offsets.
This suggests that offsets not only could exist, but do exist.
We originally drew the diffuse scattering curve in Figs.\ \ref{model:fig1} and \ref{model:fig4}
so that the peak scattering would go to zero near $19.5^{\circ}$
and near $20.5^{\circ}$, but this is not a necessity as seen from the fitted
curves in Fig.\ \ref{model:fig4}.  The reason for this is interesting and reveals some
features that help to appreciate the effect of the offsets shown in
Fig.\ \ref{model:fig4}.  Fig.\ \ref{model:fig5} shows the form factors for the (1,1) and (1,-1) rods.
Offsets lower the symmetry which only demands
that the (1,1) and (-1,-1) rods have the same form factors and also
that the (1,-1) and (-1,1) rods have the same form factors.  To
obtain the intensity, the individual form factors are first squared
and then added because the scattering from the two rods comes from
different regions in real space that scatter incoherently.  Even
though the intensity from each Bragg rod has zeros, the zeros occur
for different values of q for the (1,1) and (1,-1) rods,
so there are no zeros in the sum.  Another reason
that there are no zeros between the satellites in Fig.\ \ref{model:fig4} is that
both $\pm |q_{z}|$ values contribute to the same
powder averaged value of q.
Another feature of interest in Fig.\ \ref{model:fig5} is that the form factor is not
symmetric along $q_{z}$ as it would be for a sinc function which describes a
simple array of rods with no offsets or headgroups.  The simple sinc functions
have a maximum at the $q_{z}$ indicated by the vertical dashed line near
${0.7\mbox{\AA}}^{-1}$ in Fig.\ \ref{model:fig5}; this is the value of $q_{z}$ that gives 
$q=\frac{2 \pi}{d_{11}}$ . With non-zero offsets the maxima in the form factors 
are displaced as shown in Fig.\ \ref{model:fig5} and reflection symmetry about this 
$q_{z}$ value disappears.

\begin{figure}[ht]
\centerline {\psfig{figure=gel_model_dir/fqz11all2.ps,width=4in}}
\caption{The form factors $F(q_{z})$ along the (1,1) and (1,-1) Bragg rods
for the parameters in Table\ \protect\ref{table1}.  One vertical dashed
line at $q_{z}=0$ marks the
equator and the other vertical dashed line at $q_{z}=0.684$ marks the
value where $q=2 \pi / d_{11}$.
\label{model:fig5}}
\end{figure}

The electron densities perpendicular to the bilayer
determined from the fit in Fig.\ \ref{model:fig4} and Table\ \ref{table1} are shown 
in Fig.\ \ref{model:fig6}.  
The chain electron density has been normalized so that it gives
the total number of electrons 248, as in the chains (not including the 
first carbonyl carbon) when multiplied by the area/molecule A and then
integrated along z.  With this same normalization the number of electrons under 
the fitted headgroup peak is $n_{H}^{fit}$ = 45.0.
This may be compared to the number of electrons $n_{{PO_{4}}}$=47 in the phosphate,
$n_{choline}$=50 in the choline and $n_{back}$=67 in the backbone which
includes the first carbonyl carbons on the acyl chains, for a total number of
electrons in the head $n_{H}$=164 .  (Subtraction of that fraction of the
carbonyl electrons that are included in the slightly larger L would only
reduce this by 12.)  That this actual
$n_{H}$ is considerably larger than the $n_{H}^{fit}$ 
derived from the size of the fitted headgroup peak
suggests that the headgroups are more disordered than the tails so that
they preferentially scatter diffusely rather than with coherent peak scattering.

Fig.\ \ref{model:fig6} also shows the electron density determined from our low
angle data \cite{WSN89}.  The small mismatch with the wide angle
data in the chain region is due to use of a less accurate and
smaller measurement of ${d_{11}}$ in the earlier work.
The headgroup Gaussian from the
low angle data is larger by a factor of 1.24, containing about 56 electrons,
than the headgroup Gaussian from the wide angle fit in Fig.\ \ref{model:fig4} that contains
45.0 electrons.  Furthermore, 
56 electrons is an underestimate of the number of electrons in the headgroup
obtained from the low angle fit
because that Gaussian sits on a smooth ``bridging'' electron density
that comes from a variety of molecular substituents such as water,
chains, and headgroup contributions that are not included
in the headgroup Gaussian.  These latter headgroup contributions account
for water and chains that have been displaced from the headgroup
region and therefore the apparent headgroup Gaussian just accounts for electron
density of the headgroups in excess of the water and hydrocarbon chains.
To account for the total number of headgroup electrons,
the low angle Gaussian should be scaled by a factor of 2.9
for comparison with the wide angle headgroup Gaussian.
This is shown as the dot-dash Gaussian in Fig.\ \ref{model:fig6}.
(The difference between the low angle electron density (dashed line
in Fig.\ \ref{model:fig6}) and the dot-dash Gaussian is the sum of the electron density of
the chains and the water; this sum is positive as it must be.)
Since the dot-dash Gaussian is much larger than the solid Gaussian obtained
from the wide angle fit, it can be concluded that,
when normalized by the chain scattering,
the headgroup electrons yield less peak scattering
in the wide angle region than in the low angle region.

\begin{figure}[ht]
\centerline {\psfig{figure=gel_model_dir/denprof2.ps,width=4in}}
\caption{Electron density ${\rho}_{z}$, in units of electrons/cubic angstrom,
along the bilayer normal for half a symmetric bilayer, with the midplane
(terminal methyl trough) at z=0.  The constant solid line shows the
chain density and the solid Gaussian shows the headgroup
contribution for the parameters in Table\ \protect\ref{table1} .
The electron density determined by our previous low-angle
study \protect\cite{WSN89} is plotted with a dashed curve and
the low-angle headgroup contribution is plotted
separately (dot-dash curve) for comparison with the wide-angle
headgroup Gaussian. 
\label{model:fig6}}
\end{figure}

The quantities in Table\ \ref{table1} determined directly by the fit to the wide
angle data allow the determination of other interesting quantities
for bilayer structure shown in Table\ \ref{table2}.  The following elementary
equations give the unit cell dimensions, {\it a} and {\it b}, and
the area $A_{c}$ perpendicular to the chain and the area A
of the head group:
\begin{eqnarray}
\label{d-1}
a = 2 d_{20} , \ \ b = {d_{11} \over \cos {\theta}_{t} \sqrt{1-({d_{11} \over 2 
d_{20}})^2}}, \nonumber\\
A_{c} = {d_{20} d_{11} \over \sqrt{1-({d_{11} \over 2 d_{20}})^2}} \ \mbox{and} 
\   A = {2 A_{c} \over \cos {\theta}_{t}} .
\end{eqnarray}
Since the length of a methylene group is
1.27\AA~ along all-trans chains, the volume per methylene $V_{CH_{2}}$ equals
$A_{c} \times 1.27\AA$. Using our current measurement of $A_{c}$, the ratio of methyl 
volume to methylene volume is
determined from our previous low angle diffraction work (see Fig.\ \ref{model:fig4}
in \cite{WSN89}) to be in the range 2.0 to 2.1, which also agrees with
\cite{NW88,WW5}, and this yields $V_{CH_{3}}$. The 
hydrocarbon volume $V_{C}$ consists of two chains, each with 14 methylenes
and one methyl.  The average thickness of the hydrocarbon region $D_{C}$ is 
obtained from $V_{C}$/A. The volume of the
headgroup $V_{H}$ equals the measured volume \cite{NW78} of the entire lipid
$V_{L}$ minus the volume of the chains $V_{C}$. The volume $V_{X}$ of the unit 
cell comes from AD/2 where our recent measurement of the low angle
D spacing \cite{STN92} is given.  
Finally, the number of waters per lipid is given by $(V_{X}-V_{L})/V_{W}$ where 
$V_{W}$ is the volume of a water molecule which is 30.0\AA$^{3}$ for this 
temperature of $24^{\circ}$C. Our previous estimates for many of the quantities 
in Table II \cite{WSN89} used wide angle data that were not fully resolved from 
which we estimated a value of $d_{11}=4.12 \pm 0.02$\AA~ which is significantly
smaller than the value of $d_{11}$ from Table \ \ref{table1}.
This difference is the primary reason that our previous \cite{WSN89}
and current error bars do not overlap for the quantities
$A_{c}$, $V_{CH_{2}}$, $V_{CH_{3}}$, $V_{C}$ and $V_{H}$. There may, however, be 
variations from one sample to another that account for
these differences in these latter quantities.  We obtained a different data
set at $19^{\circ}$C that gave results consistent with the values for
most of the quantities in Table \ \ref{table1}, but which had $d_{11}=4.14$\AA.
This propagated values of $A_{c}$, $V_{CH_{2}}$, $V_{CH_{3}}$, $V_{C}$ and $V_{H}$ 
that are about midway between our previous results \cite{WSN89} and the
results in Table \ \ref{table2}.  Therefore, it might be appropriate to assign a
wider range of uncertainty to these quantities.

\begin{table}
\caption{Other Results
\label{table2}}
\begin{center}
\begin{tabular}{cc}
\hline \hline \\
$a$             & 8.4880$\pm$0.0008\AA\\
$b$             & 5.64$\pm$0.02\AA\\
$A_{c}$         & 20.40$\pm$0.04${\mbox{\AA}}^{2}$\\
$A$             & 47.9$\pm$0.2${\mbox{\AA}}^{2}$\\
$V_{{CH}_2}$    & 25.91$\pm$0.05${\mbox{\AA}}^{3}$\\
$V_{{CH}_3}$    & 53.1$\pm$1.3${\mbox{\AA}}^{3}$\\
$V_{C}$         & 829$\pm$4${\mbox{\AA}}^{3}$\\
$D_{C}$         & 17.3$\pm$0.2\AA\\
$V_{L}$         & 1148$\pm$2${\mbox{\AA}}^{3}$\\
$V_{H}$         & 319$\pm$6${\mbox{\AA}}^{3}$\\
$V_{X}$         & 1518$\pm$9${\mbox{\AA}}^{3}$\\
$D$             & 63.4$\pm$0.1\AA\\
$n_{w}$         & 12.6$\pm$0.4\\
\hline \hline \\
\end{tabular}
\end{center}
\end{table}
