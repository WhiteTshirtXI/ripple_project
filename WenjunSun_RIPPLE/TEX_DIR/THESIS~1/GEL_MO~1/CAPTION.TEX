% ********************* caption.tex **********************
\begin{figure}
\centerline {\psfig{figure=gel_model_dir/shdpt.ps,width=4in}}
\caption{Wide-angle scattering intensity versus scattering angle 
2$\theta$ (open circles)
taken with graphite monochromator and PSD 572mm from the sample.
Slits were configured to yield resolution $0.068^{\circ}$ HWHM, which is the
width of the unresolved (20) peak.  The area in black is interpreted
as the peak scattering.  The temperature was $24^{\circ}$C.
At $2 \theta = 22.3^{\circ}$ the solid square shows the scattering
from a capillary filled with water and the solid triangle shows the
scattering from a capillary in air.
\label{fig1}}
\end{figure}

\begin{figure}
\begin{center}
%\leavevmode
\raggedleft
\hspace{0.5in}
\psfig{figure=gel_model_dir/modela.ps,height=3.0in}
%\leavevmode
\raggedright
\hspace{1.2in}
\psfig{figure=gel_model_dir/modelb.ps,height=3.0in}
\end{center}
\begin{center}
\leavevmode
\psfig{figure=gel_model_dir/modelc.ps,width=2.5in}
\end{center}
\caption{Cross-sectional views of the model to fit the peak scattering.  
The drawing is to scale for the values of the parameters shown in 
Table\ \protect\ref{table1}.
(a) Projection onto the (x,z) plane.  The chains are represented
by long thin black rectangles and the head groups are represented by shorter
gray rectangles.  ${\Delta x}_{c}$ is the chain offset, ${\Delta x}_{h}$ is the
head offset, $Z_{h}$ is the peak position of the headgroup Gaussian and a is the
unit cell dimension in the x-direction.
(b) Projection onto the (y,z) plane showing the tilt angle ${\theta}_{t}$
and the length L of the chains.  ${\Delta y}_{c}$ is the chain offset,
${\Delta y}_{h}$ is the head offset and b is the unit cell
dimension in the y-direction.
(c) Projection onto the (x,y) midplane showing the body-centered
two-dimensional unit cell.  The filled (open) circles are the projections of
the hydrocarbon chains from the upper (lower) monolayer, respectively, to show
the offsets ${\Delta x}_{c}$ and ${\Delta y}_{c}$.  The scale for 
this projection is
three times the scale for the preceding two projections.
Only the z-component of the headgroup electron density distribution is
represented, so this figure does not imply that the headgroups are
perpendicular to the bilayer.
\label{fig3}}
\end{figure}

\begin{figure}
\centerline {\psfig{figure=gel_model_dir/fig3.ps,width=4in}}
\caption{The q-space pattern for the simplest model of chains tilted towards
nearest neighbors.  The locations of the Bragg rods are shown as vertical
dashed lines.  The intensities along these rods are shown as the
horizontal distances between the solid curves and their underlying dashed
lines.  The angle $\hat{\theta}$ appears in Eq.\ \protect\ref{et6}.
\label{fig9}}
\end{figure}

\begin{figure}
\centerline {\psfig{figure=gel_model_dir/fit_new.ps,width=4in}}
\caption{Peak scattering, solid circles with error bars, 
from Fig.\ \protect\ref{fig1} and 
model fits from Section 4. Longer counting times in the satellite 
region account for the smaller error bars there. The solid line running 
within the error bars of most
of the data points shows our best fit (${\chi}^{2}$=2.03). The dashed line 
shows the separate 
contribution from the (11) Bragg rod. The left hand inset expands 
the vertical scale for the 
satellite region with scattering angles between $18^{\circ}$ and $20.5^{\circ}$.
The right hand inset expands the region between $21.0^{\circ}$ 
and $21.4^{\circ}$.
The dot-dash lines in the insets show the suboptimal fit when the
offsets were set to zero (${\chi}^{2}$=5.79).
\label{fig2}}
\end{figure}

\begin{figure}
\centerline {\psfig{figure=gel_model_dir/fqz11all2.ps,width=4in}}
\caption{The form factors $F(q_{z})$ along the (1,1) and (1,-1) Bragg rods 
for the parameters in Table\ \protect\ref{table1}.  One vertical dashed 
line at $q_{z}=0$ marks the
equator and the other vertical dashed line at $q_{z}=0.684$ marks the
value where $q=2 \pi / d_{11}$.
\label{fig4}}
\end{figure}

\begin{figure}
\centerline {\psfig{figure=gel_model_dir/denprof2.ps,width=4in}}
\caption{Electron density ${\rho}_{z}$, in units of electrons/cubic angstrom, 
along the bilayer normal for half a symmetric bilayer, with the midplane
(terminal methyl trough) at z=0.  The constant solid line shows the
chain density and the solid Gaussian shows the headgroup
contribution for the parameters in Table\ \protect\ref{table1} .
The electron density determined by our previous low-angle
study \protect\cite{WSN89} is plotted with a dashed curve and
the low-angle headgroup contribution is plotted 
separately (dot-dash curve) for comparison with the wide-angle
headgroup Gaussian. \label{fig5}}
\end{figure}

\begin{figure}
\centerline {\psfig{figure=gel_model_dir/diffuse.ps,width=4in}}
\caption{Diffuse scattering S($2 \theta$) versus scattering angle $2 \theta$ 
predicted by a simple model for phosphate headgroups indicated by the inset
which represents an x-y plane (see Fig. 2c) at the z level of the headgroups.
Each hydrocarbon chain projects to one of the lattice sites in the
inset.  Each phosphate group is represented by a solid circle near one 
chain and 
the position of the other chain in the same molecule is 
represented by an open circle connected to the phosphate group by a 
broad grey line. 
\label{fig6}}
\end{figure}
