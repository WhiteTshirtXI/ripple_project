\section{Introduction}
\label{gel_model_intro}

Fully hydrated lipid bilayers as model membranes possess order 
in the sense of having a periodically layered structure along the bilayer 
normal direction in all lamellar phases and regular chain packings within 
the bilayers in the subgel and gel phases; previous x-ray studies on 
lipid bilayers have focussed upon these types of order. However, these fully 
hydrated systems are also disordered enough not to form pure crystals; 
as discussed in Chapter \ref{intro_chap}, this disorder is important and 
even vital to biological functions of real biomembranes. The 
characterization of both the order and the disorder inherent in such 
partially ordered, anisotropic systems is clearly an appropriate topic 
for biological physics.

This chapter studies the wide angle scattering from the gel phase of 
DPPC lipid bilayers. This particular thermodynamic phase of this benchmark 
lipid has been perhaps the best characterized of all the lipid phases.  
There are two reasons for adding this study.
The first is the occurrence of two small satellite peaks that have previously 
been ignored in structural analyses of unoriented samples.  We show that
refinements of the conventional model that treat the gel phase
as an ordered two-dimensional crystal can account quantitatively for all 
the peaks.  However, the conventional models can not account for all 
the scattering after background scatterings due to water and air are 
subtracted.  The peaks in the wide angle region appear to be underlain 
by extensive broad diffuse scattering caused by disorder.  
The second reason for this study is to emphasize this disorder in 
gel phase lipid bilayers and to begin to explore it.

The experimental system studied in this chapter is the fully hydrated
multilamellar vesicle (MLV) which scatters x-rays in a powder pattern.
The reasons for choosing unoriented samples, not oriented
samples, are as follows: (1) Unoriented samples are most reproducible and 
easiest to prepare in the laboratory, while oriented samples are more 
difficult to make; (2) Unoriented samples are less susceptible to
the artifacts of oriented samples, including the mosaic spread which
degrades the data, and the possible effect of the substrate on
thin samples \cite{STN92,Wac89a}; (3) Unoriented samples are usually
thicker than oriented samples, thus permitting better counting statistics 
with less radiation damage; (4) Quantitative scattering profiles can
be obtained from unoriented samples using photon-counting detectors
like the PSD (Position Sensitive Detector) and point detector; this allows
quantitative model fitting as will be shown in this chapter. In contrast,
diffraction patterns from oriented samples are usually recorded on
x-ray films, from which only peak positions and approximate line widths
can be extracted.

Because unoriented samples lose the information about anisotropy,
it has been thought that some structural parameters, such as the tilt 
angle ${\theta}_{t}$ of the hydrocarbon chains, can be obtained only from
oriented samples. This is somewhat misleading since the supposedly
missing information is contained in the accurate scattering profile
from unoriented samples; carefully model fitting will extract this
information.  One of the purposes of this chapter is to show 
that ${\theta}_{t}$ can also be obtained from unoriented powder samples, 
and that this value of ${\theta}_{t}$ agrees very well with our recent 
studies of oriented samples \cite{STN92}. The value of ${\theta}_{t}$, 
along with low angle lamellar diffraction and macroscopic volume 
measurements, is central to obtaining the ordered, average structure of 
gel phase lipid bilayers \cite{WSN89}, and it is appropriate to obtain 
it for both oriented and unoriented bilayers. However, unoriented samples 
yield considerably more than a mere confirmation of the results from 
oriented samples; as will be shown, the effective hydrocarbon chain
length and the offsets of opposing monolayers can also be obtained from
analyzing unoriented sample data.  

The layout of this chapter is as follows.  Section \ref{gel_model_data} 
presents the wide angle data.  A simple strategy for dealing with these data
from partially ordered systems is presented in Section \ref{gel_model_strategy} 
followed by a detailed presentation of a model for the average order in 
Section \ref{gel_model_model}.  The physical origin of the various peaks,
especially the satellite peaks, is discussed in Section \ref{gel_model_sate}
in terms of the simplest possible model.  In Section \ref{gel_model_fit} the 
results of detailed fits to the data are reported using the full model 
presented in Section \ref{gel_model_model} and the values of the parameters 
in the model are interpreted in terms of bilayer structure.  An initial 
attempt is then made in Section \ref{gel_model_diff} to understand the 
diffuse scattering, defined by subtracting the model and the background 
scattering from the data. A discussion that includes comparison with 
some other studies is given in Section \ref{gel_model_dis}.
