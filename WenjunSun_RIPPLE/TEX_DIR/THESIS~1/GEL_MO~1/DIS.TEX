\section{Discussion}
\label{gel_model_dis}

The gel phase has usually been thought of as a fairly well-ordered
phase because the hydrocarbon chains are nearly {\em all-trans}, in
contrast to the fluid $L_{\alpha}$ phase which is rotamerically,
{\em i.e.} conformationally, 
disordered \cite{Nag80}.  This difference in rotameric ordering
is clearly proven by the appearance
of sharp peaks in the wide angle region of the gel phase whereas the
wide angle scattering in the $L_{\alpha}$ phase consists only of broad
diffuse scattering.  However, the analysis in this chapter suggests that
diffuse scattering is at least as great as peak scattering,
implying that there is still considerable
disorder in the gel phase even though the chains are rotamerically ordered.  

Our analysis indicates that the disorder is relatively greater in the
headgroup region than in the chain region.  This is consistent with
the picture that the headgroups remain disordered.  We might mention
that we first attempted to accommodate the small satellite peaks at
$18.9^{\circ}$ and $20.1^{\circ}$ by considering supercells corresponding to
headgroup ordering, but that no satisfactory supercell was found.

Our analysis also suggests that there is disorder in the chain region. 
The usual interpretation of chain disorder in the gel phase is 
that there is free rotation of 
each chain around its long axis.  There is recent evidence that this
rotational disorder is not complete \cite{Nag93a} and the ensuing
short range correlations would contribute to the diffuse scattering.
Of course, translational fluctuations must also be considered.

The study of disorder in lipid bilayers is relatively new and this chapter
only begins to come to quantitative grips with it.  We considered
a very primitive model for headgroup disorder in Section \ref{gel_model_model} 
that
concentrates upon the positional disorder inherent in the fact that
each molecule consists of two chains and one head.  Not surprisingly,
this primitive model does not do very well at reproducing the diffuse
scattering indicated in Fig.\ \ref{model:fig1}, although it does provide a 
mechanism for
producing diffuse scattering at angles considerably smaller than
the region around the peaks.  One might expect chain disorder to dominate
diffuse scattering in the wide angle peak region and
the headgroups to dominate the scattering at angles between the
wide angle peak region and the low angle lamellar peaks.

In contrast to the study of disorder, we now believe that the study of
the order in gel phase DPPC is rather complete.  Our recent work 
\cite{STN92} on fully hydrated oriented DPPC 
bilayers obtained the tilt angle ${\theta}_{t}$ 
to be $32^{\circ}$ at $19^{\circ}$C and verified directly that the tilt is
towards nearest neighbors.  This work obtains essentially the 
same results for
fully hydrated unoriented samples; the slightly smaller value of ${\theta}_{t}$ 
obtained here is consistent with the data having been taken at 
$24^{\circ}$C and
with ${\theta}_{t}$ decreasing with increasing temperature \cite{STN92}.
The present work also obtains 
a number of additional results that could not be obtained from the
oriented samples.  These results are listed in Table I and include
the models in Figs.\ \ref{model:fig2} and \  \ref{model:fig6}.
The fitted result for the length L of
the hydrocarbon chains is in good agreement with that demanded by
stereochemistry and the vertical extent of the
headgroup positions agree with low angle analyses \cite{WSN89}.
One new feature
that we discovered is that the hydrocarbon tails in the two monolayers are
slightly offset.  This is not a surprising feature since there are no
covalent bonds between the lipids in adjacent monolayers, but it
has not been included previously in diffraction analyses.  Some consequences
of this feature for scattering are shown in Fig.\ \ref{model:fig5}.
Additional results for bilayer structure are also derived and shown in
Table II.  It should be emphasized, however, that our data require
registry between both monolayers of the bilayer, so that the effective
chain length for scattering is basically 2L, in agreement with our
less accurate results from oriented samples \cite{STN92}.
 
The key to being able to obtain so much information from unoriented samples
was obtaining quantitative data
for the satellites and incorporating these data in the analysis.
These satellites were originally reported by Mitsui and 
co-workers \cite{Mit78}.  Although the original interpretation
was incorrect, a brief correction was later noted \cite{Mit80},
but no previous work has included these satellites in a global analysis of 
the peak scattering.  

Our result that the chains are tilted toward nearest neighbors disagrees
with the conclusions of a recent study in which the DPPC bilayers were 
oriented on a glass beaker \cite{Kat92}.
Their geometry precluded tilting the sample, an option
that allows all wide angle peaks to be observed with little differential
absorption as we have recently shown \cite{STN92}.
The reported presence \cite{Kat92} of only one unique reflection that
is off the equator is similar to the result of an earlier study
at low hydration \cite{Sta79} where it was shown that this pattern
corresponds to a uniform distribution of tilt directions.  
Our data, however, clearly have two unique reflections with a sharp
(20) peak on the equator for both unoriented samples and oriented samples
on a glass substrate \cite{STN92}.
It may also be useful to compare to results on free-standing
films of DMPC \cite{Smi88} where it was carefully shown that the 
direction of tilt could be towards neighbors (as in the present chapter),
between neighbors (two unique off-equatorial reflections),
or even between the preceding two cases (three unique off-equatorial 
reflections), depending upon hydration.  Although full hydration 
conditions were not achieved, extrapolation using the reported phase 
diagram \cite{Smi88} suggests that the fully hydrated gel phase would 
have the chains tilted between neighbors rather 
than towards neighbors as we find for DPPC.  Our unpublished data for DMPC
also indicate that the tilt is towards neighbors because there is again one
narrow and one broad wide angle peak.  An explanation for this variety 
of results is that differences in free energies for different directions 
of chain tilt may be small so that the observed chain tilt direction may 
be highly sensitive to sample preparation including the influence 
of substrates.  We suggest that the unoriented samples studied in this
chapter are least susceptible to these influences, so that the results 
reported here are more appropriate for model membranes.  
