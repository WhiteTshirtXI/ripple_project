\section{Materials and methods}
\label{gel_model_matmet}

\subsection*{Phospholipid Samples}

DPPC (Lot \#160 PC-176) was purchased from Avanti Polar Lipids (Birmingham, AL) and used
without further purification.  Samples were prepared for X-ray scattering
by weighing water and lipid in a 3:1 (w:w) ratio.
The lipid was hydrated by cycling it
three times between $80^{\circ}$C and $5^{\circ}$C with 5 minutes of vortexing at
each temperature.  After hydration the sample was loaded into thin walled
1.0 mm glass X-ray capillaries (Charles Supper Co.).  The capillaries
had been precleaned by sequentially washing with chromic acid, 
acetone and copious amounts of deionized water.
After drying with nitrogen and flame sealing the capillaries
at one end, they were filled with hydrated lipid using a
1.0 ml Hamilton syringe.  Upon standing, these dispersions separate
into a lipid rich phase and a clear water rich phase.  
In order to remove air bubbles, the capillaries
were centrifuged for 10 minutes at 1100g at room temperature.
This amount of centrifugation did not overly compress the lipid, 
since upon additional standing for one week, the lipid settled 
further as indicated by the presence of a larger volume of water rich phase at
the top of the capillary.  X-ray scattering was obtained from the
lipid rich phase and samples with either weight
ratio gave the same low angle D-spacing of 63.4\AA .
In addition, the low angle D-spacing was unaffected by centrifugation.
After centrifugation the capillaries were flame-sealed above the water 
layer, and this seal was dipped in Duco cement.  
After completion of the experiment, the continued presence of an excess
water layer above the lipid was observed, confirming that the sample was
fully hydrated during the course of the experiment.  Then
the lipid was removed from the capillary, dried under nitrogen and
analyzed by thin layer chromatography using the solvent system
chloroform:methanol:water (60:30:5).   The chromatogram showed less than
0.2\% lysolecithin formed during irradiation.  Also, the positions and half-widths 
of the first and second order low angle peaks were identical before and after
48 hours of irradiation, indicating that any degradation did not affect
gel phase structure.

The lipid capillary was held upright in a custom built sample chamber made from
aluminum with $1.5\mu$ thick mylar windows (DuPont).  A calibrated 
silicon diode (Type DT-470-CU-13) coated with Dow Corning
heat sink compound was seated in a pocket in
the aluminum block directly next to the capillary. The diode was
connected to a Lake Shore Cryotronics Model DRC 84C Temperature
Controller which monitored the temperature.

\subsection*{X-ray Scattering}

Our principal measurements were carried out using a rotating anode x-ray source
run at 35 kVolts and 150 mAmps and interfaced with a Huber four circle
diffractometer via evacuated beampaths.
Monochromatic copper $K_{\alpha}$ radiation was obtained
by diffraction from a vertically bent graphite monochromator.
Several different input and diffracted beam collimation and
detection schemes were used, as described below.  On the input side,
two sets of xy-slits, separated by 500 mm, define the beam size
and divergence in the in-plane and out-of-plane directions.
The four slit openings will be referred to as $\mbox{S}_{mh}$, $\mbox{S}_{mv}$,
 $\mbox{S}_{sh}$ and $\mbox{S}_{sv}$, where the {\em m} subscript refers to the slits
nearest the monochromator, {\em s} to the slits nearest the sample, 
{\em h} is the horizontal (in-plane) opening, and {\em v} is the vertical
(out-of-plane) opening.  Slits after the sample eliminated
extraneous scattering from air.  An xy-slit set before the detector
was also used to reduce extraneous radiation and to define the
out-of-plane divergence.

For most measurements in the wide angle region, a Braun linear
position sensitive detector (PSD) was placed on the detector arm of
the diffractometer at a distance of 572 mm from 
the sample center. All vertical slit openings were 4 mm.  
$\mbox{S}_{mh}$ and $\mbox{S}_{sh}$ were
set to 0.4 mm.  Thus the beam size was less than the 1 mm
capillary diameter and the in-plane divergence was $0.092^{\circ}$ full width.
Accounting for the finite beam size, the finite sample size and the
divergence of the incoming beam leads to an estimate for the resolution
half width at half maximum (HWHM) of $0.08^{\circ}$ in 2$\theta$ which
is expected to be an overestimate.

Some additional measurements used two other setups.  First,
to verify the background scattering obtained with the PSD,
low resolution data ($0.14^{\circ}$ HWHM) were
also taken for 2$\theta$ from $1^{\circ}$ to $55^{\circ}$ using a graphite crystal
analyzer. To fully resolve the sharp (20) peak, the F-3 beamline at CHESS was
used with silicon monochromator, silicon analyzer crystal and
photomultiplier detector. This setup had a resolution of $0.004^{\circ}$ HWHM at $\lambda 
= 1.2148$ \AA.

We carried out a limited set of measurements in the low angle region
to verify that samples were in the fully hydrated state and to check
for effects of radiation damage.  These measurements used the PSD
with slits configured in the same way as for the wide angle
measurements.  Some smearing of peaks on the low angle side occurs in
this configuration; we obtained best estimates for the D-spacing by
extrapolating the fitted peak positions to the higher order limit using
model calculations of the smearing effect \cite{STN92}. While this procedure
yields only approximate values, it is reliable for comparing fresh
samples with those which have been exposed to the x-ray beam for
substantial periods.  All data reported here are from samples for
which no discernable shift in D-spacing or broadening of the low or wide
angle peaks could be detected after measurements were completed. 
