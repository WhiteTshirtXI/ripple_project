\chapter{Pleated vs. parallel chain packings in gel phase bilayers}

\section{Introduction}

In \cite{Tu96}, Tu et al. reported their Molecular Dynamics (MD) 
simulation results 
of gel phase DPPC bilayers. For the in-plane structure, despite considerable
disorder and defects, their results agree with ours \cite{Sun94} in that within
one monolayer, the average chain packing is basically 
orthorhombic (distorted hexagonal), with
chains tilting mostly towards nearest neighbors, and the tilt angle is about
33$^{\circ}$ (comparable to ours of 31.6$^{\circ}$). Their simulation 
also yielded similar offsets of chains between the two opposing monolayers.

Prior to publication, Tu et al. thought that all their results were in good
agreement with ours. We recognized, however, that there were disagreements.
The first disagreement is that the width of the (11) peak in their powder
scattering intensity is about twice as broad as our experimental
data; the second disagreement is that there are five diffraction peaks
in their oriented q-space pattern, while there are only three
observed experimentally; further examination showed the third
disagreement: in their simulation, the chains from the two monolayers 
are not parallel, as in our model, but pleat toward each other. 
They first thought that this pleated structure was not the cause of the
first two disagreements and they argued that it was a secondary chain 
packing structure that gave rise to those disagreements. But as we
looked carefully at their results, we didn't observe any secondary
structures. It is now agreed that the pleated structure was the cause 
of the disagreements. This appendix describes the argument 
I developed to demonstrate this. 

In the following sections, a simple but general model is worked out
to include both the parallel and the pleated chain packing structures.
It will be shown that the the pleated structure gives qualitatively
different scattering patterns from the parallel structure and the
pleated structure is indeed the cause of the broadness of the (11)
peak and the five diffraction peaks in Tu et al.'s simulation results.

\section{Theory}

In our simplified model, I consider only hydrocarbon chains, and assume 
that the chains are long, cylindrical rods (i.e. all-trans conformation), 
and are packed in an ideal 2-D orthorhombic lattice within the membrane
plane. Also the tilt angle is fixed to Tu's 33$^{\circ}$ and the tilting 
direction is relative to the long side of the orthorhombic unit cell (see
the angle $\sigma$ in Fig.\ref{chain}). The offsets between
the two monolayers are denoted $\Delta x0$ and $\Delta y0$ (not shown
in Fig.\ref{chain}), and it will be shown that small offsets play a rather 
minor role in the overall scattering.

\begin{figure}
\centerline {\psfig{figure=appendix_dir/pleat_klein.eps,width=4in}}
\caption{The unit cell used in this model. It shows the midplane projections
of the two opposing monolayers. Solid line arrow is the chain
tilting direction in the upper monolayer, and the dashed-line arrow is
that for the lower monolayer. Angle $\sigma$ is the angle made by the
tilting directions to $\vec{a}$. This figure shows the special case in which
the chains in both monolayers are tilted towards nearest neighbors. The chains
in opposing monolayers are pleated against each other.
\label{chain}}
\end{figure}

Given a scattering vector $\vec{q}$, the upper monolayer
gives a form factor $F_{u}(\vec{q})$ and the lower monolayer gives
$F_{l}(\vec{q})$. If the two monolayers scatter incoherently, the scattering
intensity will be

\begin{eqnarray}
\label{inco}
I(\vec{q}) = |F_{u}(\vec{q})|^2 + |F_{l}(\vec{q})|^2.
\end{eqnarray}
If the two monolayers scatter coherently (As in both ours and Tu's
results), the overall form factor is

\begin{eqnarray}
\label{form}
F(\vec{q}) = F_{u}(\vec{q}) + F_{l}(\vec{q}),
\end{eqnarray}
and the total scattering intensity is

\begin{eqnarray}
\label{inten}
I(\vec{q}) && = |F(\vec{q})|^2 \nonumber \\
&& = |F_{u}(\vec{q})|^2 + |F_{l}(\vec{q})|^2 + F^{*}_{u}(\vec{q}) F_{l}(\vec{q}) + F_{u}(\vec{q}) F^{*}_{l}(\vec{q}) \nonumber \\
&& = |F_{u}(\vec{q})|^2 + |F_{l}(\vec{q})|^2 + 2 Re[F^{*}_{u}(\vec{q}) F_{l}(\vec{q})].
\end{eqnarray}
The first two terms are the scatterings from the two monolayers.
The third term in the final equation is the interference between the two 
monolayers. It plays an important role in fine-tuning the diffraction 
profiles. The effects of the offsets will show up in this term.

Now let us proceed into the analytical calculation of the form factors.
It could be shown that the form factor for the upper monolayer is

\begin{eqnarray}
\label{formu}
F_{u}(\vec{q}) = \left\{ 
	\begin{array}{ll}
	0	& \mbox{if M+N are odd,} \\
	\frac{4}{\gamma}\ e^{i \frac{\gamma L}{2}}\ \sin(\frac{\gamma L}{2}) & \mbox{if M+N are even,}
	\end{array}
\right.
\end{eqnarray}
where: \( \gamma = (q_{x} \cos{\sigma} + q_{y} \sin{\sigma}) \tan{\theta_{t}} + q_{z} = (M \frac{2\pi}{a}\ \cos{\sigma} + N \frac{2\pi}{b}\ \sin{\sigma})\ \tan{\theta_{t}} + q_{z} \) ,\( L = l \cos{\theta_{t}}
\), $l$ is the chain length of one lipid molecule (20 \AA\ for DPPC), $\theta_{t}$
is the tilt angle, and (M,N) are the indices for scattering peaks. We are 
interested in the (2,0) and (1,1) peaks.

Similar to the calculation of the form factor of the upper monolayer, the form 
factor for the lower monolayer is

\begin{eqnarray}
\label{forml}
F_{l}(\vec{q}) = \left\{
	\begin{array}{ll}
		0       & \mbox{if M+N are odd,} \\
		e^{i \phi}\ \frac{4}{\kappa}\ e^{i \frac{\kappa L}{2}}\ \sin(\frac{\kappa L}{2}) & \mbox{if M+N are even,}
	\end{array}
\right.
\end{eqnarray}
where  \( \kappa = (q_{x}\cos{\sigma} - q_{y}\sin{\sigma}) \tan{\theta_{t}} 
- q_{z} = (M \frac{2 \pi}{a}\cos{\sigma} - N \frac{2 \pi}{b}\sin{\sigma})\ 
\tan{\theta_{t}} - q_{z} \), and \( \phi = q_{x}\ \Delta x0 + q_{y} \Delta 
y0 = M \frac{2 \pi}{a}\ \Delta x0 + N \frac{2 \pi}{b}\ \Delta y0 \). 

Combining Eqns.\ref{inten}, \ref{formu}, \ref{forml}, the overall scattering
intensity (for even M+N) of the bilayer will be

\begin{eqnarray}
\label{intenfi}
I(\vec{q}) = && \frac{16}{\gamma^2}\ \sin^{2}(\frac{1}{2} \gamma L) \nonumber \\
	&& + \frac{16}{\kappa^2}\ \sin^{2}(\frac{1}{2} \kappa L) \nonumber \\
	&& + \frac{32}{\gamma \kappa}\ \sin(\frac{1}{2} \gamma L)\ 
	\sin(\frac{1}{2} \kappa L)\ \cos[\frac{1}{2}(\kappa-\gamma) L + \phi].
\end{eqnarray}
In this simplified model, each monolayer's contribution is a square--sinc term,
with the full width inversely proportional to $L/2$, and the interference between
the two monolayers is the product of two sinc functions, modulated by a factor
which includes the phase difference of the two monolayers and the offsets. As
can be seen, the offsets only appear in the interference term.

\section{Results and discussion}

Using Eq.\ref{intenfi}, let us check two special cases: 

\begin{itemize}

\item[(a).] Collinear chains without offsets, $\phi = 0$,
$\sigma = 90^{\circ}$. Since $\kappa = - \gamma$, the two monolayers
scatter to the same $q_z$, and their interference is very strong. Eq. 
\ref{intenfi} gives an enhanced sharp scattering peak along each
of the Bragg rods we are interested in:

\begin{eqnarray}
\label{colin}
I(\vec{q}) = \frac{16}{\gamma^2}\ \sin^{2}(\gamma L),
\end{eqnarray}
which is our old result (the $q_{r}$-$q_{z}$ space pattern is shown
in Fig.\ref{sun}). 

\item[(b).] Pleated chains with no offsets. 
$\phi = 0$, $\sigma = \tan^{-1}(\frac{b}{a})$ (see Fig. \ref{chain}).
The q-space pattern is shown in Fig.\ref{klein} (using 
a=9.52\AA,\ b=4.87\AA,\ $\sigma=27.1^{\circ}$, and $\theta_{t}=33^{\circ}$).
This picture is surprisingly close to Tu's $q_r$-$q_z$ pattern, meaning 
our model caught the essential features of their simulation structure. The small
deviations from their results, e.g., the exact peak positions and intensities,
could be compensated by considering non-zero offsets, relaxing the orthorhombic
requirement, and allowing chains to be tilted towards non-nearest neighbor
directions. 

There are a few points about this pleated result:
\begin{enumerate}
\item There are five spots in q-space compared to the parallel case, which 
has only three. This fact could be understood by noticing that, with chains
pleated, two opposite monolayers scatter in different $q_z$ directions.
\item The peak widths along the $q_{z}$ direction are twice as wide as in our 
parallel case. This could not be compensated by considering any other degrees
of freedom. The physical mechanism behind this broadening is just the same
as in the above point, namely, each peak comes from one single monolayer, which
would not yield an interference-enhanced sharpening as in the parallel
case.
\item By using the unit cell in this appendix, the indexings of q-space
peaks are permuted from Tu's result.
\item The two extra (20) peaks in Tu's result are from the same 
structure which gives all other main peaks, not from a separate structure
as thought by Tu et. al..
\item Because in this pleated structure, the two monolayers scatter in
different $q_z$ directions, the offsets could not bring them together,
simply because $|\kappa|$ does not equal $|\gamma|$. 
\end{enumerate}

\end{itemize}

\begin{figure}
\centerline {\psfig{figure=appendix_dir/colli.ps,width=4in}}
\caption{The $q_r$--$q_z$ pattern for the parallel case. There are only three
peaks which are twice as sharp as in the pleated--chain case along the $q_z$
direction. The horizontal axis is the $q_r$ axis and the vertical axis is 
the $q_z$ axis. The units of both axes are in \AA$^{-1}$.
\label{sun}}
\end{figure}

\begin{figure}
\centerline {\psfig{figure=appendix_dir/md.ps,width=4in}}
\caption{The $q_r$--$q_z$ pattern for the pleated--chain case corresponding to 
Tu's MD simulation results.
The horizontal axis is the $q_r$ axis and the vertical axis is 
the $q_z$ axis. The units of both axes are in \AA$^{-1}$.
 Parameter values are derived from Tu's
$q_r$--$q_z$ pattern. a=9.52\AA,
\ b=4.87\AA,\ $\sigma= 27.1^{\circ}$, and $\theta_{t}=33^{\circ}$. 
\label{klein}}
\end{figure}
