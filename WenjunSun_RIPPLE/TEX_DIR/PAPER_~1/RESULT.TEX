\section{Results}

The upper portion of Figure \ref{wack_data} shows the scattering
intensity data in Wack and Webb \cite{Wac89a}, replotted in the 2D 
reciprocal space. The lower portion of Fig. \ref{wack_data} shows
the real space ripple structure obtained by nonlinear-least-square 
fit of the above data to the model proposed in the last section. 
The correlation between the normals of the two sides of the ripples 
and the angular scattering intensity distribution in the q-space is shown
by the dot-dashed lines in Fig. \ref{wack_data}, as discussed in 
Section II(C) and shown in Fig. \ref{q_pattern}.

The structural quantities of the ripple phase obtained from the fit
are shown in Table \ref{parameter}. The primary parameters are the
fitting parameters of the model, and the secondary parameters are
derived from the primary ones. Because $\psi$ = 5$^{\circ}$, 
the upper monolayer is slightly shifted to the left of the lower
monolayer, following the oblique direction of the ripple unit
cell (See Fig. \ref{wack_data}). The ratio of the major side length to the 
minor side length is 2.4, which is a good parameter 
to describe the degree of the ripple asymmetry. The minor side
is steeper than the major side. Both the bilayer thickness and water spacing 
along the normal direction of the major side are larger than those of 
the minor side.

The form factors, both from Wack and Webb's data \cite{Wac89a} and
from the model fit, are given in Table \ref{formfactor}. The phases
of the form factors are outputs of the model fit and listed
in the last column of Table \ref{formfactor}. 
%The agreements of the stronger peaks between the model fit and the data 
%are vary good.
With the obtained phasing information, the electron density maps, both from 
the data and the model fit, are Fourier reconstructed and shown in 
Fig. \ref{wack_map}(A) and Fig. \ref{wack_map}(B), respectively.
%Despite some minor differences in the details of the actual electron density,
%the agreements between the two maps are surprisingly well, especially
%the main structural features.

Although the electron density maps are helpful in getting the
overall view of the ripple phase structure, slice electron
density profiles are needed to obtain accurate ripple structural
quantities along some special directions. Three such slices
are taken and the electron density profiles along them are
shown in Figure \ref{slice}. Slice B is along the normal of the
major side and the two electron density profiles (data and model)
along it are shown in Fig. \ref{slice}(B), from which the bilayer
head-head spacing cross the major side is found to be about 38\AA,
and the water spacing is about 21\AA. Both the data and model maps
give the same spacings. For the minor side, two slices are taken,
one going through the center of the chain region (Slice C in
Fig. \ref{slice}(A)) and another through the center of the water
region (Slice D in Fig. \ref{slice}(A)). The electron density
profiles along slices C and D are shown in Fig. \ref{slice}(C) and
(D). The bilayer head-head spacing on the minor side is 31\AA, smaller
than that of the major side (37\AA). The water spacing of minor side
is 18\AA, also smaller than that of the major side (21\AA). Again,
both the data and model maps give the same spacings. It is noticeable
that all these spacings from the slice electron density profiles are
slightly different from the corresponding spacings directly derived from the
best values of fitting parameters (See Table \ref{parameter}). 
These differences are caused by the well-known Fourier truncation error
when only limited numbers of diffraction orders are used in the Fourier 
reconstruction. Since in Wack and Webb's data, only the form factors 
up to three lamellar orders were measured and reported, these truncation 
effects are not unexpected.

In order to test the robustness of this model and the phases obtained
from it, a few variations of this model were used to fit the same data.
When the monolayer shift angle, $\psi$, was set to either zero or
to the value of $\alpha$ (the slope of the major side), 
similar values of fitting
parameters and identical phases were obtained (not shown), although
the fits were slightly worse due to the constraints. Another variation
was to adopt the results of Wiener's \cite{MWTH} 1G hybrid model fit
to the DPPC gel phase data. Most of Wiener's best fit values were
retained while the head-head distance was set free because DMPC molecules
are two carbons shorter on each chain compared to DPPC. After incorporating
this variation into our model as the transbilayer electron density, a fit
was performed with surprising results: almost identical values of the primary
fitting parameters and identical phases were obtained (not shown). 

%This 
%implies that the ripple phase has similar local structures as the gel phase. 
%Certainly it is desirable to replace the delta function model with the 1G 
%hybrid model, but because Wack and Webb's ripple phase data only reported
%three lamellar orders, while the 1G hybrid model requires more than that not 
%to be underdetermined, we stuck to the delta function model, which turned 
%out to be working well.

We also performed x-ray experiments on a DMPC sample with almost identical
hydration level as in Wack and Webb's \cite{Wac89a}. The aim was to test
the hypothesis that the ripple phase has a gel-phase-like local bilayer
structure. The sample was partially dehydrated by a 40\% Polyvinylpyrrolidone
(PVP) aqueous solution (an aqueous solution of the polymer PVP 
exerts an osmotic pressure
on the lipid bilayers, i.e., PVP molecules compete with the bilayers for
water without entering into or in between the bilayers). At 18$^{\circ}$C, 
this sample was in the ripple phase, with a d-spacing of 58.10\AA, comparing 
well with Wack and Webb's 57.94\AA. This implies that our DMPC sample was at 
about the same hydration level as Wack and Webb's sample. We then took
the low angle data at 6$^{\circ}$C at which the sample was clearly in the
gel phase with no ripple peaks present. Five orders were recorded. The electron
density profile of this DMPC sample at 6$^{\circ}$C was reconstructed using
the first three orders, in accord with the lamellar orders in Wack and Webb's
data, and compares well with the major side slice profile of Wack and Webb's
data electron density map (See Fig. \ref{gel_rppl}). The slight difference
in the headgroup positions of the two profiles in Fig. \ref{gel_rppl} could
be due to the small difference in the d-spacings. This gel phase profile
gives a head-head bilayer thickness of 36.96\AA\ and a water spacing of
20.79\AA, agreeing well with the ripple phase's 37.86\AA\ and 20.73\AA.
The electron density profile reconstructed with the first four orders
gives a head-head spacing of 40.42\AA\ and water spacing of 17.33\AA, 
comparing with 40.0\AA\ and 18.5\AA\ given by the ripple phase model
fitting results (see Table \ref{parameter}). Five order reconstruction
gives 38.12\AA\ and 19.63\AA, close to those of the three order reconstruction.

\pagebreak
