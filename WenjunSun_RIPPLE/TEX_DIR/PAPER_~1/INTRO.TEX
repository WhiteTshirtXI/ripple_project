\section{Introduction}

Lipid molecules are known to self-assemble into lamellar bilayers when in
water, which are the basic structures of biomembranes. Moreover,
lipids are also lyotropic liquid crystals with rich phase behavior
and diversified structures. Among the lamellar phases observed in
lipid bilayer systems, the structure and formation mechanism of the 
periodically modulated 2D ripple (P$_{\beta '}$) phase are not well
characterized, despite many experimental studies 
\cite{Tar73,JanSS76,JanSS79,Luna77,Ino78,%
Ino80,Cop80,Sta82,Rup83,HenHos83,Ale85,Hic87,Zas87,Zas88b,Sir88,Morten88,%
Wac89a,HenRus91,Cev91,Wol92,Matu90,Kat95} and theoretical works
\cite{Don79,Mar84,Haw86,Car87,Gol88,McCu90,Hon91,Lub93}. 
In contrast, the two adjacent lamellar phases,
the lower temperature gel (L$_{\beta '}$) phase and the higher temperature 
liquid crystalline (L$_{\alpha}$) phase, have been better characterized.
Experimental determination of the exact ripple structure can serve both as a
criterion and a guide for theories of ripple phase formation. By comparing
the relative size of the ripple wavelength and the ripple amplitude to the 
size of lipid molecules and the bilayer thickness, some hints of the 
energetics of the ripple formation \cite{Gol88} may be obtained. In addition, 
when an accurate ripple profile is known, one can study the correlation 
between the chirality of the constituent lipid molecules and the symmetry of 
the ripples, which may in turn lead to elucidation of the microscopic origin 
of the macroscopic properties of the ripple phase \cite{Kat95,Lub93}.

The structural information of the ripple phase can be divided into two
groups. The first group consists of the 2D ripple lattice parameters, namely, 
the ripple wavelength $\lambda_r$, the bilayer packing repeat distance $d$ and
the unit cell oblique angle $\gamma$. This group of parameters can be easily 
obtained by indexing low-angle x-ray scattering peaks and have been well 
established by many research groups \cite{Tar73,JanSS79,Ino80,Ale85,Wac89a}. 
For example, Wack and Webb \cite{Wac89a} reported that for DMPC with
25\% water by weight at 18$^{\circ}$C, $\lambda_r$ = 141.7\AA, $d$ = 57.94\AA\
and $\gamma$ = 98.4$^{\circ}$.  The second group
consists of the detailed structure within the ripple unit cell, including
the shape and the amplitude of the ripples and the bilayer thickness. The
information about this second group lies in the intensity distribution of the 
x-ray scattering peaks. It is this second group which is subject to a large
 ambiguity. Until now, many possible ripple profiles have been 
proposed \cite{Tar73,JanSS79,Sta82,Don79,Mar84,Car87,Gol88,McCu90,Hon91,Lub93},
including sinusoidal, peristaltic and sawtooth, but few of them can stand the 
test of experimental facts. Published estimations on the ripple trough-to-peak 
amplitude range widely from 15\AA\ to 50\AA\ 
\cite{Tar73,JanSS79,Sta82,Zas87,Zas88b,Zas88a,Hata93},
obtained mostly from x-ray and freeze fracture electron microscopic works.

%It will also be helpful to compare the ripple structural features with that 
%of the gel and fluid phases, such as to see their relations and transitions.

%But it has been long overlooked that in a structure with periodic modulation
%like the ripple phase, the out-plane buckling of the bilayers can lead to
%an increase in the apparent d-spacing along the normal direction of the
%average membrane plane, while the structure along the local normal may remain
%less altered. The local d-spacing is simply the product of the apparent
%d-spacing and the cosine of the local ripple slope. A big increase in the
%apparent d-spacing may be just caused by a bigger slope, not necessarily
%implying a big change in the local d-spacings, which is much more relevant
%to the interbilayer interactions and the hydrocarbon chain configurations. 
%To get averaged local area per lipid molecular, lipid bilayer thickness and
%the water spacing, the amplitude and the profile of the ripples have to be
%accurately obtained.

Electron density maps are helpful and sometimes necessary to determine the 
structure of the ripple phase. In order to Fourier reconstruct the electron 
density maps from the experimentally measured form factor amplitudes, the
correct phases must be known. Unlike the amplitudes of the form factors,
the phase information is not contained in the intensity data and has to be 
sought using other means. There are two approaches to obtain the phase 
information. The more traditional approach is the ``pattern recognition''
method employed by Tardieu et al. \cite{Tar73}, which is to 
select the phase combinations which will give rise to electron density maps 
agreeing with known physical properties of the bilayer system, out of all 
possible phase combinations. Considering that there are typically 20 or so 
peaks in most of the ripple phase x-ray data, finding the right phase 
combination out of about 2$^{20}$ possibilities is a difficult task. 
The second approach involves modeling and nonlinear least square fitting.
This approach requires the following steps: 1) Build up an electron density 
model by incorporating various known physical properties 
of the bilayer; 2) Fourier transform the model electron density to get model 
form factors (or scattering intensities); 3) fit the experimentally measured
amplitudes of form factors (or intensities) to the model and get the structural
quantities and the phases. 

%The difficulty associated with this approach can be seen from the reconstructed
%electron density map of DLPC published in \cite{Tar73} by Tardieu et al.,
%where one sees rather irregular bilayers with an arrangement of the
%ripples within the unit cell which does not give rise to the typical q-space
%ripple phase patterns shown in oriented sample x-ray data \cite{Ale85,Kat95}.

%It should be emphasized that in 
%this approach, phases are outputs instead of inputs as in the first approach
%and are not essential to obtaining the structural quantities, but are very
%useful in reconstructing the electron density map from the data and comparing
%that with the map from the model fit.
%Wiener et al. \cite{WSN89} used this approach successfully to obtain many
%key structural quantities of gel phase DPPC bilayers. 
%Janiak et al. \cite{JanSS79} also followed this route and tried to solve 
%the ripple phase structures of DMPC at two different hydration levels. 
%But because their data were from low resolution measurements, which made
%the decomposition of overlapping peaks subject to large ambiguity, their
%results are not definitive. Nonetheless, they obtained a ripple amplitude
%of about 15\AA\ for DMPC at two hydration levels, which compares with
%15\AA\ for DLPC reported by Tardieu et al. \cite{Tar73}.

%In a recent experimental paper, Katsaras and Raghunathan \cite{Kat95} showed 
%that molecular chirality does not affect the asymmetry of the ripples, contrary
%to the theoretical prediction of Lubensky and MacKintosh's model \cite{Lub93}.
%Katsaras and Raghunathan employed a qualitative argument proposed by
%Tardieu \cite{Tar72}, in which a connection between ripple unit cell
%obliquity and ripple asymmetry was made. In this paper, it will be shown
%that the degree of ripple asymmetry can be more directly and quantitatively 
%obtained from the model fit.

The task of this paper is to propose a more precise electron density model 
which is consistent with both available experimental results and known 
physical properties of the lipid bilayers, and to fit Wack and Webb's
published high resolution x-ray form factor amplitude data \cite{Wac89a}
 to this model. Both accurate structural quantities and form factor phases 
are obtained from the fit. Ripple electron density maps, both from Wack 
and Webb's data and the model fit, are Fourier reconstructed using the 
phases obtained from the model fit. The major side of the ripple is 
found to have a gel-phase like structure. Comparisons with previous ripple 
phase works are presented. Implications of the model fit results in 
understanding the ripple phase structure will be discussed.
