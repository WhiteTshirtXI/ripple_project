\section{Discussion}

It is surprising to find out that such a simple model like the one
presented in this paper can be very powerful and robust. This can
be seen from the agreements, both quantitative and qualitative,
between the experimental data and the model fit (Fig. \ref{wack_data} and
Table \ref{formfactor}), and the realistic quality of the reconstructed 
electron density maps, both from the data and the model fit 
(Fig. \ref{wack_map}). We think that the success of this model is
due to the fact that it incorporates the most essential features of the rippled 
bilayers, similarly to the finding of Wiener, Suter and Nagle
\cite{WSN89} from their work solving the gel phase structure using
the modeling approach. The results of electron density modeling are not 
dependent upon the particular analytical form of the density profile but 
instead require only that certain structural features of the bilayer be 
presented as distinct entities in the model.

The first important results of this work are the structural quantities
of the ripple phase extracted from the model fit (Table \ref{parameter}). 
As far as we know, this work gives the most complete and accurate 
determinations of the ripple structure to date. The
ripple amplitude from this work, 18.6\AA, compares well with the estimated
values of 15\AA\ from Tardieu et al. \cite{Tar73} for DLPC and 15\AA\ from 
Janiak et al. \cite{JanSS79} for DMPC. This work also indicates that the 
major side of the ripple is about 2.4 times larger than the minor side, thus 
quantitatively showing the degree of ripple asymmetry. Furthermore, the 
closeness of the three angles (see Figs. \ref{trans}, \ref{wack_data}, 
\ref{wack_map} and Table \ref{parameter}), $\gamma$-90$^{\circ}$ 
(8.4$^{\circ}$), the major side slope $\alpha$ (10.2$^{\circ}$) and the 
monolayer shift angle $\psi$ (5$^{\circ}$), implies that the unit cell
obliquity originates from the asymmetric ripple profile and the way the
two opposing monolayers are shifted relative to each other. This combination
of the three angles gives a balanced and physically realistic bilayer
packing, as can be seen clearly from the reconstructed electron density
maps (Fig.\ref{wack_map}). In contrast, one highly unlikely case is shown
is Fig. \ref{trans}(B). As mentioned in the Introduction of this
paper, Tardieu \cite{Tar72} and Katsaras et al. \cite{Kat95} have made
a qualitative inference of the ripple asymmetry from the obliquity of the 
ripple unit cell. This work puts that on a more solid quantitative basis.

The second contribution of this paper is to derive the phases of the
form factors from the model fit (Table \ref{formfactor}). Determining
the correct phases is one half of the task in electron density map
reconstruction using Fourier transform (the other half is to measure accurately
the amplitudes of the form factors). As mentioned in the Introduction
, the 'pattern recognition' method employed by Tardieu et al. 
\cite{Tar73} is subject to large ambiguity and difficulty when there are 
many peaks. As has been shown in this paper, the modeling approach is more
reasonable. The correctness of the phases determined by this work
can be seen clearly from our reconstructed electron density maps 
(Fig. \ref{wack_map}), which give realistic ripple phase structures. To 
compare with previous ripple phase work and
the importance of obtaining the corrected phases, additional electron density
maps were reconstructed and are shown in Fig. \ref{others}.
Fig. \ref{others}(A) shows Tardieu et al.'s result, which does show
a bilayer structure, but the headgroup regions are rather
irregular. Figs. \ref{others}(B) and (C) are Janiak et al.'s results 
from their modeling work and it is difficult to see 
the typical bilayer structure 
of a lipid/water system.  Fig. \ref{others}(D) is reconstructed from Wack and 
Webb's data \cite{Wac89a} with random phase assignments. By comparing 
Fig. \ref{others}(D)
with Fig. \ref{wack_map}(A), one can see the importance of 
correct phase determination.

The asymmetry of the ripple profile implies that the uniformity of the 
structures within one single bilayer, which is the characteristic of flat 
bilayers like in the gel and fluid phases, is broken for the 
ripple phase. This
can be seen in more detail in the slice electron density profiles taken
from the major and the minor sides of the ripples (See Fig. \ref{slice}).
Fig. \ref{slice} shows that the major side has a thicker bilayer and a wider
water spacing along its local normal direction than that of the minor sides.
The slices from both the data map (Fig. \ref{wack_map}(A)) and the model fit
map (Fig. \ref{wack_map}(B)) yield very similar spacings both on the major
and the minor sides, although the exact
electron density profiles are slightly different in detail. This shows that
our model is good at obtaining the structural quantities despite its
simplicity, concurring well with the point we made in the first paragraph of
the Discussion. The values for these local spacings extracted from the primary
fitting parameters as shown in Table \ref{parameter} are different from the
ones obtained from the slice profiles by a few Angstroms, but the qualitative
conclusion that the major side has a thicker bilayer and a wider water spacing
remains the same. Those differences are due to the Fourier truncation 
because only up to three lamellar orders were used in reconstructing the
electron density maps shown in Fig. \ref{wack_map}, as limited by the 
experimental data.

In a recent experimental paper, Katsaras and Raghunathan \cite{Kat95} showed 
that molecular chirality does not affect the asymmetry of the ripples, contrary
to the theoretical prediction of Lubensky and MacKintosh's model \cite{Lub93}.
Katsaras and Raghunathan employed a qualitative argument proposed by
Tardieu \cite{Tar72}, in which a connection between ripple unit cell
obliquity and ripple asymmetry was made. We have shown in this paper
that the degree of ripple asymmetry can be more directly and quantitatively 
obtained from the model fit.

In testing the robustness of our model, we tried to adopt Wiener's 
\cite{MWTH} 1G hybrid model for the gel phase bilayers as the transbilayer 
profile. 
%All of the key parameters were fixed to his best fit values 
%for gel phase DPPC except the bilayer head-head spacing, which was set free. 
The fit turned out to be almost identical to the best fit we reported in this 
paper, with similar structural quantities and identical phases. 
This result suggests that the ripple phase bilayers basically retain the 
local structure of the gel phase bilayers. This point is made stronger by the
direct comparison of the gel phase DMPC (with similar hydration level
as Wack and Webb's DMPC sample) electron density profile to the
slice profile of the major side as shown in Fig. \ref{gel_rppl}. Therefore,
from these two results, we can infer that the major side of the ripple 
phase has the gel phase like structure, i.e. all trans chains, similar
local chain tilt angle, similar local bilayer thickness and water spacing.
This conclusion agrees well with the volumetric \cite{NW78} and the
calorimetric \cite{Cev87} results that the volume change and enthalpy
change at the pretransition (gel to ripple transition) have little
chain length dependence, which imply that the chains are not very much
involved in the pretransition, in contrast to the chain-melting or
main transition, at which both the volume and enthalpy changes showed
strong chain length dependence \cite{NW78,Cev87}.

While this model showed its power in obtaining the key structural quantities
and the phase information of the ripple phase, it is nonetheless a very
simply model. Close examination of Table \ref{formfactor} showed that
a few of the weaker peaks were not fitted perfectly. That discrepancy
manifested itself in the differences of the exact electron density values
between the two maps in Fig. \ref{wack_map}, and between the profiles in 
Fig. \ref{slice}, although one should notice that the structural features
between the data and fit maps/profiles are almost identical. We attribute
the discrepancy to the simplicity of our model and the limited number of
diffraction peaks in the data. The data in \cite{Wac89a} only reported
up to three lamellar orders. When we tried to fit the data to a more 
refined model like the 1G Hybrid model in \cite{WSN89} with all parameters
freed, the fits fell into some local minima and gave unrealistic ripple
structures. In another word, the fitting parameters were underdetermined 
when the data were limited. If more complete data are available, this model 
could be refined to incorporate more features, like rounded corners for the 
ripple profile and more detailed transbilayer electron density profiles like 
the ones used in the gel phase modeling work of Wiener, Suter and Nagle 
\cite{WSN89}.

It is highly desirable to systematically study the dependence of
the ripple structure on various physical and chemical variables,
e.g. the temperature, the hydration level and the chain length.
Until now, only the dependences of the ripple unit cell
lattice parameters, i.e. the ripple wavelength, the bilayer stacking
repeat and the unit cell obliquity angle, on some of these variables
have been  studied and reported \cite{Wac89a,Matu90,Cev91}. With the help
of our model, the detailed ripple features can be obtained and used
to help elucidate the mechanism of the ripple phase formation. 

In summary, we have proposed a very simple and realistic electron density 
model for the ripple phase of lecithin bilayers. Excellent fits were 
achieved when fitting the experimental data to this model. Many structural
quantities of the ripple phase and the phases of the form factors
were extracted. For the first time, an accurate electron density map of the
ripple phase, which agrees with available experimental results and known
physical properties of the lipid bilayers, was obtained. This model was found 
to be very robust when subjected to various variations. The rippled bilayers 
have an asymmetric profile and gel-phase like structure, with the two 
opposing monolayers in the same bilayer shifted relative to each other in a 
way which gives rise to the obliquity of the unit cell. This work can be 
used for further systematic studies of the ripple phase structures.

We would like to thank Dr. Ruitian Zhang, Mr. Ralf Heilmann and Mr. Horia 
Petrache for stimulating discussions. This work was supported by NIH Grant 
No. GM44976.

%----------------------------------------------------------------------------
%Close scrutiny showed
%that they were rather unsuccessful. The electron density maps reconstructed
%from their published model fit form factors suggested no bilayer
%feature which should be a part of the lipid system. Given the lack of
%details of their models in the paper \cite{JanSS79}, one can nonetheless
%guess the reasons for their unsuccessful attempts: (1) They used a symmetric
%ripple profile, without realizing that the obliquity of the ripple unit cells
%suggested asymmetric ripples; (2) Their data were low resolution x-ray
%scattering intensity measurements from unoriented samples, which made the
%indexing and decomposition of overlapping peaks very difficult. Considering
%the fact that many later high quality oriented sample data \cite{Ale85,Kat95}
%and high resolution x-ray data \cite{Wac89a} were not available back then, their
%efforts are rather respectable.
%
%Knowing accurately the detailed ripple structure within the unit cell is 
%essential to the determinations of other key structural parameters of the 
%ripple phase, like the area per lipid molecule, inter-bilayer water spacing. 
%Incorrect ripple shapes and profiles or the lack of them will lead to
%wrong or incomplete answers. Parsegian \cite{Pars83} first proposed using 
%both a sinusoidal and a symmetric triangular 
%profiles to evaluate average local area per lipid. But he used the ripple 
%amplitude (50\AA\ peak-to-trough) given by Stamatoff et al. \cite{Sta82}, 
%which was obtained from an incorrect analysis, as pointed out by Wack and 
%Webb \cite{Wac89a}. Parsegian's calculation gave an averaged local area per 
%lipid molecule of about 68\AA$^2$ for ripple phase DPPC, directly comparable 
%to the area per lipid in the fluid phase, 71\AA$^2$ \cite{Pars83}, but much 
%larger than the gel phase area per lipid, 47.9\AA$^2$ \cite{Sun94}. 
%Detailed analysis 
%showed that Parsegian's local area/lipid will require a chain tilt of about 
%50$^{\circ}$ relative to the local bilayer normal. This tilt angle is bigger 
%than the slope of the ripple (about 35$^{\circ}$ using a ripple amplitude 
%of 50\AA\ and a symmetric ripple profile), thus contradicts 
%the model of Stamatoff et al. \cite{Sta82} in which the chains are proposed 
%to be perpendicular to the average bilayer plane. Using gravimetric method 
%\cite{Luz68} , Janiak et al. \cite{JanSS79} obtained the apparent
%area per DPPC molecule in the ripple phase of about 53.5\AA$^2$, and Wack
%et al.\cite{Wac89a} gave a value of about 52.7\AA$^2$, but no average local
%area per lipid was given in either of these two works \cite{JanSS79,Wac89a},
%due to the lack of information on the ripple profile. 
