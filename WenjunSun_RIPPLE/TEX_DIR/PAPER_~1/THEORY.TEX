\section{Theory}
%----------------------------------------------------------------------------
\subsection{Ripple lattice structure}

It has been shown from x-ray experiments \cite{Tar73,JanSS79,Ino80,Ale85,%
Wac89a,Kat95} that ripples in different bilayers are registered to form a 
2-D oblique lattice, which can be adequately described by three parameters: 
the ripple wavelength $\lambda_r$, the oblique angle $\gamma$ and the stacking
repeat $d_s$ (See Figure \ref{unitcells}(i)). The lamellar repeat along 
the average bilayer
normal direction, $d$, is related to $\gamma$ and $d_s$ by

\begin{eqnarray}
\label{dspace}
d = d_s\ \sin \gamma.
\end{eqnarray}

By choosing the ripple wave vector direction as the $x$ axis and the average
bilayer normal as the $z$ axis, the ripple unit cell vectors can be 
expressed as (Figure \ref{unitcells}(ii)):

\begin{eqnarray}
\label{realunit}
\left\{ \begin{array}{ll}
\vec{a} = d_s \cos \gamma\ \hat{x} + d\ \hat{z},\\
\vec{b} = \lambda_r\ \hat{x}.
\end{array}
\right.
\end{eqnarray}

The corresponding reciprocal lattice unit cell vectors are 
(Fig. \ref{unitcells}(ii)):

\begin{eqnarray}
\label{reciunit}
\left\{ \begin{array}{ll}
\vec{A} = \frac{2 \pi}{d}\ \hat{z},\\
\vec{B} = \frac{2 \pi}{\lambda_r}\ \hat{x} - \frac{2 \pi}{\lambda_r 
\tan \gamma}\ \hat{z}.
\end{array}
\right.
\end{eqnarray}
So the q-space vector for the Bragg peak with Miller indices $(h,k)$
is:

\begin{eqnarray}
\label{qvector}
\vec{q}_{hk} =&& h \vec{A} + k \vec{B} \nonumber\\
=&& \frac{2 \pi k}{\lambda_r}\ \hat{x} + \left( \frac{2 \pi h}{d} -
\frac{2 \pi k}{\lambda_r \tan \gamma} \right) \hat{z}.
\end{eqnarray}

%----------------------------------------------------------------------------
\subsection{Electron density model and form factor}

The electron density $\rho (x,z)$ within the ripple unit cell is the 
convolution of the ripple contour function $C(x,z)$ and the trans-bilayer 
electron density profile $T(x,z)$:

\begin{eqnarray}
\label{model}
\rho (x,z) = C(x,z) \ast T(x,z),
\end{eqnarray}
where `$\ast$' stands for convolution. $C(x,z)$ is related to the ripple profile
$u(x)$ by:

\begin{eqnarray}
\label{contour}
C(x,z) = \delta [z - u(x)].
\end{eqnarray}

The form factor $F(\vec{q})$ is the Fourier transform of the electron density
expressed in Eqn.\ref{model}. Using the Convolution Theorem, we have:

\begin{eqnarray}
\label{formall}
F(\vec{q}) =&& {\cal F} \left[ C(x,z) * T(x,z) \right]\nonumber\\
=&& {\cal F} \left[ C(x,z) \right] \times {\cal F} \left[ T(x,z) \right]\nonumber\\
=&& F_C(\vec{q}) \times F_T(\vec{q}),
\end{eqnarray}
where ${\cal F} [\ ]$ represents the Fourier transform. From Eqn.\ref{formall}
we can see that the contour part of the form factor, $F_C(\vec{q})$, and the 
trans-bilayer part, $F_T(\vec{q})$, can be evaluated separately. It will be 
shown in later sections that generally $F_C(\vec{q})$ gives the angular 
scattering intensity distribution in q space.
%and $F_T(\vec{q})$ determines the radial 
%intensity distribution in the 2-D q-space.

In the following sections, detailed calculations will be performed using an
asymmetric triangular ripple profile and a simple trans-bilayer electron 
density profile.

%----------------------------------------------------------------------------
\subsection{Asymmetric triangular ripple profile}

The apparent asymmetry of ripples shown by freeze fracture electron 
microscopy \cite{Luna77,Cop80,Rup83,Zas87} and scanning tunneling microscopy 
\cite{Zas88a,Hata93}, combined with the obliquity of the ripple unit cell 
determined by x-ray investigations 
\cite{Tar73,JanSS79,Ino80,Ale85,Sir88,Wac89a,Kat95}, 
disqualifies any symmetric profiles, such as the sinusoidal or square-wave 
ones. The uneven angular scattering intensity distribution in 
space, clearly observable in oriented sample x-ray diffraction patterns 
\cite{Ale85,JanSS79}, also strongly suggests asymmetric ripple profiles.
In light of this experimental evidence, an asymmetric, 
triangular ripple profile seems like a good and simple candidate.
We will consider this profile in further details. In the rest of this paper,
we shall call the longer side of the ripple the {\it major} side, and the 
shorter side the {\it minor} side (See Fig. \ref{rppl_profile}).

As shown in Fig.\ref{rppl_profile}(i), the triangular ripple profile can 
be well described
by the amplitude $A$, the slope angle of the major side, $\alpha$, and
$x_0$, the projection of the major side onto the $x$ axis. Of these three
parameters, only two of them are independent and they are related by the 
following equation:

\begin{eqnarray}
\label{tripara}
A = x_0\ \tan \alpha.
\end{eqnarray}
We will choose $A$ and $x_0$ as the independent parameters of this model.
Using these parameters and putting the center of the major side of the ripple
at the origin of the coordinate system (see Fig.\ref{rppl_profile}(ii)), the 
ripple profile can be expressed as

\begin{eqnarray}
\label{profile}
u(x) = \left\{ \begin{array}{ll}
		-\frac{A}{\lambda_r - x_0} (x + \frac{\lambda_r}{2})
\ \ \ \mbox{if}\ - \frac{\lambda_r}{2} \le x < - \frac{x_0}{2};\\
		\ \ \ \ \ \frac{A}{x_0}\ x\ \ \ \ \ \ \ \ \ \ \ \ \mbox{if}
\ \ - \frac{x_0}{2} \le x \le 
\frac{x_0}{2};\\
		-\frac{A}{\lambda_r - x_0} (x - \frac{\lambda_r}{2})
\ \ \ \mbox{if}\ \ \ \ \  \frac{x_0}{2} < x \le \frac{\lambda_r}{2}.
		\end{array}
	\right.
\end{eqnarray}
This particular choice of writing $u(x)$ is to show explicitly the inversion
symmetry in the ripple profile, so that the following derivation of the
form factor can utilize this symmetry to get rid of unnecessary complex
phase factors.

Now, the contour part of the form factor, $F_C(\vec{q})$ (see Eqn.\ref{formall})
, is

\begin{eqnarray}
\label{formcon}
F_C(\vec{q}) =&& \frac{1}{\lambda_r}\ \int_{- \frac{\lambda_r}{2}}^{ 
\frac{\lambda_r}{2}} dx
\int_{- \frac{d}{2}}^{\frac{d}{2}} dz\ C(x,z)\ \mbox{e}^{i \vec{q} \cdot \vec{r}} \nonumber\\
=&& \frac{1}{\lambda_r}\ \int_{- \frac{\lambda_r}{2}}^{\frac{\lambda_r}{2}} dx
\int_{- \frac{d}{2}}^{\frac{d}{2}} dz\ \delta [z - u(x)] \ \mbox{e}^{i (q_x x + q_z z)} \nonumber\\
=&& \frac{1}{\lambda_r}\ \int_{- \frac{\lambda_r}{2}}^{\frac{\lambda_r}{2}} dx\ \mbox{e}^{i [q_x x + q_z u(x)]}.
\end{eqnarray}
By introducing an intermediate variable

\begin{eqnarray}
\label{w}
\omega (\vec{q}) = \frac{1}{2}\ ( q_x x_0 + q_z A),
\end{eqnarray}
and using the triangular ripple profile given in Eqn.\ref{profile}, 
Eqn.\ref{formcon} can be integrated out to be

\begin{eqnarray}
\label{formcon1}
F_C(\vec{q}) = \frac{x_0}{\lambda_r}\ \mbox{sinc}(\omega)
+ \frac{\lambda_r - x_0}{\lambda_r}\ \frac{\cos \frac{1}{2} (\frac{1}{2}
q_x \lambda_r + \omega)}{\cos \frac{1}{2} (\frac{1}{2} q_x \lambda_r -
\omega)}\ 
\mbox{sinc}(\frac{q_x \lambda_r}{2} - \omega ),
\end{eqnarray}
which is a real-valued quantity, as required by the inversion symmetry
of the ripple profile.
The first part in the parentheses of Eqn.\ref{formcon1} is the contribution from the major side of
the ripple, with scattering amplitude \(\frac{x_0}{\lambda_r}\), proportional
to its projection length onto the x axis, while the second part is from the 
minor side of the ripple, with amplitude \(\frac{\lambda_r - x_0}{\lambda_r}\).
If $x_0 > \frac{\lambda_r}{2}$, the major side should 
scatter more strongly than the minor side, and the minor side more weakly. 
It can be further shown that the maximum scattering 
of the major side occurs
along the direction determined by $\omega = 0$, which is the normal direction
of the major side of the ripple, with the direction unit vector \(\hat{n} =
- \sin \alpha \hat{x} + \cos \alpha \hat{z}\). Similarly, the maximum scattering
direction of the minor side is given by \( \frac{q_x \lambda_r}{2} - \omega 
= 0 \), and can be easily shown to be the normal direction of the minor side.
These observations can be seen clearly in Fig. \ref{q_pattern}.

Some important conclusions can be drawn from the analysis up to now:
(i) The q-space angular distribution of the scattering intensity from 
ripples is 
determined by the geometry of the ripple profile. Angular scattering maxima
occur at the average normal directions of the two sides of the ripple.
(ii) The uneven intensity distribution among the two maximum scattering
directions is due to the asymmetry of the ripple 
profile, namely, the major side of the ripple contains more scattering
material (lipid molecules), and thus scatters more strongly along its normal 
direction compared to the maximum scattering of the minor side, which 
has fewer lipid molecules. (iii) The experimentally observed uneven angular 
distribution of 
scattering intensity among Bragg peaks \cite{Ale85,Wac89a,Kat95} can be used 
to deduce the average ripple profile. These conclusions should apply to any 
ripple profiles, not limited to the triangular ones. 

Combining Eqn.\ref{qvector} and Eqn.\ref{formcon1}, the discrete contour form
factor for the Bragg peak with Miller indices $(h,k)$ can be expressed as:
\begin{eqnarray}
\label{discrete}
F_C(h,k) =&& \frac{x_0}{\lambda_r}\ \mbox{sinc}
(\omega_{hk}) + (-1)^k\ \frac{\lambda_r - x_0}{\lambda_r}\ 
\mbox{sinc}(k \pi - \omega_{hk})\nonumber\\
=&& \sin \omega_{hk}\ \left[\frac{k \pi x_0 - 
\omega_{hk} \lambda_r}{\omega_{hk} \lambda_r (k \pi - \omega_{hk})} \right],
\end{eqnarray}
where

\begin{eqnarray}
\label{whk}
\omega_{hk} =&& \frac{1}{2} (q_x^{hk} x_0 + q_z^{hk} A)\nonumber\\
=&& \pi\ \left[ \frac{h A}{d} + \frac{k}{\lambda_r}\ (x_0 -
\frac{A}{\tan \gamma}) \right].
\end{eqnarray}

The q-space pattern of $|F_C(h,k)|^2$ is shown in Fig. \ref{q_pattern}. 
The two maximum scattering 
directions and the uneven intensity distribution among them can be seen 
clearly, as well as their
relations to the ripple profile. This result agrees well with the qualitative 
features of the x-ray diffraction patterns reported in 
\cite{Ale85,Wac89a,Kat95}. 

%---------------------------------------------------------------------------
\subsection{Trans-bilayer electron density profile $T(x,z)$}

Because of the bilayer nature of the lipid water system, it is natural 
to start with a simple, symmetric transbilayer electron density profile. 
In their extensive study of gel phase DPPC bilayers using the electron
density modeling approach, Wiener, Suter and Nagle \cite{WSN89} found
out that as long as the essential structural features of the bilayers
were incorporated into the model, the precise functional form of the
electron density model did not seem to effect the final results much.
Following this finding, we model the two opposite headgroup regions to be 
two identical positive $\delta$-functions, with amplitude $\rho_H$, and
the central methyl trough to be a negative $\delta$-function, with amplitude
$\rho_M$. Because the x-rays are only sensitive to the contrast in the
electron density, we follow the minus-fluid convention by considering
only the relative electron density with the constant water density
subtracted. Since the methylene region of the bilayers has similar
electron density as the water, we treat it as zero.

In order to avoid possible conflict between adjacent rippled bilayers,
the two opposing monolayers of the same bilayer in this model are allowed 
to shift by a angle $\psi$ (See Fig. \ref{trans}). The trans-bilayer 
electron density profile, $T(x,z)$, can be expressed as:

\begin{eqnarray}
\label{transbi}
T(x,z) = \delta (x + z \tan \psi)\ \left\{ \rho_H [\delta (z-X_h\cos \psi)
+\delta (z+X_h \cos \psi)]-\rho_M \delta (z) \right\},
\end{eqnarray}
where $\rho_H$ and $\rho_M$ are defined as above and 
$X_h$ is the distance from the headgroup to the center of bilayer along
the line connecting the deflection points of the two opposing monolayers
(See Fig. \ref{trans}). Then the form factor due to this trans-bilayer 
electron density profile is:

\begin{eqnarray}
\label{form_trans}
F_T (\vec{q}) =&& \int_{- \infty}^{\infty} dx \int_{- \infty}^{\infty} dz
\ T(x,z) \ \mbox{e}^{i \vec{q} \cdot \vec{r}} \nonumber\\
=&& 2 \rho_H\ \cos (q_z X_h \cos \psi - q_x X_h \sin \psi) - \rho_M \nonumber\\
=&& \rho_M\ [R_{HM} \cos (q_z X_h \cos \psi - q_x X_h \sin \psi) -1],
\end{eqnarray}
where \(R_{HM} = \frac{2\ \rho_H}{ \rho_M}\). $F_T (\vec{q})$ is also
a real quantity, as $F_C (\vec{q})$, due to the inversion symmetry of the
trans-bilayer electron density profile relative to the center of the bilayer.
$R_{HM}$, $\psi$ and $X_h$ are the independent parameters which will enter
the model fit. The discrete form of Eqn.\ref{form_trans} is:

\begin{eqnarray}
\label{dis_trans}
F_T (h,k) = \rho_M\ [R_{HM} \cos (q_z^{hk} X_h \cos \psi 
		- q_x^{hk} X_h \sin \psi) -1],
\end{eqnarray}
where:

\begin{eqnarray}
\label{qxzhk}
\left\{ \begin{array}{ll}
q_x^{hk} =\frac{2 \pi k}{\lambda_r} \\
q_z^{hk} = 2 \pi\ \left( \frac{h}{d} - \frac{k}{\lambda_r \tan \gamma} \right).
\end{array}
\right.
\end{eqnarray}

%----------------------------------------------------------------------------
\subsection{Overall form factor and the scattering intensity}

Combining Eqns.\ref{formall},\ref{discrete},\ref{dis_trans}, the overall 
form factor for the (h,k) peak is:

\begin{eqnarray}
\label{form_dis}
F(h,k) =&& F_C (h,k) \times F_T (h,k) \nonumber\\
=&& \sin \omega_{hk} \left[\frac{k \pi x_0 -
\omega_{hk} \lambda_r}{\omega_{hk} \lambda_r (k \pi - \omega_{hk})} \right]
 \rho_M\ [R_{HM} \cos (q_z^{hk} X_h \cos \psi - q_x^{hk} X_h \sin \psi) -1].
\end{eqnarray}
$F(h,k)$ is real because the overall electron density funtion has the 
inversion symmetry as can be seen from the last three subsections.
The overall scattering intensity of the (h,k) peak is:

\begin{eqnarray}
\label{inten_dis}
I(h,k) =&& |F(h,k)|^2 \nonumber\\
=&& \rho_M^2\ \sin^2 \omega_{hk}\ \left[\frac{k \pi x_0 -
\omega_{hk} \lambda_r}{\omega_{hk} \lambda_r (k \pi - \omega_{hk})} \right]^2\ 
[R_{HM} \cos (q_z^{hk} X_h \cos \psi - q_x^{hk} X_h \sin \psi) -1]^2.
\end{eqnarray}

There will be six independent parameters when the non-linear least square
fit is performed. Two of them are related to the ripple profile, namely
the ripple amplitude $A$ (contained in $\omega_{hk}$, see Eqn.\ref{whk}) and
the horizontal projection length of the major side of the ripple, $x_0$. The
trans-bilayer part contains three, the headgroup to methyl trough electron
density ratio, $R_{HM}$, the monolayer shift angle $\psi$ and the headgroup 
position $X_h$. The sixth parameter is an overall scaling factor for $F(h,k)$ 
or $I(h,k)$ when actually fitting to the data. 
 
%----------------------------------------------------------------------------
\subsection{Fourier reconstruction of electron density map}

In order to reconstruct the electron density maps of the ripple phase
from the experimentally measured amplitudes of the form factors $|F(h,k)|$, 
correct phases have to be found out first. There are two ways of getting 
the phases, as discussed in the introduction section. The first is to
select the ``right'' one from all possible combinations. This approach
is difficult when the number of peaks is large. The second approach,
which we follow here, is to get phases from model fit. As can be seen
from the previous sections, the phases come out as outputs of our model.

The formula for reconstructing the electron density maps
from discrete form factors is derived as follows. It is easy to see that 
the general form factor satisfies 

\begin{eqnarray}
\label{formallcon2}
F(-\vec{q}) = \ F^{*}(\vec{q}),
\end{eqnarray}
as it should be because $F(\vec{q})$ is the Fourier transform of a real
function (the electron density).
The electron density is the inverse Fourier transform of $F(\vec{q})$
(keep in mind that we are dealing with 2D structures here, so that
\( \vec{r} = x \hat{x} + z \hat{z}, \vec{q} = q_x \hat{x} + q_z \hat{z}\))
and is shown as follows

\begin{eqnarray}
\label{edrecon}
\rho(\vec{r}) =&& \int d \vec{q}\ F(\vec{q})\ 
\mbox{e}^{- i \vec{q} \cdot \vec{r}}\nonumber\\
=&& \int_{- \infty}^{\infty} dq_x \int_{- \infty}^{\infty} dq_z
F(\vec{q})\ \mbox{e}^{- i \vec{q} \cdot \vec{r}}\nonumber\\
=&& \int_{- \infty}^{\infty} dq_x \int_{0}^{\infty} dq_z
\left[ F(\vec{q})\ \mbox{e}^{- i \vec{q} \cdot \vec{r}} +
F(-\vec{q})\ \mbox{e}^{i \vec{q} \cdot \vec{r}}\right]\nonumber\\
=&& \int_{- \infty}^{\infty} dq_x \int_{0}^{\infty} dq_z
\left[ F(\vec{q})\ \mbox{e}^{- i \vec{q} \cdot \vec{r}} +
F^{*}(\vec{q})\ \mbox{e}^{i \vec{q} \cdot \vec{r}}\right]\nonumber\\
=&& \int_{- \infty}^{\infty} dq_x \int_{0}^{\infty} dq_z
\ 2\ Re \left[ F(\vec{q})\ \mbox{e}^{- i \vec{q} \cdot \vec{r}}\right]\nonumber\\
=&& 2\ \int_{- \infty}^{\infty} dq_x \int_{0}^{\infty} dq_z
\ F(\vec{q}) \cos( \vec{q} \cdot \vec{r}).
\end{eqnarray}
In the last step, we used the result that $F(\vec{q})$ is real.

Ignoring the irrelevant constant prefactor, the discrete form for 
the relative electron density $\rho(\vec{r})$ is 

\begin{eqnarray}
\label{discretell}
\rho(x,z) = \sum_{h \geq 0} \sum_{k} F(h,k) \ \cos(q^{hk}_x x + q^{hk}_z z).
\end{eqnarray}
Eqn.\ref{discretell} will be the formula used to reconstruct the electron 
density maps presented in this paper. 

\pagebreak
