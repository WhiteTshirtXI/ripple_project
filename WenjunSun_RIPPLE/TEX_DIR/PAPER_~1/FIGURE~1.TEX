%------------ Figures and Tables ----------------------------
\begin{figure}
\centerline {\psfig{figure=lattice_dim.eps,height=5in}}
\vspace{11pt}
\caption{(i). Ripple unit cell lattice parameters; (ii). The real space and
the reciprocal space unit cells, and the 2D coordinate system used in this 
paper.
\label{unitcells}}
\end{figure}

\pagebreak

\begin{figure}
\centerline {\psfig{figure=ripple_profile.eps,height=5in}}
\vspace{11pt}
\caption{(i). Parameters describing the ripple profile; (ii). 
The origin of the coordinate system is chosen to be at the center
of the major side, to utilize the inversion symmetry. The ripple unit
cell is shown as indicated.
\label{rppl_profile}}
\end{figure}

\pagebreak

\begin{figure}
\centerline {\psfig{figure=q_profile.ps,height=6in}}
\vspace{11pt}
\caption{Real space ripple structure (only the ripple profile part) and its 
corresponding q-space pattern. The dot-dashed lines show the normals
of the major and minor sides. Notice that the two maximum
scattering directions are along the two normals, and that
the scattering along the normal of the major side is stronger than
along the normal of the minor side. Darker spots represent stronger
scatterings.
\label{q_pattern}}
\end{figure}

\pagebreak

\begin{figure}
\begin{center}
\leavevmode
\raggedleft 
\hspace{0.2in}
\psfig{figure=psi1.ps,height=2.5in}
\leavevmode
\raggedright 
\hspace{0.3in}
\psfig{figure=psi2.ps,height=2.5in}
\end{center}
\vspace{-0.2in}
\hspace{1.5in} (A) \hspace{2.9in} (B)
\vspace{0.2in}
\caption{Illustrations of the trans-bilayer structure. The monolayer
shift angle $\psi$ and headgroup-to-center distance X$_h$ are defined as
indicated. The halftone stripes denote the bilayers; the thick black lines 
represent the central methyl troughs of the rippled bilayers; the white spaces
between the adjacent bilayers are the water regions. The thin solid
lines represent the vertical direction and the dot-dashed lines are the
connecting lines of the deflection points of the upper and lower monolayers
of the central bilayers. (A) A reasonable case of $\psi$; (B) an unlikely case 
of $\psi$.
\label{trans}}
\end{figure}

\pagebreak

\begin{figure}
\centerline {\psfig{figure=wack_data_model3.ps,height=6in}}
\vspace{11pt}
\caption{Replotting of Wack and Webb's data [17] in q space and the 
real space
ripple structure obtained by fitting that data to our model (structural
quantities and fitted form factors obtained from the fit 
are shown in Tables \ref{parameter} and \ref{formfactor}, respectively). 
%The halftone stripes denote the bilayers; The thick black lines represent
%the central profiles of the rippled bilayers; The white space
%between the two adjacent bilayers is the water region. 
The correlation between the normals of the two sides of the ripples and the 
angular scattering intensity distribution in q space is shown by the
dot-dashed lines. Ripple unit cell lattice parameters are: d = 57.94\AA, 
$\lambda_r$ = 141.7\AA, $\gamma$ = 98.4$^{\circ}$. The sample was DMPC with 
25\% water by weight at 18$^{\circ}$C.
\label{wack_data}}
\end{figure}

\pagebreak

\begin{table}
\caption{Ripple structural quantities[obtained from the model fit$^{(1)}$.
\label{parameter}}
\vspace{6pt}
{\tabcolsep=0.35in
\begin{tabular}{c|c||c|c|c} 
\multicolumn{2}{c||}{Primary parameters} & 
\multicolumn{3}{c}{Secondary parameters} \\ \hline
Amplitude & 18.6(7)\AA & & major side & minor side \\ \cline{4-5}
x$_0$ & 103(3)\AA & length & 104.6\AA & 42.9\AA \\
$\psi$ & 5(2)$^{\circ}$ & slope & 10.2$^{\circ}$ & 25.7$^{\circ}$ \\
R$_{hm}$ & 2.2(2) & d$_l^{(2)}$ & 40.0\AA & 34.6\AA \\
X$_h$ & 20.1(2)\AA & d$_w^{(3)}$ & 18.5\AA & 13.9\AA \\
\end{tabular}
}
\vspace{5pt}
{\small
\indent (1) The sample was DMPC with 25\% water by weight at 
		18$^{\circ}$C [17]. Ripple unit cell lattice parameters 
	are: d = 57.94\AA, $\lambda_r$ = 141.7\AA, and $\gamma$ = 
	98.4$^{\circ}$, also from [17]. \\
\indent (2)  d$_l$: Bilayer head-head thickness along local normal
		directions. \\
\indent (3)  d$_w$: Water spacing along local normal
		directions.
}
\end{table}

\pagebreak

\begin{table}
\caption{Experimental absolute form factors and model
fit form factors (both amplitudes \hspace{1in} and phases).
\label{formfactor}}
\vspace{6pt}
{\tabcolsep=0.2in
\begin{tabular}{lrccc} 
h & k & Wack \& Webb[17] &
	\multicolumn{2}{c}{Model fit} \\ \cline{4-5}
& & $|$F(h,k)$|$ & $|$F(h,k)$|$ & phase \\ \hline
3  &     -2   &           29.33   &        22.30   &        + \\
3  &     -1   &           44.18   &        42.20   &        + \\
3  &      0   &           12.03   &         2.22   &        + \\
3  &      2   &           10.48   &        14.93   &        + \\
3  &      3   &           14.87   &        15.03   &        $-$ \\
3  &      4   &            9.99   &        12.51   &        + \\
2  &     -2   &           15.06   &         9.00    &       $-$ \\
2   &    -1  &            71.22   &        66.92    &       $-$ \\
2   &     0    &          39.69   &        41.02   &        $-$ \\
2  &      1     &         33.91 &          31.14   &        + \\
2  &      2    &          22.70   &        24.27    &       $-$ \\
2    &    3     &         14.16    &       17.11    &       + \\
2   &     4     &          7.78    &       10.27     &      $-$ \\
1    &   -1       &       60.83    &       41.84      &     $-$ \\
1  &      0     &        100.00    &       96.70   &        $-$ \\
1  &      1       &       26.86         &  38.12       &    + \\
1   &     3        &       7.61        &   13.86        &   + \\
0    &    1         &      5.33          &  1.40         &  $-$ \\
0     &   2          &     9.68          &  0.98        &   + \\
0      &  3           &    7.81         &   0.44         &  $-$ \\
\end{tabular}
}
\end{table}

\pagebreak

\begin{figure}
\begin{center}
\leavevmode
\raggedleft 
\hspace{0.5in}
\psfig{figure=wack_map.ps,height=2.5in}
\leavevmode
\raggedright 
\hspace{0.5in}
\psfig{figure=model3_map.ps,height=2.5in}
\end{center}
\vspace{-0.2in}
\hspace{1.5in} (A) \hspace{2.9in} (B)
\vspace{0.2in}
\caption{Electron density maps of the ripple phase. The map shown in (A)
is reconstructed by combining Wack and Webb's data and the phases obtained from
the model fit (See Table \ref{formfactor}); the map shown in (B) is 
reconstructed from 
the model fit form factors (See Table \ref{formfactor}) by using the 
same peaks reported by Wack and Webb. The widths of 
the images are both 2$\lambda_r$ = 283.4\AA. Brightness corresponds 
to higher electron density, where the narrow bright bands are the headgroups, 
and the wide black bands are the hydrocarbon chain regions and the narrow black 
bands are the water regions.
\label{wack_map}}
\end{figure}

\pagebreak

\begin{figure}
\begin{center}
\leavevmode
\raggedleft
%\hspace{0.5in}
\psfig{figure=slice_diagram.ps,height=2.0in}
\leavevmode
\raggedright
\hspace{0.35in}
\psfig{figure=slice1.ps,height=2.0in}
\end{center}
\begin{center}
\leavevmode
\raggedleft
%\hspace{0.5in}
\psfig{figure=slice2.ps,height=2.0in}
\leavevmode
\raggedright
\hspace{0.3in}
\psfig{figure=slice4.ps,height=2.0in}
\end{center}
\caption{(A) Depiction of the ripple bilayers, with three slices along
the normals denoted by B, C and D. The white and black dots are the centers
of the three slices. The electron density profiles along 
these three slices taken from the two maps shown in Fig. \ref{wack_map}
are as follows: (B) Two-bilayer profiles along slice B. The bilayer head-head 
spacing is 38\AA, and the water spacing is 21\AA; (C) two-bilayer profiles 
along slice C. The bilayer head-head spacing is 31\AA; (D) two-bilayer 
profiles along slice D. The water spacing is 18\AA. Solid lines are from 
Fig. \ref{wack_map}(A) (Wack and Webb data map) and dashed lines are from 
Fig. \ref{wack_map}(B) (model fit map).
\label{slice}}
\end{figure}

\pagebreak

\begin{figure}
\begin{center}
\leavevmode
\psfig{figure=gel_rppl.ps,height=4in}
\end{center}
\caption{Comparison of the gel phase and ripple phase electron density
profiles. Solid line: The major side slice electron density profile
from the data map (Fig. \ref{wack_map}(A)), identical to the solid line
in Fig. \ref{slice}(B). Dashed line: the gel phase electron density
profile reconstructed using three orders of low angle peaks, in accord
with the three lamellar orders reported in Wack and Webb's data. The gel
phase sample was DMPC with 40\%PVP at 6$^{\circ}$C, with a lamellar
d-spacing of 57.75\AA. 
%This sample at 18$^{\circ}$C was in the ripple
%phase and gave a d-spacing of 58.10\AA, comparing well with Wack and
%Webb's d-spacing of 57.94\AA\ for DMPC with 25\% water (by weight) at
%18$^{\circ}$C (in the ripple phase). From this we concluded that
%these two samples are at about the same hydration level, thus validating
%the comparison of the two profiles.
\label{gel_rppl}}
\end{figure}

\pagebreak

\begin{figure}
\begin{center}
\leavevmode
\raggedleft
\hspace{0.5in}
\psfig{figure=tardieu_map.ps,height=2.5in}
\leavevmode
\raggedright
\hspace{0.5in}
\psfig{figure=janiak_map1.ps,height=2.5in}
\end{center}
\vspace{-0.2in}
\hspace{1.5in} (A) \hspace{2.9in} (B)
%\vspace{0.2in}
\begin{center}
\leavevmode
\raggedleft
\hspace{0.5in}
\psfig{figure=janiak_map2.ps,height=2.5in}
\leavevmode
\raggedright
\hspace{0.5in}
\psfig{figure=random_map.ps,height=2.5in}
\end{center}
\vspace{-0.2in}
\hspace{1.5in} (C) \hspace{2.9in} (D)
\vspace{0.2in}
\caption{Electron density maps of the ripple phase. (A) Reconstructed
from the F(h,k) results in Tardieu et al. [1]. The sample was DLPC with 23\%
water (by weight) at -7$^{\circ}$C. The width of the image is 
2$\lambda_r$=170.6\AA. (B) Reconstructed from the F(h,k) results in Janiak et 
al. [3]. The sample was DMPC with 21\% water (by weight) at 20$^{\circ}$C. 
The width of the image is 2$\lambda_r$=321.8\AA. (C) Reconstructed from the 
F(h,k) results in Janiak et al. [3]. The sample was DMPC with 30\% water 
(by weight) at 20$^{\circ}$C. The width of the image is 2$\lambda_r$=236.0\AA. 
(D). Reconstructed from the $|$F(h,k)$|$ data in Wack and Webb [17] with random 
phase assignments. The sample was DMPC with 25\% water (by weight) at 
18$^{\circ}$C. The width of the image is 2$\lambda_r$=283.4\AA. Brightness 
corresponds to higher electron density.
\label{others}}
\end{figure}
