% ********************** abst.tex ***********************************
\begin{abstract}

A 2D electron density model is proposed for the ripple (P$_{\beta '}$)
phase in the lecithin/water bilayer system. The model is constructed by 
convoluting a ripple profile function with a trans-bilayer electron
density profile. Both good qualitative and quantitative agreements are
achieved when fitting experimental low-angle x-ray scattering intensity 
data into this model. Accurate ripple structures, which includes the ripple 
amplitude, the slopes of both the major and the minor sides of the ripple, 
the degree of ripple asymmetry, the bilayer head-head spacings and
the water spacings are obtained from the fitting. Correct phases of the
form factors are obtained as outputs of the model fitting and used
in reconstructing the electron density maps. The advantage of this model 
over the ``pattern recognition'' method, which is done by finding the 
correct phase combination out of all possible combinations of 20 or so 
peaks, is quite obvious. Furthermore, this model provides a more 
direct, quantitative way of obtaining the degree of ripple asymmetry,
compared to the qualitative connection between unit cell obliquity and
ripple asymmetry made by other groups.
%This in turn leads to more 
%precise determinations of other key structural parameters of the ripple phase, 
%like the area per lipid molecule, inter-bilayer water spacing and the degree 
%of ripple asymmetry. 

\end{abstract}
%\pacs{87.20.B, 87.15, 87.64.B, 78.70.C, 61.30.E, 82.65.F}

\pagebreak
%\twocolumn
