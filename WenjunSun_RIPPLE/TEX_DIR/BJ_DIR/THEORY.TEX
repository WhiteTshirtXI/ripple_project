\section{METHODS}

%----------------------------------------------------------------------------
The modeling approach first creates a functional form for the electron density 
that incorporates known and plausible properties of the bilayer but that
contains free parameters to incorporate unknown detailed structure.
The success of the method depends on this modelling step which requires some care
and is therefore the emphasis of this section.
Then, the values of the free parameters in the real space model are 
determined by routine non-linear least squares procedures so that the Fourier
components of the model provide the best fit to the measured intensities.
This last step automatically determines the phases of the model, which are
then used with the original intensity data to produce an electron density map.

\subsection{Lattice structure}

It has been shown from x-ray studies \cite{Tar73,JanSS79,Ino80,Ale85,%
Wac89a,Kat95} that ripples in different bilayers are registered to form a 
2-D oblique lattice as shown by the unit cell in Fig. 1.
The unit cell vectors can be expressed (see Fig. 1) as
\begin{eqnarray}
\label{realunit}
%\left\{ \begin{array}{ll}
\vec{a} = d \cot \gamma\ \hat{x} + d\ \hat{z} \ \mbox{and}\ \vec{b} = \lambda_r\ \hat{x}.
%\end{array}
%\right.
\end{eqnarray}
The corresponding reciprocal lattice unit cell vectors (see Fig. 1) are 
\begin{eqnarray}
\label{reciunit}
%\left\{ \begin{array}{ll}
\vec{A} = \frac{2 \pi}{d}\ \hat{z}\ \mbox{and}\ \vec{B} = \frac{2 \pi}{\lambda_r}\ \hat{x} - \frac{2 \pi}{\lambda_r \tan \gamma}\ \hat{z}.
%\end{array}
%\right.
\end{eqnarray}
so the q-space vector $\vec{q}_{hk}$ for the Bragg peak with Miller indices $(h,k)$
is:
\begin{eqnarray}
\label{qvector}
\vec{q}_{hk} =&& h \vec{A} + k \vec{B} ,
%=&& \frac{2 \pi k}{\lambda_r}\ \hat{x} + \left( \frac{2 \pi h}{d} - \frac{2 \pi k}{\lambda_r \tan \gamma} \right) \hat{z},
\end{eqnarray}
with some examples shown in Fig. 1.
%----------------------------------------------------------------------------
\subsection{Electron density model and form factor}

The ripple profile is the line z=u(x) that describes the center of the bilayer 
in the (x,z) plane, such as the one shown in Fig. 1.  
The closely related contour function is defined as
\begin{eqnarray}
\label{contour}
C(x,z) = \delta [z - u(x)].
\end{eqnarray}
The transbilayer electron density function $T_{\psi}(x,z)$ expresses the
actual electron density at points (x,z) that lie on a straight line with
slope $\psi$ from the vertical as shown in Fig. 1.  An underlying electron
density profile will be chosen similar to models for flat bilayer phases.
Then, the electron density model for $\rho (x,z)$ within
the unit cell is described as the convolution of a ripple contour
function $C(x,z)$ and the trans-bilayer electron density profile $T_{\psi}(x,z)$:
\begin{eqnarray}
\label{model}
\rho (x,z) = C(x,z) \ast T_{\psi}(x,z).
\end{eqnarray}
This convolution moves the transbilayer electron density function so that
it is centered on the ripple profile.

The form factor $F(\vec{q})$ is the Fourier transform of the electron density
expressed in Eqn.\ref{model} and is the product
\begin{eqnarray}
\label{formall}
F(\vec{q}) =&& F_C(\vec{q}) F_T(\vec{q}),
\end{eqnarray}
of the Fourier transform $F_{C}(q)$ of $C(x,z)$ and of the Fourier transform
$F_{T}(q)$ of $T_{\psi}(x,z)$.

%----------------------------------------------------------------------------
\subsection{Ripple profile}

The results of freeze fracture electron microscopy \cite{Luna77,Cop80,Rup83} and
scanning tunneling microscopy \cite{Zas88a,Hata93} strongly suggest that the
ripple profile is asymmetric.  This conclusion is further supported by 
the result
of x-ray diffraction that the ripple unit cell is oblique
($\gamma$ not equal to 90$^{\circ}$) 
\cite{Tar73,JanSS79,Ino80,Ale85,Sir88,Wac89a,Kat95}.
For our model ripple profile
u(x) we choose a triangular shape as shown in
Fig. 1.  We shall call the longer side of the ripple
the {\it major} side, and the shorter side the {\it minor} side.
The triangular ripple profile is completely described by the amplitude
$A$ and $x_0$, the projection of the major side onto the $x$ axis,
with obvious relations for other quantities such as the slope angles
of the major and minor sides.  Choosing the origin at the center
of the unit cell shown in Fig. 1 gives the ripple profile
\begin{eqnarray}
\label{profile}
u(x) = \left\{ \begin{array}{ll}
		-\frac{A}{\lambda_r - x_0} (x + \frac{\lambda_r}{2})
\ \ \ \mbox{if}\ - \frac{\lambda_r}{2} \le x < - \frac{x_0}{2};\\
		\ \ \ \ \ \frac{A}{x_0}\ x\ \ \ \ \ \ \ \ \ \ \ \ \mbox{if}
\ \ - \frac{x_0}{2} \le x \le 
\frac{x_0}{2};\\
		-\frac{A}{\lambda_r - x_0} (x - \frac{\lambda_r}{2})
\ \ \ \mbox{if}\ \ \ \ \  \frac{x_0}{2} < x \le \frac{\lambda_r}{2}.
		\end{array}
	\right.
\end{eqnarray}
This particular choice shows explicitly the inversion
symmetry in the ripple profile, so that the form factors are real.

Using Eqn.\ref{profile} the contour part of the form factor is
\begin{eqnarray}
\label{formcon}
F_C(\vec{q}) =&& \frac{1}{\lambda_r}\ \int_{- \frac{\lambda_r}
{2}}^{\frac{\lambda_r}{2}} dx\ \mbox{e}^{i [q_x x + q_z u(x)]}\nonumber \\ 
=&& \frac{x_0}{\lambda_r}\ \mbox{sinc}(\omega)
+ \frac{\lambda_r - x_0}{\lambda_r}\ \frac{\cos \frac{1}{2} (\frac{1}{2}
q_x \lambda_r + \omega)}{\cos \frac{1}{2} (\frac{1}{2} q_x \lambda_r -
\omega)}\nonumber\\ 
&& \times\ \mbox{sinc}(\frac{q_x \lambda_r}{2} - \omega ),
\end{eqnarray}
where the intermediate variable $\omega$ is defined as
\begin{eqnarray}
\label{w}
\omega (\vec{q}) = \frac{1}{2}\ ( q_x x_0 + q_z A).
\end{eqnarray}
%Combining Eqn.\ref{qvector} and Eqn.\ref{formcon}, the discrete contour form
%factor for the Bragg peak with Miller indices $(h,k)$ can be expressed as:
%\begin{eqnarray}
%\label{discrete}
%F_C(h,k) =&& \frac{x_0}{\lambda_r}\ \mbox{sinc}
%(\omega_{hk}) + (-1)^k\ \frac{\lambda_r - x_0}{\lambda_r}\ 
%\mbox{sinc}(k \pi - \omega_{hk})\nonumber\\
%=&& \sin \omega_{hk}\ \left[\frac{k \pi x_0 - 
%\omega_{hk} \lambda_r}{\omega_{hk} \lambda_r (k \pi - \omega_{hk})} \right],
%\end{eqnarray}
%where
%\begin{eqnarray}
%\label{whk}
%\omega_{hk} =&& \frac{1}{2} (q_x^{hk} x_0 + q_z^{hk} A)\nonumber\\
%=&& \pi\ \left[ \frac{h A}{d} + \frac{k}{\lambda_r}\ (x_0 -
%\frac{A}{\tan \gamma}) \right].
%\end{eqnarray}
The first term on the right hand side of Eqn.\ref{formcon} is the contribution 
from the major side of the ripple, with scattering amplitude
\(\frac{x_0}{\lambda_r}\), proportional to its projected length onto the
x axis, while the last term is the contribution from the 
minor side of the ripple, with amplitude \(\frac{\lambda_r - x_0}{\lambda_r}\).
Eqn.\ref{formcon} also shows that the maximum scattering 
from the major side occurs along that $\vec{q}$ direction determined by
$\omega = 0$ in Eqn.\ref{w}.  In real space this is also the direction normal to
the major side of the ripple.  Similarly, the maximum scattering
direction of the minor side is given by \( \frac{q_x \lambda_r}{2} - \omega 
= 0 \), and this can be shown to be the direction normal to the minor side.
These directions are illustrated by the dashed lines in Fig. 1.  
This theoretical picture for the location of the strong (h,k) peaks
agrees qualitatively with the experimental intensity pattern of Wack and
Webb \cite{Wac89a} shown in Fig. 1 as well as other x-ray data
\cite{Ale85,Kat95}.  This general pattern would also be expected to
pertain to ripple profiles that are not perfectly triangular, for example,
with rounded corners.

%---------------------------------------------------------------------------
\subsection{Trans-bilayer electron density profile $T_{\psi}(x,z)$}

In a study of gel phase DPPC bilayers using the electron
density modeling approach \cite{WSN89} it was found
that the precise functional form of the electron density model did not
much affect the results provided that the essential structural
features of bilayers were incorporated into the model.
We therefore first utilized a very simple model for $T_{\psi}(x,z)$ in
which the two opposite headgroup regions consist of
two identical positive $\delta$-functions, with amplitude $\rho_H$ and separated
by the head-head spacing $2X_h$, and
the central methyl trough consists of a negative $\delta$-function, with amplitude
$\rho_M$.  The methylene region of bilayers has an electron density
close to that of water and hence is set equal to zero in the
following model (minus fluid \cite{WKMc73}) of $T_{\psi}(x,z)$,
\begin{eqnarray}
\label{transbi}
T_{\psi}(x,z) =&& \delta (x + z \tan \psi)\ \{ \rho_H 
	[\delta (z-X_h\cos \psi) \nonumber\\
&& +\delta (z+X_h \cos \psi)]-\rho_M \delta (z) \}.
\end{eqnarray}
The form factor corresponding to this trans-bilayer electron density profile is
\begin{eqnarray}
\label{form_trans}
F_T (\vec{q}) =&& \rho_M\ [R_{HM} \cos (q_z X_h \cos \psi - 
q_x X_h \sin \psi) -1],
\end{eqnarray}
where \(R_{HM} = \frac{2\ \rho_H}{ \rho_M}\).   We will call this the
simple delta function (SDF) model.
Following \cite{WSN89} we also utilized a more realistic model, called the 1G 
hybrid model that consists of Gaussian functions for the headgroups and
the terminal methyl trough and that also allows for differences between the electron
density of methylenes and of water.  We will call this the simple 1G (S1G) model.
We also considered more complex models that modulated the electron density
along the direction perpendicular to $\vec{a}$ in real space in Fig. 1; this is
also the direction $\vec{B}$ in reciprocal space, so this modulation strongly
affects the $(0,k)$ pure ripple form factors.  Our best ripple modulated 
models (MDF and M1G) had two additional parameters, one to allow for the 
electron density across the minor side to be different by a ratio $f_1$
from the electron density across the major side and a second parameter
$f_2$, which is multiplied by delta functions ${\delta}(x \pm x_0/2)$ to 
allow for a
different electron density near the kink between the major and the minor sides.

%----------------------------------------------------------------------------
\subsection{Determination of the phases and the electron density map}

Combining Eqns. \ref{formall}, \ref{formcon} and \ref{form_trans}, the 
Lorentz-corrected scattering intensity for the (h,k) reflection of the 
SDF model is
\begin{eqnarray}
\label{inten_dis}
I(h,k) && = |F(h,k)|^2 
= \rho_M^2\ \sin^2 \omega_{hk}\ \left[\frac{k \pi x_0 -
\omega_{hk} \lambda_r}{\omega_{hk} \lambda_r (k \pi 
- \omega_{hk})} \right]^2 \nonumber\\
&& \times\ [R_{HM} \cos (q_z^{hk} X_h \cos \psi - q_x^{hk} X_h \sin \psi) -1]^2,
\end{eqnarray}
which contains six independent parameters.
Two parameters are related to the ripple profile, namely
the ripple amplitude $A$ and the horizontal projection length of the major side
of the ripple, $x_0$. The trans-bilayer profile contains three
parameters, the headgroup to methyl trough electron
density ratio, $R_{HM}$, the monolayer shift angle $\psi$ and 
$X_h$, the headgroup position relative to the center of the bilayer. 
The sixth parameter is a common scaling factor for all the
$I(h,k)$ peaks.  For the S1G model, all the trans-bilayer profile parameters
were taken from gel phase data \cite{WSN89} except for the headgroup position
$X_h$.   The MDF and M1G models had 
the two additional parameters
$f_1$ and $f_2$ mentioned above.
(The parameters ${\lambda}_r$, d and $\gamma$ are known
from indexing the (h,k) peaks.)
Standard non-linear least squares fitting procedures were employed to
obtain unknown parameters by finding the best
fit of the intensities given by Eqn.\ref{inten_dis} to the $|F(h,k)|^2$ reported
by Wack and Webb \cite{Wac89a}. 
This procedure does not require knowing any phases in advance.  However,
once the best values of the parameters in the electron density model are
determined, the form factors, and especially the phases, are determined.
Using these phases and the amplitudes of the form factors $|F(h,k)|$ from 
the intensity data \cite{Wac89a},
an experimental relative electron density map $\rho(\vec{r})$ 
is obtained using
\begin{eqnarray}
\label{discretell}
\rho(x,z) = \sum_{h \geq 0} \sum_{k} F(h,k) \ \cos(q^{hk}_x x + q^{hk}_z z).
\end{eqnarray}
